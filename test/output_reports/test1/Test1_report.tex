\documentclass[12pt, fleqn]{article}
\usepackage{pgfplots}
\usepackage{bm}
\usepackage{marginnote}
\usepackage{wallpaper}
\usepackage{lastpage}
\usepackage[left=1.3cm,right=2.0cm,top=1.8cm,bottom=5.0cm,marginparwidth=3.4cm]{geometry}
\usepackage{amsmath}
\usepackage{amssymb}
\usepackage{xcolor}
\usepackage{enumitem}
\usepackage{float}
\usepackage{textgreek}
\usepackage{textcomp}
\usepackage{fancyhdr}
\usepackage{graphicx}
\usepackage{pstricks}
\usepackage{subfigure}
\usepackage{caption}
\captionsetup{justification=centering,labelfont=bf, belowskip=12pt,aboveskip=12pt}
\usepackage{textcomp}
\setlength{\headheight}{70pt}
\setlength{\textfloatsep}{12pt}
\setlength{\intextsep}{0pt}
\pagestyle{fancy}\fancyhf{}
\renewcommand{\headrulewidth}{0pt}
\definecolor{darkBlue}{cmyk}{.80, .32, 0, 0}
\setlength{\parindent}{0cm}
\newcommand{\tab}{\hspace*{2em}}
\newcommand\BackgroundStructure{
\setlength{\unitlength}{1mm}
\setlength\fboxsep{0mm}
\setlength\fboxrule{0.5mm}
\put(10, 20pr){\fcolorbox{black}{gray!5}{\framebox(155,247){}}}
\put(165, 20){\fcolorbox{black}{gray!10}{\framebox(37,247){}}}
\put(10, 262){\fcolorbox{black}{white!10}{\framebox(192, 25){}}}
\put(175, 263){\includegraphics{}}}
\setlength{\abovedisplayskip}{0pt}
\setlength{\belowdisplayskip}{0pt}
%	----------------------------------- HEADER -----------------------------------
\fancyhead[L]{\begin{tabular}{l l | l l}
\textbf{Member:} & {Test1} & \textbf{Firm:} & {ABC Company} \\
\textbf{Project:} & {123 Maple Street, San Francisco CA} & \textbf{Engineer:} & {Jesse} \\
\textbf{Level:} & {2} & \textbf{Checker:} & {Joey}  \\
\textbf{Date:} & {2021-05-23} & \textbf{Page:} & \thepage\\
\end{tabular}}
%	---------------------------- APPLIED LOADS SECTION ---------------------------
\begin{document}
\begin{center}
\textbf{\LARGE W12x14 Design Report}
\end{center}
\section{Applied Loading}
\vspace{-30pt}
\begin{figure}[H]
\begin{center}
%% Creator: Matplotlib, PGF backend
%%
%% To include the figure in your LaTeX document, write
%%   \input{<filename>.pgf}
%%
%% Make sure the required packages are loaded in your preamble
%%   \usepackage{pgf}
%%
%% Figures using additional raster images can only be included by \input if
%% they are in the same directory as the main LaTeX file. For loading figures
%% from other directories you can use the `import` package
%%   \usepackage{import}
%%
%% and then include the figures with
%%   \import{<path to file>}{<filename>.pgf}
%%
%% Matplotlib used the following preamble
%%
\begingroup%
\makeatletter%
\begin{pgfpicture}%
\pgfpathrectangle{\pgfpointorigin}{\pgfqpoint{8.000000in}{3.000000in}}%
\pgfusepath{use as bounding box, clip}%
\begin{pgfscope}%
\pgfpathrectangle{\pgfqpoint{1.000000in}{0.825864in}}{\pgfqpoint{6.200000in}{1.814136in}}%
\pgfusepath{clip}%
\pgfsetbuttcap%
\pgfsetmiterjoin%
\definecolor{currentfill}{rgb}{0.121569,0.466667,0.705882}%
\pgfsetfillcolor{currentfill}%
\pgfsetfillopacity{0.400000}%
\pgfsetlinewidth{1.003750pt}%
\definecolor{currentstroke}{rgb}{0.121569,0.466667,0.705882}%
\pgfsetstrokecolor{currentstroke}%
\pgfsetstrokeopacity{0.400000}%
\pgfsetdash{}{0pt}%
\pgfpathmoveto{\pgfqpoint{1.281818in}{0.908325in}}%
\pgfpathlineto{\pgfqpoint{1.281818in}{2.557539in}}%
\pgfpathlineto{\pgfqpoint{6.918182in}{2.557539in}}%
\pgfpathlineto{\pgfqpoint{6.918182in}{0.908325in}}%
\pgfpathclose%
\pgfusepath{stroke,fill}%
\end{pgfscope}%
\begin{pgfscope}%
\pgfsetbuttcap%
\pgfsetmiterjoin%
\definecolor{currentfill}{rgb}{0.800000,0.800000,0.800000}%
\pgfsetfillcolor{currentfill}%
\pgfsetlinewidth{1.003750pt}%
\definecolor{currentstroke}{rgb}{0.000000,0.000000,0.000000}%
\pgfsetstrokecolor{currentstroke}%
\pgfsetdash{}{0pt}%
\pgfpathmoveto{\pgfqpoint{3.983970in}{1.636481in}}%
\pgfpathcurveto{\pgfqpoint{4.018692in}{1.601759in}}{\pgfqpoint{4.181308in}{1.601759in}}{\pgfqpoint{4.216030in}{1.636481in}}%
\pgfpathcurveto{\pgfqpoint{4.250753in}{1.671204in}}{\pgfqpoint{4.250753in}{1.794660in}}{\pgfqpoint{4.216030in}{1.829382in}}%
\pgfpathcurveto{\pgfqpoint{4.181308in}{1.864105in}}{\pgfqpoint{4.018692in}{1.864105in}}{\pgfqpoint{3.983970in}{1.829382in}}%
\pgfpathcurveto{\pgfqpoint{3.949248in}{1.794660in}}{\pgfqpoint{3.949248in}{1.671204in}}{\pgfqpoint{3.983970in}{1.636481in}}%
\pgfpathclose%
\pgfusepath{stroke,fill}%
\end{pgfscope}%
\begin{pgfscope}%
\definecolor{textcolor}{rgb}{0.000000,0.000000,0.000000}%
\pgfsetstrokecolor{textcolor}%
\pgfsetfillcolor{textcolor}%
\pgftext[x=4.100000in,y=1.732932in,,]{\color{textcolor}\rmfamily\fontsize{10.000000}{12.000000}\selectfont w\textsubscript{1}}%
\end{pgfscope}%
\begin{pgfscope}%
\pgfsetroundcap%
\pgfsetroundjoin%
\pgfsetlinewidth{1.003750pt}%
\definecolor{currentstroke}{rgb}{0.000000,0.000000,0.000000}%
\pgfsetstrokecolor{currentstroke}%
\pgfsetdash{}{0pt}%
\pgfpathmoveto{\pgfqpoint{3.697403in}{2.158833in}}%
\pgfpathquadraticcurveto{\pgfqpoint{3.697403in}{1.547451in}}{\pgfqpoint{3.697403in}{0.944962in}}%
\pgfusepath{stroke}%
\end{pgfscope}%
\begin{pgfscope}%
\pgfsetroundcap%
\pgfsetroundjoin%
\pgfsetlinewidth{1.003750pt}%
\definecolor{currentstroke}{rgb}{0.000000,0.000000,0.000000}%
\pgfsetstrokecolor{currentstroke}%
\pgfsetdash{}{0pt}%
\pgfpathmoveto{\pgfqpoint{3.780736in}{1.011629in}}%
\pgfpathlineto{\pgfqpoint{3.697403in}{0.944962in}}%
\pgfpathlineto{\pgfqpoint{3.614069in}{1.011629in}}%
\pgfusepath{stroke}%
\end{pgfscope}%
\begin{pgfscope}%
\pgfsetbuttcap%
\pgfsetmiterjoin%
\definecolor{currentfill}{rgb}{0.800000,0.800000,0.800000}%
\pgfsetfillcolor{currentfill}%
\pgfsetlinewidth{1.003750pt}%
\definecolor{currentstroke}{rgb}{0.000000,0.000000,0.000000}%
\pgfsetstrokecolor{currentstroke}%
\pgfsetdash{}{0pt}%
\pgfpathmoveto{\pgfqpoint{3.566442in}{2.223140in}}%
\pgfpathcurveto{\pgfqpoint{3.608109in}{2.181473in}}{\pgfqpoint{3.786697in}{2.181473in}}{\pgfqpoint{3.828363in}{2.223140in}}%
\pgfpathcurveto{\pgfqpoint{3.870030in}{2.264806in}}{\pgfqpoint{3.870030in}{2.412954in}}{\pgfqpoint{3.828363in}{2.454621in}}%
\pgfpathcurveto{\pgfqpoint{3.786697in}{2.496287in}}{\pgfqpoint{3.608109in}{2.496287in}}{\pgfqpoint{3.566442in}{2.454621in}}%
\pgfpathcurveto{\pgfqpoint{3.524775in}{2.412954in}}{\pgfqpoint{3.524775in}{2.264806in}}{\pgfqpoint{3.566442in}{2.223140in}}%
\pgfpathclose%
\pgfusepath{stroke,fill}%
\end{pgfscope}%
\begin{pgfscope}%
\definecolor{textcolor}{rgb}{0.000000,0.000000,0.000000}%
\pgfsetstrokecolor{textcolor}%
\pgfsetfillcolor{textcolor}%
\pgftext[x=3.697403in,y=2.297213in,,base]{\color{textcolor}\rmfamily\fontsize{12.000000}{14.400000}\selectfont \(\displaystyle P_x\)}%
\end{pgfscope}%
\begin{pgfscope}%
\pgfpathrectangle{\pgfqpoint{1.000000in}{0.330000in}}{\pgfqpoint{6.200000in}{0.604712in}}%
\pgfusepath{clip}%
\pgfsetbuttcap%
\pgfsetroundjoin%
\definecolor{currentfill}{rgb}{1.000000,0.000000,0.000000}%
\pgfsetfillcolor{currentfill}%
\pgfsetlinewidth{1.003750pt}%
\definecolor{currentstroke}{rgb}{1.000000,0.000000,0.000000}%
\pgfsetstrokecolor{currentstroke}%
\pgfsetdash{}{0pt}%
\pgfsys@defobject{currentmarker}{\pgfqpoint{-0.098209in}{-0.098209in}}{\pgfqpoint{0.098209in}{0.098209in}}{%
\pgfpathmoveto{\pgfqpoint{0.000000in}{0.098209in}}%
\pgfpathlineto{\pgfqpoint{-0.098209in}{-0.098209in}}%
\pgfpathlineto{\pgfqpoint{0.098209in}{-0.098209in}}%
\pgfpathclose%
\pgfusepath{stroke,fill}%
}%
\begin{pgfscope}%
\pgfsys@transformshift{1.281818in}{0.357487in}%
\pgfsys@useobject{currentmarker}{}%
\end{pgfscope}%
\begin{pgfscope}%
\pgfsys@transformshift{6.918182in}{0.357487in}%
\pgfsys@useobject{currentmarker}{}%
\end{pgfscope}%
\end{pgfscope}%
\begin{pgfscope}%
\pgfpathrectangle{\pgfqpoint{1.000000in}{0.330000in}}{\pgfqpoint{6.200000in}{0.604712in}}%
\pgfusepath{clip}%
\pgfsetbuttcap%
\pgfsetroundjoin%
\definecolor{currentfill}{rgb}{0.000000,0.000000,1.000000}%
\pgfsetfillcolor{currentfill}%
\pgfsetlinewidth{1.003750pt}%
\definecolor{currentstroke}{rgb}{0.000000,0.000000,1.000000}%
\pgfsetstrokecolor{currentstroke}%
\pgfsetdash{}{0pt}%
\pgfsys@defobject{currentmarker}{\pgfqpoint{-0.098209in}{-0.098209in}}{\pgfqpoint{0.098209in}{0.098209in}}{%
\pgfpathmoveto{\pgfqpoint{-0.098209in}{-0.098209in}}%
\pgfpathlineto{\pgfqpoint{0.098209in}{-0.098209in}}%
\pgfpathlineto{\pgfqpoint{0.098209in}{0.098209in}}%
\pgfpathlineto{\pgfqpoint{-0.098209in}{0.098209in}}%
\pgfpathclose%
\pgfusepath{stroke,fill}%
}%
\end{pgfscope}%
\begin{pgfscope}%
\pgfpathrectangle{\pgfqpoint{1.000000in}{0.330000in}}{\pgfqpoint{6.200000in}{0.604712in}}%
\pgfusepath{clip}%
\pgfsetrectcap%
\pgfsetroundjoin%
\pgfsetlinewidth{1.003750pt}%
\definecolor{currentstroke}{rgb}{0.000000,0.000000,0.000000}%
\pgfsetstrokecolor{currentstroke}%
\pgfsetdash{}{0pt}%
\pgfpathmoveto{\pgfqpoint{1.281818in}{0.907225in}}%
\pgfpathlineto{\pgfqpoint{6.918182in}{0.907225in}}%
\pgfusepath{stroke}%
\end{pgfscope}%
\begin{pgfscope}%
\pgfpathrectangle{\pgfqpoint{1.000000in}{0.330000in}}{\pgfqpoint{6.200000in}{0.604712in}}%
\pgfusepath{clip}%
\pgfsetrectcap%
\pgfsetroundjoin%
\pgfsetlinewidth{1.003750pt}%
\definecolor{currentstroke}{rgb}{0.000000,0.000000,0.000000}%
\pgfsetstrokecolor{currentstroke}%
\pgfsetdash{}{0pt}%
\pgfpathmoveto{\pgfqpoint{1.281818in}{0.870637in}}%
\pgfpathlineto{\pgfqpoint{6.918182in}{0.870637in}}%
\pgfusepath{stroke}%
\end{pgfscope}%
\begin{pgfscope}%
\pgfpathrectangle{\pgfqpoint{1.000000in}{0.330000in}}{\pgfqpoint{6.200000in}{0.604712in}}%
\pgfusepath{clip}%
\pgfsetrectcap%
\pgfsetroundjoin%
\pgfsetlinewidth{1.003750pt}%
\definecolor{currentstroke}{rgb}{0.000000,0.000000,0.000000}%
\pgfsetstrokecolor{currentstroke}%
\pgfsetdash{}{0pt}%
\pgfpathmoveto{\pgfqpoint{1.281818in}{0.468172in}}%
\pgfpathlineto{\pgfqpoint{6.918182in}{0.468172in}}%
\pgfusepath{stroke}%
\end{pgfscope}%
\begin{pgfscope}%
\pgfpathrectangle{\pgfqpoint{1.000000in}{0.330000in}}{\pgfqpoint{6.200000in}{0.604712in}}%
\pgfusepath{clip}%
\pgfsetrectcap%
\pgfsetroundjoin%
\pgfsetlinewidth{1.003750pt}%
\definecolor{currentstroke}{rgb}{0.000000,0.000000,0.000000}%
\pgfsetstrokecolor{currentstroke}%
\pgfsetdash{}{0pt}%
\pgfpathmoveto{\pgfqpoint{1.281818in}{0.504760in}}%
\pgfpathlineto{\pgfqpoint{6.918182in}{0.504760in}}%
\pgfusepath{stroke}%
\end{pgfscope}%
\begin{pgfscope}%
\pgfpathrectangle{\pgfqpoint{1.000000in}{0.330000in}}{\pgfqpoint{6.200000in}{0.604712in}}%
\pgfusepath{clip}%
\pgfsetrectcap%
\pgfsetroundjoin%
\pgfsetlinewidth{1.003750pt}%
\definecolor{currentstroke}{rgb}{0.000000,0.000000,0.000000}%
\pgfsetstrokecolor{currentstroke}%
\pgfsetdash{}{0pt}%
\pgfpathmoveto{\pgfqpoint{1.281818in}{0.468172in}}%
\pgfpathlineto{\pgfqpoint{1.281818in}{0.907225in}}%
\pgfusepath{stroke}%
\end{pgfscope}%
\begin{pgfscope}%
\pgfpathrectangle{\pgfqpoint{1.000000in}{0.330000in}}{\pgfqpoint{6.200000in}{0.604712in}}%
\pgfusepath{clip}%
\pgfsetrectcap%
\pgfsetroundjoin%
\pgfsetlinewidth{1.003750pt}%
\definecolor{currentstroke}{rgb}{0.000000,0.000000,0.000000}%
\pgfsetstrokecolor{currentstroke}%
\pgfsetdash{}{0pt}%
\pgfpathmoveto{\pgfqpoint{6.918182in}{0.468172in}}%
\pgfpathlineto{\pgfqpoint{6.918182in}{0.907225in}}%
\pgfusepath{stroke}%
\end{pgfscope}%
\begin{pgfscope}%
\pgfsetbuttcap%
\pgfsetmiterjoin%
\definecolor{currentfill}{rgb}{0.800000,0.800000,0.800000}%
\pgfsetfillcolor{currentfill}%
\pgfsetlinewidth{1.003750pt}%
\definecolor{currentstroke}{rgb}{0.000000,0.000000,0.000000}%
\pgfsetstrokecolor{currentstroke}%
\pgfsetdash{}{0pt}%
\pgfpathmoveto{\pgfqpoint{3.818363in}{0.591248in}}%
\pgfpathcurveto{\pgfqpoint{3.853086in}{0.556526in}}{\pgfqpoint{4.346914in}{0.556526in}}{\pgfqpoint{4.381637in}{0.591248in}}%
\pgfpathcurveto{\pgfqpoint{4.416359in}{0.625970in}}{\pgfqpoint{4.416359in}{0.749427in}}{\pgfqpoint{4.381637in}{0.784149in}}%
\pgfpathcurveto{\pgfqpoint{4.346914in}{0.818872in}}{\pgfqpoint{3.853086in}{0.818872in}}{\pgfqpoint{3.818363in}{0.784149in}}%
\pgfpathcurveto{\pgfqpoint{3.783641in}{0.749427in}}{\pgfqpoint{3.783641in}{0.625970in}}{\pgfqpoint{3.818363in}{0.591248in}}%
\pgfpathclose%
\pgfusepath{stroke,fill}%
\end{pgfscope}%
\begin{pgfscope}%
\definecolor{textcolor}{rgb}{0.000000,0.000000,0.000000}%
\pgfsetstrokecolor{textcolor}%
\pgfsetfillcolor{textcolor}%
\pgftext[x=4.100000in,y=0.687699in,,]{\color{textcolor}\rmfamily\fontsize{10.000000}{12.000000}\selectfont W12x14}%
\end{pgfscope}%
\begin{pgfscope}%
\pgfsetbuttcap%
\pgfsetmiterjoin%
\definecolor{currentfill}{rgb}{0.800000,0.800000,0.800000}%
\pgfsetfillcolor{currentfill}%
\pgfsetlinewidth{1.003750pt}%
\definecolor{currentstroke}{rgb}{0.000000,0.000000,0.000000}%
\pgfsetstrokecolor{currentstroke}%
\pgfsetdash{}{0pt}%
\pgfpathmoveto{\pgfqpoint{3.650539in}{0.187246in}}%
\pgfpathcurveto{\pgfqpoint{3.685261in}{0.152524in}}{\pgfqpoint{4.514739in}{0.152524in}}{\pgfqpoint{4.549461in}{0.187246in}}%
\pgfpathcurveto{\pgfqpoint{4.584183in}{0.221968in}}{\pgfqpoint{4.584183in}{0.345425in}}{\pgfqpoint{4.549461in}{0.380147in}}%
\pgfpathcurveto{\pgfqpoint{4.514739in}{0.414869in}}{\pgfqpoint{3.685261in}{0.414869in}}{\pgfqpoint{3.650539in}{0.380147in}}%
\pgfpathcurveto{\pgfqpoint{3.615817in}{0.345425in}}{\pgfqpoint{3.615817in}{0.221968in}}{\pgfqpoint{3.650539in}{0.187246in}}%
\pgfpathclose%
\pgfusepath{stroke,fill}%
\end{pgfscope}%
\begin{pgfscope}%
\definecolor{textcolor}{rgb}{0.000000,0.000000,0.000000}%
\pgfsetstrokecolor{textcolor}%
\pgfsetfillcolor{textcolor}%
\pgftext[x=4.100000in,y=0.283697in,,]{\color{textcolor}\rmfamily\fontsize{10.000000}{12.000000}\selectfont Span 0 = 7 ft}%
\end{pgfscope}%
\end{pgfpicture}%
\makeatother%
\endgroup%

\end{center}
\vspace{-18pt}
\caption{Applied Loads}
\end{figure}
The following distributed loads are applied to the beam. The program can handle all possible mass and force units in both metric and imperial systems simultaneously. Loads are plotted to scale according to their relative magnitudes. A "positive" load is defined as a load acting in the direction of gravity.
\begin{table}[ht]
\caption{Applied Distributed Loads}
\centering
\begin{tabular}{l l l l l l l}
\hline
Load & Start Loc. & Start Mag. & End Loc. & End Mag. & Type & Description\\
\hline
w\textsubscript{1} & 0 {\color{darkBlue}{\textbf{ft}}} & 14.2 {\color{darkBlue}{\textbf{plf}}} & 7 {\color{darkBlue}{\textbf{ft}}} & 14.2 {\color{darkBlue}{\textbf{plf}}} & D & Self weight\\
\hline
\end{tabular}
\end{table}
\begin{table}[ht]
\caption{Applied Point Loads}
\centering
\begin{tabular}{l l l l l l}
\hline
Load & Loc. & Shear & Type & Description \\
\hline
P\textsubscript{1} & 3 {\color{darkBlue}{\textbf{ft}}} & 1.6 {\color{darkBlue}{\textbf{kip}}} & D & No description\\
P\textsubscript{2} & 3 {\color{darkBlue}{\textbf{ft}}} & 2.4 {\color{darkBlue}{\textbf{kip}}} & L & No description\\
\hline
\end{tabular}
\end{table}
%	-------------------------------- LOAD COMBOS	--------------------------------
\section{Load Combinations}
The following load combinations are used for the design. Duplicate load combinations are not listed and only loads that are used on the beam are included in the load combinations (i.e. If soil load is not included as a load type in any of the applied loads, then "H" loads will not be included in the listed load combinations). S\textsubscript{DS} is input as 1.0 and \textOmega\textsubscript{0} is input as 2.5 for use in seismic load combinations. Any load designated as a pattern load is applied to spans in all possible permutations to create the most extreme loading condition. Numbers after a load indicate the span over which the pattern load is applied (i.e. L0 indicates that live load is applied only on the first span).
\begin{table}[H]
\caption{Strength (LRFD) Load Combinations}
\centering
\begin{tabular}{l l l}
\hline
Load Combo & Loads and Factors & Reference\\
\hline
LC 1 & 1.2D + 1.0L0 & ASCE7-16 \S2.3.1 (LC 3)\\
LC 2 & 0.7D & ASCE7-16 \S2.3.6 (LC 7)\\
LC 3 & 1.4D & ASCE7-16 \S2.3.1 (LC 1)\\
LC 4 & 0.9D & ASCE7-16 \S2.3.1 (LC 5)\\
LC 5 & 1.2D & ASCE7-16 \S2.3.1 (LC 3)\\
LC 6 & 1.2D + 1.6L0 & ASCE7-16 \S2.3.1 (LC 2)\\
LC 7 & 1.4D + 0.5L0 & ASCE7-16 \S2.3.6 (LC 6)\\
\hline
\end{tabular}
\end{table}
\begin{table}[H]
\caption{Deflection (ASD) Load Combinations}
\centering
\begin{tabular}{l l l}
\hline
Load Combo & Loads and Factors & Reference\\
\hline
LC 1 & 0.6D & ASCE7-16 \S2.4.1 (LC 7)\\
LC 2 & 0.4D & ASCE7-16 \S2.4.5 (LC 9)\\
LC 3 & 1.0D & ASCE7-16 \S2.4.1 (LC 1)\\
LC 4 & 1.0D + 0.75L0 & ASCE7-16 \S2.4.1 (LC 4)\\
LC 5 & 1.0L0 & L only deflection check\\
LC 6 & 1.1D + 0.75L0 & ASCE7-16 \S2.4.5 (LC 8)\\
LC 7 & 1.14D & ASCE7-16 \S2.4.5 (LC 8)\\
LC 8 & 1.0D + 1.0L0 & ASCE7-16 \S2.4.1 (LC 2)\\
\hline
\end{tabular}
\end{table}
%	---------------------- SECTIONAL & MATERIAL PROPERTIES -----------------------
\section{Sectional and Material Properties}
The following are sectional and material properties used for analysis \textbf{(W12x14, Grade A992)}:
\begin{table}[ht]
\caption{Sectional and Material Properties}
\vspace{-10pt}
\centering
\begin{tabular}{lll}
\centering
\begin{tabular}[t]{ll}
\cline{1-2}
Property & Value \\
\cline{1-2}
A\textsubscript{w} & 2.4 {\color{darkBlue}{\textbf{{\color{darkBlue}{\textbf{in}}}\textsuperscript{2}}}} \\
C\textsubscript{w} & 80.4 {\color{darkBlue}{\textbf{{\color{darkBlue}{\textbf{in}}}\textsuperscript{6}}}} \\
F\textsubscript{u} & 65 {\color{darkBlue}{\textbf{ksi}}} \\
F\textsubscript{y} & 50 {\color{darkBlue}{\textbf{ksi}}} \\
I\textsubscript{x} & 88.6 {\color{darkBlue}{\textbf{{\color{darkBlue}{\textbf{in}}}\textsuperscript{4}}}} \\
I\textsubscript{y} & 2.4 {\color{darkBlue}{\textbf{{\color{darkBlue}{\textbf{in}}}\textsuperscript{4}}}} \\
\cline{1-2}
\end{tabular}
&
\begin{tabular}[t]{ll}
\cline{1-2}
Property & Value \\
\cline{1-2}
S\textsubscript{x} & 14.9 {\color{darkBlue}{\textbf{{\color{darkBlue}{\textbf{in}}}\textsuperscript{3}}}} \\
S\textsubscript{y} & 1.2 {\color{darkBlue}{\textbf{{\color{darkBlue}{\textbf{in}}}\textsuperscript{3}}}} \\
Z\textsubscript{x} & 17.4 {\color{darkBlue}{\textbf{{\color{darkBlue}{\textbf{in}}}\textsuperscript{3}}}} \\
Z\textsubscript{y} & 1.9 {\color{darkBlue}{\textbf{{\color{darkBlue}{\textbf{in}}}\textsuperscript{3}}}} \\
r\textsubscript{x} & 4.6 {\color{darkBlue}{\textbf{in}}} \\
r\textsubscript{y} & 0.8 {\color{darkBlue}{\textbf{in}}} \\
\cline{1-2}
\end{tabular}
&
\begin{tabular}[t]{ll}
\cline{1-2}
Property & Value \\
\cline{1-2}
b\textsubscript{f} & 4.0 {\color{darkBlue}{\textbf{in}}} \\
t\textsubscript{f} & 0.2 {\color{darkBlue}{\textbf{in}}} \\
t\textsubscript{w} & 0.2 {\color{darkBlue}{\textbf{in}}} \\
h\textsubscript{0} & 11.7 {\color{darkBlue}{\textbf{in}}} \\
U.W. & 490 {\color{darkBlue}{\textbf{pcf}}} \\
\cline{1-2}
\end{tabular}
\end{tabular}
\end{table}
%	-------------------------------- BENDING CHECK -------------------------------
\section{Bending Check}
\begin{figure}[H]
\begin{center}
%% Creator: Matplotlib, PGF backend
%%
%% To include the figure in your LaTeX document, write
%%   \input{<filename>.pgf}
%%
%% Make sure the required packages are loaded in your preamble
%%   \usepackage{pgf}
%%
%% Figures using additional raster images can only be included by \input if
%% they are in the same directory as the main LaTeX file. For loading figures
%% from other directories you can use the `import` package
%%   \usepackage{import}
%%
%% and then include the figures with
%%   \import{<path to file>}{<filename>.pgf}
%%
%% Matplotlib used the following preamble
%%
\begingroup%
\makeatletter%
\begin{pgfpicture}%
\pgfpathrectangle{\pgfpointorigin}{\pgfqpoint{8.000000in}{3.000000in}}%
\pgfusepath{use as bounding box, clip}%
\begin{pgfscope}%
\pgfsetbuttcap%
\pgfsetmiterjoin%
\definecolor{currentfill}{rgb}{1.000000,1.000000,1.000000}%
\pgfsetfillcolor{currentfill}%
\pgfsetlinewidth{0.000000pt}%
\definecolor{currentstroke}{rgb}{1.000000,1.000000,1.000000}%
\pgfsetstrokecolor{currentstroke}%
\pgfsetdash{}{0pt}%
\pgfpathmoveto{\pgfqpoint{0.000000in}{0.000000in}}%
\pgfpathlineto{\pgfqpoint{8.000000in}{0.000000in}}%
\pgfpathlineto{\pgfqpoint{8.000000in}{3.000000in}}%
\pgfpathlineto{\pgfqpoint{0.000000in}{3.000000in}}%
\pgfpathclose%
\pgfusepath{fill}%
\end{pgfscope}%
\begin{pgfscope}%
\pgfsetbuttcap%
\pgfsetmiterjoin%
\definecolor{currentfill}{rgb}{1.000000,1.000000,1.000000}%
\pgfsetfillcolor{currentfill}%
\pgfsetlinewidth{0.000000pt}%
\definecolor{currentstroke}{rgb}{0.000000,0.000000,0.000000}%
\pgfsetstrokecolor{currentstroke}%
\pgfsetstrokeopacity{0.000000}%
\pgfsetdash{}{0pt}%
\pgfpathmoveto{\pgfqpoint{1.000000in}{0.330000in}}%
\pgfpathlineto{\pgfqpoint{7.200000in}{0.330000in}}%
\pgfpathlineto{\pgfqpoint{7.200000in}{2.640000in}}%
\pgfpathlineto{\pgfqpoint{1.000000in}{2.640000in}}%
\pgfpathclose%
\pgfusepath{fill}%
\end{pgfscope}%
\begin{pgfscope}%
\pgfpathrectangle{\pgfqpoint{1.000000in}{0.330000in}}{\pgfqpoint{6.200000in}{2.310000in}}%
\pgfusepath{clip}%
\pgfsetbuttcap%
\pgfsetroundjoin%
\pgfsetlinewidth{0.803000pt}%
\definecolor{currentstroke}{rgb}{0.000000,0.000000,0.000000}%
\pgfsetstrokecolor{currentstroke}%
\pgfsetdash{{0.800000pt}{1.320000pt}}{0.000000pt}%
\pgfpathmoveto{\pgfqpoint{1.281818in}{0.330000in}}%
\pgfpathlineto{\pgfqpoint{1.281818in}{2.640000in}}%
\pgfusepath{stroke}%
\end{pgfscope}%
\begin{pgfscope}%
\pgfsetbuttcap%
\pgfsetroundjoin%
\definecolor{currentfill}{rgb}{0.000000,0.000000,0.000000}%
\pgfsetfillcolor{currentfill}%
\pgfsetlinewidth{0.803000pt}%
\definecolor{currentstroke}{rgb}{0.000000,0.000000,0.000000}%
\pgfsetstrokecolor{currentstroke}%
\pgfsetdash{}{0pt}%
\pgfsys@defobject{currentmarker}{\pgfqpoint{0.000000in}{-0.048611in}}{\pgfqpoint{0.000000in}{0.000000in}}{%
\pgfpathmoveto{\pgfqpoint{0.000000in}{0.000000in}}%
\pgfpathlineto{\pgfqpoint{0.000000in}{-0.048611in}}%
\pgfusepath{stroke,fill}%
}%
\begin{pgfscope}%
\pgfsys@transformshift{1.281818in}{0.330000in}%
\pgfsys@useobject{currentmarker}{}%
\end{pgfscope}%
\end{pgfscope}%
\begin{pgfscope}%
\pgfsetbuttcap%
\pgfsetroundjoin%
\definecolor{currentfill}{rgb}{0.000000,0.000000,0.000000}%
\pgfsetfillcolor{currentfill}%
\pgfsetlinewidth{0.803000pt}%
\definecolor{currentstroke}{rgb}{0.000000,0.000000,0.000000}%
\pgfsetstrokecolor{currentstroke}%
\pgfsetdash{}{0pt}%
\pgfsys@defobject{currentmarker}{\pgfqpoint{0.000000in}{0.000000in}}{\pgfqpoint{0.000000in}{0.048611in}}{%
\pgfpathmoveto{\pgfqpoint{0.000000in}{0.000000in}}%
\pgfpathlineto{\pgfqpoint{0.000000in}{0.048611in}}%
\pgfusepath{stroke,fill}%
}%
\begin{pgfscope}%
\pgfsys@transformshift{1.281818in}{2.640000in}%
\pgfsys@useobject{currentmarker}{}%
\end{pgfscope}%
\end{pgfscope}%
\begin{pgfscope}%
\definecolor{textcolor}{rgb}{0.000000,0.000000,0.000000}%
\pgfsetstrokecolor{textcolor}%
\pgfsetfillcolor{textcolor}%
\pgftext[x=1.281818in,y=0.232778in,,top]{\color{textcolor}\rmfamily\fontsize{10.000000}{12.000000}\selectfont \(\displaystyle {0}\)}%
\end{pgfscope}%
\begin{pgfscope}%
\pgfpathrectangle{\pgfqpoint{1.000000in}{0.330000in}}{\pgfqpoint{6.200000in}{2.310000in}}%
\pgfusepath{clip}%
\pgfsetbuttcap%
\pgfsetroundjoin%
\pgfsetlinewidth{0.803000pt}%
\definecolor{currentstroke}{rgb}{0.000000,0.000000,0.000000}%
\pgfsetstrokecolor{currentstroke}%
\pgfsetdash{{0.800000pt}{1.320000pt}}{0.000000pt}%
\pgfpathmoveto{\pgfqpoint{2.087013in}{0.330000in}}%
\pgfpathlineto{\pgfqpoint{2.087013in}{2.640000in}}%
\pgfusepath{stroke}%
\end{pgfscope}%
\begin{pgfscope}%
\pgfsetbuttcap%
\pgfsetroundjoin%
\definecolor{currentfill}{rgb}{0.000000,0.000000,0.000000}%
\pgfsetfillcolor{currentfill}%
\pgfsetlinewidth{0.803000pt}%
\definecolor{currentstroke}{rgb}{0.000000,0.000000,0.000000}%
\pgfsetstrokecolor{currentstroke}%
\pgfsetdash{}{0pt}%
\pgfsys@defobject{currentmarker}{\pgfqpoint{0.000000in}{-0.048611in}}{\pgfqpoint{0.000000in}{0.000000in}}{%
\pgfpathmoveto{\pgfqpoint{0.000000in}{0.000000in}}%
\pgfpathlineto{\pgfqpoint{0.000000in}{-0.048611in}}%
\pgfusepath{stroke,fill}%
}%
\begin{pgfscope}%
\pgfsys@transformshift{2.087013in}{0.330000in}%
\pgfsys@useobject{currentmarker}{}%
\end{pgfscope}%
\end{pgfscope}%
\begin{pgfscope}%
\pgfsetbuttcap%
\pgfsetroundjoin%
\definecolor{currentfill}{rgb}{0.000000,0.000000,0.000000}%
\pgfsetfillcolor{currentfill}%
\pgfsetlinewidth{0.803000pt}%
\definecolor{currentstroke}{rgb}{0.000000,0.000000,0.000000}%
\pgfsetstrokecolor{currentstroke}%
\pgfsetdash{}{0pt}%
\pgfsys@defobject{currentmarker}{\pgfqpoint{0.000000in}{0.000000in}}{\pgfqpoint{0.000000in}{0.048611in}}{%
\pgfpathmoveto{\pgfqpoint{0.000000in}{0.000000in}}%
\pgfpathlineto{\pgfqpoint{0.000000in}{0.048611in}}%
\pgfusepath{stroke,fill}%
}%
\begin{pgfscope}%
\pgfsys@transformshift{2.087013in}{2.640000in}%
\pgfsys@useobject{currentmarker}{}%
\end{pgfscope}%
\end{pgfscope}%
\begin{pgfscope}%
\definecolor{textcolor}{rgb}{0.000000,0.000000,0.000000}%
\pgfsetstrokecolor{textcolor}%
\pgfsetfillcolor{textcolor}%
\pgftext[x=2.087013in,y=0.232778in,,top]{\color{textcolor}\rmfamily\fontsize{10.000000}{12.000000}\selectfont \(\displaystyle {1}\)}%
\end{pgfscope}%
\begin{pgfscope}%
\pgfpathrectangle{\pgfqpoint{1.000000in}{0.330000in}}{\pgfqpoint{6.200000in}{2.310000in}}%
\pgfusepath{clip}%
\pgfsetbuttcap%
\pgfsetroundjoin%
\pgfsetlinewidth{0.803000pt}%
\definecolor{currentstroke}{rgb}{0.000000,0.000000,0.000000}%
\pgfsetstrokecolor{currentstroke}%
\pgfsetdash{{0.800000pt}{1.320000pt}}{0.000000pt}%
\pgfpathmoveto{\pgfqpoint{2.892208in}{0.330000in}}%
\pgfpathlineto{\pgfqpoint{2.892208in}{2.640000in}}%
\pgfusepath{stroke}%
\end{pgfscope}%
\begin{pgfscope}%
\pgfsetbuttcap%
\pgfsetroundjoin%
\definecolor{currentfill}{rgb}{0.000000,0.000000,0.000000}%
\pgfsetfillcolor{currentfill}%
\pgfsetlinewidth{0.803000pt}%
\definecolor{currentstroke}{rgb}{0.000000,0.000000,0.000000}%
\pgfsetstrokecolor{currentstroke}%
\pgfsetdash{}{0pt}%
\pgfsys@defobject{currentmarker}{\pgfqpoint{0.000000in}{-0.048611in}}{\pgfqpoint{0.000000in}{0.000000in}}{%
\pgfpathmoveto{\pgfqpoint{0.000000in}{0.000000in}}%
\pgfpathlineto{\pgfqpoint{0.000000in}{-0.048611in}}%
\pgfusepath{stroke,fill}%
}%
\begin{pgfscope}%
\pgfsys@transformshift{2.892208in}{0.330000in}%
\pgfsys@useobject{currentmarker}{}%
\end{pgfscope}%
\end{pgfscope}%
\begin{pgfscope}%
\pgfsetbuttcap%
\pgfsetroundjoin%
\definecolor{currentfill}{rgb}{0.000000,0.000000,0.000000}%
\pgfsetfillcolor{currentfill}%
\pgfsetlinewidth{0.803000pt}%
\definecolor{currentstroke}{rgb}{0.000000,0.000000,0.000000}%
\pgfsetstrokecolor{currentstroke}%
\pgfsetdash{}{0pt}%
\pgfsys@defobject{currentmarker}{\pgfqpoint{0.000000in}{0.000000in}}{\pgfqpoint{0.000000in}{0.048611in}}{%
\pgfpathmoveto{\pgfqpoint{0.000000in}{0.000000in}}%
\pgfpathlineto{\pgfqpoint{0.000000in}{0.048611in}}%
\pgfusepath{stroke,fill}%
}%
\begin{pgfscope}%
\pgfsys@transformshift{2.892208in}{2.640000in}%
\pgfsys@useobject{currentmarker}{}%
\end{pgfscope}%
\end{pgfscope}%
\begin{pgfscope}%
\definecolor{textcolor}{rgb}{0.000000,0.000000,0.000000}%
\pgfsetstrokecolor{textcolor}%
\pgfsetfillcolor{textcolor}%
\pgftext[x=2.892208in,y=0.232778in,,top]{\color{textcolor}\rmfamily\fontsize{10.000000}{12.000000}\selectfont \(\displaystyle {2}\)}%
\end{pgfscope}%
\begin{pgfscope}%
\pgfpathrectangle{\pgfqpoint{1.000000in}{0.330000in}}{\pgfqpoint{6.200000in}{2.310000in}}%
\pgfusepath{clip}%
\pgfsetbuttcap%
\pgfsetroundjoin%
\pgfsetlinewidth{0.803000pt}%
\definecolor{currentstroke}{rgb}{0.000000,0.000000,0.000000}%
\pgfsetstrokecolor{currentstroke}%
\pgfsetdash{{0.800000pt}{1.320000pt}}{0.000000pt}%
\pgfpathmoveto{\pgfqpoint{3.697403in}{0.330000in}}%
\pgfpathlineto{\pgfqpoint{3.697403in}{2.640000in}}%
\pgfusepath{stroke}%
\end{pgfscope}%
\begin{pgfscope}%
\pgfsetbuttcap%
\pgfsetroundjoin%
\definecolor{currentfill}{rgb}{0.000000,0.000000,0.000000}%
\pgfsetfillcolor{currentfill}%
\pgfsetlinewidth{0.803000pt}%
\definecolor{currentstroke}{rgb}{0.000000,0.000000,0.000000}%
\pgfsetstrokecolor{currentstroke}%
\pgfsetdash{}{0pt}%
\pgfsys@defobject{currentmarker}{\pgfqpoint{0.000000in}{-0.048611in}}{\pgfqpoint{0.000000in}{0.000000in}}{%
\pgfpathmoveto{\pgfqpoint{0.000000in}{0.000000in}}%
\pgfpathlineto{\pgfqpoint{0.000000in}{-0.048611in}}%
\pgfusepath{stroke,fill}%
}%
\begin{pgfscope}%
\pgfsys@transformshift{3.697403in}{0.330000in}%
\pgfsys@useobject{currentmarker}{}%
\end{pgfscope}%
\end{pgfscope}%
\begin{pgfscope}%
\pgfsetbuttcap%
\pgfsetroundjoin%
\definecolor{currentfill}{rgb}{0.000000,0.000000,0.000000}%
\pgfsetfillcolor{currentfill}%
\pgfsetlinewidth{0.803000pt}%
\definecolor{currentstroke}{rgb}{0.000000,0.000000,0.000000}%
\pgfsetstrokecolor{currentstroke}%
\pgfsetdash{}{0pt}%
\pgfsys@defobject{currentmarker}{\pgfqpoint{0.000000in}{0.000000in}}{\pgfqpoint{0.000000in}{0.048611in}}{%
\pgfpathmoveto{\pgfqpoint{0.000000in}{0.000000in}}%
\pgfpathlineto{\pgfqpoint{0.000000in}{0.048611in}}%
\pgfusepath{stroke,fill}%
}%
\begin{pgfscope}%
\pgfsys@transformshift{3.697403in}{2.640000in}%
\pgfsys@useobject{currentmarker}{}%
\end{pgfscope}%
\end{pgfscope}%
\begin{pgfscope}%
\definecolor{textcolor}{rgb}{0.000000,0.000000,0.000000}%
\pgfsetstrokecolor{textcolor}%
\pgfsetfillcolor{textcolor}%
\pgftext[x=3.697403in,y=0.232778in,,top]{\color{textcolor}\rmfamily\fontsize{10.000000}{12.000000}\selectfont \(\displaystyle {3}\)}%
\end{pgfscope}%
\begin{pgfscope}%
\pgfpathrectangle{\pgfqpoint{1.000000in}{0.330000in}}{\pgfqpoint{6.200000in}{2.310000in}}%
\pgfusepath{clip}%
\pgfsetbuttcap%
\pgfsetroundjoin%
\pgfsetlinewidth{0.803000pt}%
\definecolor{currentstroke}{rgb}{0.000000,0.000000,0.000000}%
\pgfsetstrokecolor{currentstroke}%
\pgfsetdash{{0.800000pt}{1.320000pt}}{0.000000pt}%
\pgfpathmoveto{\pgfqpoint{4.502597in}{0.330000in}}%
\pgfpathlineto{\pgfqpoint{4.502597in}{2.640000in}}%
\pgfusepath{stroke}%
\end{pgfscope}%
\begin{pgfscope}%
\pgfsetbuttcap%
\pgfsetroundjoin%
\definecolor{currentfill}{rgb}{0.000000,0.000000,0.000000}%
\pgfsetfillcolor{currentfill}%
\pgfsetlinewidth{0.803000pt}%
\definecolor{currentstroke}{rgb}{0.000000,0.000000,0.000000}%
\pgfsetstrokecolor{currentstroke}%
\pgfsetdash{}{0pt}%
\pgfsys@defobject{currentmarker}{\pgfqpoint{0.000000in}{-0.048611in}}{\pgfqpoint{0.000000in}{0.000000in}}{%
\pgfpathmoveto{\pgfqpoint{0.000000in}{0.000000in}}%
\pgfpathlineto{\pgfqpoint{0.000000in}{-0.048611in}}%
\pgfusepath{stroke,fill}%
}%
\begin{pgfscope}%
\pgfsys@transformshift{4.502597in}{0.330000in}%
\pgfsys@useobject{currentmarker}{}%
\end{pgfscope}%
\end{pgfscope}%
\begin{pgfscope}%
\pgfsetbuttcap%
\pgfsetroundjoin%
\definecolor{currentfill}{rgb}{0.000000,0.000000,0.000000}%
\pgfsetfillcolor{currentfill}%
\pgfsetlinewidth{0.803000pt}%
\definecolor{currentstroke}{rgb}{0.000000,0.000000,0.000000}%
\pgfsetstrokecolor{currentstroke}%
\pgfsetdash{}{0pt}%
\pgfsys@defobject{currentmarker}{\pgfqpoint{0.000000in}{0.000000in}}{\pgfqpoint{0.000000in}{0.048611in}}{%
\pgfpathmoveto{\pgfqpoint{0.000000in}{0.000000in}}%
\pgfpathlineto{\pgfqpoint{0.000000in}{0.048611in}}%
\pgfusepath{stroke,fill}%
}%
\begin{pgfscope}%
\pgfsys@transformshift{4.502597in}{2.640000in}%
\pgfsys@useobject{currentmarker}{}%
\end{pgfscope}%
\end{pgfscope}%
\begin{pgfscope}%
\definecolor{textcolor}{rgb}{0.000000,0.000000,0.000000}%
\pgfsetstrokecolor{textcolor}%
\pgfsetfillcolor{textcolor}%
\pgftext[x=4.502597in,y=0.232778in,,top]{\color{textcolor}\rmfamily\fontsize{10.000000}{12.000000}\selectfont \(\displaystyle {4}\)}%
\end{pgfscope}%
\begin{pgfscope}%
\pgfpathrectangle{\pgfqpoint{1.000000in}{0.330000in}}{\pgfqpoint{6.200000in}{2.310000in}}%
\pgfusepath{clip}%
\pgfsetbuttcap%
\pgfsetroundjoin%
\pgfsetlinewidth{0.803000pt}%
\definecolor{currentstroke}{rgb}{0.000000,0.000000,0.000000}%
\pgfsetstrokecolor{currentstroke}%
\pgfsetdash{{0.800000pt}{1.320000pt}}{0.000000pt}%
\pgfpathmoveto{\pgfqpoint{5.307792in}{0.330000in}}%
\pgfpathlineto{\pgfqpoint{5.307792in}{2.640000in}}%
\pgfusepath{stroke}%
\end{pgfscope}%
\begin{pgfscope}%
\pgfsetbuttcap%
\pgfsetroundjoin%
\definecolor{currentfill}{rgb}{0.000000,0.000000,0.000000}%
\pgfsetfillcolor{currentfill}%
\pgfsetlinewidth{0.803000pt}%
\definecolor{currentstroke}{rgb}{0.000000,0.000000,0.000000}%
\pgfsetstrokecolor{currentstroke}%
\pgfsetdash{}{0pt}%
\pgfsys@defobject{currentmarker}{\pgfqpoint{0.000000in}{-0.048611in}}{\pgfqpoint{0.000000in}{0.000000in}}{%
\pgfpathmoveto{\pgfqpoint{0.000000in}{0.000000in}}%
\pgfpathlineto{\pgfqpoint{0.000000in}{-0.048611in}}%
\pgfusepath{stroke,fill}%
}%
\begin{pgfscope}%
\pgfsys@transformshift{5.307792in}{0.330000in}%
\pgfsys@useobject{currentmarker}{}%
\end{pgfscope}%
\end{pgfscope}%
\begin{pgfscope}%
\pgfsetbuttcap%
\pgfsetroundjoin%
\definecolor{currentfill}{rgb}{0.000000,0.000000,0.000000}%
\pgfsetfillcolor{currentfill}%
\pgfsetlinewidth{0.803000pt}%
\definecolor{currentstroke}{rgb}{0.000000,0.000000,0.000000}%
\pgfsetstrokecolor{currentstroke}%
\pgfsetdash{}{0pt}%
\pgfsys@defobject{currentmarker}{\pgfqpoint{0.000000in}{0.000000in}}{\pgfqpoint{0.000000in}{0.048611in}}{%
\pgfpathmoveto{\pgfqpoint{0.000000in}{0.000000in}}%
\pgfpathlineto{\pgfqpoint{0.000000in}{0.048611in}}%
\pgfusepath{stroke,fill}%
}%
\begin{pgfscope}%
\pgfsys@transformshift{5.307792in}{2.640000in}%
\pgfsys@useobject{currentmarker}{}%
\end{pgfscope}%
\end{pgfscope}%
\begin{pgfscope}%
\definecolor{textcolor}{rgb}{0.000000,0.000000,0.000000}%
\pgfsetstrokecolor{textcolor}%
\pgfsetfillcolor{textcolor}%
\pgftext[x=5.307792in,y=0.232778in,,top]{\color{textcolor}\rmfamily\fontsize{10.000000}{12.000000}\selectfont \(\displaystyle {5}\)}%
\end{pgfscope}%
\begin{pgfscope}%
\pgfpathrectangle{\pgfqpoint{1.000000in}{0.330000in}}{\pgfqpoint{6.200000in}{2.310000in}}%
\pgfusepath{clip}%
\pgfsetbuttcap%
\pgfsetroundjoin%
\pgfsetlinewidth{0.803000pt}%
\definecolor{currentstroke}{rgb}{0.000000,0.000000,0.000000}%
\pgfsetstrokecolor{currentstroke}%
\pgfsetdash{{0.800000pt}{1.320000pt}}{0.000000pt}%
\pgfpathmoveto{\pgfqpoint{6.112987in}{0.330000in}}%
\pgfpathlineto{\pgfqpoint{6.112987in}{2.640000in}}%
\pgfusepath{stroke}%
\end{pgfscope}%
\begin{pgfscope}%
\pgfsetbuttcap%
\pgfsetroundjoin%
\definecolor{currentfill}{rgb}{0.000000,0.000000,0.000000}%
\pgfsetfillcolor{currentfill}%
\pgfsetlinewidth{0.803000pt}%
\definecolor{currentstroke}{rgb}{0.000000,0.000000,0.000000}%
\pgfsetstrokecolor{currentstroke}%
\pgfsetdash{}{0pt}%
\pgfsys@defobject{currentmarker}{\pgfqpoint{0.000000in}{-0.048611in}}{\pgfqpoint{0.000000in}{0.000000in}}{%
\pgfpathmoveto{\pgfqpoint{0.000000in}{0.000000in}}%
\pgfpathlineto{\pgfqpoint{0.000000in}{-0.048611in}}%
\pgfusepath{stroke,fill}%
}%
\begin{pgfscope}%
\pgfsys@transformshift{6.112987in}{0.330000in}%
\pgfsys@useobject{currentmarker}{}%
\end{pgfscope}%
\end{pgfscope}%
\begin{pgfscope}%
\pgfsetbuttcap%
\pgfsetroundjoin%
\definecolor{currentfill}{rgb}{0.000000,0.000000,0.000000}%
\pgfsetfillcolor{currentfill}%
\pgfsetlinewidth{0.803000pt}%
\definecolor{currentstroke}{rgb}{0.000000,0.000000,0.000000}%
\pgfsetstrokecolor{currentstroke}%
\pgfsetdash{}{0pt}%
\pgfsys@defobject{currentmarker}{\pgfqpoint{0.000000in}{0.000000in}}{\pgfqpoint{0.000000in}{0.048611in}}{%
\pgfpathmoveto{\pgfqpoint{0.000000in}{0.000000in}}%
\pgfpathlineto{\pgfqpoint{0.000000in}{0.048611in}}%
\pgfusepath{stroke,fill}%
}%
\begin{pgfscope}%
\pgfsys@transformshift{6.112987in}{2.640000in}%
\pgfsys@useobject{currentmarker}{}%
\end{pgfscope}%
\end{pgfscope}%
\begin{pgfscope}%
\definecolor{textcolor}{rgb}{0.000000,0.000000,0.000000}%
\pgfsetstrokecolor{textcolor}%
\pgfsetfillcolor{textcolor}%
\pgftext[x=6.112987in,y=0.232778in,,top]{\color{textcolor}\rmfamily\fontsize{10.000000}{12.000000}\selectfont \(\displaystyle {6}\)}%
\end{pgfscope}%
\begin{pgfscope}%
\pgfpathrectangle{\pgfqpoint{1.000000in}{0.330000in}}{\pgfqpoint{6.200000in}{2.310000in}}%
\pgfusepath{clip}%
\pgfsetbuttcap%
\pgfsetroundjoin%
\pgfsetlinewidth{0.803000pt}%
\definecolor{currentstroke}{rgb}{0.000000,0.000000,0.000000}%
\pgfsetstrokecolor{currentstroke}%
\pgfsetdash{{0.800000pt}{1.320000pt}}{0.000000pt}%
\pgfpathmoveto{\pgfqpoint{6.918182in}{0.330000in}}%
\pgfpathlineto{\pgfqpoint{6.918182in}{2.640000in}}%
\pgfusepath{stroke}%
\end{pgfscope}%
\begin{pgfscope}%
\pgfsetbuttcap%
\pgfsetroundjoin%
\definecolor{currentfill}{rgb}{0.000000,0.000000,0.000000}%
\pgfsetfillcolor{currentfill}%
\pgfsetlinewidth{0.803000pt}%
\definecolor{currentstroke}{rgb}{0.000000,0.000000,0.000000}%
\pgfsetstrokecolor{currentstroke}%
\pgfsetdash{}{0pt}%
\pgfsys@defobject{currentmarker}{\pgfqpoint{0.000000in}{-0.048611in}}{\pgfqpoint{0.000000in}{0.000000in}}{%
\pgfpathmoveto{\pgfqpoint{0.000000in}{0.000000in}}%
\pgfpathlineto{\pgfqpoint{0.000000in}{-0.048611in}}%
\pgfusepath{stroke,fill}%
}%
\begin{pgfscope}%
\pgfsys@transformshift{6.918182in}{0.330000in}%
\pgfsys@useobject{currentmarker}{}%
\end{pgfscope}%
\end{pgfscope}%
\begin{pgfscope}%
\pgfsetbuttcap%
\pgfsetroundjoin%
\definecolor{currentfill}{rgb}{0.000000,0.000000,0.000000}%
\pgfsetfillcolor{currentfill}%
\pgfsetlinewidth{0.803000pt}%
\definecolor{currentstroke}{rgb}{0.000000,0.000000,0.000000}%
\pgfsetstrokecolor{currentstroke}%
\pgfsetdash{}{0pt}%
\pgfsys@defobject{currentmarker}{\pgfqpoint{0.000000in}{0.000000in}}{\pgfqpoint{0.000000in}{0.048611in}}{%
\pgfpathmoveto{\pgfqpoint{0.000000in}{0.000000in}}%
\pgfpathlineto{\pgfqpoint{0.000000in}{0.048611in}}%
\pgfusepath{stroke,fill}%
}%
\begin{pgfscope}%
\pgfsys@transformshift{6.918182in}{2.640000in}%
\pgfsys@useobject{currentmarker}{}%
\end{pgfscope}%
\end{pgfscope}%
\begin{pgfscope}%
\definecolor{textcolor}{rgb}{0.000000,0.000000,0.000000}%
\pgfsetstrokecolor{textcolor}%
\pgfsetfillcolor{textcolor}%
\pgftext[x=6.918182in,y=0.232778in,,top]{\color{textcolor}\rmfamily\fontsize{10.000000}{12.000000}\selectfont \(\displaystyle {7}\)}%
\end{pgfscope}%
\begin{pgfscope}%
\pgfpathrectangle{\pgfqpoint{1.000000in}{0.330000in}}{\pgfqpoint{6.200000in}{2.310000in}}%
\pgfusepath{clip}%
\pgfsetbuttcap%
\pgfsetroundjoin%
\pgfsetlinewidth{0.803000pt}%
\definecolor{currentstroke}{rgb}{0.000000,0.000000,0.000000}%
\pgfsetstrokecolor{currentstroke}%
\pgfsetdash{{0.800000pt}{1.320000pt}}{0.000000pt}%
\pgfpathmoveto{\pgfqpoint{1.000000in}{0.435000in}}%
\pgfpathlineto{\pgfqpoint{7.200000in}{0.435000in}}%
\pgfusepath{stroke}%
\end{pgfscope}%
\begin{pgfscope}%
\pgfsetbuttcap%
\pgfsetroundjoin%
\definecolor{currentfill}{rgb}{0.000000,0.000000,0.000000}%
\pgfsetfillcolor{currentfill}%
\pgfsetlinewidth{0.803000pt}%
\definecolor{currentstroke}{rgb}{0.000000,0.000000,0.000000}%
\pgfsetstrokecolor{currentstroke}%
\pgfsetdash{}{0pt}%
\pgfsys@defobject{currentmarker}{\pgfqpoint{-0.048611in}{0.000000in}}{\pgfqpoint{-0.000000in}{0.000000in}}{%
\pgfpathmoveto{\pgfqpoint{-0.000000in}{0.000000in}}%
\pgfpathlineto{\pgfqpoint{-0.048611in}{0.000000in}}%
\pgfusepath{stroke,fill}%
}%
\begin{pgfscope}%
\pgfsys@transformshift{1.000000in}{0.435000in}%
\pgfsys@useobject{currentmarker}{}%
\end{pgfscope}%
\end{pgfscope}%
\begin{pgfscope}%
\pgfsetbuttcap%
\pgfsetroundjoin%
\definecolor{currentfill}{rgb}{0.000000,0.000000,0.000000}%
\pgfsetfillcolor{currentfill}%
\pgfsetlinewidth{0.803000pt}%
\definecolor{currentstroke}{rgb}{0.000000,0.000000,0.000000}%
\pgfsetstrokecolor{currentstroke}%
\pgfsetdash{}{0pt}%
\pgfsys@defobject{currentmarker}{\pgfqpoint{0.000000in}{0.000000in}}{\pgfqpoint{0.048611in}{0.000000in}}{%
\pgfpathmoveto{\pgfqpoint{0.000000in}{0.000000in}}%
\pgfpathlineto{\pgfqpoint{0.048611in}{0.000000in}}%
\pgfusepath{stroke,fill}%
}%
\begin{pgfscope}%
\pgfsys@transformshift{7.200000in}{0.435000in}%
\pgfsys@useobject{currentmarker}{}%
\end{pgfscope}%
\end{pgfscope}%
\begin{pgfscope}%
\definecolor{textcolor}{rgb}{0.000000,0.000000,0.000000}%
\pgfsetstrokecolor{textcolor}%
\pgfsetfillcolor{textcolor}%
\pgftext[x=0.833333in, y=0.386775in, left, base]{\color{textcolor}\rmfamily\fontsize{10.000000}{12.000000}\selectfont \(\displaystyle {0}\)}%
\end{pgfscope}%
\begin{pgfscope}%
\pgfpathrectangle{\pgfqpoint{1.000000in}{0.330000in}}{\pgfqpoint{6.200000in}{2.310000in}}%
\pgfusepath{clip}%
\pgfsetbuttcap%
\pgfsetroundjoin%
\pgfsetlinewidth{0.803000pt}%
\definecolor{currentstroke}{rgb}{0.000000,0.000000,0.000000}%
\pgfsetstrokecolor{currentstroke}%
\pgfsetdash{{0.800000pt}{1.320000pt}}{0.000000pt}%
\pgfpathmoveto{\pgfqpoint{1.000000in}{0.855988in}}%
\pgfpathlineto{\pgfqpoint{7.200000in}{0.855988in}}%
\pgfusepath{stroke}%
\end{pgfscope}%
\begin{pgfscope}%
\pgfsetbuttcap%
\pgfsetroundjoin%
\definecolor{currentfill}{rgb}{0.000000,0.000000,0.000000}%
\pgfsetfillcolor{currentfill}%
\pgfsetlinewidth{0.803000pt}%
\definecolor{currentstroke}{rgb}{0.000000,0.000000,0.000000}%
\pgfsetstrokecolor{currentstroke}%
\pgfsetdash{}{0pt}%
\pgfsys@defobject{currentmarker}{\pgfqpoint{-0.048611in}{0.000000in}}{\pgfqpoint{-0.000000in}{0.000000in}}{%
\pgfpathmoveto{\pgfqpoint{-0.000000in}{0.000000in}}%
\pgfpathlineto{\pgfqpoint{-0.048611in}{0.000000in}}%
\pgfusepath{stroke,fill}%
}%
\begin{pgfscope}%
\pgfsys@transformshift{1.000000in}{0.855988in}%
\pgfsys@useobject{currentmarker}{}%
\end{pgfscope}%
\end{pgfscope}%
\begin{pgfscope}%
\pgfsetbuttcap%
\pgfsetroundjoin%
\definecolor{currentfill}{rgb}{0.000000,0.000000,0.000000}%
\pgfsetfillcolor{currentfill}%
\pgfsetlinewidth{0.803000pt}%
\definecolor{currentstroke}{rgb}{0.000000,0.000000,0.000000}%
\pgfsetstrokecolor{currentstroke}%
\pgfsetdash{}{0pt}%
\pgfsys@defobject{currentmarker}{\pgfqpoint{0.000000in}{0.000000in}}{\pgfqpoint{0.048611in}{0.000000in}}{%
\pgfpathmoveto{\pgfqpoint{0.000000in}{0.000000in}}%
\pgfpathlineto{\pgfqpoint{0.048611in}{0.000000in}}%
\pgfusepath{stroke,fill}%
}%
\begin{pgfscope}%
\pgfsys@transformshift{7.200000in}{0.855988in}%
\pgfsys@useobject{currentmarker}{}%
\end{pgfscope}%
\end{pgfscope}%
\begin{pgfscope}%
\definecolor{textcolor}{rgb}{0.000000,0.000000,0.000000}%
\pgfsetstrokecolor{textcolor}%
\pgfsetfillcolor{textcolor}%
\pgftext[x=0.833333in, y=0.807763in, left, base]{\color{textcolor}\rmfamily\fontsize{10.000000}{12.000000}\selectfont \(\displaystyle {2}\)}%
\end{pgfscope}%
\begin{pgfscope}%
\pgfpathrectangle{\pgfqpoint{1.000000in}{0.330000in}}{\pgfqpoint{6.200000in}{2.310000in}}%
\pgfusepath{clip}%
\pgfsetbuttcap%
\pgfsetroundjoin%
\pgfsetlinewidth{0.803000pt}%
\definecolor{currentstroke}{rgb}{0.000000,0.000000,0.000000}%
\pgfsetstrokecolor{currentstroke}%
\pgfsetdash{{0.800000pt}{1.320000pt}}{0.000000pt}%
\pgfpathmoveto{\pgfqpoint{1.000000in}{1.276976in}}%
\pgfpathlineto{\pgfqpoint{7.200000in}{1.276976in}}%
\pgfusepath{stroke}%
\end{pgfscope}%
\begin{pgfscope}%
\pgfsetbuttcap%
\pgfsetroundjoin%
\definecolor{currentfill}{rgb}{0.000000,0.000000,0.000000}%
\pgfsetfillcolor{currentfill}%
\pgfsetlinewidth{0.803000pt}%
\definecolor{currentstroke}{rgb}{0.000000,0.000000,0.000000}%
\pgfsetstrokecolor{currentstroke}%
\pgfsetdash{}{0pt}%
\pgfsys@defobject{currentmarker}{\pgfqpoint{-0.048611in}{0.000000in}}{\pgfqpoint{-0.000000in}{0.000000in}}{%
\pgfpathmoveto{\pgfqpoint{-0.000000in}{0.000000in}}%
\pgfpathlineto{\pgfqpoint{-0.048611in}{0.000000in}}%
\pgfusepath{stroke,fill}%
}%
\begin{pgfscope}%
\pgfsys@transformshift{1.000000in}{1.276976in}%
\pgfsys@useobject{currentmarker}{}%
\end{pgfscope}%
\end{pgfscope}%
\begin{pgfscope}%
\pgfsetbuttcap%
\pgfsetroundjoin%
\definecolor{currentfill}{rgb}{0.000000,0.000000,0.000000}%
\pgfsetfillcolor{currentfill}%
\pgfsetlinewidth{0.803000pt}%
\definecolor{currentstroke}{rgb}{0.000000,0.000000,0.000000}%
\pgfsetstrokecolor{currentstroke}%
\pgfsetdash{}{0pt}%
\pgfsys@defobject{currentmarker}{\pgfqpoint{0.000000in}{0.000000in}}{\pgfqpoint{0.048611in}{0.000000in}}{%
\pgfpathmoveto{\pgfqpoint{0.000000in}{0.000000in}}%
\pgfpathlineto{\pgfqpoint{0.048611in}{0.000000in}}%
\pgfusepath{stroke,fill}%
}%
\begin{pgfscope}%
\pgfsys@transformshift{7.200000in}{1.276976in}%
\pgfsys@useobject{currentmarker}{}%
\end{pgfscope}%
\end{pgfscope}%
\begin{pgfscope}%
\definecolor{textcolor}{rgb}{0.000000,0.000000,0.000000}%
\pgfsetstrokecolor{textcolor}%
\pgfsetfillcolor{textcolor}%
\pgftext[x=0.833333in, y=1.228751in, left, base]{\color{textcolor}\rmfamily\fontsize{10.000000}{12.000000}\selectfont \(\displaystyle {4}\)}%
\end{pgfscope}%
\begin{pgfscope}%
\pgfpathrectangle{\pgfqpoint{1.000000in}{0.330000in}}{\pgfqpoint{6.200000in}{2.310000in}}%
\pgfusepath{clip}%
\pgfsetbuttcap%
\pgfsetroundjoin%
\pgfsetlinewidth{0.803000pt}%
\definecolor{currentstroke}{rgb}{0.000000,0.000000,0.000000}%
\pgfsetstrokecolor{currentstroke}%
\pgfsetdash{{0.800000pt}{1.320000pt}}{0.000000pt}%
\pgfpathmoveto{\pgfqpoint{1.000000in}{1.697965in}}%
\pgfpathlineto{\pgfqpoint{7.200000in}{1.697965in}}%
\pgfusepath{stroke}%
\end{pgfscope}%
\begin{pgfscope}%
\pgfsetbuttcap%
\pgfsetroundjoin%
\definecolor{currentfill}{rgb}{0.000000,0.000000,0.000000}%
\pgfsetfillcolor{currentfill}%
\pgfsetlinewidth{0.803000pt}%
\definecolor{currentstroke}{rgb}{0.000000,0.000000,0.000000}%
\pgfsetstrokecolor{currentstroke}%
\pgfsetdash{}{0pt}%
\pgfsys@defobject{currentmarker}{\pgfqpoint{-0.048611in}{0.000000in}}{\pgfqpoint{-0.000000in}{0.000000in}}{%
\pgfpathmoveto{\pgfqpoint{-0.000000in}{0.000000in}}%
\pgfpathlineto{\pgfqpoint{-0.048611in}{0.000000in}}%
\pgfusepath{stroke,fill}%
}%
\begin{pgfscope}%
\pgfsys@transformshift{1.000000in}{1.697965in}%
\pgfsys@useobject{currentmarker}{}%
\end{pgfscope}%
\end{pgfscope}%
\begin{pgfscope}%
\pgfsetbuttcap%
\pgfsetroundjoin%
\definecolor{currentfill}{rgb}{0.000000,0.000000,0.000000}%
\pgfsetfillcolor{currentfill}%
\pgfsetlinewidth{0.803000pt}%
\definecolor{currentstroke}{rgb}{0.000000,0.000000,0.000000}%
\pgfsetstrokecolor{currentstroke}%
\pgfsetdash{}{0pt}%
\pgfsys@defobject{currentmarker}{\pgfqpoint{0.000000in}{0.000000in}}{\pgfqpoint{0.048611in}{0.000000in}}{%
\pgfpathmoveto{\pgfqpoint{0.000000in}{0.000000in}}%
\pgfpathlineto{\pgfqpoint{0.048611in}{0.000000in}}%
\pgfusepath{stroke,fill}%
}%
\begin{pgfscope}%
\pgfsys@transformshift{7.200000in}{1.697965in}%
\pgfsys@useobject{currentmarker}{}%
\end{pgfscope}%
\end{pgfscope}%
\begin{pgfscope}%
\definecolor{textcolor}{rgb}{0.000000,0.000000,0.000000}%
\pgfsetstrokecolor{textcolor}%
\pgfsetfillcolor{textcolor}%
\pgftext[x=0.833333in, y=1.649739in, left, base]{\color{textcolor}\rmfamily\fontsize{10.000000}{12.000000}\selectfont \(\displaystyle {6}\)}%
\end{pgfscope}%
\begin{pgfscope}%
\pgfpathrectangle{\pgfqpoint{1.000000in}{0.330000in}}{\pgfqpoint{6.200000in}{2.310000in}}%
\pgfusepath{clip}%
\pgfsetbuttcap%
\pgfsetroundjoin%
\pgfsetlinewidth{0.803000pt}%
\definecolor{currentstroke}{rgb}{0.000000,0.000000,0.000000}%
\pgfsetstrokecolor{currentstroke}%
\pgfsetdash{{0.800000pt}{1.320000pt}}{0.000000pt}%
\pgfpathmoveto{\pgfqpoint{1.000000in}{2.118953in}}%
\pgfpathlineto{\pgfqpoint{7.200000in}{2.118953in}}%
\pgfusepath{stroke}%
\end{pgfscope}%
\begin{pgfscope}%
\pgfsetbuttcap%
\pgfsetroundjoin%
\definecolor{currentfill}{rgb}{0.000000,0.000000,0.000000}%
\pgfsetfillcolor{currentfill}%
\pgfsetlinewidth{0.803000pt}%
\definecolor{currentstroke}{rgb}{0.000000,0.000000,0.000000}%
\pgfsetstrokecolor{currentstroke}%
\pgfsetdash{}{0pt}%
\pgfsys@defobject{currentmarker}{\pgfqpoint{-0.048611in}{0.000000in}}{\pgfqpoint{-0.000000in}{0.000000in}}{%
\pgfpathmoveto{\pgfqpoint{-0.000000in}{0.000000in}}%
\pgfpathlineto{\pgfqpoint{-0.048611in}{0.000000in}}%
\pgfusepath{stroke,fill}%
}%
\begin{pgfscope}%
\pgfsys@transformshift{1.000000in}{2.118953in}%
\pgfsys@useobject{currentmarker}{}%
\end{pgfscope}%
\end{pgfscope}%
\begin{pgfscope}%
\pgfsetbuttcap%
\pgfsetroundjoin%
\definecolor{currentfill}{rgb}{0.000000,0.000000,0.000000}%
\pgfsetfillcolor{currentfill}%
\pgfsetlinewidth{0.803000pt}%
\definecolor{currentstroke}{rgb}{0.000000,0.000000,0.000000}%
\pgfsetstrokecolor{currentstroke}%
\pgfsetdash{}{0pt}%
\pgfsys@defobject{currentmarker}{\pgfqpoint{0.000000in}{0.000000in}}{\pgfqpoint{0.048611in}{0.000000in}}{%
\pgfpathmoveto{\pgfqpoint{0.000000in}{0.000000in}}%
\pgfpathlineto{\pgfqpoint{0.048611in}{0.000000in}}%
\pgfusepath{stroke,fill}%
}%
\begin{pgfscope}%
\pgfsys@transformshift{7.200000in}{2.118953in}%
\pgfsys@useobject{currentmarker}{}%
\end{pgfscope}%
\end{pgfscope}%
\begin{pgfscope}%
\definecolor{textcolor}{rgb}{0.000000,0.000000,0.000000}%
\pgfsetstrokecolor{textcolor}%
\pgfsetfillcolor{textcolor}%
\pgftext[x=0.833333in, y=2.070727in, left, base]{\color{textcolor}\rmfamily\fontsize{10.000000}{12.000000}\selectfont \(\displaystyle {8}\)}%
\end{pgfscope}%
\begin{pgfscope}%
\pgfpathrectangle{\pgfqpoint{1.000000in}{0.330000in}}{\pgfqpoint{6.200000in}{2.310000in}}%
\pgfusepath{clip}%
\pgfsetbuttcap%
\pgfsetroundjoin%
\pgfsetlinewidth{0.803000pt}%
\definecolor{currentstroke}{rgb}{0.000000,0.000000,0.000000}%
\pgfsetstrokecolor{currentstroke}%
\pgfsetdash{{0.800000pt}{1.320000pt}}{0.000000pt}%
\pgfpathmoveto{\pgfqpoint{1.000000in}{2.539941in}}%
\pgfpathlineto{\pgfqpoint{7.200000in}{2.539941in}}%
\pgfusepath{stroke}%
\end{pgfscope}%
\begin{pgfscope}%
\pgfsetbuttcap%
\pgfsetroundjoin%
\definecolor{currentfill}{rgb}{0.000000,0.000000,0.000000}%
\pgfsetfillcolor{currentfill}%
\pgfsetlinewidth{0.803000pt}%
\definecolor{currentstroke}{rgb}{0.000000,0.000000,0.000000}%
\pgfsetstrokecolor{currentstroke}%
\pgfsetdash{}{0pt}%
\pgfsys@defobject{currentmarker}{\pgfqpoint{-0.048611in}{0.000000in}}{\pgfqpoint{-0.000000in}{0.000000in}}{%
\pgfpathmoveto{\pgfqpoint{-0.000000in}{0.000000in}}%
\pgfpathlineto{\pgfqpoint{-0.048611in}{0.000000in}}%
\pgfusepath{stroke,fill}%
}%
\begin{pgfscope}%
\pgfsys@transformshift{1.000000in}{2.539941in}%
\pgfsys@useobject{currentmarker}{}%
\end{pgfscope}%
\end{pgfscope}%
\begin{pgfscope}%
\pgfsetbuttcap%
\pgfsetroundjoin%
\definecolor{currentfill}{rgb}{0.000000,0.000000,0.000000}%
\pgfsetfillcolor{currentfill}%
\pgfsetlinewidth{0.803000pt}%
\definecolor{currentstroke}{rgb}{0.000000,0.000000,0.000000}%
\pgfsetstrokecolor{currentstroke}%
\pgfsetdash{}{0pt}%
\pgfsys@defobject{currentmarker}{\pgfqpoint{0.000000in}{0.000000in}}{\pgfqpoint{0.048611in}{0.000000in}}{%
\pgfpathmoveto{\pgfqpoint{0.000000in}{0.000000in}}%
\pgfpathlineto{\pgfqpoint{0.048611in}{0.000000in}}%
\pgfusepath{stroke,fill}%
}%
\begin{pgfscope}%
\pgfsys@transformshift{7.200000in}{2.539941in}%
\pgfsys@useobject{currentmarker}{}%
\end{pgfscope}%
\end{pgfscope}%
\begin{pgfscope}%
\definecolor{textcolor}{rgb}{0.000000,0.000000,0.000000}%
\pgfsetstrokecolor{textcolor}%
\pgfsetfillcolor{textcolor}%
\pgftext[x=0.763888in, y=2.491716in, left, base]{\color{textcolor}\rmfamily\fontsize{10.000000}{12.000000}\selectfont \(\displaystyle {10}\)}%
\end{pgfscope}%
\begin{pgfscope}%
\pgfpathrectangle{\pgfqpoint{1.000000in}{0.330000in}}{\pgfqpoint{6.200000in}{2.310000in}}%
\pgfusepath{clip}%
\pgfsetrectcap%
\pgfsetroundjoin%
\pgfsetlinewidth{1.505625pt}%
\definecolor{currentstroke}{rgb}{0.121569,0.466667,0.705882}%
\pgfsetstrokecolor{currentstroke}%
\pgfsetdash{}{0pt}%
\pgfpathmoveto{\pgfqpoint{1.281818in}{0.435000in}}%
\pgfpathlineto{\pgfqpoint{1.281818in}{0.435000in}}%
\pgfpathlineto{\pgfqpoint{1.986364in}{0.899278in}}%
\pgfpathlineto{\pgfqpoint{2.690909in}{1.360810in}}%
\pgfpathlineto{\pgfqpoint{3.395455in}{1.819596in}}%
\pgfpathlineto{\pgfqpoint{3.697403in}{2.015380in}}%
\pgfpathlineto{\pgfqpoint{4.368398in}{1.690867in}}%
\pgfpathlineto{\pgfqpoint{5.039394in}{1.363862in}}%
\pgfpathlineto{\pgfqpoint{5.710390in}{1.034367in}}%
\pgfpathlineto{\pgfqpoint{6.381385in}{0.702380in}}%
\pgfpathlineto{\pgfqpoint{6.918182in}{0.435000in}}%
\pgfpathlineto{\pgfqpoint{6.918182in}{0.435000in}}%
\pgfusepath{stroke}%
\end{pgfscope}%
\begin{pgfscope}%
\pgfpathrectangle{\pgfqpoint{1.000000in}{0.330000in}}{\pgfqpoint{6.200000in}{2.310000in}}%
\pgfusepath{clip}%
\pgfsetrectcap%
\pgfsetroundjoin%
\pgfsetlinewidth{1.505625pt}%
\definecolor{currentstroke}{rgb}{1.000000,0.498039,0.054902}%
\pgfsetstrokecolor{currentstroke}%
\pgfsetdash{}{0pt}%
\pgfpathmoveto{\pgfqpoint{1.281818in}{0.435000in}}%
\pgfpathlineto{\pgfqpoint{1.281818in}{0.435000in}}%
\pgfpathlineto{\pgfqpoint{2.120563in}{0.581822in}}%
\pgfpathlineto{\pgfqpoint{2.959307in}{0.726374in}}%
\pgfpathlineto{\pgfqpoint{3.697403in}{0.851703in}}%
\pgfpathlineto{\pgfqpoint{4.536147in}{0.746410in}}%
\pgfpathlineto{\pgfqpoint{5.374892in}{0.638847in}}%
\pgfpathlineto{\pgfqpoint{6.213636in}{0.529013in}}%
\pgfpathlineto{\pgfqpoint{6.918182in}{0.435000in}}%
\pgfpathlineto{\pgfqpoint{6.918182in}{0.435000in}}%
\pgfusepath{stroke}%
\end{pgfscope}%
\begin{pgfscope}%
\pgfpathrectangle{\pgfqpoint{1.000000in}{0.330000in}}{\pgfqpoint{6.200000in}{2.310000in}}%
\pgfusepath{clip}%
\pgfsetrectcap%
\pgfsetroundjoin%
\pgfsetlinewidth{1.505625pt}%
\definecolor{currentstroke}{rgb}{0.172549,0.627451,0.172549}%
\pgfsetstrokecolor{currentstroke}%
\pgfsetdash{}{0pt}%
\pgfpathmoveto{\pgfqpoint{1.281818in}{0.435000in}}%
\pgfpathlineto{\pgfqpoint{1.281818in}{0.435000in}}%
\pgfpathlineto{\pgfqpoint{1.885714in}{0.646882in}}%
\pgfpathlineto{\pgfqpoint{2.489610in}{0.856409in}}%
\pgfpathlineto{\pgfqpoint{3.093506in}{1.063583in}}%
\pgfpathlineto{\pgfqpoint{3.697403in}{1.268405in}}%
\pgfpathlineto{\pgfqpoint{4.301299in}{1.117241in}}%
\pgfpathlineto{\pgfqpoint{4.905195in}{0.963724in}}%
\pgfpathlineto{\pgfqpoint{5.509091in}{0.807852in}}%
\pgfpathlineto{\pgfqpoint{6.112987in}{0.649626in}}%
\pgfpathlineto{\pgfqpoint{6.716883in}{0.489047in}}%
\pgfpathlineto{\pgfqpoint{6.918182in}{0.435000in}}%
\pgfpathlineto{\pgfqpoint{6.918182in}{0.435000in}}%
\pgfusepath{stroke}%
\end{pgfscope}%
\begin{pgfscope}%
\pgfpathrectangle{\pgfqpoint{1.000000in}{0.330000in}}{\pgfqpoint{6.200000in}{2.310000in}}%
\pgfusepath{clip}%
\pgfsetrectcap%
\pgfsetroundjoin%
\pgfsetlinewidth{1.505625pt}%
\definecolor{currentstroke}{rgb}{0.839216,0.152941,0.156863}%
\pgfsetstrokecolor{currentstroke}%
\pgfsetdash{}{0pt}%
\pgfpathmoveto{\pgfqpoint{1.281818in}{0.435000in}}%
\pgfpathlineto{\pgfqpoint{1.281818in}{0.435000in}}%
\pgfpathlineto{\pgfqpoint{2.019913in}{0.601273in}}%
\pgfpathlineto{\pgfqpoint{2.758009in}{0.765285in}}%
\pgfpathlineto{\pgfqpoint{3.496104in}{0.927037in}}%
\pgfpathlineto{\pgfqpoint{3.697403in}{0.970761in}}%
\pgfpathlineto{\pgfqpoint{4.435498in}{0.851783in}}%
\pgfpathlineto{\pgfqpoint{5.173593in}{0.730546in}}%
\pgfpathlineto{\pgfqpoint{5.911688in}{0.607048in}}%
\pgfpathlineto{\pgfqpoint{6.649784in}{0.481289in}}%
\pgfpathlineto{\pgfqpoint{6.918182in}{0.435000in}}%
\pgfpathlineto{\pgfqpoint{6.918182in}{0.435000in}}%
\pgfusepath{stroke}%
\end{pgfscope}%
\begin{pgfscope}%
\pgfpathrectangle{\pgfqpoint{1.000000in}{0.330000in}}{\pgfqpoint{6.200000in}{2.310000in}}%
\pgfusepath{clip}%
\pgfsetrectcap%
\pgfsetroundjoin%
\pgfsetlinewidth{1.505625pt}%
\definecolor{currentstroke}{rgb}{0.580392,0.403922,0.741176}%
\pgfsetstrokecolor{currentstroke}%
\pgfsetdash{}{0pt}%
\pgfpathmoveto{\pgfqpoint{1.281818in}{0.435000in}}%
\pgfpathlineto{\pgfqpoint{1.281818in}{0.435000in}}%
\pgfpathlineto{\pgfqpoint{1.919264in}{0.626643in}}%
\pgfpathlineto{\pgfqpoint{2.556710in}{0.816038in}}%
\pgfpathlineto{\pgfqpoint{3.194156in}{1.003186in}}%
\pgfpathlineto{\pgfqpoint{3.697403in}{1.149347in}}%
\pgfpathlineto{\pgfqpoint{4.334848in}{1.012521in}}%
\pgfpathlineto{\pgfqpoint{4.972294in}{0.873446in}}%
\pgfpathlineto{\pgfqpoint{5.609740in}{0.732124in}}%
\pgfpathlineto{\pgfqpoint{6.247186in}{0.588553in}}%
\pgfpathlineto{\pgfqpoint{6.851082in}{0.450466in}}%
\pgfpathlineto{\pgfqpoint{6.918182in}{0.435000in}}%
\pgfpathlineto{\pgfqpoint{6.918182in}{0.435000in}}%
\pgfusepath{stroke}%
\end{pgfscope}%
\begin{pgfscope}%
\pgfpathrectangle{\pgfqpoint{1.000000in}{0.330000in}}{\pgfqpoint{6.200000in}{2.310000in}}%
\pgfusepath{clip}%
\pgfsetrectcap%
\pgfsetroundjoin%
\pgfsetlinewidth{1.505625pt}%
\definecolor{currentstroke}{rgb}{0.549020,0.337255,0.294118}%
\pgfsetstrokecolor{currentstroke}%
\pgfsetdash{}{0pt}%
\pgfpathmoveto{\pgfqpoint{1.281818in}{0.435000in}}%
\pgfpathlineto{\pgfqpoint{1.281818in}{0.435000in}}%
\pgfpathlineto{\pgfqpoint{2.019913in}{1.080091in}}%
\pgfpathlineto{\pgfqpoint{2.758009in}{1.722168in}}%
\pgfpathlineto{\pgfqpoint{3.496104in}{2.361231in}}%
\pgfpathlineto{\pgfqpoint{3.697403in}{2.535000in}}%
\pgfpathlineto{\pgfqpoint{4.401948in}{2.080528in}}%
\pgfpathlineto{\pgfqpoint{5.106494in}{1.623311in}}%
\pgfpathlineto{\pgfqpoint{5.811039in}{1.163347in}}%
\pgfpathlineto{\pgfqpoint{6.515584in}{0.700637in}}%
\pgfpathlineto{\pgfqpoint{6.918182in}{0.435000in}}%
\pgfpathlineto{\pgfqpoint{6.918182in}{0.435000in}}%
\pgfusepath{stroke}%
\end{pgfscope}%
\begin{pgfscope}%
\pgfpathrectangle{\pgfqpoint{1.000000in}{0.330000in}}{\pgfqpoint{6.200000in}{2.310000in}}%
\pgfusepath{clip}%
\pgfsetrectcap%
\pgfsetroundjoin%
\pgfsetlinewidth{1.505625pt}%
\definecolor{currentstroke}{rgb}{0.890196,0.466667,0.760784}%
\pgfsetstrokecolor{currentstroke}%
\pgfsetdash{}{0pt}%
\pgfpathmoveto{\pgfqpoint{1.281818in}{0.435000in}}%
\pgfpathlineto{\pgfqpoint{1.281818in}{0.435000in}}%
\pgfpathlineto{\pgfqpoint{1.919264in}{0.772852in}}%
\pgfpathlineto{\pgfqpoint{2.556710in}{1.108081in}}%
\pgfpathlineto{\pgfqpoint{3.194156in}{1.440688in}}%
\pgfpathlineto{\pgfqpoint{3.697403in}{1.701422in}}%
\pgfpathlineto{\pgfqpoint{4.301299in}{1.469067in}}%
\pgfpathlineto{\pgfqpoint{4.905195in}{1.234359in}}%
\pgfpathlineto{\pgfqpoint{5.509091in}{0.997297in}}%
\pgfpathlineto{\pgfqpoint{6.112987in}{0.757881in}}%
\pgfpathlineto{\pgfqpoint{6.716883in}{0.516111in}}%
\pgfpathlineto{\pgfqpoint{6.918182in}{0.435000in}}%
\pgfpathlineto{\pgfqpoint{6.918182in}{0.435000in}}%
\pgfusepath{stroke}%
\end{pgfscope}%
\begin{pgfscope}%
\pgfsetroundcap%
\pgfsetroundjoin%
\pgfsetlinewidth{1.003750pt}%
\definecolor{currentstroke}{rgb}{0.000000,0.000000,0.000000}%
\pgfsetstrokecolor{currentstroke}%
\pgfsetdash{}{0pt}%
\pgfpathmoveto{\pgfqpoint{4.130599in}{2.535000in}}%
\pgfpathquadraticcurveto{\pgfqpoint{3.927895in}{2.535000in}}{\pgfqpoint{3.725191in}{2.535000in}}%
\pgfusepath{stroke}%
\end{pgfscope}%
\begin{pgfscope}%
\pgfsetbuttcap%
\pgfsetmiterjoin%
\definecolor{currentfill}{rgb}{0.800000,0.800000,0.800000}%
\pgfsetfillcolor{currentfill}%
\pgfsetlinewidth{1.003750pt}%
\definecolor{currentstroke}{rgb}{0.000000,0.000000,0.000000}%
\pgfsetstrokecolor{currentstroke}%
\pgfsetdash{}{0pt}%
\pgfpathmoveto{\pgfqpoint{4.188351in}{2.438549in}}%
\pgfpathcurveto{\pgfqpoint{4.223073in}{2.403827in}}{\pgfqpoint{5.086291in}{2.403827in}}{\pgfqpoint{5.121014in}{2.438549in}}%
\pgfpathcurveto{\pgfqpoint{5.155736in}{2.473272in}}{\pgfqpoint{5.155736in}{2.596728in}}{\pgfqpoint{5.121014in}{2.631451in}}%
\pgfpathcurveto{\pgfqpoint{5.086291in}{2.666173in}}{\pgfqpoint{4.223073in}{2.666173in}}{\pgfqpoint{4.188351in}{2.631451in}}%
\pgfpathcurveto{\pgfqpoint{4.153629in}{2.596728in}}{\pgfqpoint{4.153629in}{2.473272in}}{\pgfqpoint{4.188351in}{2.438549in}}%
\pgfpathclose%
\pgfusepath{stroke,fill}%
\end{pgfscope}%
\begin{pgfscope}%
\definecolor{textcolor}{rgb}{0.000000,0.000000,0.000000}%
\pgfsetstrokecolor{textcolor}%
\pgfsetfillcolor{textcolor}%
\pgftext[x=5.086291in,y=2.535000in,right,]{\color{textcolor}\rmfamily\fontsize{10.000000}{12.000000}\selectfont \(\displaystyle M_u =\) 10.0 kft}%
\end{pgfscope}%
\begin{pgfscope}%
\pgfsetbuttcap%
\pgfsetmiterjoin%
\definecolor{currentfill}{rgb}{0.800000,0.800000,0.800000}%
\pgfsetfillcolor{currentfill}%
\pgfsetlinewidth{1.003750pt}%
\definecolor{currentstroke}{rgb}{0.000000,0.000000,0.000000}%
\pgfsetstrokecolor{currentstroke}%
\pgfsetdash{}{0pt}%
\pgfpathmoveto{\pgfqpoint{0.965278in}{0.358599in}}%
\pgfpathcurveto{\pgfqpoint{1.000000in}{0.323877in}}{\pgfqpoint{2.720682in}{0.323877in}}{\pgfqpoint{2.755404in}{0.358599in}}%
\pgfpathcurveto{\pgfqpoint{2.790127in}{0.393321in}}{\pgfqpoint{2.790127in}{0.668784in}}{\pgfqpoint{2.755404in}{0.703506in}}%
\pgfpathcurveto{\pgfqpoint{2.720682in}{0.738228in}}{\pgfqpoint{1.000000in}{0.738228in}}{\pgfqpoint{0.965278in}{0.703506in}}%
\pgfpathcurveto{\pgfqpoint{0.930556in}{0.668784in}}{\pgfqpoint{0.930556in}{0.393321in}}{\pgfqpoint{0.965278in}{0.358599in}}%
\pgfpathclose%
\pgfusepath{stroke,fill}%
\end{pgfscope}%
\begin{pgfscope}%
\definecolor{textcolor}{rgb}{0.000000,0.000000,0.000000}%
\pgfsetstrokecolor{textcolor}%
\pgfsetfillcolor{textcolor}%
\pgftext[x=1.000000in, y=0.580049in, left, base]{\color{textcolor}\rmfamily\fontsize{10.000000}{12.000000}\selectfont Max combo: 1.2D + 1.6L0}%
\end{pgfscope}%
\begin{pgfscope}%
\definecolor{textcolor}{rgb}{0.000000,0.000000,0.000000}%
\pgfsetstrokecolor{textcolor}%
\pgfsetfillcolor{textcolor}%
\pgftext[x=1.000000in, y=0.428043in, left, base]{\color{textcolor}\rmfamily\fontsize{10.000000}{12.000000}\selectfont ASCE7-16 Sec. 2.3.1 (LC 2)}%
\end{pgfscope}%
\end{pgfpicture}%
\makeatother%
\endgroup%

\end{center}
\caption{Moment Demand Envelope}
\end{figure}
L\textsubscript{p}, the limiting laterally unbraced length for the limit state of yielding, is calculated per AISC/ANSI 360-16 Eq. F2-5 as follows:
\begin{flalign*}
L_p = 1.76\cdot r_y \cdot \sqrt{\frac{E}{F_y}}  = 1.76\cdot 0.753 {\color{darkBlue}{\mathbf{ \; in}}} \cdot \sqrt{\frac{29000 {\color{darkBlue}{\mathbf{ \; ksi}}}}{50 {\color{darkBlue}{\mathbf{ \; ksi}}}}}  = \mathbf{2.7 {\color{darkBlue}{\mathbf{ \; ft}}}}
\end{flalign*}
r\textsubscript{ts}, a coefficient used in the calculation of L\textsubscript{r} and C\textsubscript{b}, is calculated per AISC/ANSI 360-16 Eq. F2-7 as follows:
\begin{flalign*}
r_{{ts}} = \sqrt{\frac{\sqrt{I_y \cdot C_w}}{S_x}}  = \sqrt{\frac{\sqrt{2.36 {\color{darkBlue}{\mathbf{ \; {\color{darkBlue}{\mathbf{ \; in}}}^{4}}}} \cdot 80.4 {\color{darkBlue}{\mathbf{ \; {\color{darkBlue}{\mathbf{ \; in}}}^{6}}}}}}{14.9 {\color{darkBlue}{\mathbf{ \; {\color{darkBlue}{\mathbf{ \; in}}}^{3}}}}}}  = \mathbf{1.0 {\color{darkBlue}{\mathbf{ \; in}}}}
\end{flalign*}
L\textsubscript{r}, the limiting unbraced length for the limit state of inelastic lateral-torsional buckling, is calculated per AISC/ANSI 360-16 Eq. F2-6 as follows:
\begin{flalign*}
L_r &= 1.95\cdot r_{ts} \cdot \frac{E}{0.7\cdot F_y} \sqrt{\frac{J \cdot c}{S_x \cdot h_0}+\sqrt{{\left(\frac{J \cdot c}{S_x \cdot h_0}\right)}^2+6.76{\left(\frac{0.7\cdot F_y}{E}\right)}^2}} \\ &= 1.95\cdot 1.0 {\color{darkBlue}{\mathbf{ \; in}}} \cdot \frac{29000 {\color{darkBlue}{\mathbf{ \; ksi}}}}{0.7\cdot 50 {\color{darkBlue}{\mathbf{ \; ksi}}}} \sqrt{\frac{0.07 {\color{darkBlue}{\mathbf{ \; {\color{darkBlue}{\mathbf{ \; in}}}^{4}}}} \cdot 1}{14.9 {\color{darkBlue}{\mathbf{ \; {\color{darkBlue}{\mathbf{ \; in}}}^{3}}}} \cdot 11.7 {\color{darkBlue}{\mathbf{ \; in}}}}+\sqrt{{\left(\frac{0.07 {\color{darkBlue}{\mathbf{ \; {\color{darkBlue}{\mathbf{ \; in}}}^{4}}}} \cdot 1}{14.9 {\color{darkBlue}{\mathbf{ \; {\color{darkBlue}{\mathbf{ \; in}}}^{3}}}} \cdot 11.7 {\color{darkBlue}{\mathbf{ \; in}}}}\right)}^2+6.76{\left(\frac{0.7\cdot 50 {\color{darkBlue}{\mathbf{ \; ksi}}}}{29000 {\color{darkBlue}{\mathbf{ \; ksi}}}}\right)}^2}} \\ &= \mathbf{7.7 {\color{darkBlue}{\mathbf{ \; ft}}}}
\end{flalign*}
\textlambda\textsubscript{web}, the web width-to-thickness ratio, is calculated per {AISC/ANSI 360-16 Table B4.1b} as follows:
\begin{flalign*}
\lambda_{{web}} = \frac{h}{t_w}  = \frac{10.85 {\color{darkBlue}{\mathbf{ \; in}}}}{0.2 {\color{darkBlue}{\mathbf{ \; in}}}}  = \mathbf{54.2 }
\end{flalign*}
\textlambda\textsubscript{P-web}, the limiting width-to-thickness ratio for compact/noncompact web, is calculated per {AISC/ANSI 360-16 Table B4.1b} as follows:
\begin{flalign*}
\lambda_{P-web} = 3.76\cdot \sqrt{\frac{E}{F_y}} = 3.76\cdot \sqrt{\frac{29000 {\color{darkBlue}{\mathbf{ \; ksi}}}}{50 {\color{darkBlue}{\mathbf{ \; ksi}}}}} = \mathbf{90.6}
\end{flalign*}
\textlambda\textsubscript{R-web}, the limiting width-to-thickness ratio for noncompact/slender web, is calculated per {AISC/ANSI 360-16 Table B4.1b} as follows:
\begin{flalign*}
\lambda_{R-web} = 5.7\cdot \sqrt{\frac{E}{F_y}} = 5.7\cdot \sqrt{\frac{29000 {\color{darkBlue}{\mathbf{ \; ksi}}}}{50 {\color{darkBlue}{\mathbf{ \; ksi}}}}} = \mathbf{137.3}
\end{flalign*}
\textlambda\textsubscript{web} $<$ \textlambda\textsubscript{P-web} \textrightarrow \; \textbf{Compact Web}
\\\\
\textlambda\textsubscript{flange}, the flange width-to-thickness ratio, is calculated per {AISC/ANSI 360-16 Table B4.1b} as follows:
\begin{flalign*}
\lambda_{{flange}} = \frac{b}{t}  = \frac{1.99 {\color{darkBlue}{\mathbf{ \; in}}}}{0.225 {\color{darkBlue}{\mathbf{ \; in}}}}  = \mathbf{8.8 }
\end{flalign*}
\textlambda\textsubscript{P-flange}, the limiting width-to-thickness ratio for compact/noncompact flange, is calculated per {AISC/ANSI 360-16 Table B4.1b} as follows:
\begin{flalign*}
\lambda_{P-flange} = 0.38\cdot \sqrt{\frac{E}{F_y}} = 0.38\cdot \sqrt{\frac{29000 {\color{darkBlue}{\mathbf{ \; ksi}}}}{50 {\color{darkBlue}{\mathbf{ \; ksi}}}}} = \mathbf{9.2}
\end{flalign*}
\textlambda\textsubscript{R-flange}, the limiting width-to-thickness ratio for noncompact/slender flange, is calculated per {AISC/ANSI 360-16 Table B4.1b} as follows:
\begin{flalign*}
\lambda_{R-flange} = 1\cdot \sqrt{\frac{E}{F_y}} = 1\cdot \sqrt{\frac{29000 {\color{darkBlue}{\mathbf{ \; ksi}}}}{50 {\color{darkBlue}{\mathbf{ \; ksi}}}}} = \mathbf{24.1}
\end{flalign*}
\textlambda\textsubscript{flange} $<$ \textlambda\textsubscript{P-flange} \textrightarrow \; \textbf{Compact Flange}
\\\\
Since \(\mathbf{{L_p} < {L_b} <= {L_r}}\) and the beam's flanges are \textbf{compact}, controlling limit state for flexure is \textbf{LTB (not to exceed capacity based on yielding)}.
\\\\
M\textsubscript{p}, the plastic bending moment, is calculated per AISC/ANSI 360-16 Eq. F2-1 as follows:
\begin{flalign*}
M_p = F_y \cdot Z_x  = 50 {\color{darkBlue}{\mathbf{ \; ksi}}} \cdot 17.4 {\color{darkBlue}{\mathbf{ \; {\color{darkBlue}{\mathbf{ \; in}}}^{3}}}}  = \mathbf{72.5 {\color{darkBlue}{\mathbf{ \; kft}}}}
\end{flalign*}
C\textsubscript{b}, the lateral-torsional buckling modification factor in the critical unbraced span for the critical load combination, is calculated per AISC/ANSI 360- 16 Sec. F1 as follows:
\\
\begin{flalign*}
C_b &= \frac{12.5\cdot M_{max}}{2.5\cdot M_{max}+3\cdot M_A+4\cdot M_B+3\cdot M_C} \\ &= \frac{12.5\cdot 10.0 {\color{darkBlue}{\mathbf{ \; kft}}}}{2.5\cdot 10.0 {\color{darkBlue}{\mathbf{ \; kft}}}+3\cdot 5.7 {\color{darkBlue}{\mathbf{ \; kft}}}+4\cdot 8.6 {\color{darkBlue}{\mathbf{ \; kft}}}+3\cdot 4.3 {\color{darkBlue}{\mathbf{ \; kft}}}} \\ &= \mathbf{1.4 }
\end{flalign*}
\\
For brevity, the C\textsubscript{b} calculation is not shown for each span. The following figure illustrates the value of C\textsubscript{b} for each span.
\begin{figure}[H]
\begin{center}
%% Creator: Matplotlib, PGF backend
%%
%% To include the figure in your LaTeX document, write
%%   \input{<filename>.pgf}
%%
%% Make sure the required packages are loaded in your preamble
%%   \usepackage{pgf}
%%
%% Figures using additional raster images can only be included by \input if
%% they are in the same directory as the main LaTeX file. For loading figures
%% from other directories you can use the `import` package
%%   \usepackage{import}
%%
%% and then include the figures with
%%   \import{<path to file>}{<filename>.pgf}
%%
%% Matplotlib used the following preamble
%%
\begingroup%
\makeatletter%
\begin{pgfpicture}%
\pgfpathrectangle{\pgfpointorigin}{\pgfqpoint{8.000000in}{1.000000in}}%
\pgfusepath{use as bounding box, clip}%
\begin{pgfscope}%
\pgfpathrectangle{\pgfqpoint{1.000000in}{0.110000in}}{\pgfqpoint{6.200000in}{0.770000in}}%
\pgfusepath{clip}%
\pgfsetrectcap%
\pgfsetroundjoin%
\pgfsetlinewidth{0.752812pt}%
\definecolor{currentstroke}{rgb}{0.000000,0.000000,0.000000}%
\pgfsetstrokecolor{currentstroke}%
\pgfsetdash{}{0pt}%
\pgfpathmoveto{\pgfqpoint{1.281818in}{0.880000in}}%
\pgfpathlineto{\pgfqpoint{6.918182in}{0.880000in}}%
\pgfusepath{stroke}%
\end{pgfscope}%
\begin{pgfscope}%
\pgfpathrectangle{\pgfqpoint{1.000000in}{0.110000in}}{\pgfqpoint{6.200000in}{0.770000in}}%
\pgfusepath{clip}%
\pgfsetrectcap%
\pgfsetroundjoin%
\pgfsetlinewidth{0.752812pt}%
\definecolor{currentstroke}{rgb}{0.000000,0.000000,0.000000}%
\pgfsetstrokecolor{currentstroke}%
\pgfsetdash{}{0pt}%
\pgfpathmoveto{\pgfqpoint{1.281818in}{0.865441in}}%
\pgfpathlineto{\pgfqpoint{6.918182in}{0.865441in}}%
\pgfusepath{stroke}%
\end{pgfscope}%
\begin{pgfscope}%
\pgfpathrectangle{\pgfqpoint{1.000000in}{0.110000in}}{\pgfqpoint{6.200000in}{0.770000in}}%
\pgfusepath{clip}%
\pgfsetrectcap%
\pgfsetroundjoin%
\pgfsetlinewidth{0.752812pt}%
\definecolor{currentstroke}{rgb}{0.000000,0.000000,0.000000}%
\pgfsetstrokecolor{currentstroke}%
\pgfsetdash{}{0pt}%
\pgfpathmoveto{\pgfqpoint{1.281818in}{0.110000in}}%
\pgfpathlineto{\pgfqpoint{6.918182in}{0.110000in}}%
\pgfusepath{stroke}%
\end{pgfscope}%
\begin{pgfscope}%
\pgfpathrectangle{\pgfqpoint{1.000000in}{0.110000in}}{\pgfqpoint{6.200000in}{0.770000in}}%
\pgfusepath{clip}%
\pgfsetrectcap%
\pgfsetroundjoin%
\pgfsetlinewidth{0.752812pt}%
\definecolor{currentstroke}{rgb}{0.000000,0.000000,0.000000}%
\pgfsetstrokecolor{currentstroke}%
\pgfsetdash{}{0pt}%
\pgfpathmoveto{\pgfqpoint{1.281818in}{0.124559in}}%
\pgfpathlineto{\pgfqpoint{6.918182in}{0.124559in}}%
\pgfusepath{stroke}%
\end{pgfscope}%
\begin{pgfscope}%
\pgfpathrectangle{\pgfqpoint{1.000000in}{0.110000in}}{\pgfqpoint{6.200000in}{0.770000in}}%
\pgfusepath{clip}%
\pgfsetbuttcap%
\pgfsetroundjoin%
\pgfsetlinewidth{1.505625pt}%
\definecolor{currentstroke}{rgb}{1.000000,0.000000,0.000000}%
\pgfsetstrokecolor{currentstroke}%
\pgfsetdash{{1.500000pt}{2.475000pt}}{0.000000pt}%
\pgfpathmoveto{\pgfqpoint{1.281818in}{0.110000in}}%
\pgfpathlineto{\pgfqpoint{1.281818in}{0.880000in}}%
\pgfusepath{stroke}%
\end{pgfscope}%
\begin{pgfscope}%
\pgfpathrectangle{\pgfqpoint{1.000000in}{0.110000in}}{\pgfqpoint{6.200000in}{0.770000in}}%
\pgfusepath{clip}%
\pgfsetbuttcap%
\pgfsetroundjoin%
\pgfsetlinewidth{1.505625pt}%
\definecolor{currentstroke}{rgb}{1.000000,0.000000,0.000000}%
\pgfsetstrokecolor{currentstroke}%
\pgfsetdash{{1.500000pt}{2.475000pt}}{0.000000pt}%
\pgfpathmoveto{\pgfqpoint{6.918182in}{0.110000in}}%
\pgfpathlineto{\pgfqpoint{6.918182in}{0.880000in}}%
\pgfusepath{stroke}%
\end{pgfscope}%
\begin{pgfscope}%
\pgfsetbuttcap%
\pgfsetmiterjoin%
\definecolor{currentfill}{rgb}{0.800000,0.800000,0.800000}%
\pgfsetfillcolor{currentfill}%
\pgfsetlinewidth{1.003750pt}%
\definecolor{currentstroke}{rgb}{0.000000,0.000000,0.000000}%
\pgfsetstrokecolor{currentstroke}%
\pgfsetdash{}{0pt}%
\pgfpathmoveto{\pgfqpoint{3.757214in}{0.396932in}}%
\pgfpathcurveto{\pgfqpoint{3.791936in}{0.362210in}}{\pgfqpoint{4.408064in}{0.362210in}}{\pgfqpoint{4.442786in}{0.396932in}}%
\pgfpathcurveto{\pgfqpoint{4.477508in}{0.431654in}}{\pgfqpoint{4.477508in}{0.555111in}}{\pgfqpoint{4.442786in}{0.589833in}}%
\pgfpathcurveto{\pgfqpoint{4.408064in}{0.624555in}}{\pgfqpoint{3.791936in}{0.624555in}}{\pgfqpoint{3.757214in}{0.589833in}}%
\pgfpathcurveto{\pgfqpoint{3.722492in}{0.555111in}}{\pgfqpoint{3.722492in}{0.431654in}}{\pgfqpoint{3.757214in}{0.396932in}}%
\pgfpathclose%
\pgfusepath{stroke,fill}%
\end{pgfscope}%
\begin{pgfscope}%
\definecolor{textcolor}{rgb}{0.000000,0.000000,0.000000}%
\pgfsetstrokecolor{textcolor}%
\pgfsetfillcolor{textcolor}%
\pgftext[x=4.100000in,y=0.493382in,,]{\color{textcolor}\rmfamily\fontsize{10.000000}{12.000000}\selectfont C\textsubscript{b} = 1.39}%
\end{pgfscope}%
\end{pgfpicture}%
\makeatother%
\endgroup%

\end{center}
\caption{C\textsubscript{b} Along Member}
\end{figure}
F\textsubscript{cr}, the buckling stress for the critical section in the critical unbraced span, is calculated per AISC/ANSI 360- 16 Eq. F2-4 as follows:
\begin{flalign*}
F_{cr} & = \cfrac{C_b \cdot \pi^2 \cdot {{E}}} {\left(\cfrac{L_b}{r_{ts}}\right)^2} \cdot \sqrt{1 + 0.078 \cdot \cfrac{{J} \cdot {c}}{{S_x} \cdot h_0} \cdot \left(\cfrac{L_b}{r_{ts}}\right)^2} \\ & = \cfrac{1.39  \cdot \pi^2 \cdot {{29000 {\color{darkBlue}{\mathbf{ \; ksi}}}}}} {\left(\cfrac{7 {\color{darkBlue}{\mathbf{ \; ft}}}}{1.0 {\color{darkBlue}{\mathbf{ \; in}}}}\right)^2} \cdot \sqrt{1 + 0.078 \cdot \cfrac{{0.0704 {\color{darkBlue}{\mathbf{ \; {\color{darkBlue}{\mathbf{ \; in}}}^{4}}}}} \cdot {1}}{{14.9 {\color{darkBlue}{\mathbf{ \; {\color{darkBlue}{\mathbf{ \; in}}}^{3}}}}} \cdot 11.7 {\color{darkBlue}{\mathbf{ \; in}}}} \cdot \left(\cfrac{7 {\color{darkBlue}{\mathbf{ \; ft}}}}{1.0 {\color{darkBlue}{\mathbf{ \; in}}}}\right)^2} = \mathbf{58.2 {\color{darkBlue}{\mathbf{ \; ksi}}}}
\end{flalign*}
\\
\textphi\textsubscript{b}, the resistance factor for bending, is determined per AISC/ANSI 360-16 {\S}F1a as \textbf{0.9}.
\\\\
\textphi\textsubscript{b}M\textsubscript{n}, the design flexural strength, is calculated per AISC/ANSI 360-16 Eq. F2-2 as follows:
\begin{flalign*}
{\phi_b}{M_{n}} & = {\phi_b} \cdot {C_b} \cdot {{M_p} - 0.7 \cdot {F_{y}} \cdot {S_{x}} \cdot \frac{{L_b} - {L_p}}{{L_r} - {L_p}}} < {\phi_b} \cdot {M_p} \\ & = {0.9} \cdot {1.39 } \cdot {{72.5 {\color{darkBlue}{\mathbf{ \; kft}}}} - 0.7 \cdot {50 {\color{darkBlue}{\mathbf{ \; ksi}}}} \cdot {14.9 {\color{darkBlue}{\mathbf{ \; {\color{darkBlue}{\mathbf{ \; in}}}^{3}}}}} \cdot \frac{{{7 {\color{darkBlue}{\mathbf{ \; ft}}}}} - {2.7 {\color{darkBlue}{\mathbf{ \; ft}}}}}{{7.7 {\color{darkBlue}{\mathbf{ \; ft}}}} - {2.7 {\color{darkBlue}{\mathbf{ \; ft}}}}}} < {0.9} \cdot {72.5 {\color{darkBlue}{\mathbf{ \; kft}}}} = \mathbf{59.8 {\color{darkBlue}{\mathbf{ \; kft}}}}
\end{flalign*}
\vspace{-20pt}
{\setlength{\mathindent}{0cm}
\begin{flalign*}
\mathbf{|M_u| = 10.0 {\color{darkBlue}{\mathbf{ \; kft}}}  \;  < \phi_b \cdot M_n = 59.8 {\color{darkBlue}{\mathbf{ \; kft}}}  \;  (DCR = 0.17 - OK)}
\end{flalign*}
\textbf{(LTB controls)}
%	-------------------------------- SHEAR CHECK ---------------------------------
\section{Shear Check}
\begin{figure}[H]
\begin{center}
%% Creator: Matplotlib, PGF backend
%%
%% To include the figure in your LaTeX document, write
%%   \input{<filename>.pgf}
%%
%% Make sure the required packages are loaded in your preamble
%%   \usepackage{pgf}
%%
%% Figures using additional raster images can only be included by \input if
%% they are in the same directory as the main LaTeX file. For loading figures
%% from other directories you can use the `import` package
%%   \usepackage{import}
%%
%% and then include the figures with
%%   \import{<path to file>}{<filename>.pgf}
%%
%% Matplotlib used the following preamble
%%
\begingroup%
\makeatletter%
\begin{pgfpicture}%
\pgfpathrectangle{\pgfpointorigin}{\pgfqpoint{8.000000in}{3.000000in}}%
\pgfusepath{use as bounding box, clip}%
\begin{pgfscope}%
\pgfsetbuttcap%
\pgfsetmiterjoin%
\definecolor{currentfill}{rgb}{1.000000,1.000000,1.000000}%
\pgfsetfillcolor{currentfill}%
\pgfsetlinewidth{0.000000pt}%
\definecolor{currentstroke}{rgb}{1.000000,1.000000,1.000000}%
\pgfsetstrokecolor{currentstroke}%
\pgfsetdash{}{0pt}%
\pgfpathmoveto{\pgfqpoint{0.000000in}{0.000000in}}%
\pgfpathlineto{\pgfqpoint{8.000000in}{0.000000in}}%
\pgfpathlineto{\pgfqpoint{8.000000in}{3.000000in}}%
\pgfpathlineto{\pgfqpoint{0.000000in}{3.000000in}}%
\pgfpathclose%
\pgfusepath{fill}%
\end{pgfscope}%
\begin{pgfscope}%
\pgfsetbuttcap%
\pgfsetmiterjoin%
\definecolor{currentfill}{rgb}{1.000000,1.000000,1.000000}%
\pgfsetfillcolor{currentfill}%
\pgfsetlinewidth{0.000000pt}%
\definecolor{currentstroke}{rgb}{0.000000,0.000000,0.000000}%
\pgfsetstrokecolor{currentstroke}%
\pgfsetstrokeopacity{0.000000}%
\pgfsetdash{}{0pt}%
\pgfpathmoveto{\pgfqpoint{1.000000in}{0.330000in}}%
\pgfpathlineto{\pgfqpoint{7.200000in}{0.330000in}}%
\pgfpathlineto{\pgfqpoint{7.200000in}{2.640000in}}%
\pgfpathlineto{\pgfqpoint{1.000000in}{2.640000in}}%
\pgfpathclose%
\pgfusepath{fill}%
\end{pgfscope}%
\begin{pgfscope}%
\pgfpathrectangle{\pgfqpoint{1.000000in}{0.330000in}}{\pgfqpoint{6.200000in}{2.310000in}}%
\pgfusepath{clip}%
\pgfsetbuttcap%
\pgfsetroundjoin%
\pgfsetlinewidth{0.803000pt}%
\definecolor{currentstroke}{rgb}{0.000000,0.000000,0.000000}%
\pgfsetstrokecolor{currentstroke}%
\pgfsetdash{{0.800000pt}{1.320000pt}}{0.000000pt}%
\pgfpathmoveto{\pgfqpoint{1.281818in}{0.330000in}}%
\pgfpathlineto{\pgfqpoint{1.281818in}{2.640000in}}%
\pgfusepath{stroke}%
\end{pgfscope}%
\begin{pgfscope}%
\pgfsetbuttcap%
\pgfsetroundjoin%
\definecolor{currentfill}{rgb}{0.000000,0.000000,0.000000}%
\pgfsetfillcolor{currentfill}%
\pgfsetlinewidth{0.803000pt}%
\definecolor{currentstroke}{rgb}{0.000000,0.000000,0.000000}%
\pgfsetstrokecolor{currentstroke}%
\pgfsetdash{}{0pt}%
\pgfsys@defobject{currentmarker}{\pgfqpoint{0.000000in}{-0.048611in}}{\pgfqpoint{0.000000in}{0.000000in}}{%
\pgfpathmoveto{\pgfqpoint{0.000000in}{0.000000in}}%
\pgfpathlineto{\pgfqpoint{0.000000in}{-0.048611in}}%
\pgfusepath{stroke,fill}%
}%
\begin{pgfscope}%
\pgfsys@transformshift{1.281818in}{0.330000in}%
\pgfsys@useobject{currentmarker}{}%
\end{pgfscope}%
\end{pgfscope}%
\begin{pgfscope}%
\pgfsetbuttcap%
\pgfsetroundjoin%
\definecolor{currentfill}{rgb}{0.000000,0.000000,0.000000}%
\pgfsetfillcolor{currentfill}%
\pgfsetlinewidth{0.803000pt}%
\definecolor{currentstroke}{rgb}{0.000000,0.000000,0.000000}%
\pgfsetstrokecolor{currentstroke}%
\pgfsetdash{}{0pt}%
\pgfsys@defobject{currentmarker}{\pgfqpoint{0.000000in}{0.000000in}}{\pgfqpoint{0.000000in}{0.048611in}}{%
\pgfpathmoveto{\pgfqpoint{0.000000in}{0.000000in}}%
\pgfpathlineto{\pgfqpoint{0.000000in}{0.048611in}}%
\pgfusepath{stroke,fill}%
}%
\begin{pgfscope}%
\pgfsys@transformshift{1.281818in}{2.640000in}%
\pgfsys@useobject{currentmarker}{}%
\end{pgfscope}%
\end{pgfscope}%
\begin{pgfscope}%
\definecolor{textcolor}{rgb}{0.000000,0.000000,0.000000}%
\pgfsetstrokecolor{textcolor}%
\pgfsetfillcolor{textcolor}%
\pgftext[x=1.281818in,y=0.232778in,,top]{\color{textcolor}\rmfamily\fontsize{10.000000}{12.000000}\selectfont \(\displaystyle {0}\)}%
\end{pgfscope}%
\begin{pgfscope}%
\pgfpathrectangle{\pgfqpoint{1.000000in}{0.330000in}}{\pgfqpoint{6.200000in}{2.310000in}}%
\pgfusepath{clip}%
\pgfsetbuttcap%
\pgfsetroundjoin%
\pgfsetlinewidth{0.803000pt}%
\definecolor{currentstroke}{rgb}{0.000000,0.000000,0.000000}%
\pgfsetstrokecolor{currentstroke}%
\pgfsetdash{{0.800000pt}{1.320000pt}}{0.000000pt}%
\pgfpathmoveto{\pgfqpoint{2.087013in}{0.330000in}}%
\pgfpathlineto{\pgfqpoint{2.087013in}{2.640000in}}%
\pgfusepath{stroke}%
\end{pgfscope}%
\begin{pgfscope}%
\pgfsetbuttcap%
\pgfsetroundjoin%
\definecolor{currentfill}{rgb}{0.000000,0.000000,0.000000}%
\pgfsetfillcolor{currentfill}%
\pgfsetlinewidth{0.803000pt}%
\definecolor{currentstroke}{rgb}{0.000000,0.000000,0.000000}%
\pgfsetstrokecolor{currentstroke}%
\pgfsetdash{}{0pt}%
\pgfsys@defobject{currentmarker}{\pgfqpoint{0.000000in}{-0.048611in}}{\pgfqpoint{0.000000in}{0.000000in}}{%
\pgfpathmoveto{\pgfqpoint{0.000000in}{0.000000in}}%
\pgfpathlineto{\pgfqpoint{0.000000in}{-0.048611in}}%
\pgfusepath{stroke,fill}%
}%
\begin{pgfscope}%
\pgfsys@transformshift{2.087013in}{0.330000in}%
\pgfsys@useobject{currentmarker}{}%
\end{pgfscope}%
\end{pgfscope}%
\begin{pgfscope}%
\pgfsetbuttcap%
\pgfsetroundjoin%
\definecolor{currentfill}{rgb}{0.000000,0.000000,0.000000}%
\pgfsetfillcolor{currentfill}%
\pgfsetlinewidth{0.803000pt}%
\definecolor{currentstroke}{rgb}{0.000000,0.000000,0.000000}%
\pgfsetstrokecolor{currentstroke}%
\pgfsetdash{}{0pt}%
\pgfsys@defobject{currentmarker}{\pgfqpoint{0.000000in}{0.000000in}}{\pgfqpoint{0.000000in}{0.048611in}}{%
\pgfpathmoveto{\pgfqpoint{0.000000in}{0.000000in}}%
\pgfpathlineto{\pgfqpoint{0.000000in}{0.048611in}}%
\pgfusepath{stroke,fill}%
}%
\begin{pgfscope}%
\pgfsys@transformshift{2.087013in}{2.640000in}%
\pgfsys@useobject{currentmarker}{}%
\end{pgfscope}%
\end{pgfscope}%
\begin{pgfscope}%
\definecolor{textcolor}{rgb}{0.000000,0.000000,0.000000}%
\pgfsetstrokecolor{textcolor}%
\pgfsetfillcolor{textcolor}%
\pgftext[x=2.087013in,y=0.232778in,,top]{\color{textcolor}\rmfamily\fontsize{10.000000}{12.000000}\selectfont \(\displaystyle {1}\)}%
\end{pgfscope}%
\begin{pgfscope}%
\pgfpathrectangle{\pgfqpoint{1.000000in}{0.330000in}}{\pgfqpoint{6.200000in}{2.310000in}}%
\pgfusepath{clip}%
\pgfsetbuttcap%
\pgfsetroundjoin%
\pgfsetlinewidth{0.803000pt}%
\definecolor{currentstroke}{rgb}{0.000000,0.000000,0.000000}%
\pgfsetstrokecolor{currentstroke}%
\pgfsetdash{{0.800000pt}{1.320000pt}}{0.000000pt}%
\pgfpathmoveto{\pgfqpoint{2.892208in}{0.330000in}}%
\pgfpathlineto{\pgfqpoint{2.892208in}{2.640000in}}%
\pgfusepath{stroke}%
\end{pgfscope}%
\begin{pgfscope}%
\pgfsetbuttcap%
\pgfsetroundjoin%
\definecolor{currentfill}{rgb}{0.000000,0.000000,0.000000}%
\pgfsetfillcolor{currentfill}%
\pgfsetlinewidth{0.803000pt}%
\definecolor{currentstroke}{rgb}{0.000000,0.000000,0.000000}%
\pgfsetstrokecolor{currentstroke}%
\pgfsetdash{}{0pt}%
\pgfsys@defobject{currentmarker}{\pgfqpoint{0.000000in}{-0.048611in}}{\pgfqpoint{0.000000in}{0.000000in}}{%
\pgfpathmoveto{\pgfqpoint{0.000000in}{0.000000in}}%
\pgfpathlineto{\pgfqpoint{0.000000in}{-0.048611in}}%
\pgfusepath{stroke,fill}%
}%
\begin{pgfscope}%
\pgfsys@transformshift{2.892208in}{0.330000in}%
\pgfsys@useobject{currentmarker}{}%
\end{pgfscope}%
\end{pgfscope}%
\begin{pgfscope}%
\pgfsetbuttcap%
\pgfsetroundjoin%
\definecolor{currentfill}{rgb}{0.000000,0.000000,0.000000}%
\pgfsetfillcolor{currentfill}%
\pgfsetlinewidth{0.803000pt}%
\definecolor{currentstroke}{rgb}{0.000000,0.000000,0.000000}%
\pgfsetstrokecolor{currentstroke}%
\pgfsetdash{}{0pt}%
\pgfsys@defobject{currentmarker}{\pgfqpoint{0.000000in}{0.000000in}}{\pgfqpoint{0.000000in}{0.048611in}}{%
\pgfpathmoveto{\pgfqpoint{0.000000in}{0.000000in}}%
\pgfpathlineto{\pgfqpoint{0.000000in}{0.048611in}}%
\pgfusepath{stroke,fill}%
}%
\begin{pgfscope}%
\pgfsys@transformshift{2.892208in}{2.640000in}%
\pgfsys@useobject{currentmarker}{}%
\end{pgfscope}%
\end{pgfscope}%
\begin{pgfscope}%
\definecolor{textcolor}{rgb}{0.000000,0.000000,0.000000}%
\pgfsetstrokecolor{textcolor}%
\pgfsetfillcolor{textcolor}%
\pgftext[x=2.892208in,y=0.232778in,,top]{\color{textcolor}\rmfamily\fontsize{10.000000}{12.000000}\selectfont \(\displaystyle {2}\)}%
\end{pgfscope}%
\begin{pgfscope}%
\pgfpathrectangle{\pgfqpoint{1.000000in}{0.330000in}}{\pgfqpoint{6.200000in}{2.310000in}}%
\pgfusepath{clip}%
\pgfsetbuttcap%
\pgfsetroundjoin%
\pgfsetlinewidth{0.803000pt}%
\definecolor{currentstroke}{rgb}{0.000000,0.000000,0.000000}%
\pgfsetstrokecolor{currentstroke}%
\pgfsetdash{{0.800000pt}{1.320000pt}}{0.000000pt}%
\pgfpathmoveto{\pgfqpoint{3.697403in}{0.330000in}}%
\pgfpathlineto{\pgfqpoint{3.697403in}{2.640000in}}%
\pgfusepath{stroke}%
\end{pgfscope}%
\begin{pgfscope}%
\pgfsetbuttcap%
\pgfsetroundjoin%
\definecolor{currentfill}{rgb}{0.000000,0.000000,0.000000}%
\pgfsetfillcolor{currentfill}%
\pgfsetlinewidth{0.803000pt}%
\definecolor{currentstroke}{rgb}{0.000000,0.000000,0.000000}%
\pgfsetstrokecolor{currentstroke}%
\pgfsetdash{}{0pt}%
\pgfsys@defobject{currentmarker}{\pgfqpoint{0.000000in}{-0.048611in}}{\pgfqpoint{0.000000in}{0.000000in}}{%
\pgfpathmoveto{\pgfqpoint{0.000000in}{0.000000in}}%
\pgfpathlineto{\pgfqpoint{0.000000in}{-0.048611in}}%
\pgfusepath{stroke,fill}%
}%
\begin{pgfscope}%
\pgfsys@transformshift{3.697403in}{0.330000in}%
\pgfsys@useobject{currentmarker}{}%
\end{pgfscope}%
\end{pgfscope}%
\begin{pgfscope}%
\pgfsetbuttcap%
\pgfsetroundjoin%
\definecolor{currentfill}{rgb}{0.000000,0.000000,0.000000}%
\pgfsetfillcolor{currentfill}%
\pgfsetlinewidth{0.803000pt}%
\definecolor{currentstroke}{rgb}{0.000000,0.000000,0.000000}%
\pgfsetstrokecolor{currentstroke}%
\pgfsetdash{}{0pt}%
\pgfsys@defobject{currentmarker}{\pgfqpoint{0.000000in}{0.000000in}}{\pgfqpoint{0.000000in}{0.048611in}}{%
\pgfpathmoveto{\pgfqpoint{0.000000in}{0.000000in}}%
\pgfpathlineto{\pgfqpoint{0.000000in}{0.048611in}}%
\pgfusepath{stroke,fill}%
}%
\begin{pgfscope}%
\pgfsys@transformshift{3.697403in}{2.640000in}%
\pgfsys@useobject{currentmarker}{}%
\end{pgfscope}%
\end{pgfscope}%
\begin{pgfscope}%
\definecolor{textcolor}{rgb}{0.000000,0.000000,0.000000}%
\pgfsetstrokecolor{textcolor}%
\pgfsetfillcolor{textcolor}%
\pgftext[x=3.697403in,y=0.232778in,,top]{\color{textcolor}\rmfamily\fontsize{10.000000}{12.000000}\selectfont \(\displaystyle {3}\)}%
\end{pgfscope}%
\begin{pgfscope}%
\pgfpathrectangle{\pgfqpoint{1.000000in}{0.330000in}}{\pgfqpoint{6.200000in}{2.310000in}}%
\pgfusepath{clip}%
\pgfsetbuttcap%
\pgfsetroundjoin%
\pgfsetlinewidth{0.803000pt}%
\definecolor{currentstroke}{rgb}{0.000000,0.000000,0.000000}%
\pgfsetstrokecolor{currentstroke}%
\pgfsetdash{{0.800000pt}{1.320000pt}}{0.000000pt}%
\pgfpathmoveto{\pgfqpoint{4.502597in}{0.330000in}}%
\pgfpathlineto{\pgfqpoint{4.502597in}{2.640000in}}%
\pgfusepath{stroke}%
\end{pgfscope}%
\begin{pgfscope}%
\pgfsetbuttcap%
\pgfsetroundjoin%
\definecolor{currentfill}{rgb}{0.000000,0.000000,0.000000}%
\pgfsetfillcolor{currentfill}%
\pgfsetlinewidth{0.803000pt}%
\definecolor{currentstroke}{rgb}{0.000000,0.000000,0.000000}%
\pgfsetstrokecolor{currentstroke}%
\pgfsetdash{}{0pt}%
\pgfsys@defobject{currentmarker}{\pgfqpoint{0.000000in}{-0.048611in}}{\pgfqpoint{0.000000in}{0.000000in}}{%
\pgfpathmoveto{\pgfqpoint{0.000000in}{0.000000in}}%
\pgfpathlineto{\pgfqpoint{0.000000in}{-0.048611in}}%
\pgfusepath{stroke,fill}%
}%
\begin{pgfscope}%
\pgfsys@transformshift{4.502597in}{0.330000in}%
\pgfsys@useobject{currentmarker}{}%
\end{pgfscope}%
\end{pgfscope}%
\begin{pgfscope}%
\pgfsetbuttcap%
\pgfsetroundjoin%
\definecolor{currentfill}{rgb}{0.000000,0.000000,0.000000}%
\pgfsetfillcolor{currentfill}%
\pgfsetlinewidth{0.803000pt}%
\definecolor{currentstroke}{rgb}{0.000000,0.000000,0.000000}%
\pgfsetstrokecolor{currentstroke}%
\pgfsetdash{}{0pt}%
\pgfsys@defobject{currentmarker}{\pgfqpoint{0.000000in}{0.000000in}}{\pgfqpoint{0.000000in}{0.048611in}}{%
\pgfpathmoveto{\pgfqpoint{0.000000in}{0.000000in}}%
\pgfpathlineto{\pgfqpoint{0.000000in}{0.048611in}}%
\pgfusepath{stroke,fill}%
}%
\begin{pgfscope}%
\pgfsys@transformshift{4.502597in}{2.640000in}%
\pgfsys@useobject{currentmarker}{}%
\end{pgfscope}%
\end{pgfscope}%
\begin{pgfscope}%
\definecolor{textcolor}{rgb}{0.000000,0.000000,0.000000}%
\pgfsetstrokecolor{textcolor}%
\pgfsetfillcolor{textcolor}%
\pgftext[x=4.502597in,y=0.232778in,,top]{\color{textcolor}\rmfamily\fontsize{10.000000}{12.000000}\selectfont \(\displaystyle {4}\)}%
\end{pgfscope}%
\begin{pgfscope}%
\pgfpathrectangle{\pgfqpoint{1.000000in}{0.330000in}}{\pgfqpoint{6.200000in}{2.310000in}}%
\pgfusepath{clip}%
\pgfsetbuttcap%
\pgfsetroundjoin%
\pgfsetlinewidth{0.803000pt}%
\definecolor{currentstroke}{rgb}{0.000000,0.000000,0.000000}%
\pgfsetstrokecolor{currentstroke}%
\pgfsetdash{{0.800000pt}{1.320000pt}}{0.000000pt}%
\pgfpathmoveto{\pgfqpoint{5.307792in}{0.330000in}}%
\pgfpathlineto{\pgfqpoint{5.307792in}{2.640000in}}%
\pgfusepath{stroke}%
\end{pgfscope}%
\begin{pgfscope}%
\pgfsetbuttcap%
\pgfsetroundjoin%
\definecolor{currentfill}{rgb}{0.000000,0.000000,0.000000}%
\pgfsetfillcolor{currentfill}%
\pgfsetlinewidth{0.803000pt}%
\definecolor{currentstroke}{rgb}{0.000000,0.000000,0.000000}%
\pgfsetstrokecolor{currentstroke}%
\pgfsetdash{}{0pt}%
\pgfsys@defobject{currentmarker}{\pgfqpoint{0.000000in}{-0.048611in}}{\pgfqpoint{0.000000in}{0.000000in}}{%
\pgfpathmoveto{\pgfqpoint{0.000000in}{0.000000in}}%
\pgfpathlineto{\pgfqpoint{0.000000in}{-0.048611in}}%
\pgfusepath{stroke,fill}%
}%
\begin{pgfscope}%
\pgfsys@transformshift{5.307792in}{0.330000in}%
\pgfsys@useobject{currentmarker}{}%
\end{pgfscope}%
\end{pgfscope}%
\begin{pgfscope}%
\pgfsetbuttcap%
\pgfsetroundjoin%
\definecolor{currentfill}{rgb}{0.000000,0.000000,0.000000}%
\pgfsetfillcolor{currentfill}%
\pgfsetlinewidth{0.803000pt}%
\definecolor{currentstroke}{rgb}{0.000000,0.000000,0.000000}%
\pgfsetstrokecolor{currentstroke}%
\pgfsetdash{}{0pt}%
\pgfsys@defobject{currentmarker}{\pgfqpoint{0.000000in}{0.000000in}}{\pgfqpoint{0.000000in}{0.048611in}}{%
\pgfpathmoveto{\pgfqpoint{0.000000in}{0.000000in}}%
\pgfpathlineto{\pgfqpoint{0.000000in}{0.048611in}}%
\pgfusepath{stroke,fill}%
}%
\begin{pgfscope}%
\pgfsys@transformshift{5.307792in}{2.640000in}%
\pgfsys@useobject{currentmarker}{}%
\end{pgfscope}%
\end{pgfscope}%
\begin{pgfscope}%
\definecolor{textcolor}{rgb}{0.000000,0.000000,0.000000}%
\pgfsetstrokecolor{textcolor}%
\pgfsetfillcolor{textcolor}%
\pgftext[x=5.307792in,y=0.232778in,,top]{\color{textcolor}\rmfamily\fontsize{10.000000}{12.000000}\selectfont \(\displaystyle {5}\)}%
\end{pgfscope}%
\begin{pgfscope}%
\pgfpathrectangle{\pgfqpoint{1.000000in}{0.330000in}}{\pgfqpoint{6.200000in}{2.310000in}}%
\pgfusepath{clip}%
\pgfsetbuttcap%
\pgfsetroundjoin%
\pgfsetlinewidth{0.803000pt}%
\definecolor{currentstroke}{rgb}{0.000000,0.000000,0.000000}%
\pgfsetstrokecolor{currentstroke}%
\pgfsetdash{{0.800000pt}{1.320000pt}}{0.000000pt}%
\pgfpathmoveto{\pgfqpoint{6.112987in}{0.330000in}}%
\pgfpathlineto{\pgfqpoint{6.112987in}{2.640000in}}%
\pgfusepath{stroke}%
\end{pgfscope}%
\begin{pgfscope}%
\pgfsetbuttcap%
\pgfsetroundjoin%
\definecolor{currentfill}{rgb}{0.000000,0.000000,0.000000}%
\pgfsetfillcolor{currentfill}%
\pgfsetlinewidth{0.803000pt}%
\definecolor{currentstroke}{rgb}{0.000000,0.000000,0.000000}%
\pgfsetstrokecolor{currentstroke}%
\pgfsetdash{}{0pt}%
\pgfsys@defobject{currentmarker}{\pgfqpoint{0.000000in}{-0.048611in}}{\pgfqpoint{0.000000in}{0.000000in}}{%
\pgfpathmoveto{\pgfqpoint{0.000000in}{0.000000in}}%
\pgfpathlineto{\pgfqpoint{0.000000in}{-0.048611in}}%
\pgfusepath{stroke,fill}%
}%
\begin{pgfscope}%
\pgfsys@transformshift{6.112987in}{0.330000in}%
\pgfsys@useobject{currentmarker}{}%
\end{pgfscope}%
\end{pgfscope}%
\begin{pgfscope}%
\pgfsetbuttcap%
\pgfsetroundjoin%
\definecolor{currentfill}{rgb}{0.000000,0.000000,0.000000}%
\pgfsetfillcolor{currentfill}%
\pgfsetlinewidth{0.803000pt}%
\definecolor{currentstroke}{rgb}{0.000000,0.000000,0.000000}%
\pgfsetstrokecolor{currentstroke}%
\pgfsetdash{}{0pt}%
\pgfsys@defobject{currentmarker}{\pgfqpoint{0.000000in}{0.000000in}}{\pgfqpoint{0.000000in}{0.048611in}}{%
\pgfpathmoveto{\pgfqpoint{0.000000in}{0.000000in}}%
\pgfpathlineto{\pgfqpoint{0.000000in}{0.048611in}}%
\pgfusepath{stroke,fill}%
}%
\begin{pgfscope}%
\pgfsys@transformshift{6.112987in}{2.640000in}%
\pgfsys@useobject{currentmarker}{}%
\end{pgfscope}%
\end{pgfscope}%
\begin{pgfscope}%
\definecolor{textcolor}{rgb}{0.000000,0.000000,0.000000}%
\pgfsetstrokecolor{textcolor}%
\pgfsetfillcolor{textcolor}%
\pgftext[x=6.112987in,y=0.232778in,,top]{\color{textcolor}\rmfamily\fontsize{10.000000}{12.000000}\selectfont \(\displaystyle {6}\)}%
\end{pgfscope}%
\begin{pgfscope}%
\pgfpathrectangle{\pgfqpoint{1.000000in}{0.330000in}}{\pgfqpoint{6.200000in}{2.310000in}}%
\pgfusepath{clip}%
\pgfsetbuttcap%
\pgfsetroundjoin%
\pgfsetlinewidth{0.803000pt}%
\definecolor{currentstroke}{rgb}{0.000000,0.000000,0.000000}%
\pgfsetstrokecolor{currentstroke}%
\pgfsetdash{{0.800000pt}{1.320000pt}}{0.000000pt}%
\pgfpathmoveto{\pgfqpoint{6.918182in}{0.330000in}}%
\pgfpathlineto{\pgfqpoint{6.918182in}{2.640000in}}%
\pgfusepath{stroke}%
\end{pgfscope}%
\begin{pgfscope}%
\pgfsetbuttcap%
\pgfsetroundjoin%
\definecolor{currentfill}{rgb}{0.000000,0.000000,0.000000}%
\pgfsetfillcolor{currentfill}%
\pgfsetlinewidth{0.803000pt}%
\definecolor{currentstroke}{rgb}{0.000000,0.000000,0.000000}%
\pgfsetstrokecolor{currentstroke}%
\pgfsetdash{}{0pt}%
\pgfsys@defobject{currentmarker}{\pgfqpoint{0.000000in}{-0.048611in}}{\pgfqpoint{0.000000in}{0.000000in}}{%
\pgfpathmoveto{\pgfqpoint{0.000000in}{0.000000in}}%
\pgfpathlineto{\pgfqpoint{0.000000in}{-0.048611in}}%
\pgfusepath{stroke,fill}%
}%
\begin{pgfscope}%
\pgfsys@transformshift{6.918182in}{0.330000in}%
\pgfsys@useobject{currentmarker}{}%
\end{pgfscope}%
\end{pgfscope}%
\begin{pgfscope}%
\pgfsetbuttcap%
\pgfsetroundjoin%
\definecolor{currentfill}{rgb}{0.000000,0.000000,0.000000}%
\pgfsetfillcolor{currentfill}%
\pgfsetlinewidth{0.803000pt}%
\definecolor{currentstroke}{rgb}{0.000000,0.000000,0.000000}%
\pgfsetstrokecolor{currentstroke}%
\pgfsetdash{}{0pt}%
\pgfsys@defobject{currentmarker}{\pgfqpoint{0.000000in}{0.000000in}}{\pgfqpoint{0.000000in}{0.048611in}}{%
\pgfpathmoveto{\pgfqpoint{0.000000in}{0.000000in}}%
\pgfpathlineto{\pgfqpoint{0.000000in}{0.048611in}}%
\pgfusepath{stroke,fill}%
}%
\begin{pgfscope}%
\pgfsys@transformshift{6.918182in}{2.640000in}%
\pgfsys@useobject{currentmarker}{}%
\end{pgfscope}%
\end{pgfscope}%
\begin{pgfscope}%
\definecolor{textcolor}{rgb}{0.000000,0.000000,0.000000}%
\pgfsetstrokecolor{textcolor}%
\pgfsetfillcolor{textcolor}%
\pgftext[x=6.918182in,y=0.232778in,,top]{\color{textcolor}\rmfamily\fontsize{10.000000}{12.000000}\selectfont \(\displaystyle {7}\)}%
\end{pgfscope}%
\begin{pgfscope}%
\pgfpathrectangle{\pgfqpoint{1.000000in}{0.330000in}}{\pgfqpoint{6.200000in}{2.310000in}}%
\pgfusepath{clip}%
\pgfsetbuttcap%
\pgfsetroundjoin%
\pgfsetlinewidth{0.803000pt}%
\definecolor{currentstroke}{rgb}{0.000000,0.000000,0.000000}%
\pgfsetstrokecolor{currentstroke}%
\pgfsetdash{{0.800000pt}{1.320000pt}}{0.000000pt}%
\pgfpathmoveto{\pgfqpoint{1.000000in}{0.623713in}}%
\pgfpathlineto{\pgfqpoint{7.200000in}{0.623713in}}%
\pgfusepath{stroke}%
\end{pgfscope}%
\begin{pgfscope}%
\pgfsetbuttcap%
\pgfsetroundjoin%
\definecolor{currentfill}{rgb}{0.000000,0.000000,0.000000}%
\pgfsetfillcolor{currentfill}%
\pgfsetlinewidth{0.803000pt}%
\definecolor{currentstroke}{rgb}{0.000000,0.000000,0.000000}%
\pgfsetstrokecolor{currentstroke}%
\pgfsetdash{}{0pt}%
\pgfsys@defobject{currentmarker}{\pgfqpoint{-0.048611in}{0.000000in}}{\pgfqpoint{-0.000000in}{0.000000in}}{%
\pgfpathmoveto{\pgfqpoint{-0.000000in}{0.000000in}}%
\pgfpathlineto{\pgfqpoint{-0.048611in}{0.000000in}}%
\pgfusepath{stroke,fill}%
}%
\begin{pgfscope}%
\pgfsys@transformshift{1.000000in}{0.623713in}%
\pgfsys@useobject{currentmarker}{}%
\end{pgfscope}%
\end{pgfscope}%
\begin{pgfscope}%
\pgfsetbuttcap%
\pgfsetroundjoin%
\definecolor{currentfill}{rgb}{0.000000,0.000000,0.000000}%
\pgfsetfillcolor{currentfill}%
\pgfsetlinewidth{0.803000pt}%
\definecolor{currentstroke}{rgb}{0.000000,0.000000,0.000000}%
\pgfsetstrokecolor{currentstroke}%
\pgfsetdash{}{0pt}%
\pgfsys@defobject{currentmarker}{\pgfqpoint{0.000000in}{0.000000in}}{\pgfqpoint{0.048611in}{0.000000in}}{%
\pgfpathmoveto{\pgfqpoint{0.000000in}{0.000000in}}%
\pgfpathlineto{\pgfqpoint{0.048611in}{0.000000in}}%
\pgfusepath{stroke,fill}%
}%
\begin{pgfscope}%
\pgfsys@transformshift{7.200000in}{0.623713in}%
\pgfsys@useobject{currentmarker}{}%
\end{pgfscope}%
\end{pgfscope}%
\begin{pgfscope}%
\definecolor{textcolor}{rgb}{0.000000,0.000000,0.000000}%
\pgfsetstrokecolor{textcolor}%
\pgfsetfillcolor{textcolor}%
\pgftext[x=0.725308in, y=0.575488in, left, base]{\color{textcolor}\rmfamily\fontsize{10.000000}{12.000000}\selectfont \(\displaystyle {\ensuremath{-}2}\)}%
\end{pgfscope}%
\begin{pgfscope}%
\pgfpathrectangle{\pgfqpoint{1.000000in}{0.330000in}}{\pgfqpoint{6.200000in}{2.310000in}}%
\pgfusepath{clip}%
\pgfsetbuttcap%
\pgfsetroundjoin%
\pgfsetlinewidth{0.803000pt}%
\definecolor{currentstroke}{rgb}{0.000000,0.000000,0.000000}%
\pgfsetstrokecolor{currentstroke}%
\pgfsetdash{{0.800000pt}{1.320000pt}}{0.000000pt}%
\pgfpathmoveto{\pgfqpoint{1.000000in}{0.980900in}}%
\pgfpathlineto{\pgfqpoint{7.200000in}{0.980900in}}%
\pgfusepath{stroke}%
\end{pgfscope}%
\begin{pgfscope}%
\pgfsetbuttcap%
\pgfsetroundjoin%
\definecolor{currentfill}{rgb}{0.000000,0.000000,0.000000}%
\pgfsetfillcolor{currentfill}%
\pgfsetlinewidth{0.803000pt}%
\definecolor{currentstroke}{rgb}{0.000000,0.000000,0.000000}%
\pgfsetstrokecolor{currentstroke}%
\pgfsetdash{}{0pt}%
\pgfsys@defobject{currentmarker}{\pgfqpoint{-0.048611in}{0.000000in}}{\pgfqpoint{-0.000000in}{0.000000in}}{%
\pgfpathmoveto{\pgfqpoint{-0.000000in}{0.000000in}}%
\pgfpathlineto{\pgfqpoint{-0.048611in}{0.000000in}}%
\pgfusepath{stroke,fill}%
}%
\begin{pgfscope}%
\pgfsys@transformshift{1.000000in}{0.980900in}%
\pgfsys@useobject{currentmarker}{}%
\end{pgfscope}%
\end{pgfscope}%
\begin{pgfscope}%
\pgfsetbuttcap%
\pgfsetroundjoin%
\definecolor{currentfill}{rgb}{0.000000,0.000000,0.000000}%
\pgfsetfillcolor{currentfill}%
\pgfsetlinewidth{0.803000pt}%
\definecolor{currentstroke}{rgb}{0.000000,0.000000,0.000000}%
\pgfsetstrokecolor{currentstroke}%
\pgfsetdash{}{0pt}%
\pgfsys@defobject{currentmarker}{\pgfqpoint{0.000000in}{0.000000in}}{\pgfqpoint{0.048611in}{0.000000in}}{%
\pgfpathmoveto{\pgfqpoint{0.000000in}{0.000000in}}%
\pgfpathlineto{\pgfqpoint{0.048611in}{0.000000in}}%
\pgfusepath{stroke,fill}%
}%
\begin{pgfscope}%
\pgfsys@transformshift{7.200000in}{0.980900in}%
\pgfsys@useobject{currentmarker}{}%
\end{pgfscope}%
\end{pgfscope}%
\begin{pgfscope}%
\definecolor{textcolor}{rgb}{0.000000,0.000000,0.000000}%
\pgfsetstrokecolor{textcolor}%
\pgfsetfillcolor{textcolor}%
\pgftext[x=0.725308in, y=0.932675in, left, base]{\color{textcolor}\rmfamily\fontsize{10.000000}{12.000000}\selectfont \(\displaystyle {\ensuremath{-}1}\)}%
\end{pgfscope}%
\begin{pgfscope}%
\pgfpathrectangle{\pgfqpoint{1.000000in}{0.330000in}}{\pgfqpoint{6.200000in}{2.310000in}}%
\pgfusepath{clip}%
\pgfsetbuttcap%
\pgfsetroundjoin%
\pgfsetlinewidth{0.803000pt}%
\definecolor{currentstroke}{rgb}{0.000000,0.000000,0.000000}%
\pgfsetstrokecolor{currentstroke}%
\pgfsetdash{{0.800000pt}{1.320000pt}}{0.000000pt}%
\pgfpathmoveto{\pgfqpoint{1.000000in}{1.338087in}}%
\pgfpathlineto{\pgfqpoint{7.200000in}{1.338087in}}%
\pgfusepath{stroke}%
\end{pgfscope}%
\begin{pgfscope}%
\pgfsetbuttcap%
\pgfsetroundjoin%
\definecolor{currentfill}{rgb}{0.000000,0.000000,0.000000}%
\pgfsetfillcolor{currentfill}%
\pgfsetlinewidth{0.803000pt}%
\definecolor{currentstroke}{rgb}{0.000000,0.000000,0.000000}%
\pgfsetstrokecolor{currentstroke}%
\pgfsetdash{}{0pt}%
\pgfsys@defobject{currentmarker}{\pgfqpoint{-0.048611in}{0.000000in}}{\pgfqpoint{-0.000000in}{0.000000in}}{%
\pgfpathmoveto{\pgfqpoint{-0.000000in}{0.000000in}}%
\pgfpathlineto{\pgfqpoint{-0.048611in}{0.000000in}}%
\pgfusepath{stroke,fill}%
}%
\begin{pgfscope}%
\pgfsys@transformshift{1.000000in}{1.338087in}%
\pgfsys@useobject{currentmarker}{}%
\end{pgfscope}%
\end{pgfscope}%
\begin{pgfscope}%
\pgfsetbuttcap%
\pgfsetroundjoin%
\definecolor{currentfill}{rgb}{0.000000,0.000000,0.000000}%
\pgfsetfillcolor{currentfill}%
\pgfsetlinewidth{0.803000pt}%
\definecolor{currentstroke}{rgb}{0.000000,0.000000,0.000000}%
\pgfsetstrokecolor{currentstroke}%
\pgfsetdash{}{0pt}%
\pgfsys@defobject{currentmarker}{\pgfqpoint{0.000000in}{0.000000in}}{\pgfqpoint{0.048611in}{0.000000in}}{%
\pgfpathmoveto{\pgfqpoint{0.000000in}{0.000000in}}%
\pgfpathlineto{\pgfqpoint{0.048611in}{0.000000in}}%
\pgfusepath{stroke,fill}%
}%
\begin{pgfscope}%
\pgfsys@transformshift{7.200000in}{1.338087in}%
\pgfsys@useobject{currentmarker}{}%
\end{pgfscope}%
\end{pgfscope}%
\begin{pgfscope}%
\definecolor{textcolor}{rgb}{0.000000,0.000000,0.000000}%
\pgfsetstrokecolor{textcolor}%
\pgfsetfillcolor{textcolor}%
\pgftext[x=0.833333in, y=1.289861in, left, base]{\color{textcolor}\rmfamily\fontsize{10.000000}{12.000000}\selectfont \(\displaystyle {0}\)}%
\end{pgfscope}%
\begin{pgfscope}%
\pgfpathrectangle{\pgfqpoint{1.000000in}{0.330000in}}{\pgfqpoint{6.200000in}{2.310000in}}%
\pgfusepath{clip}%
\pgfsetbuttcap%
\pgfsetroundjoin%
\pgfsetlinewidth{0.803000pt}%
\definecolor{currentstroke}{rgb}{0.000000,0.000000,0.000000}%
\pgfsetstrokecolor{currentstroke}%
\pgfsetdash{{0.800000pt}{1.320000pt}}{0.000000pt}%
\pgfpathmoveto{\pgfqpoint{1.000000in}{1.695273in}}%
\pgfpathlineto{\pgfqpoint{7.200000in}{1.695273in}}%
\pgfusepath{stroke}%
\end{pgfscope}%
\begin{pgfscope}%
\pgfsetbuttcap%
\pgfsetroundjoin%
\definecolor{currentfill}{rgb}{0.000000,0.000000,0.000000}%
\pgfsetfillcolor{currentfill}%
\pgfsetlinewidth{0.803000pt}%
\definecolor{currentstroke}{rgb}{0.000000,0.000000,0.000000}%
\pgfsetstrokecolor{currentstroke}%
\pgfsetdash{}{0pt}%
\pgfsys@defobject{currentmarker}{\pgfqpoint{-0.048611in}{0.000000in}}{\pgfqpoint{-0.000000in}{0.000000in}}{%
\pgfpathmoveto{\pgfqpoint{-0.000000in}{0.000000in}}%
\pgfpathlineto{\pgfqpoint{-0.048611in}{0.000000in}}%
\pgfusepath{stroke,fill}%
}%
\begin{pgfscope}%
\pgfsys@transformshift{1.000000in}{1.695273in}%
\pgfsys@useobject{currentmarker}{}%
\end{pgfscope}%
\end{pgfscope}%
\begin{pgfscope}%
\pgfsetbuttcap%
\pgfsetroundjoin%
\definecolor{currentfill}{rgb}{0.000000,0.000000,0.000000}%
\pgfsetfillcolor{currentfill}%
\pgfsetlinewidth{0.803000pt}%
\definecolor{currentstroke}{rgb}{0.000000,0.000000,0.000000}%
\pgfsetstrokecolor{currentstroke}%
\pgfsetdash{}{0pt}%
\pgfsys@defobject{currentmarker}{\pgfqpoint{0.000000in}{0.000000in}}{\pgfqpoint{0.048611in}{0.000000in}}{%
\pgfpathmoveto{\pgfqpoint{0.000000in}{0.000000in}}%
\pgfpathlineto{\pgfqpoint{0.048611in}{0.000000in}}%
\pgfusepath{stroke,fill}%
}%
\begin{pgfscope}%
\pgfsys@transformshift{7.200000in}{1.695273in}%
\pgfsys@useobject{currentmarker}{}%
\end{pgfscope}%
\end{pgfscope}%
\begin{pgfscope}%
\definecolor{textcolor}{rgb}{0.000000,0.000000,0.000000}%
\pgfsetstrokecolor{textcolor}%
\pgfsetfillcolor{textcolor}%
\pgftext[x=0.833333in, y=1.647048in, left, base]{\color{textcolor}\rmfamily\fontsize{10.000000}{12.000000}\selectfont \(\displaystyle {1}\)}%
\end{pgfscope}%
\begin{pgfscope}%
\pgfpathrectangle{\pgfqpoint{1.000000in}{0.330000in}}{\pgfqpoint{6.200000in}{2.310000in}}%
\pgfusepath{clip}%
\pgfsetbuttcap%
\pgfsetroundjoin%
\pgfsetlinewidth{0.803000pt}%
\definecolor{currentstroke}{rgb}{0.000000,0.000000,0.000000}%
\pgfsetstrokecolor{currentstroke}%
\pgfsetdash{{0.800000pt}{1.320000pt}}{0.000000pt}%
\pgfpathmoveto{\pgfqpoint{1.000000in}{2.052460in}}%
\pgfpathlineto{\pgfqpoint{7.200000in}{2.052460in}}%
\pgfusepath{stroke}%
\end{pgfscope}%
\begin{pgfscope}%
\pgfsetbuttcap%
\pgfsetroundjoin%
\definecolor{currentfill}{rgb}{0.000000,0.000000,0.000000}%
\pgfsetfillcolor{currentfill}%
\pgfsetlinewidth{0.803000pt}%
\definecolor{currentstroke}{rgb}{0.000000,0.000000,0.000000}%
\pgfsetstrokecolor{currentstroke}%
\pgfsetdash{}{0pt}%
\pgfsys@defobject{currentmarker}{\pgfqpoint{-0.048611in}{0.000000in}}{\pgfqpoint{-0.000000in}{0.000000in}}{%
\pgfpathmoveto{\pgfqpoint{-0.000000in}{0.000000in}}%
\pgfpathlineto{\pgfqpoint{-0.048611in}{0.000000in}}%
\pgfusepath{stroke,fill}%
}%
\begin{pgfscope}%
\pgfsys@transformshift{1.000000in}{2.052460in}%
\pgfsys@useobject{currentmarker}{}%
\end{pgfscope}%
\end{pgfscope}%
\begin{pgfscope}%
\pgfsetbuttcap%
\pgfsetroundjoin%
\definecolor{currentfill}{rgb}{0.000000,0.000000,0.000000}%
\pgfsetfillcolor{currentfill}%
\pgfsetlinewidth{0.803000pt}%
\definecolor{currentstroke}{rgb}{0.000000,0.000000,0.000000}%
\pgfsetstrokecolor{currentstroke}%
\pgfsetdash{}{0pt}%
\pgfsys@defobject{currentmarker}{\pgfqpoint{0.000000in}{0.000000in}}{\pgfqpoint{0.048611in}{0.000000in}}{%
\pgfpathmoveto{\pgfqpoint{0.000000in}{0.000000in}}%
\pgfpathlineto{\pgfqpoint{0.048611in}{0.000000in}}%
\pgfusepath{stroke,fill}%
}%
\begin{pgfscope}%
\pgfsys@transformshift{7.200000in}{2.052460in}%
\pgfsys@useobject{currentmarker}{}%
\end{pgfscope}%
\end{pgfscope}%
\begin{pgfscope}%
\definecolor{textcolor}{rgb}{0.000000,0.000000,0.000000}%
\pgfsetstrokecolor{textcolor}%
\pgfsetfillcolor{textcolor}%
\pgftext[x=0.833333in, y=2.004234in, left, base]{\color{textcolor}\rmfamily\fontsize{10.000000}{12.000000}\selectfont \(\displaystyle {2}\)}%
\end{pgfscope}%
\begin{pgfscope}%
\pgfpathrectangle{\pgfqpoint{1.000000in}{0.330000in}}{\pgfqpoint{6.200000in}{2.310000in}}%
\pgfusepath{clip}%
\pgfsetbuttcap%
\pgfsetroundjoin%
\pgfsetlinewidth{0.803000pt}%
\definecolor{currentstroke}{rgb}{0.000000,0.000000,0.000000}%
\pgfsetstrokecolor{currentstroke}%
\pgfsetdash{{0.800000pt}{1.320000pt}}{0.000000pt}%
\pgfpathmoveto{\pgfqpoint{1.000000in}{2.409646in}}%
\pgfpathlineto{\pgfqpoint{7.200000in}{2.409646in}}%
\pgfusepath{stroke}%
\end{pgfscope}%
\begin{pgfscope}%
\pgfsetbuttcap%
\pgfsetroundjoin%
\definecolor{currentfill}{rgb}{0.000000,0.000000,0.000000}%
\pgfsetfillcolor{currentfill}%
\pgfsetlinewidth{0.803000pt}%
\definecolor{currentstroke}{rgb}{0.000000,0.000000,0.000000}%
\pgfsetstrokecolor{currentstroke}%
\pgfsetdash{}{0pt}%
\pgfsys@defobject{currentmarker}{\pgfqpoint{-0.048611in}{0.000000in}}{\pgfqpoint{-0.000000in}{0.000000in}}{%
\pgfpathmoveto{\pgfqpoint{-0.000000in}{0.000000in}}%
\pgfpathlineto{\pgfqpoint{-0.048611in}{0.000000in}}%
\pgfusepath{stroke,fill}%
}%
\begin{pgfscope}%
\pgfsys@transformshift{1.000000in}{2.409646in}%
\pgfsys@useobject{currentmarker}{}%
\end{pgfscope}%
\end{pgfscope}%
\begin{pgfscope}%
\pgfsetbuttcap%
\pgfsetroundjoin%
\definecolor{currentfill}{rgb}{0.000000,0.000000,0.000000}%
\pgfsetfillcolor{currentfill}%
\pgfsetlinewidth{0.803000pt}%
\definecolor{currentstroke}{rgb}{0.000000,0.000000,0.000000}%
\pgfsetstrokecolor{currentstroke}%
\pgfsetdash{}{0pt}%
\pgfsys@defobject{currentmarker}{\pgfqpoint{0.000000in}{0.000000in}}{\pgfqpoint{0.048611in}{0.000000in}}{%
\pgfpathmoveto{\pgfqpoint{0.000000in}{0.000000in}}%
\pgfpathlineto{\pgfqpoint{0.048611in}{0.000000in}}%
\pgfusepath{stroke,fill}%
}%
\begin{pgfscope}%
\pgfsys@transformshift{7.200000in}{2.409646in}%
\pgfsys@useobject{currentmarker}{}%
\end{pgfscope}%
\end{pgfscope}%
\begin{pgfscope}%
\definecolor{textcolor}{rgb}{0.000000,0.000000,0.000000}%
\pgfsetstrokecolor{textcolor}%
\pgfsetfillcolor{textcolor}%
\pgftext[x=0.833333in, y=2.361421in, left, base]{\color{textcolor}\rmfamily\fontsize{10.000000}{12.000000}\selectfont \(\displaystyle {3}\)}%
\end{pgfscope}%
\begin{pgfscope}%
\pgfpathrectangle{\pgfqpoint{1.000000in}{0.330000in}}{\pgfqpoint{6.200000in}{2.310000in}}%
\pgfusepath{clip}%
\pgfsetrectcap%
\pgfsetroundjoin%
\pgfsetlinewidth{1.505625pt}%
\definecolor{currentstroke}{rgb}{0.121569,0.466667,0.705882}%
\pgfsetstrokecolor{currentstroke}%
\pgfsetdash{}{0pt}%
\pgfpathmoveto{\pgfqpoint{1.281818in}{1.338087in}}%
\pgfpathlineto{\pgfqpoint{1.281818in}{2.241086in}}%
\pgfpathlineto{\pgfqpoint{3.630303in}{2.223334in}}%
\pgfpathlineto{\pgfqpoint{3.697403in}{0.679781in}}%
\pgfpathlineto{\pgfqpoint{6.918182in}{0.655435in}}%
\pgfpathlineto{\pgfqpoint{6.918182in}{1.338087in}}%
\pgfpathlineto{\pgfqpoint{6.918182in}{1.338087in}}%
\pgfusepath{stroke}%
\end{pgfscope}%
\begin{pgfscope}%
\pgfpathrectangle{\pgfqpoint{1.000000in}{0.330000in}}{\pgfqpoint{6.200000in}{2.310000in}}%
\pgfusepath{clip}%
\pgfsetrectcap%
\pgfsetroundjoin%
\pgfsetlinewidth{1.505625pt}%
\definecolor{currentstroke}{rgb}{1.000000,0.498039,0.054902}%
\pgfsetstrokecolor{currentstroke}%
\pgfsetdash{}{0pt}%
\pgfpathmoveto{\pgfqpoint{1.281818in}{1.338087in}}%
\pgfpathlineto{\pgfqpoint{1.281818in}{1.579087in}}%
\pgfpathlineto{\pgfqpoint{3.630303in}{1.568732in}}%
\pgfpathlineto{\pgfqpoint{3.697403in}{1.168387in}}%
\pgfpathlineto{\pgfqpoint{6.918182in}{1.154185in}}%
\pgfpathlineto{\pgfqpoint{6.918182in}{1.338087in}}%
\pgfpathlineto{\pgfqpoint{6.918182in}{1.338087in}}%
\pgfusepath{stroke}%
\end{pgfscope}%
\begin{pgfscope}%
\pgfpathrectangle{\pgfqpoint{1.000000in}{0.330000in}}{\pgfqpoint{6.200000in}{2.310000in}}%
\pgfusepath{clip}%
\pgfsetrectcap%
\pgfsetroundjoin%
\pgfsetlinewidth{1.505625pt}%
\definecolor{currentstroke}{rgb}{0.172549,0.627451,0.172549}%
\pgfsetstrokecolor{currentstroke}%
\pgfsetdash{}{0pt}%
\pgfpathmoveto{\pgfqpoint{1.281818in}{1.338087in}}%
\pgfpathlineto{\pgfqpoint{1.281818in}{1.820088in}}%
\pgfpathlineto{\pgfqpoint{3.630303in}{1.799377in}}%
\pgfpathlineto{\pgfqpoint{3.697403in}{0.998687in}}%
\pgfpathlineto{\pgfqpoint{6.918182in}{0.970284in}}%
\pgfpathlineto{\pgfqpoint{6.918182in}{1.338087in}}%
\pgfpathlineto{\pgfqpoint{6.918182in}{1.338087in}}%
\pgfusepath{stroke}%
\end{pgfscope}%
\begin{pgfscope}%
\pgfpathrectangle{\pgfqpoint{1.000000in}{0.330000in}}{\pgfqpoint{6.200000in}{2.310000in}}%
\pgfusepath{clip}%
\pgfsetrectcap%
\pgfsetroundjoin%
\pgfsetlinewidth{1.505625pt}%
\definecolor{currentstroke}{rgb}{0.839216,0.152941,0.156863}%
\pgfsetstrokecolor{currentstroke}%
\pgfsetdash{}{0pt}%
\pgfpathmoveto{\pgfqpoint{1.281818in}{1.338087in}}%
\pgfpathlineto{\pgfqpoint{1.281818in}{1.647945in}}%
\pgfpathlineto{\pgfqpoint{3.630303in}{1.634630in}}%
\pgfpathlineto{\pgfqpoint{3.697403in}{1.119901in}}%
\pgfpathlineto{\pgfqpoint{6.918182in}{1.101642in}}%
\pgfpathlineto{\pgfqpoint{6.918182in}{1.338087in}}%
\pgfpathlineto{\pgfqpoint{6.918182in}{1.338087in}}%
\pgfusepath{stroke}%
\end{pgfscope}%
\begin{pgfscope}%
\pgfpathrectangle{\pgfqpoint{1.000000in}{0.330000in}}{\pgfqpoint{6.200000in}{2.310000in}}%
\pgfusepath{clip}%
\pgfsetrectcap%
\pgfsetroundjoin%
\pgfsetlinewidth{1.505625pt}%
\definecolor{currentstroke}{rgb}{0.580392,0.403922,0.741176}%
\pgfsetstrokecolor{currentstroke}%
\pgfsetdash{}{0pt}%
\pgfpathmoveto{\pgfqpoint{1.281818in}{1.338087in}}%
\pgfpathlineto{\pgfqpoint{1.281818in}{1.751231in}}%
\pgfpathlineto{\pgfqpoint{3.630303in}{1.733478in}}%
\pgfpathlineto{\pgfqpoint{3.697403in}{1.047173in}}%
\pgfpathlineto{\pgfqpoint{6.918182in}{1.022827in}}%
\pgfpathlineto{\pgfqpoint{6.918182in}{1.338087in}}%
\pgfpathlineto{\pgfqpoint{6.918182in}{1.338087in}}%
\pgfusepath{stroke}%
\end{pgfscope}%
\begin{pgfscope}%
\pgfpathrectangle{\pgfqpoint{1.000000in}{0.330000in}}{\pgfqpoint{6.200000in}{2.310000in}}%
\pgfusepath{clip}%
\pgfsetrectcap%
\pgfsetroundjoin%
\pgfsetlinewidth{1.505625pt}%
\definecolor{currentstroke}{rgb}{0.549020,0.337255,0.294118}%
\pgfsetstrokecolor{currentstroke}%
\pgfsetdash{}{0pt}%
\pgfpathmoveto{\pgfqpoint{1.281818in}{1.338087in}}%
\pgfpathlineto{\pgfqpoint{1.281818in}{2.535000in}}%
\pgfpathlineto{\pgfqpoint{3.630303in}{2.517248in}}%
\pgfpathlineto{\pgfqpoint{3.697403in}{0.459346in}}%
\pgfpathlineto{\pgfqpoint{6.918182in}{0.435000in}}%
\pgfpathlineto{\pgfqpoint{6.918182in}{1.338087in}}%
\pgfpathlineto{\pgfqpoint{6.918182in}{1.338087in}}%
\pgfusepath{stroke}%
\end{pgfscope}%
\begin{pgfscope}%
\pgfpathrectangle{\pgfqpoint{1.000000in}{0.330000in}}{\pgfqpoint{6.200000in}{2.310000in}}%
\pgfusepath{clip}%
\pgfsetrectcap%
\pgfsetroundjoin%
\pgfsetlinewidth{1.505625pt}%
\definecolor{currentstroke}{rgb}{0.890196,0.466667,0.760784}%
\pgfsetstrokecolor{currentstroke}%
\pgfsetdash{}{0pt}%
\pgfpathmoveto{\pgfqpoint{1.281818in}{1.338087in}}%
\pgfpathlineto{\pgfqpoint{1.281818in}{2.065016in}}%
\pgfpathlineto{\pgfqpoint{3.630303in}{2.044305in}}%
\pgfpathlineto{\pgfqpoint{3.697403in}{0.814991in}}%
\pgfpathlineto{\pgfqpoint{6.918182in}{0.786588in}}%
\pgfpathlineto{\pgfqpoint{6.918182in}{1.338087in}}%
\pgfpathlineto{\pgfqpoint{6.918182in}{1.338087in}}%
\pgfusepath{stroke}%
\end{pgfscope}%
\begin{pgfscope}%
\pgfsetroundcap%
\pgfsetroundjoin%
\pgfsetlinewidth{1.003750pt}%
\definecolor{currentstroke}{rgb}{0.000000,0.000000,0.000000}%
\pgfsetstrokecolor{currentstroke}%
\pgfsetdash{}{0pt}%
\pgfpathmoveto{\pgfqpoint{1.818946in}{2.535000in}}%
\pgfpathquadraticcurveto{\pgfqpoint{1.564266in}{2.535000in}}{\pgfqpoint{1.309586in}{2.535000in}}%
\pgfusepath{stroke}%
\end{pgfscope}%
\begin{pgfscope}%
\pgfsetbuttcap%
\pgfsetmiterjoin%
\definecolor{currentfill}{rgb}{0.800000,0.800000,0.800000}%
\pgfsetfillcolor{currentfill}%
\pgfsetlinewidth{1.003750pt}%
\definecolor{currentstroke}{rgb}{0.000000,0.000000,0.000000}%
\pgfsetstrokecolor{currentstroke}%
\pgfsetdash{}{0pt}%
\pgfpathmoveto{\pgfqpoint{1.876644in}{2.438549in}}%
\pgfpathcurveto{\pgfqpoint{1.911366in}{2.403827in}}{\pgfqpoint{2.670707in}{2.403827in}}{\pgfqpoint{2.705429in}{2.438549in}}%
\pgfpathcurveto{\pgfqpoint{2.740152in}{2.473272in}}{\pgfqpoint{2.740152in}{2.596728in}}{\pgfqpoint{2.705429in}{2.631451in}}%
\pgfpathcurveto{\pgfqpoint{2.670707in}{2.666173in}}{\pgfqpoint{1.911366in}{2.666173in}}{\pgfqpoint{1.876644in}{2.631451in}}%
\pgfpathcurveto{\pgfqpoint{1.841922in}{2.596728in}}{\pgfqpoint{1.841922in}{2.473272in}}{\pgfqpoint{1.876644in}{2.438549in}}%
\pgfpathclose%
\pgfusepath{stroke,fill}%
\end{pgfscope}%
\begin{pgfscope}%
\definecolor{textcolor}{rgb}{0.000000,0.000000,0.000000}%
\pgfsetstrokecolor{textcolor}%
\pgfsetfillcolor{textcolor}%
\pgftext[x=2.670707in,y=2.535000in,right,]{\color{textcolor}\rmfamily\fontsize{10.000000}{12.000000}\selectfont \(\displaystyle V_u =\) 3.4 kip}%
\end{pgfscope}%
\begin{pgfscope}%
\pgfsetbuttcap%
\pgfsetmiterjoin%
\definecolor{currentfill}{rgb}{0.800000,0.800000,0.800000}%
\pgfsetfillcolor{currentfill}%
\pgfsetlinewidth{1.003750pt}%
\definecolor{currentstroke}{rgb}{0.000000,0.000000,0.000000}%
\pgfsetstrokecolor{currentstroke}%
\pgfsetdash{}{0pt}%
\pgfpathmoveto{\pgfqpoint{0.965278in}{0.358599in}}%
\pgfpathcurveto{\pgfqpoint{1.000000in}{0.323877in}}{\pgfqpoint{2.720682in}{0.323877in}}{\pgfqpoint{2.755404in}{0.358599in}}%
\pgfpathcurveto{\pgfqpoint{2.790127in}{0.393321in}}{\pgfqpoint{2.790127in}{0.668784in}}{\pgfqpoint{2.755404in}{0.703506in}}%
\pgfpathcurveto{\pgfqpoint{2.720682in}{0.738228in}}{\pgfqpoint{1.000000in}{0.738228in}}{\pgfqpoint{0.965278in}{0.703506in}}%
\pgfpathcurveto{\pgfqpoint{0.930556in}{0.668784in}}{\pgfqpoint{0.930556in}{0.393321in}}{\pgfqpoint{0.965278in}{0.358599in}}%
\pgfpathclose%
\pgfusepath{stroke,fill}%
\end{pgfscope}%
\begin{pgfscope}%
\definecolor{textcolor}{rgb}{0.000000,0.000000,0.000000}%
\pgfsetstrokecolor{textcolor}%
\pgfsetfillcolor{textcolor}%
\pgftext[x=1.000000in, y=0.580049in, left, base]{\color{textcolor}\rmfamily\fontsize{10.000000}{12.000000}\selectfont Max combo: 1.2D + 1.6L0}%
\end{pgfscope}%
\begin{pgfscope}%
\definecolor{textcolor}{rgb}{0.000000,0.000000,0.000000}%
\pgfsetstrokecolor{textcolor}%
\pgfsetfillcolor{textcolor}%
\pgftext[x=1.000000in, y=0.428043in, left, base]{\color{textcolor}\rmfamily\fontsize{10.000000}{12.000000}\selectfont ASCE7-16 Sec. 2.3.1 (LC 2)}%
\end{pgfscope}%
\end{pgfpicture}%
\makeatother%
\endgroup%

\end{center}
\caption{Shear Demand Envelope}
\end{figure}
C\textsubscript{v1}, the web shear strength coefficient, is calculated per AISC/ANSI 360-16 Eq. G2-3 as follows, based on the ratio of the clear distance between flanges to web thickness:
\begin{flalign*}
\frac{h}{t_w} = \frac{10.8 {\color{darkBlue}{\mathbf{ \; in}}}}{0.2 {\color{darkBlue}{\mathbf{ \; in}}}} = \mathbf{54.2 } <= 1.1\sqrt{\frac{k_v E}{F_y}} = 1.1\sqrt{\frac{5.34 29000 {\color{darkBlue}{\mathbf{ \; ksi}}}}{50 {\color{darkBlue}{\mathbf{ \; ksi}}}}} = \mathbf{61.2 } \rightarrow C_{v1} = \mathbf{1.0}
\end{flalign*}
\textphi\textsubscript{v}, the resistance factor for shear, is calculated per AISC/ANSI 360-16 {\S}G1.a as follows:
\begin{flalign*}
\frac{h}{t_w} = \frac{10.8 {\color{darkBlue}{\mathbf{ \; in}}}}{0.2 {\color{darkBlue}{\mathbf{ \; in}}}} = \mathbf{54.2 } > 2.24\cdot \sqrt{\frac{E}{F_y}} = 2.24\cdot \sqrt{\frac{29000 {\color{darkBlue}{\mathbf{ \; ksi}}}}{50 {\color{darkBlue}{\mathbf{ \; ksi}}}}} = \mathbf{53.9} \rightarrow \phi_v = \mathbf{0.9}
\end{flalign*}
\textphi\textsubscript{v}V\textsubscript{n}, the design shear strength, is calculated per AISC/ANSI 360-16 Eq. G2-1 as follows:
\begin{flalign*}
\phi_v V_n = 0.6\cdot F_y \cdot A_w \cdot C_{v1}  = 0.6\cdot 50 {\color{darkBlue}{\mathbf{ \; ksi}}} \cdot 2.38 {\color{darkBlue}{\mathbf{ \; {\color{darkBlue}{\mathbf{ \; in}}}^{2}}}} \cdot 1.0  = \mathbf{71.4 {\color{darkBlue}{\mathbf{ \; kip}}}}
\end{flalign*}
\vspace{-26pt}
{\setlength{\mathindent}{0cm}
\begin{flalign*}
\mathbf{|V_u| = 3.4 {\color{darkBlue}{\mathbf{ \; kip}}}  \;  < \phi_v \cdot V_n = 71.4 {\color{darkBlue}{\mathbf{ \; kip}}}  \;  (DCR = 0.05 - OK)}
\end{flalign*}
%	----------------------------- DEFLECTION CHECK -------------------------------
\section{Deflection Check}
\begin{figure}[H]
\begin{center}
%% Creator: Matplotlib, PGF backend
%%
%% To include the figure in your LaTeX document, write
%%   \input{<filename>.pgf}
%%
%% Make sure the required packages are loaded in your preamble
%%   \usepackage{pgf}
%%
%% Figures using additional raster images can only be included by \input if
%% they are in the same directory as the main LaTeX file. For loading figures
%% from other directories you can use the `import` package
%%   \usepackage{import}
%%
%% and then include the figures with
%%   \import{<path to file>}{<filename>.pgf}
%%
%% Matplotlib used the following preamble
%%
\begingroup%
\makeatletter%
\begin{pgfpicture}%
\pgfpathrectangle{\pgfpointorigin}{\pgfqpoint{8.000000in}{3.000000in}}%
\pgfusepath{use as bounding box, clip}%
\begin{pgfscope}%
\pgfsetbuttcap%
\pgfsetmiterjoin%
\definecolor{currentfill}{rgb}{1.000000,1.000000,1.000000}%
\pgfsetfillcolor{currentfill}%
\pgfsetlinewidth{0.000000pt}%
\definecolor{currentstroke}{rgb}{1.000000,1.000000,1.000000}%
\pgfsetstrokecolor{currentstroke}%
\pgfsetdash{}{0pt}%
\pgfpathmoveto{\pgfqpoint{0.000000in}{0.000000in}}%
\pgfpathlineto{\pgfqpoint{8.000000in}{0.000000in}}%
\pgfpathlineto{\pgfqpoint{8.000000in}{3.000000in}}%
\pgfpathlineto{\pgfqpoint{0.000000in}{3.000000in}}%
\pgfpathclose%
\pgfusepath{fill}%
\end{pgfscope}%
\begin{pgfscope}%
\pgfsetbuttcap%
\pgfsetmiterjoin%
\definecolor{currentfill}{rgb}{1.000000,1.000000,1.000000}%
\pgfsetfillcolor{currentfill}%
\pgfsetlinewidth{0.000000pt}%
\definecolor{currentstroke}{rgb}{0.000000,0.000000,0.000000}%
\pgfsetstrokecolor{currentstroke}%
\pgfsetstrokeopacity{0.000000}%
\pgfsetdash{}{0pt}%
\pgfpathmoveto{\pgfqpoint{1.000000in}{0.330000in}}%
\pgfpathlineto{\pgfqpoint{7.200000in}{0.330000in}}%
\pgfpathlineto{\pgfqpoint{7.200000in}{2.640000in}}%
\pgfpathlineto{\pgfqpoint{1.000000in}{2.640000in}}%
\pgfpathclose%
\pgfusepath{fill}%
\end{pgfscope}%
\begin{pgfscope}%
\pgfpathrectangle{\pgfqpoint{1.000000in}{0.330000in}}{\pgfqpoint{6.200000in}{2.310000in}}%
\pgfusepath{clip}%
\pgfsetbuttcap%
\pgfsetroundjoin%
\pgfsetlinewidth{0.803000pt}%
\definecolor{currentstroke}{rgb}{0.000000,0.000000,0.000000}%
\pgfsetstrokecolor{currentstroke}%
\pgfsetdash{{0.800000pt}{1.320000pt}}{0.000000pt}%
\pgfpathmoveto{\pgfqpoint{1.281818in}{0.330000in}}%
\pgfpathlineto{\pgfqpoint{1.281818in}{2.640000in}}%
\pgfusepath{stroke}%
\end{pgfscope}%
\begin{pgfscope}%
\pgfsetbuttcap%
\pgfsetroundjoin%
\definecolor{currentfill}{rgb}{0.000000,0.000000,0.000000}%
\pgfsetfillcolor{currentfill}%
\pgfsetlinewidth{0.803000pt}%
\definecolor{currentstroke}{rgb}{0.000000,0.000000,0.000000}%
\pgfsetstrokecolor{currentstroke}%
\pgfsetdash{}{0pt}%
\pgfsys@defobject{currentmarker}{\pgfqpoint{0.000000in}{-0.048611in}}{\pgfqpoint{0.000000in}{0.000000in}}{%
\pgfpathmoveto{\pgfqpoint{0.000000in}{0.000000in}}%
\pgfpathlineto{\pgfqpoint{0.000000in}{-0.048611in}}%
\pgfusepath{stroke,fill}%
}%
\begin{pgfscope}%
\pgfsys@transformshift{1.281818in}{0.330000in}%
\pgfsys@useobject{currentmarker}{}%
\end{pgfscope}%
\end{pgfscope}%
\begin{pgfscope}%
\pgfsetbuttcap%
\pgfsetroundjoin%
\definecolor{currentfill}{rgb}{0.000000,0.000000,0.000000}%
\pgfsetfillcolor{currentfill}%
\pgfsetlinewidth{0.803000pt}%
\definecolor{currentstroke}{rgb}{0.000000,0.000000,0.000000}%
\pgfsetstrokecolor{currentstroke}%
\pgfsetdash{}{0pt}%
\pgfsys@defobject{currentmarker}{\pgfqpoint{0.000000in}{0.000000in}}{\pgfqpoint{0.000000in}{0.048611in}}{%
\pgfpathmoveto{\pgfqpoint{0.000000in}{0.000000in}}%
\pgfpathlineto{\pgfqpoint{0.000000in}{0.048611in}}%
\pgfusepath{stroke,fill}%
}%
\begin{pgfscope}%
\pgfsys@transformshift{1.281818in}{2.640000in}%
\pgfsys@useobject{currentmarker}{}%
\end{pgfscope}%
\end{pgfscope}%
\begin{pgfscope}%
\definecolor{textcolor}{rgb}{0.000000,0.000000,0.000000}%
\pgfsetstrokecolor{textcolor}%
\pgfsetfillcolor{textcolor}%
\pgftext[x=1.281818in,y=0.232778in,,top]{\color{textcolor}\rmfamily\fontsize{10.000000}{12.000000}\selectfont \(\displaystyle {0}\)}%
\end{pgfscope}%
\begin{pgfscope}%
\pgfpathrectangle{\pgfqpoint{1.000000in}{0.330000in}}{\pgfqpoint{6.200000in}{2.310000in}}%
\pgfusepath{clip}%
\pgfsetbuttcap%
\pgfsetroundjoin%
\pgfsetlinewidth{0.803000pt}%
\definecolor{currentstroke}{rgb}{0.000000,0.000000,0.000000}%
\pgfsetstrokecolor{currentstroke}%
\pgfsetdash{{0.800000pt}{1.320000pt}}{0.000000pt}%
\pgfpathmoveto{\pgfqpoint{2.087013in}{0.330000in}}%
\pgfpathlineto{\pgfqpoint{2.087013in}{2.640000in}}%
\pgfusepath{stroke}%
\end{pgfscope}%
\begin{pgfscope}%
\pgfsetbuttcap%
\pgfsetroundjoin%
\definecolor{currentfill}{rgb}{0.000000,0.000000,0.000000}%
\pgfsetfillcolor{currentfill}%
\pgfsetlinewidth{0.803000pt}%
\definecolor{currentstroke}{rgb}{0.000000,0.000000,0.000000}%
\pgfsetstrokecolor{currentstroke}%
\pgfsetdash{}{0pt}%
\pgfsys@defobject{currentmarker}{\pgfqpoint{0.000000in}{-0.048611in}}{\pgfqpoint{0.000000in}{0.000000in}}{%
\pgfpathmoveto{\pgfqpoint{0.000000in}{0.000000in}}%
\pgfpathlineto{\pgfqpoint{0.000000in}{-0.048611in}}%
\pgfusepath{stroke,fill}%
}%
\begin{pgfscope}%
\pgfsys@transformshift{2.087013in}{0.330000in}%
\pgfsys@useobject{currentmarker}{}%
\end{pgfscope}%
\end{pgfscope}%
\begin{pgfscope}%
\pgfsetbuttcap%
\pgfsetroundjoin%
\definecolor{currentfill}{rgb}{0.000000,0.000000,0.000000}%
\pgfsetfillcolor{currentfill}%
\pgfsetlinewidth{0.803000pt}%
\definecolor{currentstroke}{rgb}{0.000000,0.000000,0.000000}%
\pgfsetstrokecolor{currentstroke}%
\pgfsetdash{}{0pt}%
\pgfsys@defobject{currentmarker}{\pgfqpoint{0.000000in}{0.000000in}}{\pgfqpoint{0.000000in}{0.048611in}}{%
\pgfpathmoveto{\pgfqpoint{0.000000in}{0.000000in}}%
\pgfpathlineto{\pgfqpoint{0.000000in}{0.048611in}}%
\pgfusepath{stroke,fill}%
}%
\begin{pgfscope}%
\pgfsys@transformshift{2.087013in}{2.640000in}%
\pgfsys@useobject{currentmarker}{}%
\end{pgfscope}%
\end{pgfscope}%
\begin{pgfscope}%
\definecolor{textcolor}{rgb}{0.000000,0.000000,0.000000}%
\pgfsetstrokecolor{textcolor}%
\pgfsetfillcolor{textcolor}%
\pgftext[x=2.087013in,y=0.232778in,,top]{\color{textcolor}\rmfamily\fontsize{10.000000}{12.000000}\selectfont \(\displaystyle {1}\)}%
\end{pgfscope}%
\begin{pgfscope}%
\pgfpathrectangle{\pgfqpoint{1.000000in}{0.330000in}}{\pgfqpoint{6.200000in}{2.310000in}}%
\pgfusepath{clip}%
\pgfsetbuttcap%
\pgfsetroundjoin%
\pgfsetlinewidth{0.803000pt}%
\definecolor{currentstroke}{rgb}{0.000000,0.000000,0.000000}%
\pgfsetstrokecolor{currentstroke}%
\pgfsetdash{{0.800000pt}{1.320000pt}}{0.000000pt}%
\pgfpathmoveto{\pgfqpoint{2.892208in}{0.330000in}}%
\pgfpathlineto{\pgfqpoint{2.892208in}{2.640000in}}%
\pgfusepath{stroke}%
\end{pgfscope}%
\begin{pgfscope}%
\pgfsetbuttcap%
\pgfsetroundjoin%
\definecolor{currentfill}{rgb}{0.000000,0.000000,0.000000}%
\pgfsetfillcolor{currentfill}%
\pgfsetlinewidth{0.803000pt}%
\definecolor{currentstroke}{rgb}{0.000000,0.000000,0.000000}%
\pgfsetstrokecolor{currentstroke}%
\pgfsetdash{}{0pt}%
\pgfsys@defobject{currentmarker}{\pgfqpoint{0.000000in}{-0.048611in}}{\pgfqpoint{0.000000in}{0.000000in}}{%
\pgfpathmoveto{\pgfqpoint{0.000000in}{0.000000in}}%
\pgfpathlineto{\pgfqpoint{0.000000in}{-0.048611in}}%
\pgfusepath{stroke,fill}%
}%
\begin{pgfscope}%
\pgfsys@transformshift{2.892208in}{0.330000in}%
\pgfsys@useobject{currentmarker}{}%
\end{pgfscope}%
\end{pgfscope}%
\begin{pgfscope}%
\pgfsetbuttcap%
\pgfsetroundjoin%
\definecolor{currentfill}{rgb}{0.000000,0.000000,0.000000}%
\pgfsetfillcolor{currentfill}%
\pgfsetlinewidth{0.803000pt}%
\definecolor{currentstroke}{rgb}{0.000000,0.000000,0.000000}%
\pgfsetstrokecolor{currentstroke}%
\pgfsetdash{}{0pt}%
\pgfsys@defobject{currentmarker}{\pgfqpoint{0.000000in}{0.000000in}}{\pgfqpoint{0.000000in}{0.048611in}}{%
\pgfpathmoveto{\pgfqpoint{0.000000in}{0.000000in}}%
\pgfpathlineto{\pgfqpoint{0.000000in}{0.048611in}}%
\pgfusepath{stroke,fill}%
}%
\begin{pgfscope}%
\pgfsys@transformshift{2.892208in}{2.640000in}%
\pgfsys@useobject{currentmarker}{}%
\end{pgfscope}%
\end{pgfscope}%
\begin{pgfscope}%
\definecolor{textcolor}{rgb}{0.000000,0.000000,0.000000}%
\pgfsetstrokecolor{textcolor}%
\pgfsetfillcolor{textcolor}%
\pgftext[x=2.892208in,y=0.232778in,,top]{\color{textcolor}\rmfamily\fontsize{10.000000}{12.000000}\selectfont \(\displaystyle {2}\)}%
\end{pgfscope}%
\begin{pgfscope}%
\pgfpathrectangle{\pgfqpoint{1.000000in}{0.330000in}}{\pgfqpoint{6.200000in}{2.310000in}}%
\pgfusepath{clip}%
\pgfsetbuttcap%
\pgfsetroundjoin%
\pgfsetlinewidth{0.803000pt}%
\definecolor{currentstroke}{rgb}{0.000000,0.000000,0.000000}%
\pgfsetstrokecolor{currentstroke}%
\pgfsetdash{{0.800000pt}{1.320000pt}}{0.000000pt}%
\pgfpathmoveto{\pgfqpoint{3.697403in}{0.330000in}}%
\pgfpathlineto{\pgfqpoint{3.697403in}{2.640000in}}%
\pgfusepath{stroke}%
\end{pgfscope}%
\begin{pgfscope}%
\pgfsetbuttcap%
\pgfsetroundjoin%
\definecolor{currentfill}{rgb}{0.000000,0.000000,0.000000}%
\pgfsetfillcolor{currentfill}%
\pgfsetlinewidth{0.803000pt}%
\definecolor{currentstroke}{rgb}{0.000000,0.000000,0.000000}%
\pgfsetstrokecolor{currentstroke}%
\pgfsetdash{}{0pt}%
\pgfsys@defobject{currentmarker}{\pgfqpoint{0.000000in}{-0.048611in}}{\pgfqpoint{0.000000in}{0.000000in}}{%
\pgfpathmoveto{\pgfqpoint{0.000000in}{0.000000in}}%
\pgfpathlineto{\pgfqpoint{0.000000in}{-0.048611in}}%
\pgfusepath{stroke,fill}%
}%
\begin{pgfscope}%
\pgfsys@transformshift{3.697403in}{0.330000in}%
\pgfsys@useobject{currentmarker}{}%
\end{pgfscope}%
\end{pgfscope}%
\begin{pgfscope}%
\pgfsetbuttcap%
\pgfsetroundjoin%
\definecolor{currentfill}{rgb}{0.000000,0.000000,0.000000}%
\pgfsetfillcolor{currentfill}%
\pgfsetlinewidth{0.803000pt}%
\definecolor{currentstroke}{rgb}{0.000000,0.000000,0.000000}%
\pgfsetstrokecolor{currentstroke}%
\pgfsetdash{}{0pt}%
\pgfsys@defobject{currentmarker}{\pgfqpoint{0.000000in}{0.000000in}}{\pgfqpoint{0.000000in}{0.048611in}}{%
\pgfpathmoveto{\pgfqpoint{0.000000in}{0.000000in}}%
\pgfpathlineto{\pgfqpoint{0.000000in}{0.048611in}}%
\pgfusepath{stroke,fill}%
}%
\begin{pgfscope}%
\pgfsys@transformshift{3.697403in}{2.640000in}%
\pgfsys@useobject{currentmarker}{}%
\end{pgfscope}%
\end{pgfscope}%
\begin{pgfscope}%
\definecolor{textcolor}{rgb}{0.000000,0.000000,0.000000}%
\pgfsetstrokecolor{textcolor}%
\pgfsetfillcolor{textcolor}%
\pgftext[x=3.697403in,y=0.232778in,,top]{\color{textcolor}\rmfamily\fontsize{10.000000}{12.000000}\selectfont \(\displaystyle {3}\)}%
\end{pgfscope}%
\begin{pgfscope}%
\pgfpathrectangle{\pgfqpoint{1.000000in}{0.330000in}}{\pgfqpoint{6.200000in}{2.310000in}}%
\pgfusepath{clip}%
\pgfsetbuttcap%
\pgfsetroundjoin%
\pgfsetlinewidth{0.803000pt}%
\definecolor{currentstroke}{rgb}{0.000000,0.000000,0.000000}%
\pgfsetstrokecolor{currentstroke}%
\pgfsetdash{{0.800000pt}{1.320000pt}}{0.000000pt}%
\pgfpathmoveto{\pgfqpoint{4.502597in}{0.330000in}}%
\pgfpathlineto{\pgfqpoint{4.502597in}{2.640000in}}%
\pgfusepath{stroke}%
\end{pgfscope}%
\begin{pgfscope}%
\pgfsetbuttcap%
\pgfsetroundjoin%
\definecolor{currentfill}{rgb}{0.000000,0.000000,0.000000}%
\pgfsetfillcolor{currentfill}%
\pgfsetlinewidth{0.803000pt}%
\definecolor{currentstroke}{rgb}{0.000000,0.000000,0.000000}%
\pgfsetstrokecolor{currentstroke}%
\pgfsetdash{}{0pt}%
\pgfsys@defobject{currentmarker}{\pgfqpoint{0.000000in}{-0.048611in}}{\pgfqpoint{0.000000in}{0.000000in}}{%
\pgfpathmoveto{\pgfqpoint{0.000000in}{0.000000in}}%
\pgfpathlineto{\pgfqpoint{0.000000in}{-0.048611in}}%
\pgfusepath{stroke,fill}%
}%
\begin{pgfscope}%
\pgfsys@transformshift{4.502597in}{0.330000in}%
\pgfsys@useobject{currentmarker}{}%
\end{pgfscope}%
\end{pgfscope}%
\begin{pgfscope}%
\pgfsetbuttcap%
\pgfsetroundjoin%
\definecolor{currentfill}{rgb}{0.000000,0.000000,0.000000}%
\pgfsetfillcolor{currentfill}%
\pgfsetlinewidth{0.803000pt}%
\definecolor{currentstroke}{rgb}{0.000000,0.000000,0.000000}%
\pgfsetstrokecolor{currentstroke}%
\pgfsetdash{}{0pt}%
\pgfsys@defobject{currentmarker}{\pgfqpoint{0.000000in}{0.000000in}}{\pgfqpoint{0.000000in}{0.048611in}}{%
\pgfpathmoveto{\pgfqpoint{0.000000in}{0.000000in}}%
\pgfpathlineto{\pgfqpoint{0.000000in}{0.048611in}}%
\pgfusepath{stroke,fill}%
}%
\begin{pgfscope}%
\pgfsys@transformshift{4.502597in}{2.640000in}%
\pgfsys@useobject{currentmarker}{}%
\end{pgfscope}%
\end{pgfscope}%
\begin{pgfscope}%
\definecolor{textcolor}{rgb}{0.000000,0.000000,0.000000}%
\pgfsetstrokecolor{textcolor}%
\pgfsetfillcolor{textcolor}%
\pgftext[x=4.502597in,y=0.232778in,,top]{\color{textcolor}\rmfamily\fontsize{10.000000}{12.000000}\selectfont \(\displaystyle {4}\)}%
\end{pgfscope}%
\begin{pgfscope}%
\pgfpathrectangle{\pgfqpoint{1.000000in}{0.330000in}}{\pgfqpoint{6.200000in}{2.310000in}}%
\pgfusepath{clip}%
\pgfsetbuttcap%
\pgfsetroundjoin%
\pgfsetlinewidth{0.803000pt}%
\definecolor{currentstroke}{rgb}{0.000000,0.000000,0.000000}%
\pgfsetstrokecolor{currentstroke}%
\pgfsetdash{{0.800000pt}{1.320000pt}}{0.000000pt}%
\pgfpathmoveto{\pgfqpoint{5.307792in}{0.330000in}}%
\pgfpathlineto{\pgfqpoint{5.307792in}{2.640000in}}%
\pgfusepath{stroke}%
\end{pgfscope}%
\begin{pgfscope}%
\pgfsetbuttcap%
\pgfsetroundjoin%
\definecolor{currentfill}{rgb}{0.000000,0.000000,0.000000}%
\pgfsetfillcolor{currentfill}%
\pgfsetlinewidth{0.803000pt}%
\definecolor{currentstroke}{rgb}{0.000000,0.000000,0.000000}%
\pgfsetstrokecolor{currentstroke}%
\pgfsetdash{}{0pt}%
\pgfsys@defobject{currentmarker}{\pgfqpoint{0.000000in}{-0.048611in}}{\pgfqpoint{0.000000in}{0.000000in}}{%
\pgfpathmoveto{\pgfqpoint{0.000000in}{0.000000in}}%
\pgfpathlineto{\pgfqpoint{0.000000in}{-0.048611in}}%
\pgfusepath{stroke,fill}%
}%
\begin{pgfscope}%
\pgfsys@transformshift{5.307792in}{0.330000in}%
\pgfsys@useobject{currentmarker}{}%
\end{pgfscope}%
\end{pgfscope}%
\begin{pgfscope}%
\pgfsetbuttcap%
\pgfsetroundjoin%
\definecolor{currentfill}{rgb}{0.000000,0.000000,0.000000}%
\pgfsetfillcolor{currentfill}%
\pgfsetlinewidth{0.803000pt}%
\definecolor{currentstroke}{rgb}{0.000000,0.000000,0.000000}%
\pgfsetstrokecolor{currentstroke}%
\pgfsetdash{}{0pt}%
\pgfsys@defobject{currentmarker}{\pgfqpoint{0.000000in}{0.000000in}}{\pgfqpoint{0.000000in}{0.048611in}}{%
\pgfpathmoveto{\pgfqpoint{0.000000in}{0.000000in}}%
\pgfpathlineto{\pgfqpoint{0.000000in}{0.048611in}}%
\pgfusepath{stroke,fill}%
}%
\begin{pgfscope}%
\pgfsys@transformshift{5.307792in}{2.640000in}%
\pgfsys@useobject{currentmarker}{}%
\end{pgfscope}%
\end{pgfscope}%
\begin{pgfscope}%
\definecolor{textcolor}{rgb}{0.000000,0.000000,0.000000}%
\pgfsetstrokecolor{textcolor}%
\pgfsetfillcolor{textcolor}%
\pgftext[x=5.307792in,y=0.232778in,,top]{\color{textcolor}\rmfamily\fontsize{10.000000}{12.000000}\selectfont \(\displaystyle {5}\)}%
\end{pgfscope}%
\begin{pgfscope}%
\pgfpathrectangle{\pgfqpoint{1.000000in}{0.330000in}}{\pgfqpoint{6.200000in}{2.310000in}}%
\pgfusepath{clip}%
\pgfsetbuttcap%
\pgfsetroundjoin%
\pgfsetlinewidth{0.803000pt}%
\definecolor{currentstroke}{rgb}{0.000000,0.000000,0.000000}%
\pgfsetstrokecolor{currentstroke}%
\pgfsetdash{{0.800000pt}{1.320000pt}}{0.000000pt}%
\pgfpathmoveto{\pgfqpoint{6.112987in}{0.330000in}}%
\pgfpathlineto{\pgfqpoint{6.112987in}{2.640000in}}%
\pgfusepath{stroke}%
\end{pgfscope}%
\begin{pgfscope}%
\pgfsetbuttcap%
\pgfsetroundjoin%
\definecolor{currentfill}{rgb}{0.000000,0.000000,0.000000}%
\pgfsetfillcolor{currentfill}%
\pgfsetlinewidth{0.803000pt}%
\definecolor{currentstroke}{rgb}{0.000000,0.000000,0.000000}%
\pgfsetstrokecolor{currentstroke}%
\pgfsetdash{}{0pt}%
\pgfsys@defobject{currentmarker}{\pgfqpoint{0.000000in}{-0.048611in}}{\pgfqpoint{0.000000in}{0.000000in}}{%
\pgfpathmoveto{\pgfqpoint{0.000000in}{0.000000in}}%
\pgfpathlineto{\pgfqpoint{0.000000in}{-0.048611in}}%
\pgfusepath{stroke,fill}%
}%
\begin{pgfscope}%
\pgfsys@transformshift{6.112987in}{0.330000in}%
\pgfsys@useobject{currentmarker}{}%
\end{pgfscope}%
\end{pgfscope}%
\begin{pgfscope}%
\pgfsetbuttcap%
\pgfsetroundjoin%
\definecolor{currentfill}{rgb}{0.000000,0.000000,0.000000}%
\pgfsetfillcolor{currentfill}%
\pgfsetlinewidth{0.803000pt}%
\definecolor{currentstroke}{rgb}{0.000000,0.000000,0.000000}%
\pgfsetstrokecolor{currentstroke}%
\pgfsetdash{}{0pt}%
\pgfsys@defobject{currentmarker}{\pgfqpoint{0.000000in}{0.000000in}}{\pgfqpoint{0.000000in}{0.048611in}}{%
\pgfpathmoveto{\pgfqpoint{0.000000in}{0.000000in}}%
\pgfpathlineto{\pgfqpoint{0.000000in}{0.048611in}}%
\pgfusepath{stroke,fill}%
}%
\begin{pgfscope}%
\pgfsys@transformshift{6.112987in}{2.640000in}%
\pgfsys@useobject{currentmarker}{}%
\end{pgfscope}%
\end{pgfscope}%
\begin{pgfscope}%
\definecolor{textcolor}{rgb}{0.000000,0.000000,0.000000}%
\pgfsetstrokecolor{textcolor}%
\pgfsetfillcolor{textcolor}%
\pgftext[x=6.112987in,y=0.232778in,,top]{\color{textcolor}\rmfamily\fontsize{10.000000}{12.000000}\selectfont \(\displaystyle {6}\)}%
\end{pgfscope}%
\begin{pgfscope}%
\pgfpathrectangle{\pgfqpoint{1.000000in}{0.330000in}}{\pgfqpoint{6.200000in}{2.310000in}}%
\pgfusepath{clip}%
\pgfsetbuttcap%
\pgfsetroundjoin%
\pgfsetlinewidth{0.803000pt}%
\definecolor{currentstroke}{rgb}{0.000000,0.000000,0.000000}%
\pgfsetstrokecolor{currentstroke}%
\pgfsetdash{{0.800000pt}{1.320000pt}}{0.000000pt}%
\pgfpathmoveto{\pgfqpoint{6.918182in}{0.330000in}}%
\pgfpathlineto{\pgfqpoint{6.918182in}{2.640000in}}%
\pgfusepath{stroke}%
\end{pgfscope}%
\begin{pgfscope}%
\pgfsetbuttcap%
\pgfsetroundjoin%
\definecolor{currentfill}{rgb}{0.000000,0.000000,0.000000}%
\pgfsetfillcolor{currentfill}%
\pgfsetlinewidth{0.803000pt}%
\definecolor{currentstroke}{rgb}{0.000000,0.000000,0.000000}%
\pgfsetstrokecolor{currentstroke}%
\pgfsetdash{}{0pt}%
\pgfsys@defobject{currentmarker}{\pgfqpoint{0.000000in}{-0.048611in}}{\pgfqpoint{0.000000in}{0.000000in}}{%
\pgfpathmoveto{\pgfqpoint{0.000000in}{0.000000in}}%
\pgfpathlineto{\pgfqpoint{0.000000in}{-0.048611in}}%
\pgfusepath{stroke,fill}%
}%
\begin{pgfscope}%
\pgfsys@transformshift{6.918182in}{0.330000in}%
\pgfsys@useobject{currentmarker}{}%
\end{pgfscope}%
\end{pgfscope}%
\begin{pgfscope}%
\pgfsetbuttcap%
\pgfsetroundjoin%
\definecolor{currentfill}{rgb}{0.000000,0.000000,0.000000}%
\pgfsetfillcolor{currentfill}%
\pgfsetlinewidth{0.803000pt}%
\definecolor{currentstroke}{rgb}{0.000000,0.000000,0.000000}%
\pgfsetstrokecolor{currentstroke}%
\pgfsetdash{}{0pt}%
\pgfsys@defobject{currentmarker}{\pgfqpoint{0.000000in}{0.000000in}}{\pgfqpoint{0.000000in}{0.048611in}}{%
\pgfpathmoveto{\pgfqpoint{0.000000in}{0.000000in}}%
\pgfpathlineto{\pgfqpoint{0.000000in}{0.048611in}}%
\pgfusepath{stroke,fill}%
}%
\begin{pgfscope}%
\pgfsys@transformshift{6.918182in}{2.640000in}%
\pgfsys@useobject{currentmarker}{}%
\end{pgfscope}%
\end{pgfscope}%
\begin{pgfscope}%
\definecolor{textcolor}{rgb}{0.000000,0.000000,0.000000}%
\pgfsetstrokecolor{textcolor}%
\pgfsetfillcolor{textcolor}%
\pgftext[x=6.918182in,y=0.232778in,,top]{\color{textcolor}\rmfamily\fontsize{10.000000}{12.000000}\selectfont \(\displaystyle {7}\)}%
\end{pgfscope}%
\begin{pgfscope}%
\pgfpathrectangle{\pgfqpoint{1.000000in}{0.330000in}}{\pgfqpoint{6.200000in}{2.310000in}}%
\pgfusepath{clip}%
\pgfsetbuttcap%
\pgfsetroundjoin%
\pgfsetlinewidth{0.803000pt}%
\definecolor{currentstroke}{rgb}{0.000000,0.000000,0.000000}%
\pgfsetstrokecolor{currentstroke}%
\pgfsetdash{{0.800000pt}{1.320000pt}}{0.000000pt}%
\pgfpathmoveto{\pgfqpoint{1.000000in}{0.619991in}}%
\pgfpathlineto{\pgfqpoint{7.200000in}{0.619991in}}%
\pgfusepath{stroke}%
\end{pgfscope}%
\begin{pgfscope}%
\pgfsetbuttcap%
\pgfsetroundjoin%
\definecolor{currentfill}{rgb}{0.000000,0.000000,0.000000}%
\pgfsetfillcolor{currentfill}%
\pgfsetlinewidth{0.803000pt}%
\definecolor{currentstroke}{rgb}{0.000000,0.000000,0.000000}%
\pgfsetstrokecolor{currentstroke}%
\pgfsetdash{}{0pt}%
\pgfsys@defobject{currentmarker}{\pgfqpoint{-0.048611in}{0.000000in}}{\pgfqpoint{-0.000000in}{0.000000in}}{%
\pgfpathmoveto{\pgfqpoint{-0.000000in}{0.000000in}}%
\pgfpathlineto{\pgfqpoint{-0.048611in}{0.000000in}}%
\pgfusepath{stroke,fill}%
}%
\begin{pgfscope}%
\pgfsys@transformshift{1.000000in}{0.619991in}%
\pgfsys@useobject{currentmarker}{}%
\end{pgfscope}%
\end{pgfscope}%
\begin{pgfscope}%
\pgfsetbuttcap%
\pgfsetroundjoin%
\definecolor{currentfill}{rgb}{0.000000,0.000000,0.000000}%
\pgfsetfillcolor{currentfill}%
\pgfsetlinewidth{0.803000pt}%
\definecolor{currentstroke}{rgb}{0.000000,0.000000,0.000000}%
\pgfsetstrokecolor{currentstroke}%
\pgfsetdash{}{0pt}%
\pgfsys@defobject{currentmarker}{\pgfqpoint{0.000000in}{0.000000in}}{\pgfqpoint{0.048611in}{0.000000in}}{%
\pgfpathmoveto{\pgfqpoint{0.000000in}{0.000000in}}%
\pgfpathlineto{\pgfqpoint{0.048611in}{0.000000in}}%
\pgfusepath{stroke,fill}%
}%
\begin{pgfscope}%
\pgfsys@transformshift{7.200000in}{0.619991in}%
\pgfsys@useobject{currentmarker}{}%
\end{pgfscope}%
\end{pgfscope}%
\begin{pgfscope}%
\definecolor{textcolor}{rgb}{0.000000,0.000000,0.000000}%
\pgfsetstrokecolor{textcolor}%
\pgfsetfillcolor{textcolor}%
\pgftext[x=0.478394in, y=0.571766in, left, base]{\color{textcolor}\rmfamily\fontsize{10.000000}{12.000000}\selectfont \(\displaystyle {\ensuremath{-}0.020}\)}%
\end{pgfscope}%
\begin{pgfscope}%
\pgfpathrectangle{\pgfqpoint{1.000000in}{0.330000in}}{\pgfqpoint{6.200000in}{2.310000in}}%
\pgfusepath{clip}%
\pgfsetbuttcap%
\pgfsetroundjoin%
\pgfsetlinewidth{0.803000pt}%
\definecolor{currentstroke}{rgb}{0.000000,0.000000,0.000000}%
\pgfsetstrokecolor{currentstroke}%
\pgfsetdash{{0.800000pt}{1.320000pt}}{0.000000pt}%
\pgfpathmoveto{\pgfqpoint{1.000000in}{1.098743in}}%
\pgfpathlineto{\pgfqpoint{7.200000in}{1.098743in}}%
\pgfusepath{stroke}%
\end{pgfscope}%
\begin{pgfscope}%
\pgfsetbuttcap%
\pgfsetroundjoin%
\definecolor{currentfill}{rgb}{0.000000,0.000000,0.000000}%
\pgfsetfillcolor{currentfill}%
\pgfsetlinewidth{0.803000pt}%
\definecolor{currentstroke}{rgb}{0.000000,0.000000,0.000000}%
\pgfsetstrokecolor{currentstroke}%
\pgfsetdash{}{0pt}%
\pgfsys@defobject{currentmarker}{\pgfqpoint{-0.048611in}{0.000000in}}{\pgfqpoint{-0.000000in}{0.000000in}}{%
\pgfpathmoveto{\pgfqpoint{-0.000000in}{0.000000in}}%
\pgfpathlineto{\pgfqpoint{-0.048611in}{0.000000in}}%
\pgfusepath{stroke,fill}%
}%
\begin{pgfscope}%
\pgfsys@transformshift{1.000000in}{1.098743in}%
\pgfsys@useobject{currentmarker}{}%
\end{pgfscope}%
\end{pgfscope}%
\begin{pgfscope}%
\pgfsetbuttcap%
\pgfsetroundjoin%
\definecolor{currentfill}{rgb}{0.000000,0.000000,0.000000}%
\pgfsetfillcolor{currentfill}%
\pgfsetlinewidth{0.803000pt}%
\definecolor{currentstroke}{rgb}{0.000000,0.000000,0.000000}%
\pgfsetstrokecolor{currentstroke}%
\pgfsetdash{}{0pt}%
\pgfsys@defobject{currentmarker}{\pgfqpoint{0.000000in}{0.000000in}}{\pgfqpoint{0.048611in}{0.000000in}}{%
\pgfpathmoveto{\pgfqpoint{0.000000in}{0.000000in}}%
\pgfpathlineto{\pgfqpoint{0.048611in}{0.000000in}}%
\pgfusepath{stroke,fill}%
}%
\begin{pgfscope}%
\pgfsys@transformshift{7.200000in}{1.098743in}%
\pgfsys@useobject{currentmarker}{}%
\end{pgfscope}%
\end{pgfscope}%
\begin{pgfscope}%
\definecolor{textcolor}{rgb}{0.000000,0.000000,0.000000}%
\pgfsetstrokecolor{textcolor}%
\pgfsetfillcolor{textcolor}%
\pgftext[x=0.478394in, y=1.050518in, left, base]{\color{textcolor}\rmfamily\fontsize{10.000000}{12.000000}\selectfont \(\displaystyle {\ensuremath{-}0.015}\)}%
\end{pgfscope}%
\begin{pgfscope}%
\pgfpathrectangle{\pgfqpoint{1.000000in}{0.330000in}}{\pgfqpoint{6.200000in}{2.310000in}}%
\pgfusepath{clip}%
\pgfsetbuttcap%
\pgfsetroundjoin%
\pgfsetlinewidth{0.803000pt}%
\definecolor{currentstroke}{rgb}{0.000000,0.000000,0.000000}%
\pgfsetstrokecolor{currentstroke}%
\pgfsetdash{{0.800000pt}{1.320000pt}}{0.000000pt}%
\pgfpathmoveto{\pgfqpoint{1.000000in}{1.577496in}}%
\pgfpathlineto{\pgfqpoint{7.200000in}{1.577496in}}%
\pgfusepath{stroke}%
\end{pgfscope}%
\begin{pgfscope}%
\pgfsetbuttcap%
\pgfsetroundjoin%
\definecolor{currentfill}{rgb}{0.000000,0.000000,0.000000}%
\pgfsetfillcolor{currentfill}%
\pgfsetlinewidth{0.803000pt}%
\definecolor{currentstroke}{rgb}{0.000000,0.000000,0.000000}%
\pgfsetstrokecolor{currentstroke}%
\pgfsetdash{}{0pt}%
\pgfsys@defobject{currentmarker}{\pgfqpoint{-0.048611in}{0.000000in}}{\pgfqpoint{-0.000000in}{0.000000in}}{%
\pgfpathmoveto{\pgfqpoint{-0.000000in}{0.000000in}}%
\pgfpathlineto{\pgfqpoint{-0.048611in}{0.000000in}}%
\pgfusepath{stroke,fill}%
}%
\begin{pgfscope}%
\pgfsys@transformshift{1.000000in}{1.577496in}%
\pgfsys@useobject{currentmarker}{}%
\end{pgfscope}%
\end{pgfscope}%
\begin{pgfscope}%
\pgfsetbuttcap%
\pgfsetroundjoin%
\definecolor{currentfill}{rgb}{0.000000,0.000000,0.000000}%
\pgfsetfillcolor{currentfill}%
\pgfsetlinewidth{0.803000pt}%
\definecolor{currentstroke}{rgb}{0.000000,0.000000,0.000000}%
\pgfsetstrokecolor{currentstroke}%
\pgfsetdash{}{0pt}%
\pgfsys@defobject{currentmarker}{\pgfqpoint{0.000000in}{0.000000in}}{\pgfqpoint{0.048611in}{0.000000in}}{%
\pgfpathmoveto{\pgfqpoint{0.000000in}{0.000000in}}%
\pgfpathlineto{\pgfqpoint{0.048611in}{0.000000in}}%
\pgfusepath{stroke,fill}%
}%
\begin{pgfscope}%
\pgfsys@transformshift{7.200000in}{1.577496in}%
\pgfsys@useobject{currentmarker}{}%
\end{pgfscope}%
\end{pgfscope}%
\begin{pgfscope}%
\definecolor{textcolor}{rgb}{0.000000,0.000000,0.000000}%
\pgfsetstrokecolor{textcolor}%
\pgfsetfillcolor{textcolor}%
\pgftext[x=0.478394in, y=1.529270in, left, base]{\color{textcolor}\rmfamily\fontsize{10.000000}{12.000000}\selectfont \(\displaystyle {\ensuremath{-}0.010}\)}%
\end{pgfscope}%
\begin{pgfscope}%
\pgfpathrectangle{\pgfqpoint{1.000000in}{0.330000in}}{\pgfqpoint{6.200000in}{2.310000in}}%
\pgfusepath{clip}%
\pgfsetbuttcap%
\pgfsetroundjoin%
\pgfsetlinewidth{0.803000pt}%
\definecolor{currentstroke}{rgb}{0.000000,0.000000,0.000000}%
\pgfsetstrokecolor{currentstroke}%
\pgfsetdash{{0.800000pt}{1.320000pt}}{0.000000pt}%
\pgfpathmoveto{\pgfqpoint{1.000000in}{2.056248in}}%
\pgfpathlineto{\pgfqpoint{7.200000in}{2.056248in}}%
\pgfusepath{stroke}%
\end{pgfscope}%
\begin{pgfscope}%
\pgfsetbuttcap%
\pgfsetroundjoin%
\definecolor{currentfill}{rgb}{0.000000,0.000000,0.000000}%
\pgfsetfillcolor{currentfill}%
\pgfsetlinewidth{0.803000pt}%
\definecolor{currentstroke}{rgb}{0.000000,0.000000,0.000000}%
\pgfsetstrokecolor{currentstroke}%
\pgfsetdash{}{0pt}%
\pgfsys@defobject{currentmarker}{\pgfqpoint{-0.048611in}{0.000000in}}{\pgfqpoint{-0.000000in}{0.000000in}}{%
\pgfpathmoveto{\pgfqpoint{-0.000000in}{0.000000in}}%
\pgfpathlineto{\pgfqpoint{-0.048611in}{0.000000in}}%
\pgfusepath{stroke,fill}%
}%
\begin{pgfscope}%
\pgfsys@transformshift{1.000000in}{2.056248in}%
\pgfsys@useobject{currentmarker}{}%
\end{pgfscope}%
\end{pgfscope}%
\begin{pgfscope}%
\pgfsetbuttcap%
\pgfsetroundjoin%
\definecolor{currentfill}{rgb}{0.000000,0.000000,0.000000}%
\pgfsetfillcolor{currentfill}%
\pgfsetlinewidth{0.803000pt}%
\definecolor{currentstroke}{rgb}{0.000000,0.000000,0.000000}%
\pgfsetstrokecolor{currentstroke}%
\pgfsetdash{}{0pt}%
\pgfsys@defobject{currentmarker}{\pgfqpoint{0.000000in}{0.000000in}}{\pgfqpoint{0.048611in}{0.000000in}}{%
\pgfpathmoveto{\pgfqpoint{0.000000in}{0.000000in}}%
\pgfpathlineto{\pgfqpoint{0.048611in}{0.000000in}}%
\pgfusepath{stroke,fill}%
}%
\begin{pgfscope}%
\pgfsys@transformshift{7.200000in}{2.056248in}%
\pgfsys@useobject{currentmarker}{}%
\end{pgfscope}%
\end{pgfscope}%
\begin{pgfscope}%
\definecolor{textcolor}{rgb}{0.000000,0.000000,0.000000}%
\pgfsetstrokecolor{textcolor}%
\pgfsetfillcolor{textcolor}%
\pgftext[x=0.478394in, y=2.008022in, left, base]{\color{textcolor}\rmfamily\fontsize{10.000000}{12.000000}\selectfont \(\displaystyle {\ensuremath{-}0.005}\)}%
\end{pgfscope}%
\begin{pgfscope}%
\pgfpathrectangle{\pgfqpoint{1.000000in}{0.330000in}}{\pgfqpoint{6.200000in}{2.310000in}}%
\pgfusepath{clip}%
\pgfsetbuttcap%
\pgfsetroundjoin%
\pgfsetlinewidth{0.803000pt}%
\definecolor{currentstroke}{rgb}{0.000000,0.000000,0.000000}%
\pgfsetstrokecolor{currentstroke}%
\pgfsetdash{{0.800000pt}{1.320000pt}}{0.000000pt}%
\pgfpathmoveto{\pgfqpoint{1.000000in}{2.535000in}}%
\pgfpathlineto{\pgfqpoint{7.200000in}{2.535000in}}%
\pgfusepath{stroke}%
\end{pgfscope}%
\begin{pgfscope}%
\pgfsetbuttcap%
\pgfsetroundjoin%
\definecolor{currentfill}{rgb}{0.000000,0.000000,0.000000}%
\pgfsetfillcolor{currentfill}%
\pgfsetlinewidth{0.803000pt}%
\definecolor{currentstroke}{rgb}{0.000000,0.000000,0.000000}%
\pgfsetstrokecolor{currentstroke}%
\pgfsetdash{}{0pt}%
\pgfsys@defobject{currentmarker}{\pgfqpoint{-0.048611in}{0.000000in}}{\pgfqpoint{-0.000000in}{0.000000in}}{%
\pgfpathmoveto{\pgfqpoint{-0.000000in}{0.000000in}}%
\pgfpathlineto{\pgfqpoint{-0.048611in}{0.000000in}}%
\pgfusepath{stroke,fill}%
}%
\begin{pgfscope}%
\pgfsys@transformshift{1.000000in}{2.535000in}%
\pgfsys@useobject{currentmarker}{}%
\end{pgfscope}%
\end{pgfscope}%
\begin{pgfscope}%
\pgfsetbuttcap%
\pgfsetroundjoin%
\definecolor{currentfill}{rgb}{0.000000,0.000000,0.000000}%
\pgfsetfillcolor{currentfill}%
\pgfsetlinewidth{0.803000pt}%
\definecolor{currentstroke}{rgb}{0.000000,0.000000,0.000000}%
\pgfsetstrokecolor{currentstroke}%
\pgfsetdash{}{0pt}%
\pgfsys@defobject{currentmarker}{\pgfqpoint{0.000000in}{0.000000in}}{\pgfqpoint{0.048611in}{0.000000in}}{%
\pgfpathmoveto{\pgfqpoint{0.000000in}{0.000000in}}%
\pgfpathlineto{\pgfqpoint{0.048611in}{0.000000in}}%
\pgfusepath{stroke,fill}%
}%
\begin{pgfscope}%
\pgfsys@transformshift{7.200000in}{2.535000in}%
\pgfsys@useobject{currentmarker}{}%
\end{pgfscope}%
\end{pgfscope}%
\begin{pgfscope}%
\definecolor{textcolor}{rgb}{0.000000,0.000000,0.000000}%
\pgfsetstrokecolor{textcolor}%
\pgfsetfillcolor{textcolor}%
\pgftext[x=0.586419in, y=2.486775in, left, base]{\color{textcolor}\rmfamily\fontsize{10.000000}{12.000000}\selectfont \(\displaystyle {0.000}\)}%
\end{pgfscope}%
\begin{pgfscope}%
\pgfpathrectangle{\pgfqpoint{1.000000in}{0.330000in}}{\pgfqpoint{6.200000in}{2.310000in}}%
\pgfusepath{clip}%
\pgfsetrectcap%
\pgfsetroundjoin%
\pgfsetlinewidth{1.505625pt}%
\definecolor{currentstroke}{rgb}{0.121569,0.466667,0.705882}%
\pgfsetstrokecolor{currentstroke}%
\pgfsetdash{}{0pt}%
\pgfpathmoveto{\pgfqpoint{1.281818in}{2.535000in}}%
\pgfpathlineto{\pgfqpoint{1.717965in}{2.413047in}}%
\pgfpathlineto{\pgfqpoint{1.952814in}{2.349491in}}%
\pgfpathlineto{\pgfqpoint{2.154113in}{2.297093in}}%
\pgfpathlineto{\pgfqpoint{2.355411in}{2.247220in}}%
\pgfpathlineto{\pgfqpoint{2.523160in}{2.208000in}}%
\pgfpathlineto{\pgfqpoint{2.690909in}{2.171260in}}%
\pgfpathlineto{\pgfqpoint{2.825108in}{2.143873in}}%
\pgfpathlineto{\pgfqpoint{2.959307in}{2.118453in}}%
\pgfpathlineto{\pgfqpoint{3.093506in}{2.095164in}}%
\pgfpathlineto{\pgfqpoint{3.227706in}{2.074175in}}%
\pgfpathlineto{\pgfqpoint{3.361905in}{2.055652in}}%
\pgfpathlineto{\pgfqpoint{3.496104in}{2.039761in}}%
\pgfpathlineto{\pgfqpoint{3.630303in}{2.026668in}}%
\pgfpathlineto{\pgfqpoint{3.697403in}{2.021222in}}%
\pgfpathlineto{\pgfqpoint{3.798052in}{2.019639in}}%
\pgfpathlineto{\pgfqpoint{3.932251in}{2.020106in}}%
\pgfpathlineto{\pgfqpoint{4.066450in}{2.023404in}}%
\pgfpathlineto{\pgfqpoint{4.200649in}{2.029413in}}%
\pgfpathlineto{\pgfqpoint{4.334848in}{2.038008in}}%
\pgfpathlineto{\pgfqpoint{4.469048in}{2.049068in}}%
\pgfpathlineto{\pgfqpoint{4.603247in}{2.062468in}}%
\pgfpathlineto{\pgfqpoint{4.737446in}{2.078084in}}%
\pgfpathlineto{\pgfqpoint{4.871645in}{2.095793in}}%
\pgfpathlineto{\pgfqpoint{5.005844in}{2.115468in}}%
\pgfpathlineto{\pgfqpoint{5.173593in}{2.142638in}}%
\pgfpathlineto{\pgfqpoint{5.341342in}{2.172439in}}%
\pgfpathlineto{\pgfqpoint{5.509091in}{2.204626in}}%
\pgfpathlineto{\pgfqpoint{5.676840in}{2.238950in}}%
\pgfpathlineto{\pgfqpoint{5.878139in}{2.282611in}}%
\pgfpathlineto{\pgfqpoint{6.079437in}{2.328559in}}%
\pgfpathlineto{\pgfqpoint{6.314286in}{2.384479in}}%
\pgfpathlineto{\pgfqpoint{6.616234in}{2.458968in}}%
\pgfpathlineto{\pgfqpoint{6.918182in}{2.535000in}}%
\pgfpathlineto{\pgfqpoint{6.918182in}{2.535000in}}%
\pgfusepath{stroke}%
\end{pgfscope}%
\begin{pgfscope}%
\pgfpathrectangle{\pgfqpoint{1.000000in}{0.330000in}}{\pgfqpoint{6.200000in}{2.310000in}}%
\pgfusepath{clip}%
\pgfsetrectcap%
\pgfsetroundjoin%
\pgfsetlinewidth{1.505625pt}%
\definecolor{currentstroke}{rgb}{1.000000,0.498039,0.054902}%
\pgfsetstrokecolor{currentstroke}%
\pgfsetdash{}{0pt}%
\pgfpathmoveto{\pgfqpoint{1.281818in}{2.535000in}}%
\pgfpathlineto{\pgfqpoint{1.785065in}{2.441447in}}%
\pgfpathlineto{\pgfqpoint{2.053463in}{2.393675in}}%
\pgfpathlineto{\pgfqpoint{2.288312in}{2.354019in}}%
\pgfpathlineto{\pgfqpoint{2.489610in}{2.322104in}}%
\pgfpathlineto{\pgfqpoint{2.690909in}{2.292506in}}%
\pgfpathlineto{\pgfqpoint{2.858658in}{2.269885in}}%
\pgfpathlineto{\pgfqpoint{3.026407in}{2.249354in}}%
\pgfpathlineto{\pgfqpoint{3.194156in}{2.231132in}}%
\pgfpathlineto{\pgfqpoint{3.361905in}{2.215435in}}%
\pgfpathlineto{\pgfqpoint{3.496104in}{2.204841in}}%
\pgfpathlineto{\pgfqpoint{3.630303in}{2.196112in}}%
\pgfpathlineto{\pgfqpoint{3.697403in}{2.192481in}}%
\pgfpathlineto{\pgfqpoint{3.831602in}{2.191323in}}%
\pgfpathlineto{\pgfqpoint{3.965801in}{2.192113in}}%
\pgfpathlineto{\pgfqpoint{4.100000in}{2.194771in}}%
\pgfpathlineto{\pgfqpoint{4.267749in}{2.200596in}}%
\pgfpathlineto{\pgfqpoint{4.435498in}{2.209053in}}%
\pgfpathlineto{\pgfqpoint{4.603247in}{2.219979in}}%
\pgfpathlineto{\pgfqpoint{4.770996in}{2.233214in}}%
\pgfpathlineto{\pgfqpoint{4.938745in}{2.248595in}}%
\pgfpathlineto{\pgfqpoint{5.106494in}{2.265960in}}%
\pgfpathlineto{\pgfqpoint{5.307792in}{2.289187in}}%
\pgfpathlineto{\pgfqpoint{5.509091in}{2.314751in}}%
\pgfpathlineto{\pgfqpoint{5.710390in}{2.342367in}}%
\pgfpathlineto{\pgfqpoint{5.945238in}{2.376796in}}%
\pgfpathlineto{\pgfqpoint{6.213636in}{2.418499in}}%
\pgfpathlineto{\pgfqpoint{6.549134in}{2.473152in}}%
\pgfpathlineto{\pgfqpoint{6.918182in}{2.535000in}}%
\pgfpathlineto{\pgfqpoint{6.918182in}{2.535000in}}%
\pgfusepath{stroke}%
\end{pgfscope}%
\begin{pgfscope}%
\pgfpathrectangle{\pgfqpoint{1.000000in}{0.330000in}}{\pgfqpoint{6.200000in}{2.310000in}}%
\pgfusepath{clip}%
\pgfsetrectcap%
\pgfsetroundjoin%
\pgfsetlinewidth{1.505625pt}%
\definecolor{currentstroke}{rgb}{0.172549,0.627451,0.172549}%
\pgfsetstrokecolor{currentstroke}%
\pgfsetdash{}{0pt}%
\pgfpathmoveto{\pgfqpoint{1.281818in}{2.535000in}}%
\pgfpathlineto{\pgfqpoint{1.650866in}{2.362609in}}%
\pgfpathlineto{\pgfqpoint{1.852165in}{2.270763in}}%
\pgfpathlineto{\pgfqpoint{2.019913in}{2.196300in}}%
\pgfpathlineto{\pgfqpoint{2.187662in}{2.124317in}}%
\pgfpathlineto{\pgfqpoint{2.321861in}{2.068888in}}%
\pgfpathlineto{\pgfqpoint{2.456061in}{2.015682in}}%
\pgfpathlineto{\pgfqpoint{2.590260in}{1.964980in}}%
\pgfpathlineto{\pgfqpoint{2.724459in}{1.917063in}}%
\pgfpathlineto{\pgfqpoint{2.825108in}{1.883122in}}%
\pgfpathlineto{\pgfqpoint{2.925758in}{1.851024in}}%
\pgfpathlineto{\pgfqpoint{3.026407in}{1.820886in}}%
\pgfpathlineto{\pgfqpoint{3.127056in}{1.792825in}}%
\pgfpathlineto{\pgfqpoint{3.227706in}{1.766959in}}%
\pgfpathlineto{\pgfqpoint{3.328355in}{1.743404in}}%
\pgfpathlineto{\pgfqpoint{3.429004in}{1.722279in}}%
\pgfpathlineto{\pgfqpoint{3.529654in}{1.703698in}}%
\pgfpathlineto{\pgfqpoint{3.630303in}{1.687780in}}%
\pgfpathlineto{\pgfqpoint{3.697403in}{1.678703in}}%
\pgfpathlineto{\pgfqpoint{3.798052in}{1.676066in}}%
\pgfpathlineto{\pgfqpoint{3.898701in}{1.676198in}}%
\pgfpathlineto{\pgfqpoint{3.999351in}{1.679014in}}%
\pgfpathlineto{\pgfqpoint{4.100000in}{1.684428in}}%
\pgfpathlineto{\pgfqpoint{4.200649in}{1.692354in}}%
\pgfpathlineto{\pgfqpoint{4.301299in}{1.702706in}}%
\pgfpathlineto{\pgfqpoint{4.401948in}{1.715396in}}%
\pgfpathlineto{\pgfqpoint{4.502597in}{1.730339in}}%
\pgfpathlineto{\pgfqpoint{4.603247in}{1.747447in}}%
\pgfpathlineto{\pgfqpoint{4.703896in}{1.766632in}}%
\pgfpathlineto{\pgfqpoint{4.804545in}{1.787808in}}%
\pgfpathlineto{\pgfqpoint{4.905195in}{1.810887in}}%
\pgfpathlineto{\pgfqpoint{5.005844in}{1.835780in}}%
\pgfpathlineto{\pgfqpoint{5.140043in}{1.871642in}}%
\pgfpathlineto{\pgfqpoint{5.274242in}{1.910364in}}%
\pgfpathlineto{\pgfqpoint{5.408442in}{1.951736in}}%
\pgfpathlineto{\pgfqpoint{5.542641in}{1.995546in}}%
\pgfpathlineto{\pgfqpoint{5.676840in}{2.041584in}}%
\pgfpathlineto{\pgfqpoint{5.811039in}{2.089636in}}%
\pgfpathlineto{\pgfqpoint{5.978788in}{2.152210in}}%
\pgfpathlineto{\pgfqpoint{6.146537in}{2.217182in}}%
\pgfpathlineto{\pgfqpoint{6.347835in}{2.297721in}}%
\pgfpathlineto{\pgfqpoint{6.582684in}{2.394313in}}%
\pgfpathlineto{\pgfqpoint{6.918182in}{2.535000in}}%
\pgfpathlineto{\pgfqpoint{6.918182in}{2.535000in}}%
\pgfusepath{stroke}%
\end{pgfscope}%
\begin{pgfscope}%
\pgfpathrectangle{\pgfqpoint{1.000000in}{0.330000in}}{\pgfqpoint{6.200000in}{2.310000in}}%
\pgfusepath{clip}%
\pgfsetrectcap%
\pgfsetroundjoin%
\pgfsetlinewidth{1.505625pt}%
\definecolor{currentstroke}{rgb}{0.839216,0.152941,0.156863}%
\pgfsetstrokecolor{currentstroke}%
\pgfsetdash{}{0pt}%
\pgfpathmoveto{\pgfqpoint{1.281818in}{2.535000in}}%
\pgfpathlineto{\pgfqpoint{1.583766in}{2.241021in}}%
\pgfpathlineto{\pgfqpoint{1.751515in}{2.080224in}}%
\pgfpathlineto{\pgfqpoint{1.919264in}{1.922695in}}%
\pgfpathlineto{\pgfqpoint{2.053463in}{1.799801in}}%
\pgfpathlineto{\pgfqpoint{2.187662in}{1.680328in}}%
\pgfpathlineto{\pgfqpoint{2.288312in}{1.593323in}}%
\pgfpathlineto{\pgfqpoint{2.388961in}{1.508822in}}%
\pgfpathlineto{\pgfqpoint{2.489610in}{1.427074in}}%
\pgfpathlineto{\pgfqpoint{2.590260in}{1.348327in}}%
\pgfpathlineto{\pgfqpoint{2.690909in}{1.272829in}}%
\pgfpathlineto{\pgfqpoint{2.791558in}{1.200826in}}%
\pgfpathlineto{\pgfqpoint{2.892208in}{1.132567in}}%
\pgfpathlineto{\pgfqpoint{2.959307in}{1.089263in}}%
\pgfpathlineto{\pgfqpoint{3.026407in}{1.047806in}}%
\pgfpathlineto{\pgfqpoint{3.093506in}{1.008268in}}%
\pgfpathlineto{\pgfqpoint{3.160606in}{0.970724in}}%
\pgfpathlineto{\pgfqpoint{3.227706in}{0.935246in}}%
\pgfpathlineto{\pgfqpoint{3.294805in}{0.901907in}}%
\pgfpathlineto{\pgfqpoint{3.361905in}{0.870779in}}%
\pgfpathlineto{\pgfqpoint{3.429004in}{0.841937in}}%
\pgfpathlineto{\pgfqpoint{3.496104in}{0.815453in}}%
\pgfpathlineto{\pgfqpoint{3.563203in}{0.791399in}}%
\pgfpathlineto{\pgfqpoint{3.630303in}{0.769848in}}%
\pgfpathlineto{\pgfqpoint{3.697403in}{0.750874in}}%
\pgfpathlineto{\pgfqpoint{3.764502in}{0.746740in}}%
\pgfpathlineto{\pgfqpoint{3.831602in}{0.745202in}}%
\pgfpathlineto{\pgfqpoint{3.898701in}{0.746203in}}%
\pgfpathlineto{\pgfqpoint{3.965801in}{0.749692in}}%
\pgfpathlineto{\pgfqpoint{4.032900in}{0.755612in}}%
\pgfpathlineto{\pgfqpoint{4.100000in}{0.763911in}}%
\pgfpathlineto{\pgfqpoint{4.167100in}{0.774534in}}%
\pgfpathlineto{\pgfqpoint{4.234199in}{0.787426in}}%
\pgfpathlineto{\pgfqpoint{4.301299in}{0.802533in}}%
\pgfpathlineto{\pgfqpoint{4.368398in}{0.819801in}}%
\pgfpathlineto{\pgfqpoint{4.435498in}{0.839175in}}%
\pgfpathlineto{\pgfqpoint{4.502597in}{0.860600in}}%
\pgfpathlineto{\pgfqpoint{4.569697in}{0.884023in}}%
\pgfpathlineto{\pgfqpoint{4.636797in}{0.909389in}}%
\pgfpathlineto{\pgfqpoint{4.703896in}{0.936643in}}%
\pgfpathlineto{\pgfqpoint{4.770996in}{0.965729in}}%
\pgfpathlineto{\pgfqpoint{4.838095in}{0.996595in}}%
\pgfpathlineto{\pgfqpoint{4.905195in}{1.029184in}}%
\pgfpathlineto{\pgfqpoint{5.005844in}{1.081180in}}%
\pgfpathlineto{\pgfqpoint{5.106494in}{1.136747in}}%
\pgfpathlineto{\pgfqpoint{5.207143in}{1.195698in}}%
\pgfpathlineto{\pgfqpoint{5.307792in}{1.257848in}}%
\pgfpathlineto{\pgfqpoint{5.408442in}{1.323011in}}%
\pgfpathlineto{\pgfqpoint{5.509091in}{1.391001in}}%
\pgfpathlineto{\pgfqpoint{5.609740in}{1.461633in}}%
\pgfpathlineto{\pgfqpoint{5.710390in}{1.534719in}}%
\pgfpathlineto{\pgfqpoint{5.811039in}{1.610072in}}%
\pgfpathlineto{\pgfqpoint{5.945238in}{1.713748in}}%
\pgfpathlineto{\pgfqpoint{6.079437in}{1.820679in}}%
\pgfpathlineto{\pgfqpoint{6.213636in}{1.930421in}}%
\pgfpathlineto{\pgfqpoint{6.381385in}{2.070872in}}%
\pgfpathlineto{\pgfqpoint{6.582684in}{2.243067in}}%
\pgfpathlineto{\pgfqpoint{6.851082in}{2.476380in}}%
\pgfpathlineto{\pgfqpoint{6.918182in}{2.535000in}}%
\pgfpathlineto{\pgfqpoint{6.918182in}{2.535000in}}%
\pgfusepath{stroke}%
\end{pgfscope}%
\begin{pgfscope}%
\pgfpathrectangle{\pgfqpoint{1.000000in}{0.330000in}}{\pgfqpoint{6.200000in}{2.310000in}}%
\pgfusepath{clip}%
\pgfsetrectcap%
\pgfsetroundjoin%
\pgfsetlinewidth{1.505625pt}%
\definecolor{currentstroke}{rgb}{0.580392,0.403922,0.741176}%
\pgfsetstrokecolor{currentstroke}%
\pgfsetdash{}{0pt}%
\pgfpathmoveto{\pgfqpoint{1.281818in}{2.535000in}}%
\pgfpathlineto{\pgfqpoint{1.617316in}{2.309049in}}%
\pgfpathlineto{\pgfqpoint{1.818615in}{2.176136in}}%
\pgfpathlineto{\pgfqpoint{1.986364in}{2.068127in}}%
\pgfpathlineto{\pgfqpoint{2.120563in}{1.984096in}}%
\pgfpathlineto{\pgfqpoint{2.254762in}{1.902621in}}%
\pgfpathlineto{\pgfqpoint{2.388961in}{1.824111in}}%
\pgfpathlineto{\pgfqpoint{2.489610in}{1.767419in}}%
\pgfpathlineto{\pgfqpoint{2.590260in}{1.712797in}}%
\pgfpathlineto{\pgfqpoint{2.690909in}{1.660417in}}%
\pgfpathlineto{\pgfqpoint{2.791558in}{1.610452in}}%
\pgfpathlineto{\pgfqpoint{2.892208in}{1.563074in}}%
\pgfpathlineto{\pgfqpoint{2.992857in}{1.518456in}}%
\pgfpathlineto{\pgfqpoint{3.093506in}{1.476771in}}%
\pgfpathlineto{\pgfqpoint{3.194156in}{1.438190in}}%
\pgfpathlineto{\pgfqpoint{3.294805in}{1.402887in}}%
\pgfpathlineto{\pgfqpoint{3.361905in}{1.381257in}}%
\pgfpathlineto{\pgfqpoint{3.429004in}{1.361211in}}%
\pgfpathlineto{\pgfqpoint{3.496104in}{1.342801in}}%
\pgfpathlineto{\pgfqpoint{3.563203in}{1.326078in}}%
\pgfpathlineto{\pgfqpoint{3.630303in}{1.311092in}}%
\pgfpathlineto{\pgfqpoint{3.697403in}{1.297895in}}%
\pgfpathlineto{\pgfqpoint{3.764502in}{1.295143in}}%
\pgfpathlineto{\pgfqpoint{3.831602in}{1.294194in}}%
\pgfpathlineto{\pgfqpoint{3.898701in}{1.295007in}}%
\pgfpathlineto{\pgfqpoint{3.965801in}{1.297546in}}%
\pgfpathlineto{\pgfqpoint{4.032900in}{1.301771in}}%
\pgfpathlineto{\pgfqpoint{4.100000in}{1.307644in}}%
\pgfpathlineto{\pgfqpoint{4.167100in}{1.315127in}}%
\pgfpathlineto{\pgfqpoint{4.234199in}{1.324181in}}%
\pgfpathlineto{\pgfqpoint{4.301299in}{1.334769in}}%
\pgfpathlineto{\pgfqpoint{4.401948in}{1.353442in}}%
\pgfpathlineto{\pgfqpoint{4.502597in}{1.375349in}}%
\pgfpathlineto{\pgfqpoint{4.603247in}{1.400360in}}%
\pgfpathlineto{\pgfqpoint{4.703896in}{1.428347in}}%
\pgfpathlineto{\pgfqpoint{4.804545in}{1.459180in}}%
\pgfpathlineto{\pgfqpoint{4.905195in}{1.492730in}}%
\pgfpathlineto{\pgfqpoint{5.005844in}{1.528867in}}%
\pgfpathlineto{\pgfqpoint{5.106494in}{1.567461in}}%
\pgfpathlineto{\pgfqpoint{5.207143in}{1.608385in}}%
\pgfpathlineto{\pgfqpoint{5.307792in}{1.651507in}}%
\pgfpathlineto{\pgfqpoint{5.408442in}{1.696700in}}%
\pgfpathlineto{\pgfqpoint{5.509091in}{1.743833in}}%
\pgfpathlineto{\pgfqpoint{5.643290in}{1.809472in}}%
\pgfpathlineto{\pgfqpoint{5.777489in}{1.878024in}}%
\pgfpathlineto{\pgfqpoint{5.911688in}{1.949182in}}%
\pgfpathlineto{\pgfqpoint{6.045887in}{2.022640in}}%
\pgfpathlineto{\pgfqpoint{6.213636in}{2.117230in}}%
\pgfpathlineto{\pgfqpoint{6.381385in}{2.214336in}}%
\pgfpathlineto{\pgfqpoint{6.582684in}{2.333339in}}%
\pgfpathlineto{\pgfqpoint{6.851082in}{2.494514in}}%
\pgfpathlineto{\pgfqpoint{6.918182in}{2.535000in}}%
\pgfpathlineto{\pgfqpoint{6.918182in}{2.535000in}}%
\pgfusepath{stroke}%
\end{pgfscope}%
\begin{pgfscope}%
\pgfpathrectangle{\pgfqpoint{1.000000in}{0.330000in}}{\pgfqpoint{6.200000in}{2.310000in}}%
\pgfusepath{clip}%
\pgfsetrectcap%
\pgfsetroundjoin%
\pgfsetlinewidth{1.505625pt}%
\definecolor{currentstroke}{rgb}{0.549020,0.337255,0.294118}%
\pgfsetstrokecolor{currentstroke}%
\pgfsetdash{}{0pt}%
\pgfpathmoveto{\pgfqpoint{1.281818in}{2.535000in}}%
\pgfpathlineto{\pgfqpoint{1.583766in}{2.226182in}}%
\pgfpathlineto{\pgfqpoint{1.751515in}{2.057271in}}%
\pgfpathlineto{\pgfqpoint{1.885714in}{1.924557in}}%
\pgfpathlineto{\pgfqpoint{2.019913in}{1.794663in}}%
\pgfpathlineto{\pgfqpoint{2.154113in}{1.668210in}}%
\pgfpathlineto{\pgfqpoint{2.254762in}{1.575999in}}%
\pgfpathlineto{\pgfqpoint{2.355411in}{1.486334in}}%
\pgfpathlineto{\pgfqpoint{2.456061in}{1.399474in}}%
\pgfpathlineto{\pgfqpoint{2.556710in}{1.315680in}}%
\pgfpathlineto{\pgfqpoint{2.657359in}{1.235212in}}%
\pgfpathlineto{\pgfqpoint{2.758009in}{1.158331in}}%
\pgfpathlineto{\pgfqpoint{2.858658in}{1.085296in}}%
\pgfpathlineto{\pgfqpoint{2.925758in}{1.038871in}}%
\pgfpathlineto{\pgfqpoint{2.992857in}{0.994348in}}%
\pgfpathlineto{\pgfqpoint{3.059957in}{0.951804in}}%
\pgfpathlineto{\pgfqpoint{3.127056in}{0.911314in}}%
\pgfpathlineto{\pgfqpoint{3.194156in}{0.872957in}}%
\pgfpathlineto{\pgfqpoint{3.261255in}{0.836808in}}%
\pgfpathlineto{\pgfqpoint{3.328355in}{0.802944in}}%
\pgfpathlineto{\pgfqpoint{3.395455in}{0.771443in}}%
\pgfpathlineto{\pgfqpoint{3.462554in}{0.742379in}}%
\pgfpathlineto{\pgfqpoint{3.529654in}{0.715831in}}%
\pgfpathlineto{\pgfqpoint{3.596753in}{0.691873in}}%
\pgfpathlineto{\pgfqpoint{3.630303in}{0.680890in}}%
\pgfpathlineto{\pgfqpoint{3.697403in}{0.660963in}}%
\pgfpathlineto{\pgfqpoint{3.764502in}{0.656612in}}%
\pgfpathlineto{\pgfqpoint{3.831602in}{0.654986in}}%
\pgfpathlineto{\pgfqpoint{3.898701in}{0.656029in}}%
\pgfpathlineto{\pgfqpoint{3.965801in}{0.659684in}}%
\pgfpathlineto{\pgfqpoint{4.032900in}{0.665893in}}%
\pgfpathlineto{\pgfqpoint{4.100000in}{0.674601in}}%
\pgfpathlineto{\pgfqpoint{4.167100in}{0.685750in}}%
\pgfpathlineto{\pgfqpoint{4.234199in}{0.699282in}}%
\pgfpathlineto{\pgfqpoint{4.301299in}{0.715142in}}%
\pgfpathlineto{\pgfqpoint{4.368398in}{0.733271in}}%
\pgfpathlineto{\pgfqpoint{4.435498in}{0.753613in}}%
\pgfpathlineto{\pgfqpoint{4.502597in}{0.776111in}}%
\pgfpathlineto{\pgfqpoint{4.569697in}{0.800707in}}%
\pgfpathlineto{\pgfqpoint{4.636797in}{0.827344in}}%
\pgfpathlineto{\pgfqpoint{4.703896in}{0.855964in}}%
\pgfpathlineto{\pgfqpoint{4.770996in}{0.886511in}}%
\pgfpathlineto{\pgfqpoint{4.838095in}{0.918926in}}%
\pgfpathlineto{\pgfqpoint{4.905195in}{0.953152in}}%
\pgfpathlineto{\pgfqpoint{4.972294in}{0.989133in}}%
\pgfpathlineto{\pgfqpoint{5.072944in}{1.046265in}}%
\pgfpathlineto{\pgfqpoint{5.173593in}{1.107019in}}%
\pgfpathlineto{\pgfqpoint{5.274242in}{1.171200in}}%
\pgfpathlineto{\pgfqpoint{5.374892in}{1.238613in}}%
\pgfpathlineto{\pgfqpoint{5.475541in}{1.309062in}}%
\pgfpathlineto{\pgfqpoint{5.576190in}{1.382352in}}%
\pgfpathlineto{\pgfqpoint{5.676840in}{1.458287in}}%
\pgfpathlineto{\pgfqpoint{5.777489in}{1.536671in}}%
\pgfpathlineto{\pgfqpoint{5.878139in}{1.617307in}}%
\pgfpathlineto{\pgfqpoint{6.012338in}{1.727987in}}%
\pgfpathlineto{\pgfqpoint{6.146537in}{1.841855in}}%
\pgfpathlineto{\pgfqpoint{6.280736in}{1.958445in}}%
\pgfpathlineto{\pgfqpoint{6.448485in}{2.107295in}}%
\pgfpathlineto{\pgfqpoint{6.649784in}{2.289273in}}%
\pgfpathlineto{\pgfqpoint{6.918182in}{2.535000in}}%
\pgfpathlineto{\pgfqpoint{6.918182in}{2.535000in}}%
\pgfusepath{stroke}%
\end{pgfscope}%
\begin{pgfscope}%
\pgfpathrectangle{\pgfqpoint{1.000000in}{0.330000in}}{\pgfqpoint{6.200000in}{2.310000in}}%
\pgfusepath{clip}%
\pgfsetrectcap%
\pgfsetroundjoin%
\pgfsetlinewidth{1.505625pt}%
\definecolor{currentstroke}{rgb}{0.890196,0.466667,0.760784}%
\pgfsetstrokecolor{currentstroke}%
\pgfsetdash{}{0pt}%
\pgfpathmoveto{\pgfqpoint{1.281818in}{2.535000in}}%
\pgfpathlineto{\pgfqpoint{1.650866in}{2.338475in}}%
\pgfpathlineto{\pgfqpoint{1.852165in}{2.233769in}}%
\pgfpathlineto{\pgfqpoint{2.019913in}{2.148882in}}%
\pgfpathlineto{\pgfqpoint{2.187662in}{2.066822in}}%
\pgfpathlineto{\pgfqpoint{2.321861in}{2.003632in}}%
\pgfpathlineto{\pgfqpoint{2.456061in}{1.942977in}}%
\pgfpathlineto{\pgfqpoint{2.590260in}{1.885177in}}%
\pgfpathlineto{\pgfqpoint{2.690909in}{1.843893in}}%
\pgfpathlineto{\pgfqpoint{2.791558in}{1.804530in}}%
\pgfpathlineto{\pgfqpoint{2.892208in}{1.767223in}}%
\pgfpathlineto{\pgfqpoint{2.992857in}{1.732106in}}%
\pgfpathlineto{\pgfqpoint{3.093506in}{1.699312in}}%
\pgfpathlineto{\pgfqpoint{3.194156in}{1.668976in}}%
\pgfpathlineto{\pgfqpoint{3.294805in}{1.641231in}}%
\pgfpathlineto{\pgfqpoint{3.395455in}{1.616210in}}%
\pgfpathlineto{\pgfqpoint{3.496104in}{1.594046in}}%
\pgfpathlineto{\pgfqpoint{3.596753in}{1.574872in}}%
\pgfpathlineto{\pgfqpoint{3.697403in}{1.558821in}}%
\pgfpathlineto{\pgfqpoint{3.798052in}{1.555815in}}%
\pgfpathlineto{\pgfqpoint{3.898701in}{1.555966in}}%
\pgfpathlineto{\pgfqpoint{3.999351in}{1.559176in}}%
\pgfpathlineto{\pgfqpoint{4.100000in}{1.565348in}}%
\pgfpathlineto{\pgfqpoint{4.200649in}{1.574384in}}%
\pgfpathlineto{\pgfqpoint{4.301299in}{1.586185in}}%
\pgfpathlineto{\pgfqpoint{4.401948in}{1.600652in}}%
\pgfpathlineto{\pgfqpoint{4.502597in}{1.617686in}}%
\pgfpathlineto{\pgfqpoint{4.603247in}{1.637189in}}%
\pgfpathlineto{\pgfqpoint{4.703896in}{1.659061in}}%
\pgfpathlineto{\pgfqpoint{4.804545in}{1.683201in}}%
\pgfpathlineto{\pgfqpoint{4.905195in}{1.709511in}}%
\pgfpathlineto{\pgfqpoint{5.005844in}{1.737890in}}%
\pgfpathlineto{\pgfqpoint{5.106494in}{1.768237in}}%
\pgfpathlineto{\pgfqpoint{5.207143in}{1.800451in}}%
\pgfpathlineto{\pgfqpoint{5.341342in}{1.846135in}}%
\pgfpathlineto{\pgfqpoint{5.475541in}{1.894718in}}%
\pgfpathlineto{\pgfqpoint{5.609740in}{1.945962in}}%
\pgfpathlineto{\pgfqpoint{5.743939in}{1.999624in}}%
\pgfpathlineto{\pgfqpoint{5.878139in}{2.055461in}}%
\pgfpathlineto{\pgfqpoint{6.045887in}{2.127946in}}%
\pgfpathlineto{\pgfqpoint{6.213636in}{2.202972in}}%
\pgfpathlineto{\pgfqpoint{6.414935in}{2.295686in}}%
\pgfpathlineto{\pgfqpoint{6.649784in}{2.406496in}}%
\pgfpathlineto{\pgfqpoint{6.918182in}{2.535000in}}%
\pgfpathlineto{\pgfqpoint{6.918182in}{2.535000in}}%
\pgfusepath{stroke}%
\end{pgfscope}%
\begin{pgfscope}%
\pgfpathrectangle{\pgfqpoint{1.000000in}{0.330000in}}{\pgfqpoint{6.200000in}{2.310000in}}%
\pgfusepath{clip}%
\pgfsetrectcap%
\pgfsetroundjoin%
\pgfsetlinewidth{1.505625pt}%
\definecolor{currentstroke}{rgb}{0.498039,0.498039,0.498039}%
\pgfsetstrokecolor{currentstroke}%
\pgfsetdash{}{0pt}%
\pgfpathmoveto{\pgfqpoint{1.281818in}{2.535000in}}%
\pgfpathlineto{\pgfqpoint{1.583766in}{2.190136in}}%
\pgfpathlineto{\pgfqpoint{1.751515in}{2.001499in}}%
\pgfpathlineto{\pgfqpoint{1.885714in}{1.853279in}}%
\pgfpathlineto{\pgfqpoint{2.019913in}{1.708202in}}%
\pgfpathlineto{\pgfqpoint{2.154113in}{1.566961in}}%
\pgfpathlineto{\pgfqpoint{2.254762in}{1.463964in}}%
\pgfpathlineto{\pgfqpoint{2.355411in}{1.363805in}}%
\pgfpathlineto{\pgfqpoint{2.456061in}{1.266776in}}%
\pgfpathlineto{\pgfqpoint{2.556710in}{1.173168in}}%
\pgfpathlineto{\pgfqpoint{2.657359in}{1.083273in}}%
\pgfpathlineto{\pgfqpoint{2.758009in}{0.997380in}}%
\pgfpathlineto{\pgfqpoint{2.825108in}{0.942486in}}%
\pgfpathlineto{\pgfqpoint{2.892208in}{0.889586in}}%
\pgfpathlineto{\pgfqpoint{2.959307in}{0.838766in}}%
\pgfpathlineto{\pgfqpoint{3.026407in}{0.790113in}}%
\pgfpathlineto{\pgfqpoint{3.093506in}{0.743711in}}%
\pgfpathlineto{\pgfqpoint{3.160606in}{0.699648in}}%
\pgfpathlineto{\pgfqpoint{3.227706in}{0.658008in}}%
\pgfpathlineto{\pgfqpoint{3.294805in}{0.618878in}}%
\pgfpathlineto{\pgfqpoint{3.361905in}{0.582344in}}%
\pgfpathlineto{\pgfqpoint{3.429004in}{0.548490in}}%
\pgfpathlineto{\pgfqpoint{3.496104in}{0.517403in}}%
\pgfpathlineto{\pgfqpoint{3.563203in}{0.489168in}}%
\pgfpathlineto{\pgfqpoint{3.630303in}{0.463871in}}%
\pgfpathlineto{\pgfqpoint{3.697403in}{0.441598in}}%
\pgfpathlineto{\pgfqpoint{3.764502in}{0.436776in}}%
\pgfpathlineto{\pgfqpoint{3.831602in}{0.435000in}}%
\pgfpathlineto{\pgfqpoint{3.898701in}{0.436205in}}%
\pgfpathlineto{\pgfqpoint{3.965801in}{0.440328in}}%
\pgfpathlineto{\pgfqpoint{4.032900in}{0.447305in}}%
\pgfpathlineto{\pgfqpoint{4.100000in}{0.457072in}}%
\pgfpathlineto{\pgfqpoint{4.167100in}{0.469565in}}%
\pgfpathlineto{\pgfqpoint{4.234199in}{0.484721in}}%
\pgfpathlineto{\pgfqpoint{4.301299in}{0.502475in}}%
\pgfpathlineto{\pgfqpoint{4.368398in}{0.522764in}}%
\pgfpathlineto{\pgfqpoint{4.435498in}{0.545522in}}%
\pgfpathlineto{\pgfqpoint{4.502597in}{0.570688in}}%
\pgfpathlineto{\pgfqpoint{4.569697in}{0.598195in}}%
\pgfpathlineto{\pgfqpoint{4.636797in}{0.627980in}}%
\pgfpathlineto{\pgfqpoint{4.703896in}{0.659979in}}%
\pgfpathlineto{\pgfqpoint{4.770996in}{0.694128in}}%
\pgfpathlineto{\pgfqpoint{4.838095in}{0.730362in}}%
\pgfpathlineto{\pgfqpoint{4.905195in}{0.768617in}}%
\pgfpathlineto{\pgfqpoint{4.972294in}{0.808828in}}%
\pgfpathlineto{\pgfqpoint{5.039394in}{0.850931in}}%
\pgfpathlineto{\pgfqpoint{5.106494in}{0.894862in}}%
\pgfpathlineto{\pgfqpoint{5.207143in}{0.964044in}}%
\pgfpathlineto{\pgfqpoint{5.307792in}{1.036974in}}%
\pgfpathlineto{\pgfqpoint{5.408442in}{1.113436in}}%
\pgfpathlineto{\pgfqpoint{5.509091in}{1.193210in}}%
\pgfpathlineto{\pgfqpoint{5.609740in}{1.276077in}}%
\pgfpathlineto{\pgfqpoint{5.710390in}{1.361819in}}%
\pgfpathlineto{\pgfqpoint{5.811039in}{1.450217in}}%
\pgfpathlineto{\pgfqpoint{5.911688in}{1.541052in}}%
\pgfpathlineto{\pgfqpoint{6.045887in}{1.665575in}}%
\pgfpathlineto{\pgfqpoint{6.180087in}{1.793519in}}%
\pgfpathlineto{\pgfqpoint{6.314286in}{1.924361in}}%
\pgfpathlineto{\pgfqpoint{6.482035in}{2.091194in}}%
\pgfpathlineto{\pgfqpoint{6.683333in}{2.294857in}}%
\pgfpathlineto{\pgfqpoint{6.918182in}{2.535000in}}%
\pgfpathlineto{\pgfqpoint{6.918182in}{2.535000in}}%
\pgfusepath{stroke}%
\end{pgfscope}%
\begin{pgfscope}%
\pgfsetroundcap%
\pgfsetroundjoin%
\pgfsetlinewidth{1.003750pt}%
\definecolor{currentstroke}{rgb}{0.000000,0.000000,0.000000}%
\pgfsetstrokecolor{currentstroke}%
\pgfsetdash{}{0pt}%
\pgfpathmoveto{\pgfqpoint{4.298568in}{0.435000in}}%
\pgfpathquadraticcurveto{\pgfqpoint{4.078979in}{0.435000in}}{\pgfqpoint{3.859391in}{0.435000in}}%
\pgfusepath{stroke}%
\end{pgfscope}%
\begin{pgfscope}%
\pgfsetbuttcap%
\pgfsetmiterjoin%
\definecolor{currentfill}{rgb}{0.800000,0.800000,0.800000}%
\pgfsetfillcolor{currentfill}%
\pgfsetlinewidth{1.003750pt}%
\definecolor{currentstroke}{rgb}{0.000000,0.000000,0.000000}%
\pgfsetstrokecolor{currentstroke}%
\pgfsetdash{}{0pt}%
\pgfpathmoveto{\pgfqpoint{4.356293in}{0.338549in}}%
\pgfpathcurveto{\pgfqpoint{4.391015in}{0.303827in}}{\pgfqpoint{5.220491in}{0.303827in}}{\pgfqpoint{5.255213in}{0.338549in}}%
\pgfpathcurveto{\pgfqpoint{5.289935in}{0.373272in}}{\pgfqpoint{5.289935in}{0.496728in}}{\pgfqpoint{5.255213in}{0.531451in}}%
\pgfpathcurveto{\pgfqpoint{5.220491in}{0.566173in}}{\pgfqpoint{4.391015in}{0.566173in}}{\pgfqpoint{4.356293in}{0.531451in}}%
\pgfpathcurveto{\pgfqpoint{4.321570in}{0.496728in}}{\pgfqpoint{4.321570in}{0.373272in}}{\pgfqpoint{4.356293in}{0.338549in}}%
\pgfpathclose%
\pgfusepath{stroke,fill}%
\end{pgfscope}%
\begin{pgfscope}%
\definecolor{textcolor}{rgb}{0.000000,0.000000,0.000000}%
\pgfsetstrokecolor{textcolor}%
\pgfsetfillcolor{textcolor}%
\pgftext[x=5.220491in,y=0.435000in,right,]{\color{textcolor}\rmfamily\fontsize{10.000000}{12.000000}\selectfont \(\displaystyle \Delta =\) -0.0 inch}%
\end{pgfscope}%
\begin{pgfscope}%
\pgfsetbuttcap%
\pgfsetmiterjoin%
\definecolor{currentfill}{rgb}{0.800000,0.800000,0.800000}%
\pgfsetfillcolor{currentfill}%
\pgfsetlinewidth{1.003750pt}%
\definecolor{currentstroke}{rgb}{0.000000,0.000000,0.000000}%
\pgfsetstrokecolor{currentstroke}%
\pgfsetdash{}{0pt}%
\pgfpathmoveto{\pgfqpoint{0.965278in}{0.358599in}}%
\pgfpathcurveto{\pgfqpoint{1.000000in}{0.323877in}}{\pgfqpoint{2.720682in}{0.323877in}}{\pgfqpoint{2.755404in}{0.358599in}}%
\pgfpathcurveto{\pgfqpoint{2.790127in}{0.393321in}}{\pgfqpoint{2.790127in}{0.668784in}}{\pgfqpoint{2.755404in}{0.703506in}}%
\pgfpathcurveto{\pgfqpoint{2.720682in}{0.738228in}}{\pgfqpoint{1.000000in}{0.738228in}}{\pgfqpoint{0.965278in}{0.703506in}}%
\pgfpathcurveto{\pgfqpoint{0.930556in}{0.668784in}}{\pgfqpoint{0.930556in}{0.393321in}}{\pgfqpoint{0.965278in}{0.358599in}}%
\pgfpathclose%
\pgfusepath{stroke,fill}%
\end{pgfscope}%
\begin{pgfscope}%
\definecolor{textcolor}{rgb}{0.000000,0.000000,0.000000}%
\pgfsetstrokecolor{textcolor}%
\pgfsetfillcolor{textcolor}%
\pgftext[x=1.000000in, y=0.580049in, left, base]{\color{textcolor}\rmfamily\fontsize{10.000000}{12.000000}\selectfont Max combo: 1.0D + 1.0L0}%
\end{pgfscope}%
\begin{pgfscope}%
\definecolor{textcolor}{rgb}{0.000000,0.000000,0.000000}%
\pgfsetstrokecolor{textcolor}%
\pgfsetfillcolor{textcolor}%
\pgftext[x=1.000000in, y=0.428043in, left, base]{\color{textcolor}\rmfamily\fontsize{10.000000}{12.000000}\selectfont ASCE7-16 Sec. 2.4.1 (LC 2)}%
\end{pgfscope}%
\begin{pgfscope}%
\pgfsetroundcap%
\pgfsetroundjoin%
\pgfsetlinewidth{1.003750pt}%
\definecolor{currentstroke}{rgb}{0.000000,0.000000,0.000000}%
\pgfsetstrokecolor{currentstroke}%
\pgfsetdash{}{0pt}%
\pgfpathmoveto{\pgfqpoint{4.298568in}{1.294194in}}%
\pgfpathquadraticcurveto{\pgfqpoint{4.078979in}{1.294194in}}{\pgfqpoint{3.859391in}{1.294194in}}%
\pgfusepath{stroke}%
\end{pgfscope}%
\begin{pgfscope}%
\pgfsetbuttcap%
\pgfsetmiterjoin%
\definecolor{currentfill}{rgb}{0.800000,0.800000,0.800000}%
\pgfsetfillcolor{currentfill}%
\pgfsetlinewidth{1.003750pt}%
\definecolor{currentstroke}{rgb}{0.000000,0.000000,0.000000}%
\pgfsetstrokecolor{currentstroke}%
\pgfsetdash{}{0pt}%
\pgfpathmoveto{\pgfqpoint{4.356293in}{1.197743in}}%
\pgfpathcurveto{\pgfqpoint{4.391015in}{1.163021in}}{\pgfqpoint{5.220491in}{1.163021in}}{\pgfqpoint{5.255213in}{1.197743in}}%
\pgfpathcurveto{\pgfqpoint{5.289935in}{1.232465in}}{\pgfqpoint{5.289935in}{1.355922in}}{\pgfqpoint{5.255213in}{1.390644in}}%
\pgfpathcurveto{\pgfqpoint{5.220491in}{1.425367in}}{\pgfqpoint{4.391015in}{1.425367in}}{\pgfqpoint{4.356293in}{1.390644in}}%
\pgfpathcurveto{\pgfqpoint{4.321570in}{1.355922in}}{\pgfqpoint{4.321570in}{1.232465in}}{\pgfqpoint{4.356293in}{1.197743in}}%
\pgfpathclose%
\pgfusepath{stroke,fill}%
\end{pgfscope}%
\begin{pgfscope}%
\definecolor{textcolor}{rgb}{0.000000,0.000000,0.000000}%
\pgfsetstrokecolor{textcolor}%
\pgfsetfillcolor{textcolor}%
\pgftext[x=5.220491in,y=1.294194in,right,]{\color{textcolor}\rmfamily\fontsize{10.000000}{12.000000}\selectfont \(\displaystyle \Delta =\) -0.0 inch}%
\end{pgfscope}%
\begin{pgfscope}%
\pgfsetbuttcap%
\pgfsetmiterjoin%
\definecolor{currentfill}{rgb}{0.800000,0.800000,0.800000}%
\pgfsetfillcolor{currentfill}%
\pgfsetlinewidth{1.003750pt}%
\definecolor{currentstroke}{rgb}{0.000000,0.000000,0.000000}%
\pgfsetstrokecolor{currentstroke}%
\pgfsetdash{}{0pt}%
\pgfpathmoveto{\pgfqpoint{0.965278in}{0.375574in}}%
\pgfpathcurveto{\pgfqpoint{1.000000in}{0.340852in}}{\pgfqpoint{2.390820in}{0.340852in}}{\pgfqpoint{2.425542in}{0.375574in}}%
\pgfpathcurveto{\pgfqpoint{2.460265in}{0.410297in}}{\pgfqpoint{2.460265in}{0.676500in}}{\pgfqpoint{2.425542in}{0.711222in}}%
\pgfpathcurveto{\pgfqpoint{2.390820in}{0.745944in}}{\pgfqpoint{1.000000in}{0.745944in}}{\pgfqpoint{0.965278in}{0.711222in}}%
\pgfpathcurveto{\pgfqpoint{0.930556in}{0.676500in}}{\pgfqpoint{0.930556in}{0.410297in}}{\pgfqpoint{0.965278in}{0.375574in}}%
\pgfpathclose%
\pgfusepath{stroke,fill}%
\end{pgfscope}%
\begin{pgfscope}%
\definecolor{textcolor}{rgb}{0.000000,0.000000,0.000000}%
\pgfsetstrokecolor{textcolor}%
\pgfsetfillcolor{textcolor}%
\pgftext[x=1.000000in, y=0.580049in, left, base]{\color{textcolor}\rmfamily\fontsize{10.000000}{12.000000}\selectfont Max combo: 1.0L0}%
\end{pgfscope}%
\begin{pgfscope}%
\definecolor{textcolor}{rgb}{0.000000,0.000000,0.000000}%
\pgfsetstrokecolor{textcolor}%
\pgfsetfillcolor{textcolor}%
\pgftext[x=1.000000in, y=0.437303in, left, base]{\color{textcolor}\rmfamily\fontsize{10.000000}{12.000000}\selectfont L only deflection check}%
\end{pgfscope}%
\end{pgfpicture}%
\makeatother%
\endgroup%

\end{center}
\caption{Deflection Envelope}
\end{figure}
Tl Deflection Check: 
$\Delta_{max} = -0.02 {\color{darkBlue}{\mathbf{ \; in}}} = \cfrac{L}{3830} < \cfrac{L}{1.0}  \;  \mathbf{(OK)}$\\
\bigbreak
Ll Deflection Check: 
$\Delta_{max} = -0.01 {\color{darkBlue}{\mathbf{ \; in}}} = \cfrac{L}{6482} < \cfrac{L}{1.0}  \;  \mathbf{(OK)}$\\
\bigbreak
\vspace{-30pt}
%	---------------------------------- REACTIONS ---------------------------------
\section{Reactions}
The following is a summary of service-level reactions at each support:
\begin{table}[ht]
\caption{Reactions at Supports}
\centering
\begin{tabular}{l l l l }
\hline
Loc. & Type & D & L0\\
\hline
0 {\color{darkBlue}{\textbf{ft}}} & Shear & 1.0 {\color{darkBlue}{\textbf{kip}}} & 1.4 {\color{darkBlue}{\textbf{kip}}}\\ 
7 {\color{darkBlue}{\textbf{ft}}} & Shear & 0.7 {\color{darkBlue}{\textbf{kip}}} & 1.0 {\color{darkBlue}{\textbf{kip}}}\\ 
\hline
\end{tabular}
\end{table}
\end{document}