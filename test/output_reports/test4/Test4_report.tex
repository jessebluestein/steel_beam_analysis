\documentclass[12pt, fleqn]{article}
\usepackage{pgfplots}
\usepackage{bm}
\usepackage{marginnote}
\usepackage{wallpaper}
\usepackage{lastpage}
\usepackage[left=1.3cm,right=2.0cm,top=1.8cm,bottom=5.0cm,marginparwidth=3.4cm]{geometry}
\usepackage{amsmath}
\usepackage{amssymb}
\usepackage{xcolor}
\usepackage{enumitem}
\usepackage{float}
\usepackage{textgreek}
\usepackage{textcomp}
\usepackage{fancyhdr}
\usepackage{graphicx}
\usepackage{pstricks}
\usepackage{subfigure}
\usepackage{caption}
\captionsetup{justification=centering,labelfont=bf, belowskip=12pt,aboveskip=12pt}
\usepackage{textcomp}
\setlength{\headheight}{70pt}
\setlength{\textfloatsep}{12pt}
\setlength{\intextsep}{0pt}
\pagestyle{fancy}\fancyhf{}
\renewcommand{\headrulewidth}{0pt}
\definecolor{darkBlue}{cmyk}{.80, .32, 0, 0}
\setlength{\parindent}{0cm}
\newcommand{\tab}{\hspace*{2em}}
\newcommand\BackgroundStructure{
\setlength{\unitlength}{1mm}
\setlength\fboxsep{0mm}
\setlength\fboxrule{0.5mm}
\put(10, 20pr){\fcolorbox{black}{gray!5}{\framebox(155,247){}}}
\put(165, 20){\fcolorbox{black}{gray!10}{\framebox(37,247){}}}
\put(10, 262){\fcolorbox{black}{white!10}{\framebox(192, 25){}}}
\put(175, 263){\includegraphics{}}}
\setlength{\abovedisplayskip}{0pt}
\setlength{\belowdisplayskip}{0pt}
%	----------------------------------- HEADER -----------------------------------
\fancyhead[L]{\begin{tabular}{l l | l l}
\textbf{Member:} & {Test4} & \textbf{Firm:} & {ABC Company} \\
\textbf{Project:} & {123 Maple Street, San Francisco CA} & \textbf{Engineer:} & {Jesse} \\
\textbf{Level:} & {2} & \textbf{Checker:} & {Joey}  \\
\textbf{Date:} & {2021-05-23} & \textbf{Page:} & \thepage\\
\end{tabular}}
%	---------------------------- APPLIED LOADS SECTION ---------------------------
\begin{document}
\begin{center}
\textbf{\LARGE W10x22 Design Report}
\end{center}
\section{Applied Loading}
\vspace{-30pt}
\begin{figure}[H]
\begin{center}
%% Creator: Matplotlib, PGF backend
%%
%% To include the figure in your LaTeX document, write
%%   \input{<filename>.pgf}
%%
%% Make sure the required packages are loaded in your preamble
%%   \usepackage{pgf}
%%
%% Figures using additional raster images can only be included by \input if
%% they are in the same directory as the main LaTeX file. For loading figures
%% from other directories you can use the `import` package
%%   \usepackage{import}
%%
%% and then include the figures with
%%   \import{<path to file>}{<filename>.pgf}
%%
%% Matplotlib used the following preamble
%%
\begingroup%
\makeatletter%
\begin{pgfpicture}%
\pgfpathrectangle{\pgfpointorigin}{\pgfqpoint{8.000000in}{3.000000in}}%
\pgfusepath{use as bounding box, clip}%
\begin{pgfscope}%
\pgfpathrectangle{\pgfqpoint{1.000000in}{0.825864in}}{\pgfqpoint{6.200000in}{1.814136in}}%
\pgfusepath{clip}%
\pgfsetbuttcap%
\pgfsetmiterjoin%
\definecolor{currentfill}{rgb}{0.121569,0.466667,0.705882}%
\pgfsetfillcolor{currentfill}%
\pgfsetfillopacity{0.400000}%
\pgfsetlinewidth{1.003750pt}%
\definecolor{currentstroke}{rgb}{0.121569,0.466667,0.705882}%
\pgfsetstrokecolor{currentstroke}%
\pgfsetstrokeopacity{0.400000}%
\pgfsetdash{}{0pt}%
\pgfpathmoveto{\pgfqpoint{1.281818in}{0.908325in}}%
\pgfpathlineto{\pgfqpoint{1.281818in}{1.187695in}}%
\pgfpathlineto{\pgfqpoint{6.692727in}{1.187695in}}%
\pgfpathlineto{\pgfqpoint{6.692727in}{0.908325in}}%
\pgfpathclose%
\pgfusepath{stroke,fill}%
\end{pgfscope}%
\begin{pgfscope}%
\pgfpathrectangle{\pgfqpoint{1.000000in}{0.825864in}}{\pgfqpoint{6.200000in}{1.814136in}}%
\pgfusepath{clip}%
\pgfsetbuttcap%
\pgfsetmiterjoin%
\definecolor{currentfill}{rgb}{1.000000,0.498039,0.054902}%
\pgfsetfillcolor{currentfill}%
\pgfsetfillopacity{0.400000}%
\pgfsetlinewidth{1.003750pt}%
\definecolor{currentstroke}{rgb}{1.000000,0.498039,0.054902}%
\pgfsetstrokecolor{currentstroke}%
\pgfsetstrokeopacity{0.400000}%
\pgfsetdash{}{0pt}%
\pgfpathmoveto{\pgfqpoint{1.281818in}{1.187695in}}%
\pgfpathlineto{\pgfqpoint{1.281818in}{1.872617in}}%
\pgfpathlineto{\pgfqpoint{4.980220in}{2.557539in}}%
\pgfpathlineto{\pgfqpoint{4.980220in}{1.187695in}}%
\pgfpathclose%
\pgfusepath{stroke,fill}%
\end{pgfscope}%
\begin{pgfscope}%
\pgfpathrectangle{\pgfqpoint{1.000000in}{0.825864in}}{\pgfqpoint{6.200000in}{1.814136in}}%
\pgfusepath{clip}%
\pgfsetbuttcap%
\pgfsetmiterjoin%
\definecolor{currentfill}{rgb}{0.172549,0.627451,0.172549}%
\pgfsetfillcolor{currentfill}%
\pgfsetfillopacity{0.400000}%
\pgfsetlinewidth{1.003750pt}%
\definecolor{currentstroke}{rgb}{0.172549,0.627451,0.172549}%
\pgfsetstrokecolor{currentstroke}%
\pgfsetstrokeopacity{0.400000}%
\pgfsetdash{}{0pt}%
\pgfpathmoveto{\pgfqpoint{5.790909in}{1.187695in}}%
\pgfpathlineto{\pgfqpoint{5.790909in}{1.697333in}}%
\pgfpathlineto{\pgfqpoint{6.918182in}{1.697333in}}%
\pgfpathlineto{\pgfqpoint{6.918182in}{1.187695in}}%
\pgfpathclose%
\pgfusepath{stroke,fill}%
\end{pgfscope}%
\begin{pgfscope}%
\pgfsetbuttcap%
\pgfsetmiterjoin%
\definecolor{currentfill}{rgb}{0.800000,0.800000,0.800000}%
\pgfsetfillcolor{currentfill}%
\pgfsetlinewidth{1.003750pt}%
\definecolor{currentstroke}{rgb}{0.000000,0.000000,0.000000}%
\pgfsetstrokecolor{currentstroke}%
\pgfsetdash{}{0pt}%
\pgfpathmoveto{\pgfqpoint{3.871242in}{0.951559in}}%
\pgfpathcurveto{\pgfqpoint{3.905965in}{0.916837in}}{\pgfqpoint{4.068581in}{0.916837in}}{\pgfqpoint{4.103303in}{0.951559in}}%
\pgfpathcurveto{\pgfqpoint{4.138025in}{0.986282in}}{\pgfqpoint{4.138025in}{1.109738in}}{\pgfqpoint{4.103303in}{1.144460in}}%
\pgfpathcurveto{\pgfqpoint{4.068581in}{1.179183in}}{\pgfqpoint{3.905965in}{1.179183in}}{\pgfqpoint{3.871242in}{1.144460in}}%
\pgfpathcurveto{\pgfqpoint{3.836520in}{1.109738in}}{\pgfqpoint{3.836520in}{0.986282in}}{\pgfqpoint{3.871242in}{0.951559in}}%
\pgfpathclose%
\pgfusepath{stroke,fill}%
\end{pgfscope}%
\begin{pgfscope}%
\definecolor{textcolor}{rgb}{0.000000,0.000000,0.000000}%
\pgfsetstrokecolor{textcolor}%
\pgfsetfillcolor{textcolor}%
\pgftext[x=3.987273in,y=1.048010in,,]{\color{textcolor}\rmfamily\fontsize{10.000000}{12.000000}\selectfont w\textsubscript{1}}%
\end{pgfscope}%
\begin{pgfscope}%
\pgfsetbuttcap%
\pgfsetmiterjoin%
\definecolor{currentfill}{rgb}{0.800000,0.800000,0.800000}%
\pgfsetfillcolor{currentfill}%
\pgfsetlinewidth{1.003750pt}%
\definecolor{currentstroke}{rgb}{0.000000,0.000000,0.000000}%
\pgfsetstrokecolor{currentstroke}%
\pgfsetdash{}{0pt}%
\pgfpathmoveto{\pgfqpoint{3.014989in}{1.776167in}}%
\pgfpathcurveto{\pgfqpoint{3.049711in}{1.741444in}}{\pgfqpoint{3.212327in}{1.741444in}}{\pgfqpoint{3.247049in}{1.776167in}}%
\pgfpathcurveto{\pgfqpoint{3.281771in}{1.810889in}}{\pgfqpoint{3.281771in}{1.934346in}}{\pgfqpoint{3.247049in}{1.969068in}}%
\pgfpathcurveto{\pgfqpoint{3.212327in}{2.003790in}}{\pgfqpoint{3.049711in}{2.003790in}}{\pgfqpoint{3.014989in}{1.969068in}}%
\pgfpathcurveto{\pgfqpoint{2.980266in}{1.934346in}}{\pgfqpoint{2.980266in}{1.810889in}}{\pgfqpoint{3.014989in}{1.776167in}}%
\pgfpathclose%
\pgfusepath{stroke,fill}%
\end{pgfscope}%
\begin{pgfscope}%
\definecolor{textcolor}{rgb}{0.000000,0.000000,0.000000}%
\pgfsetstrokecolor{textcolor}%
\pgfsetfillcolor{textcolor}%
\pgftext[x=3.131019in,y=1.872617in,,]{\color{textcolor}\rmfamily\fontsize{10.000000}{12.000000}\selectfont w\textsubscript{2}}%
\end{pgfscope}%
\begin{pgfscope}%
\pgfsetbuttcap%
\pgfsetmiterjoin%
\definecolor{currentfill}{rgb}{0.800000,0.800000,0.800000}%
\pgfsetfillcolor{currentfill}%
\pgfsetlinewidth{1.003750pt}%
\definecolor{currentstroke}{rgb}{0.000000,0.000000,0.000000}%
\pgfsetstrokecolor{currentstroke}%
\pgfsetdash{}{0pt}%
\pgfpathmoveto{\pgfqpoint{6.238515in}{1.346064in}}%
\pgfpathcurveto{\pgfqpoint{6.273237in}{1.311341in}}{\pgfqpoint{6.435854in}{1.311341in}}{\pgfqpoint{6.470576in}{1.346064in}}%
\pgfpathcurveto{\pgfqpoint{6.505298in}{1.380786in}}{\pgfqpoint{6.505298in}{1.504242in}}{\pgfqpoint{6.470576in}{1.538965in}}%
\pgfpathcurveto{\pgfqpoint{6.435854in}{1.573687in}}{\pgfqpoint{6.273237in}{1.573687in}}{\pgfqpoint{6.238515in}{1.538965in}}%
\pgfpathcurveto{\pgfqpoint{6.203793in}{1.504242in}}{\pgfqpoint{6.203793in}{1.380786in}}{\pgfqpoint{6.238515in}{1.346064in}}%
\pgfpathclose%
\pgfusepath{stroke,fill}%
\end{pgfscope}%
\begin{pgfscope}%
\definecolor{textcolor}{rgb}{0.000000,0.000000,0.000000}%
\pgfsetstrokecolor{textcolor}%
\pgfsetfillcolor{textcolor}%
\pgftext[x=6.354545in,y=1.442514in,,]{\color{textcolor}\rmfamily\fontsize{10.000000}{12.000000}\selectfont w\textsubscript{3}}%
\end{pgfscope}%
\begin{pgfscope}%
\pgfsetroundcap%
\pgfsetroundjoin%
\pgfsetlinewidth{1.003750pt}%
\definecolor{currentstroke}{rgb}{0.000000,0.000000,0.000000}%
\pgfsetstrokecolor{currentstroke}%
\pgfsetdash{}{0pt}%
\pgfpathmoveto{\pgfqpoint{1.281818in}{2.158833in}}%
\pgfpathquadraticcurveto{\pgfqpoint{1.281818in}{1.547451in}}{\pgfqpoint{1.281818in}{0.944962in}}%
\pgfusepath{stroke}%
\end{pgfscope}%
\begin{pgfscope}%
\pgfsetroundcap%
\pgfsetroundjoin%
\pgfsetlinewidth{1.003750pt}%
\definecolor{currentstroke}{rgb}{0.000000,0.000000,0.000000}%
\pgfsetstrokecolor{currentstroke}%
\pgfsetdash{}{0pt}%
\pgfpathmoveto{\pgfqpoint{1.365152in}{1.011629in}}%
\pgfpathlineto{\pgfqpoint{1.281818in}{0.944962in}}%
\pgfpathlineto{\pgfqpoint{1.198485in}{1.011629in}}%
\pgfusepath{stroke}%
\end{pgfscope}%
\begin{pgfscope}%
\pgfsetbuttcap%
\pgfsetmiterjoin%
\definecolor{currentfill}{rgb}{0.800000,0.800000,0.800000}%
\pgfsetfillcolor{currentfill}%
\pgfsetlinewidth{1.003750pt}%
\definecolor{currentstroke}{rgb}{0.000000,0.000000,0.000000}%
\pgfsetstrokecolor{currentstroke}%
\pgfsetdash{}{0pt}%
\pgfpathmoveto{\pgfqpoint{1.150857in}{2.223140in}}%
\pgfpathcurveto{\pgfqpoint{1.192524in}{2.181473in}}{\pgfqpoint{1.371112in}{2.181473in}}{\pgfqpoint{1.412779in}{2.223140in}}%
\pgfpathcurveto{\pgfqpoint{1.454446in}{2.264806in}}{\pgfqpoint{1.454446in}{2.412954in}}{\pgfqpoint{1.412779in}{2.454621in}}%
\pgfpathcurveto{\pgfqpoint{1.371112in}{2.496287in}}{\pgfqpoint{1.192524in}{2.496287in}}{\pgfqpoint{1.150857in}{2.454621in}}%
\pgfpathcurveto{\pgfqpoint{1.109191in}{2.412954in}}{\pgfqpoint{1.109191in}{2.264806in}}{\pgfqpoint{1.150857in}{2.223140in}}%
\pgfpathclose%
\pgfusepath{stroke,fill}%
\end{pgfscope}%
\begin{pgfscope}%
\definecolor{textcolor}{rgb}{0.000000,0.000000,0.000000}%
\pgfsetstrokecolor{textcolor}%
\pgfsetfillcolor{textcolor}%
\pgftext[x=1.281818in,y=2.297213in,,base]{\color{textcolor}\rmfamily\fontsize{12.000000}{14.400000}\selectfont \(\displaystyle P_x\)}%
\end{pgfscope}%
\begin{pgfscope}%
\pgfsetroundcap%
\pgfsetroundjoin%
\pgfsetlinewidth{1.003750pt}%
\definecolor{currentstroke}{rgb}{0.000000,0.000000,0.000000}%
\pgfsetstrokecolor{currentstroke}%
\pgfsetdash{}{0pt}%
\pgfpathmoveto{\pgfqpoint{2.409091in}{2.158833in}}%
\pgfpathquadraticcurveto{\pgfqpoint{2.409091in}{1.547451in}}{\pgfqpoint{2.409091in}{0.944962in}}%
\pgfusepath{stroke}%
\end{pgfscope}%
\begin{pgfscope}%
\pgfsetroundcap%
\pgfsetroundjoin%
\pgfsetlinewidth{1.003750pt}%
\definecolor{currentstroke}{rgb}{0.000000,0.000000,0.000000}%
\pgfsetstrokecolor{currentstroke}%
\pgfsetdash{}{0pt}%
\pgfpathmoveto{\pgfqpoint{2.492424in}{1.011629in}}%
\pgfpathlineto{\pgfqpoint{2.409091in}{0.944962in}}%
\pgfpathlineto{\pgfqpoint{2.325758in}{1.011629in}}%
\pgfusepath{stroke}%
\end{pgfscope}%
\begin{pgfscope}%
\pgfsetbuttcap%
\pgfsetmiterjoin%
\definecolor{currentfill}{rgb}{0.800000,0.800000,0.800000}%
\pgfsetfillcolor{currentfill}%
\pgfsetlinewidth{1.003750pt}%
\definecolor{currentstroke}{rgb}{0.000000,0.000000,0.000000}%
\pgfsetstrokecolor{currentstroke}%
\pgfsetdash{}{0pt}%
\pgfpathmoveto{\pgfqpoint{2.278130in}{2.223140in}}%
\pgfpathcurveto{\pgfqpoint{2.319797in}{2.181473in}}{\pgfqpoint{2.498385in}{2.181473in}}{\pgfqpoint{2.540052in}{2.223140in}}%
\pgfpathcurveto{\pgfqpoint{2.581718in}{2.264806in}}{\pgfqpoint{2.581718in}{2.412954in}}{\pgfqpoint{2.540052in}{2.454621in}}%
\pgfpathcurveto{\pgfqpoint{2.498385in}{2.496287in}}{\pgfqpoint{2.319797in}{2.496287in}}{\pgfqpoint{2.278130in}{2.454621in}}%
\pgfpathcurveto{\pgfqpoint{2.236464in}{2.412954in}}{\pgfqpoint{2.236464in}{2.264806in}}{\pgfqpoint{2.278130in}{2.223140in}}%
\pgfpathclose%
\pgfusepath{stroke,fill}%
\end{pgfscope}%
\begin{pgfscope}%
\definecolor{textcolor}{rgb}{0.000000,0.000000,0.000000}%
\pgfsetstrokecolor{textcolor}%
\pgfsetfillcolor{textcolor}%
\pgftext[x=2.409091in,y=2.297213in,,base]{\color{textcolor}\rmfamily\fontsize{12.000000}{14.400000}\selectfont \(\displaystyle P_x\)}%
\end{pgfscope}%
\begin{pgfscope}%
\pgfpathrectangle{\pgfqpoint{1.000000in}{0.330000in}}{\pgfqpoint{6.200000in}{0.604712in}}%
\pgfusepath{clip}%
\pgfsetbuttcap%
\pgfsetroundjoin%
\definecolor{currentfill}{rgb}{1.000000,0.000000,0.000000}%
\pgfsetfillcolor{currentfill}%
\pgfsetlinewidth{1.003750pt}%
\definecolor{currentstroke}{rgb}{1.000000,0.000000,0.000000}%
\pgfsetstrokecolor{currentstroke}%
\pgfsetdash{}{0pt}%
\pgfsys@defobject{currentmarker}{\pgfqpoint{-0.098209in}{-0.098209in}}{\pgfqpoint{0.098209in}{0.098209in}}{%
\pgfpathmoveto{\pgfqpoint{0.000000in}{0.098209in}}%
\pgfpathlineto{\pgfqpoint{-0.098209in}{-0.098209in}}%
\pgfpathlineto{\pgfqpoint{0.098209in}{-0.098209in}}%
\pgfpathclose%
\pgfusepath{stroke,fill}%
}%
\begin{pgfscope}%
\pgfsys@transformshift{1.281818in}{0.357487in}%
\pgfsys@useobject{currentmarker}{}%
\end{pgfscope}%
\begin{pgfscope}%
\pgfsys@transformshift{4.663636in}{0.357487in}%
\pgfsys@useobject{currentmarker}{}%
\end{pgfscope}%
\end{pgfscope}%
\begin{pgfscope}%
\pgfpathrectangle{\pgfqpoint{1.000000in}{0.330000in}}{\pgfqpoint{6.200000in}{0.604712in}}%
\pgfusepath{clip}%
\pgfsetbuttcap%
\pgfsetroundjoin%
\definecolor{currentfill}{rgb}{0.000000,0.000000,1.000000}%
\pgfsetfillcolor{currentfill}%
\pgfsetlinewidth{1.003750pt}%
\definecolor{currentstroke}{rgb}{0.000000,0.000000,1.000000}%
\pgfsetstrokecolor{currentstroke}%
\pgfsetdash{}{0pt}%
\pgfsys@defobject{currentmarker}{\pgfqpoint{-0.098209in}{-0.098209in}}{\pgfqpoint{0.098209in}{0.098209in}}{%
\pgfpathmoveto{\pgfqpoint{-0.098209in}{-0.098209in}}%
\pgfpathlineto{\pgfqpoint{0.098209in}{-0.098209in}}%
\pgfpathlineto{\pgfqpoint{0.098209in}{0.098209in}}%
\pgfpathlineto{\pgfqpoint{-0.098209in}{0.098209in}}%
\pgfpathclose%
\pgfusepath{stroke,fill}%
}%
\begin{pgfscope}%
\pgfsys@transformshift{6.918182in}{0.357487in}%
\pgfsys@useobject{currentmarker}{}%
\end{pgfscope}%
\end{pgfscope}%
\begin{pgfscope}%
\pgfpathrectangle{\pgfqpoint{1.000000in}{0.330000in}}{\pgfqpoint{6.200000in}{0.604712in}}%
\pgfusepath{clip}%
\pgfsetrectcap%
\pgfsetroundjoin%
\pgfsetlinewidth{1.003750pt}%
\definecolor{currentstroke}{rgb}{0.000000,0.000000,0.000000}%
\pgfsetstrokecolor{currentstroke}%
\pgfsetdash{}{0pt}%
\pgfpathmoveto{\pgfqpoint{1.281818in}{0.907225in}}%
\pgfpathlineto{\pgfqpoint{6.918182in}{0.907225in}}%
\pgfusepath{stroke}%
\end{pgfscope}%
\begin{pgfscope}%
\pgfpathrectangle{\pgfqpoint{1.000000in}{0.330000in}}{\pgfqpoint{6.200000in}{0.604712in}}%
\pgfusepath{clip}%
\pgfsetrectcap%
\pgfsetroundjoin%
\pgfsetlinewidth{1.003750pt}%
\definecolor{currentstroke}{rgb}{0.000000,0.000000,0.000000}%
\pgfsetstrokecolor{currentstroke}%
\pgfsetdash{}{0pt}%
\pgfpathmoveto{\pgfqpoint{1.281818in}{0.871825in}}%
\pgfpathlineto{\pgfqpoint{6.918182in}{0.871825in}}%
\pgfusepath{stroke}%
\end{pgfscope}%
\begin{pgfscope}%
\pgfpathrectangle{\pgfqpoint{1.000000in}{0.330000in}}{\pgfqpoint{6.200000in}{0.604712in}}%
\pgfusepath{clip}%
\pgfsetrectcap%
\pgfsetroundjoin%
\pgfsetlinewidth{1.003750pt}%
\definecolor{currentstroke}{rgb}{0.000000,0.000000,0.000000}%
\pgfsetstrokecolor{currentstroke}%
\pgfsetdash{}{0pt}%
\pgfpathmoveto{\pgfqpoint{1.281818in}{0.482427in}}%
\pgfpathlineto{\pgfqpoint{6.918182in}{0.482427in}}%
\pgfusepath{stroke}%
\end{pgfscope}%
\begin{pgfscope}%
\pgfpathrectangle{\pgfqpoint{1.000000in}{0.330000in}}{\pgfqpoint{6.200000in}{0.604712in}}%
\pgfusepath{clip}%
\pgfsetrectcap%
\pgfsetroundjoin%
\pgfsetlinewidth{1.003750pt}%
\definecolor{currentstroke}{rgb}{0.000000,0.000000,0.000000}%
\pgfsetstrokecolor{currentstroke}%
\pgfsetdash{}{0pt}%
\pgfpathmoveto{\pgfqpoint{1.281818in}{0.517827in}}%
\pgfpathlineto{\pgfqpoint{6.918182in}{0.517827in}}%
\pgfusepath{stroke}%
\end{pgfscope}%
\begin{pgfscope}%
\pgfpathrectangle{\pgfqpoint{1.000000in}{0.330000in}}{\pgfqpoint{6.200000in}{0.604712in}}%
\pgfusepath{clip}%
\pgfsetrectcap%
\pgfsetroundjoin%
\pgfsetlinewidth{1.003750pt}%
\definecolor{currentstroke}{rgb}{0.000000,0.000000,0.000000}%
\pgfsetstrokecolor{currentstroke}%
\pgfsetdash{}{0pt}%
\pgfpathmoveto{\pgfqpoint{1.281818in}{0.482427in}}%
\pgfpathlineto{\pgfqpoint{1.281818in}{0.907225in}}%
\pgfusepath{stroke}%
\end{pgfscope}%
\begin{pgfscope}%
\pgfpathrectangle{\pgfqpoint{1.000000in}{0.330000in}}{\pgfqpoint{6.200000in}{0.604712in}}%
\pgfusepath{clip}%
\pgfsetrectcap%
\pgfsetroundjoin%
\pgfsetlinewidth{1.003750pt}%
\definecolor{currentstroke}{rgb}{0.000000,0.000000,0.000000}%
\pgfsetstrokecolor{currentstroke}%
\pgfsetdash{}{0pt}%
\pgfpathmoveto{\pgfqpoint{6.918182in}{0.482427in}}%
\pgfpathlineto{\pgfqpoint{6.918182in}{0.907225in}}%
\pgfusepath{stroke}%
\end{pgfscope}%
\begin{pgfscope}%
\pgfsetbuttcap%
\pgfsetmiterjoin%
\definecolor{currentfill}{rgb}{0.800000,0.800000,0.800000}%
\pgfsetfillcolor{currentfill}%
\pgfsetlinewidth{1.003750pt}%
\definecolor{currentstroke}{rgb}{0.000000,0.000000,0.000000}%
\pgfsetstrokecolor{currentstroke}%
\pgfsetdash{}{0pt}%
\pgfpathmoveto{\pgfqpoint{3.818363in}{0.598376in}}%
\pgfpathcurveto{\pgfqpoint{3.853086in}{0.563653in}}{\pgfqpoint{4.346914in}{0.563653in}}{\pgfqpoint{4.381637in}{0.598376in}}%
\pgfpathcurveto{\pgfqpoint{4.416359in}{0.633098in}}{\pgfqpoint{4.416359in}{0.756555in}}{\pgfqpoint{4.381637in}{0.791277in}}%
\pgfpathcurveto{\pgfqpoint{4.346914in}{0.825999in}}{\pgfqpoint{3.853086in}{0.825999in}}{\pgfqpoint{3.818363in}{0.791277in}}%
\pgfpathcurveto{\pgfqpoint{3.783641in}{0.756555in}}{\pgfqpoint{3.783641in}{0.633098in}}{\pgfqpoint{3.818363in}{0.598376in}}%
\pgfpathclose%
\pgfusepath{stroke,fill}%
\end{pgfscope}%
\begin{pgfscope}%
\definecolor{textcolor}{rgb}{0.000000,0.000000,0.000000}%
\pgfsetstrokecolor{textcolor}%
\pgfsetfillcolor{textcolor}%
\pgftext[x=4.100000in,y=0.694826in,,]{\color{textcolor}\rmfamily\fontsize{10.000000}{12.000000}\selectfont W10x22}%
\end{pgfscope}%
\begin{pgfscope}%
\pgfsetbuttcap%
\pgfsetmiterjoin%
\definecolor{currentfill}{rgb}{0.800000,0.800000,0.800000}%
\pgfsetfillcolor{currentfill}%
\pgfsetlinewidth{1.003750pt}%
\definecolor{currentstroke}{rgb}{0.000000,0.000000,0.000000}%
\pgfsetstrokecolor{currentstroke}%
\pgfsetdash{}{0pt}%
\pgfpathmoveto{\pgfqpoint{2.488544in}{0.177743in}}%
\pgfpathcurveto{\pgfqpoint{2.523266in}{0.143020in}}{\pgfqpoint{3.422188in}{0.143020in}}{\pgfqpoint{3.456910in}{0.177743in}}%
\pgfpathcurveto{\pgfqpoint{3.491633in}{0.212465in}}{\pgfqpoint{3.491633in}{0.335922in}}{\pgfqpoint{3.456910in}{0.370644in}}%
\pgfpathcurveto{\pgfqpoint{3.422188in}{0.405366in}}{\pgfqpoint{2.523266in}{0.405366in}}{\pgfqpoint{2.488544in}{0.370644in}}%
\pgfpathcurveto{\pgfqpoint{2.453822in}{0.335922in}}{\pgfqpoint{2.453822in}{0.212465in}}{\pgfqpoint{2.488544in}{0.177743in}}%
\pgfpathclose%
\pgfusepath{stroke,fill}%
\end{pgfscope}%
\begin{pgfscope}%
\definecolor{textcolor}{rgb}{0.000000,0.000000,0.000000}%
\pgfsetstrokecolor{textcolor}%
\pgfsetfillcolor{textcolor}%
\pgftext[x=2.972727in,y=0.274193in,,]{\color{textcolor}\rmfamily\fontsize{10.000000}{12.000000}\selectfont Span 0 = 15 ft}%
\end{pgfscope}%
\begin{pgfscope}%
\pgfsetbuttcap%
\pgfsetmiterjoin%
\definecolor{currentfill}{rgb}{0.800000,0.800000,0.800000}%
\pgfsetfillcolor{currentfill}%
\pgfsetlinewidth{1.003750pt}%
\definecolor{currentstroke}{rgb}{0.000000,0.000000,0.000000}%
\pgfsetstrokecolor{currentstroke}%
\pgfsetdash{}{0pt}%
\pgfpathmoveto{\pgfqpoint{5.306726in}{0.177743in}}%
\pgfpathcurveto{\pgfqpoint{5.341448in}{0.143020in}}{\pgfqpoint{6.240370in}{0.143020in}}{\pgfqpoint{6.275092in}{0.177743in}}%
\pgfpathcurveto{\pgfqpoint{6.309814in}{0.212465in}}{\pgfqpoint{6.309814in}{0.335922in}}{\pgfqpoint{6.275092in}{0.370644in}}%
\pgfpathcurveto{\pgfqpoint{6.240370in}{0.405366in}}{\pgfqpoint{5.341448in}{0.405366in}}{\pgfqpoint{5.306726in}{0.370644in}}%
\pgfpathcurveto{\pgfqpoint{5.272004in}{0.335922in}}{\pgfqpoint{5.272004in}{0.212465in}}{\pgfqpoint{5.306726in}{0.177743in}}%
\pgfpathclose%
\pgfusepath{stroke,fill}%
\end{pgfscope}%
\begin{pgfscope}%
\definecolor{textcolor}{rgb}{0.000000,0.000000,0.000000}%
\pgfsetstrokecolor{textcolor}%
\pgfsetfillcolor{textcolor}%
\pgftext[x=5.790909in,y=0.274193in,,]{\color{textcolor}\rmfamily\fontsize{10.000000}{12.000000}\selectfont Span 1 = 10 ft}%
\end{pgfscope}%
\end{pgfpicture}%
\makeatother%
\endgroup%

\end{center}
\vspace{-18pt}
\caption{Applied Loads}
\end{figure}
The following distributed loads are applied to the beam. The program can handle all possible mass and force units in both metric and imperial systems simultaneously. Loads are plotted to scale according to their relative magnitudes. A "positive" load is defined as a load acting in the direction of gravity.
\begin{table}[ht]
\caption{Applied Distributed Loads}
\centering
\begin{tabular}{l l l l l l l}
\hline
Load & Start Loc. & Start Mag. & End Loc. & End Mag. & Type & Description\\
\hline
w\textsubscript{1} & 0 {\color{darkBlue}{\textbf{ft}}} & 2.0 {\color{darkBlue}{\textbf{kN/m}}} & 8 {\color{darkBlue}{\textbf{yd}}} & 2.0 {\color{darkBlue}{\textbf{kN/m}}} & D & Force line load (mixed units)\\
w\textsubscript{2} & 0 {\color{darkBlue}{\textbf{ft}}} & 500.0 {\color{darkBlue}{\textbf{kg/m}}} & 5 {\color{darkBlue}{\textbf{m}}} & 1000.0 {\color{darkBlue}{\textbf{kg/m}}} & L & Mass line load (mixed units)\\
w\textsubscript{3} & 20 {\color{darkBlue}{\textbf{ft}}} & 0.25 {\color{darkBlue}{\textbf{klf}}} & 25 {\color{darkBlue}{\textbf{ft}}} & 0.25 {\color{darkBlue}{\textbf{klf}}} & L & KLF line load\\
\hline
\end{tabular}
\end{table}
\begin{table}[ht]
\caption{Applied Point Loads}
\centering
\begin{tabular}{l l l l l l}
\hline
Load & Loc. & Shear & Type & Description \\
\hline
P\textsubscript{1} & 0 {\color{darkBlue}{\textbf{ft}}} & 2 {\color{darkBlue}{\textbf{kip}}} & D & Post load\\
P\textsubscript{2} & 0 {\color{darkBlue}{\textbf{ft}}} & 4 {\color{darkBlue}{\textbf{kip}}} & L & Post load\\
P\textsubscript{3} & 5 {\color{darkBlue}{\textbf{ft}}} & 2 {\color{darkBlue}{\textbf{kip}}} & D & Post load\\
P\textsubscript{4} & 5 {\color{darkBlue}{\textbf{ft}}} & 4 {\color{darkBlue}{\textbf{kip}}} & L & Post load\\
\hline
\end{tabular}
\end{table}
%	-------------------------------- LOAD COMBOS	--------------------------------
\section{Load Combinations}
The following load combinations are used for the design. Duplicate load combinations are not listed and only loads that are used on the beam are included in the load combinations (i.e. If soil load is not included as a load type in any of the applied loads, then "H" loads will not be included in the listed load combinations). S\textsubscript{DS} is input as 1.0 and \textOmega\textsubscript{0} is input as 2.5 for use in seismic load combinations. Any load designated as a pattern load is applied to spans in all possible permutations to create the most extreme loading condition. Numbers after a load indicate the span over which the pattern load is applied (i.e. L0 indicates that live load is applied only on the first span).
\begin{table}[H]
\caption{Strength (LRFD) Load Combinations}
\centering
\begin{tabular}{l l l}
\hline
Load Combo & Loads and Factors & Reference\\
\hline
LC 1 & 0.7D & ASCE7-16 \S2.3.6 (LC 7)\\
LC 2 & 1.2D + 1.0L1 & ASCE7-16 \S2.3.1 (LC 3)\\
LC 3 & 0.9D & ASCE7-16 \S2.3.1 (LC 5)\\
LC 4 & 1.2D & ASCE7-16 \S2.3.1 (LC 3)\\
LC 5 & 1.2D + 1.6L1 & ASCE7-16 \S2.3.1 (LC 2)\\
LC 6 & 1.2D + 1.0L0 + 1.0L1 & ASCE7-16 \S2.3.1 (LC 3)\\
LC 7 & 1.2D + 1.6L0 + 1.6L1 & ASCE7-16 \S2.3.1 (LC 2)\\
LC 8 & 1.4D + 0.5L0 + 0.5L1 & ASCE7-16 \S2.3.6 (LC 6)\\
LC 9 & 1.2D + 1.0L0 & ASCE7-16 \S2.3.1 (LC 3)\\
LC 10 & 1.4D + 0.5L0 & ASCE7-16 \S2.3.6 (LC 6)\\
LC 11 & 1.4D + 0.5L1 & ASCE7-16 \S2.3.6 (LC 6)\\
LC 12 & 1.2D + 1.6L0 & ASCE7-16 \S2.3.1 (LC 2)\\
LC 13 & 1.4D & ASCE7-16 \S2.3.1 (LC 1)\\
\hline
\end{tabular}
\end{table}
\begin{table}[H]
\caption{Deflection (ASD) Load Combinations}
\centering
\begin{tabular}{l l l}
\hline
Load Combo & Loads and Factors & Reference\\
\hline
LC 1 & 1.0D + 1.0L0 + 1.0L1 & ASCE7-16 \S2.4.1 (LC 2)\\
LC 2 & 1.0D + 1.0L0 & ASCE7-16 \S2.4.1 (LC 2)\\
LC 3 & 1.0L1 & L only deflection check\\
LC 4 & 1.14D & ASCE7-16 \S2.4.5 (LC 8)\\
LC 5 & 1.1D + 0.75L1 & ASCE7-16 \S2.4.5 (LC 8)\\
LC 6 & 1.0D & ASCE7-16 \S2.4.1 (LC 1)\\
LC 7 & 1.0L0 + 1.0L1 & L only deflection check\\
LC 8 & 1.0D + 0.75L0 + 0.75L1 & ASCE7-16 \S2.4.1 (LC 4)\\
LC 9 & 1.0D + 0.75L0 & ASCE7-16 \S2.4.1 (LC 4)\\
LC 10 & 1.0D + 1.0L1 & ASCE7-16 \S2.4.1 (LC 2)\\
LC 11 & 1.1D + 0.75L0 & ASCE7-16 \S2.4.5 (LC 8)\\
LC 12 & 1.1D + 0.75L0 + 0.75L1 & ASCE7-16 \S2.4.5 (LC 8)\\
LC 13 & 0.6D & ASCE7-16 \S2.4.1 (LC 7)\\
LC 14 & 1.0L0 & L only deflection check\\
LC 15 & 1.0D + 0.75L1 & ASCE7-16 \S2.4.1 (LC 4)\\
LC 16 & 0.4D & ASCE7-16 \S2.4.5 (LC 9)\\
\hline
\end{tabular}
\end{table}
%	---------------------- SECTIONAL & MATERIAL PROPERTIES -----------------------
\section{Sectional and Material Properties}
The following are sectional and material properties used for analysis \textbf{(W10x22, Grade A992)}:
\begin{table}[ht]
\caption{Sectional and Material Properties}
\vspace{-10pt}
\centering
\begin{tabular}{lll}
\centering
\begin{tabular}[t]{ll}
\cline{1-2}
Property & Value \\
\cline{1-2}
A\textsubscript{w} & 2.4 {\color{darkBlue}{\textbf{{\color{darkBlue}{\textbf{in}}}\textsuperscript{2}}}} \\
C\textsubscript{w} & 275 {\color{darkBlue}{\textbf{{\color{darkBlue}{\textbf{in}}}\textsuperscript{6}}}} \\
F\textsubscript{u} & 65 {\color{darkBlue}{\textbf{ksi}}} \\
F\textsubscript{y} & 50 {\color{darkBlue}{\textbf{ksi}}} \\
I\textsubscript{x} & 118 {\color{darkBlue}{\textbf{{\color{darkBlue}{\textbf{in}}}\textsuperscript{4}}}} \\
I\textsubscript{y} & 11.4 {\color{darkBlue}{\textbf{{\color{darkBlue}{\textbf{in}}}\textsuperscript{4}}}} \\
\cline{1-2}
\end{tabular}
&
\begin{tabular}[t]{ll}
\cline{1-2}
Property & Value \\
\cline{1-2}
S\textsubscript{x} & 23.2 {\color{darkBlue}{\textbf{{\color{darkBlue}{\textbf{in}}}\textsuperscript{3}}}} \\
S\textsubscript{y} & 4.0 {\color{darkBlue}{\textbf{{\color{darkBlue}{\textbf{in}}}\textsuperscript{3}}}} \\
Z\textsubscript{x} & 26 {\color{darkBlue}{\textbf{{\color{darkBlue}{\textbf{in}}}\textsuperscript{3}}}} \\
Z\textsubscript{y} & 6.1 {\color{darkBlue}{\textbf{{\color{darkBlue}{\textbf{in}}}\textsuperscript{3}}}} \\
r\textsubscript{x} & 4.3 {\color{darkBlue}{\textbf{in}}} \\
r\textsubscript{y} & 1.3 {\color{darkBlue}{\textbf{in}}} \\
\cline{1-2}
\end{tabular}
&
\begin{tabular}[t]{ll}
\cline{1-2}
Property & Value \\
\cline{1-2}
b\textsubscript{f} & 5.8 {\color{darkBlue}{\textbf{in}}} \\
t\textsubscript{f} & 0.4 {\color{darkBlue}{\textbf{in}}} \\
t\textsubscript{w} & 0.2 {\color{darkBlue}{\textbf{in}}} \\
h\textsubscript{0} & 9.8 {\color{darkBlue}{\textbf{in}}} \\
U.W. & 490 {\color{darkBlue}{\textbf{pcf}}} \\
\cline{1-2}
\end{tabular}
\end{tabular}
\end{table}
%	-------------------------------- BENDING CHECK -------------------------------
\section{Bending Check}
\begin{figure}[H]
\begin{center}
%% Creator: Matplotlib, PGF backend
%%
%% To include the figure in your LaTeX document, write
%%   \input{<filename>.pgf}
%%
%% Make sure the required packages are loaded in your preamble
%%   \usepackage{pgf}
%%
%% Figures using additional raster images can only be included by \input if
%% they are in the same directory as the main LaTeX file. For loading figures
%% from other directories you can use the `import` package
%%   \usepackage{import}
%%
%% and then include the figures with
%%   \import{<path to file>}{<filename>.pgf}
%%
%% Matplotlib used the following preamble
%%
\begingroup%
\makeatletter%
\begin{pgfpicture}%
\pgfpathrectangle{\pgfpointorigin}{\pgfqpoint{8.000000in}{3.000000in}}%
\pgfusepath{use as bounding box, clip}%
\begin{pgfscope}%
\pgfsetbuttcap%
\pgfsetmiterjoin%
\definecolor{currentfill}{rgb}{1.000000,1.000000,1.000000}%
\pgfsetfillcolor{currentfill}%
\pgfsetlinewidth{0.000000pt}%
\definecolor{currentstroke}{rgb}{1.000000,1.000000,1.000000}%
\pgfsetstrokecolor{currentstroke}%
\pgfsetdash{}{0pt}%
\pgfpathmoveto{\pgfqpoint{0.000000in}{0.000000in}}%
\pgfpathlineto{\pgfqpoint{8.000000in}{0.000000in}}%
\pgfpathlineto{\pgfqpoint{8.000000in}{3.000000in}}%
\pgfpathlineto{\pgfqpoint{0.000000in}{3.000000in}}%
\pgfpathclose%
\pgfusepath{fill}%
\end{pgfscope}%
\begin{pgfscope}%
\pgfsetbuttcap%
\pgfsetmiterjoin%
\definecolor{currentfill}{rgb}{1.000000,1.000000,1.000000}%
\pgfsetfillcolor{currentfill}%
\pgfsetlinewidth{0.000000pt}%
\definecolor{currentstroke}{rgb}{0.000000,0.000000,0.000000}%
\pgfsetstrokecolor{currentstroke}%
\pgfsetstrokeopacity{0.000000}%
\pgfsetdash{}{0pt}%
\pgfpathmoveto{\pgfqpoint{1.000000in}{0.330000in}}%
\pgfpathlineto{\pgfqpoint{7.200000in}{0.330000in}}%
\pgfpathlineto{\pgfqpoint{7.200000in}{2.640000in}}%
\pgfpathlineto{\pgfqpoint{1.000000in}{2.640000in}}%
\pgfpathclose%
\pgfusepath{fill}%
\end{pgfscope}%
\begin{pgfscope}%
\pgfpathrectangle{\pgfqpoint{1.000000in}{0.330000in}}{\pgfqpoint{6.200000in}{2.310000in}}%
\pgfusepath{clip}%
\pgfsetbuttcap%
\pgfsetroundjoin%
\pgfsetlinewidth{0.803000pt}%
\definecolor{currentstroke}{rgb}{0.000000,0.000000,0.000000}%
\pgfsetstrokecolor{currentstroke}%
\pgfsetdash{{0.800000pt}{1.320000pt}}{0.000000pt}%
\pgfpathmoveto{\pgfqpoint{1.281818in}{0.330000in}}%
\pgfpathlineto{\pgfqpoint{1.281818in}{2.640000in}}%
\pgfusepath{stroke}%
\end{pgfscope}%
\begin{pgfscope}%
\pgfsetbuttcap%
\pgfsetroundjoin%
\definecolor{currentfill}{rgb}{0.000000,0.000000,0.000000}%
\pgfsetfillcolor{currentfill}%
\pgfsetlinewidth{0.803000pt}%
\definecolor{currentstroke}{rgb}{0.000000,0.000000,0.000000}%
\pgfsetstrokecolor{currentstroke}%
\pgfsetdash{}{0pt}%
\pgfsys@defobject{currentmarker}{\pgfqpoint{0.000000in}{-0.048611in}}{\pgfqpoint{0.000000in}{0.000000in}}{%
\pgfpathmoveto{\pgfqpoint{0.000000in}{0.000000in}}%
\pgfpathlineto{\pgfqpoint{0.000000in}{-0.048611in}}%
\pgfusepath{stroke,fill}%
}%
\begin{pgfscope}%
\pgfsys@transformshift{1.281818in}{0.330000in}%
\pgfsys@useobject{currentmarker}{}%
\end{pgfscope}%
\end{pgfscope}%
\begin{pgfscope}%
\pgfsetbuttcap%
\pgfsetroundjoin%
\definecolor{currentfill}{rgb}{0.000000,0.000000,0.000000}%
\pgfsetfillcolor{currentfill}%
\pgfsetlinewidth{0.803000pt}%
\definecolor{currentstroke}{rgb}{0.000000,0.000000,0.000000}%
\pgfsetstrokecolor{currentstroke}%
\pgfsetdash{}{0pt}%
\pgfsys@defobject{currentmarker}{\pgfqpoint{0.000000in}{0.000000in}}{\pgfqpoint{0.000000in}{0.048611in}}{%
\pgfpathmoveto{\pgfqpoint{0.000000in}{0.000000in}}%
\pgfpathlineto{\pgfqpoint{0.000000in}{0.048611in}}%
\pgfusepath{stroke,fill}%
}%
\begin{pgfscope}%
\pgfsys@transformshift{1.281818in}{2.640000in}%
\pgfsys@useobject{currentmarker}{}%
\end{pgfscope}%
\end{pgfscope}%
\begin{pgfscope}%
\definecolor{textcolor}{rgb}{0.000000,0.000000,0.000000}%
\pgfsetstrokecolor{textcolor}%
\pgfsetfillcolor{textcolor}%
\pgftext[x=1.281818in,y=0.232778in,,top]{\color{textcolor}\rmfamily\fontsize{10.000000}{12.000000}\selectfont \(\displaystyle {0}\)}%
\end{pgfscope}%
\begin{pgfscope}%
\pgfpathrectangle{\pgfqpoint{1.000000in}{0.330000in}}{\pgfqpoint{6.200000in}{2.310000in}}%
\pgfusepath{clip}%
\pgfsetbuttcap%
\pgfsetroundjoin%
\pgfsetlinewidth{0.803000pt}%
\definecolor{currentstroke}{rgb}{0.000000,0.000000,0.000000}%
\pgfsetstrokecolor{currentstroke}%
\pgfsetdash{{0.800000pt}{1.320000pt}}{0.000000pt}%
\pgfpathmoveto{\pgfqpoint{2.409091in}{0.330000in}}%
\pgfpathlineto{\pgfqpoint{2.409091in}{2.640000in}}%
\pgfusepath{stroke}%
\end{pgfscope}%
\begin{pgfscope}%
\pgfsetbuttcap%
\pgfsetroundjoin%
\definecolor{currentfill}{rgb}{0.000000,0.000000,0.000000}%
\pgfsetfillcolor{currentfill}%
\pgfsetlinewidth{0.803000pt}%
\definecolor{currentstroke}{rgb}{0.000000,0.000000,0.000000}%
\pgfsetstrokecolor{currentstroke}%
\pgfsetdash{}{0pt}%
\pgfsys@defobject{currentmarker}{\pgfqpoint{0.000000in}{-0.048611in}}{\pgfqpoint{0.000000in}{0.000000in}}{%
\pgfpathmoveto{\pgfqpoint{0.000000in}{0.000000in}}%
\pgfpathlineto{\pgfqpoint{0.000000in}{-0.048611in}}%
\pgfusepath{stroke,fill}%
}%
\begin{pgfscope}%
\pgfsys@transformshift{2.409091in}{0.330000in}%
\pgfsys@useobject{currentmarker}{}%
\end{pgfscope}%
\end{pgfscope}%
\begin{pgfscope}%
\pgfsetbuttcap%
\pgfsetroundjoin%
\definecolor{currentfill}{rgb}{0.000000,0.000000,0.000000}%
\pgfsetfillcolor{currentfill}%
\pgfsetlinewidth{0.803000pt}%
\definecolor{currentstroke}{rgb}{0.000000,0.000000,0.000000}%
\pgfsetstrokecolor{currentstroke}%
\pgfsetdash{}{0pt}%
\pgfsys@defobject{currentmarker}{\pgfqpoint{0.000000in}{0.000000in}}{\pgfqpoint{0.000000in}{0.048611in}}{%
\pgfpathmoveto{\pgfqpoint{0.000000in}{0.000000in}}%
\pgfpathlineto{\pgfqpoint{0.000000in}{0.048611in}}%
\pgfusepath{stroke,fill}%
}%
\begin{pgfscope}%
\pgfsys@transformshift{2.409091in}{2.640000in}%
\pgfsys@useobject{currentmarker}{}%
\end{pgfscope}%
\end{pgfscope}%
\begin{pgfscope}%
\definecolor{textcolor}{rgb}{0.000000,0.000000,0.000000}%
\pgfsetstrokecolor{textcolor}%
\pgfsetfillcolor{textcolor}%
\pgftext[x=2.409091in,y=0.232778in,,top]{\color{textcolor}\rmfamily\fontsize{10.000000}{12.000000}\selectfont \(\displaystyle {5}\)}%
\end{pgfscope}%
\begin{pgfscope}%
\pgfpathrectangle{\pgfqpoint{1.000000in}{0.330000in}}{\pgfqpoint{6.200000in}{2.310000in}}%
\pgfusepath{clip}%
\pgfsetbuttcap%
\pgfsetroundjoin%
\pgfsetlinewidth{0.803000pt}%
\definecolor{currentstroke}{rgb}{0.000000,0.000000,0.000000}%
\pgfsetstrokecolor{currentstroke}%
\pgfsetdash{{0.800000pt}{1.320000pt}}{0.000000pt}%
\pgfpathmoveto{\pgfqpoint{3.536364in}{0.330000in}}%
\pgfpathlineto{\pgfqpoint{3.536364in}{2.640000in}}%
\pgfusepath{stroke}%
\end{pgfscope}%
\begin{pgfscope}%
\pgfsetbuttcap%
\pgfsetroundjoin%
\definecolor{currentfill}{rgb}{0.000000,0.000000,0.000000}%
\pgfsetfillcolor{currentfill}%
\pgfsetlinewidth{0.803000pt}%
\definecolor{currentstroke}{rgb}{0.000000,0.000000,0.000000}%
\pgfsetstrokecolor{currentstroke}%
\pgfsetdash{}{0pt}%
\pgfsys@defobject{currentmarker}{\pgfqpoint{0.000000in}{-0.048611in}}{\pgfqpoint{0.000000in}{0.000000in}}{%
\pgfpathmoveto{\pgfqpoint{0.000000in}{0.000000in}}%
\pgfpathlineto{\pgfqpoint{0.000000in}{-0.048611in}}%
\pgfusepath{stroke,fill}%
}%
\begin{pgfscope}%
\pgfsys@transformshift{3.536364in}{0.330000in}%
\pgfsys@useobject{currentmarker}{}%
\end{pgfscope}%
\end{pgfscope}%
\begin{pgfscope}%
\pgfsetbuttcap%
\pgfsetroundjoin%
\definecolor{currentfill}{rgb}{0.000000,0.000000,0.000000}%
\pgfsetfillcolor{currentfill}%
\pgfsetlinewidth{0.803000pt}%
\definecolor{currentstroke}{rgb}{0.000000,0.000000,0.000000}%
\pgfsetstrokecolor{currentstroke}%
\pgfsetdash{}{0pt}%
\pgfsys@defobject{currentmarker}{\pgfqpoint{0.000000in}{0.000000in}}{\pgfqpoint{0.000000in}{0.048611in}}{%
\pgfpathmoveto{\pgfqpoint{0.000000in}{0.000000in}}%
\pgfpathlineto{\pgfqpoint{0.000000in}{0.048611in}}%
\pgfusepath{stroke,fill}%
}%
\begin{pgfscope}%
\pgfsys@transformshift{3.536364in}{2.640000in}%
\pgfsys@useobject{currentmarker}{}%
\end{pgfscope}%
\end{pgfscope}%
\begin{pgfscope}%
\definecolor{textcolor}{rgb}{0.000000,0.000000,0.000000}%
\pgfsetstrokecolor{textcolor}%
\pgfsetfillcolor{textcolor}%
\pgftext[x=3.536364in,y=0.232778in,,top]{\color{textcolor}\rmfamily\fontsize{10.000000}{12.000000}\selectfont \(\displaystyle {10}\)}%
\end{pgfscope}%
\begin{pgfscope}%
\pgfpathrectangle{\pgfqpoint{1.000000in}{0.330000in}}{\pgfqpoint{6.200000in}{2.310000in}}%
\pgfusepath{clip}%
\pgfsetbuttcap%
\pgfsetroundjoin%
\pgfsetlinewidth{0.803000pt}%
\definecolor{currentstroke}{rgb}{0.000000,0.000000,0.000000}%
\pgfsetstrokecolor{currentstroke}%
\pgfsetdash{{0.800000pt}{1.320000pt}}{0.000000pt}%
\pgfpathmoveto{\pgfqpoint{4.663636in}{0.330000in}}%
\pgfpathlineto{\pgfqpoint{4.663636in}{2.640000in}}%
\pgfusepath{stroke}%
\end{pgfscope}%
\begin{pgfscope}%
\pgfsetbuttcap%
\pgfsetroundjoin%
\definecolor{currentfill}{rgb}{0.000000,0.000000,0.000000}%
\pgfsetfillcolor{currentfill}%
\pgfsetlinewidth{0.803000pt}%
\definecolor{currentstroke}{rgb}{0.000000,0.000000,0.000000}%
\pgfsetstrokecolor{currentstroke}%
\pgfsetdash{}{0pt}%
\pgfsys@defobject{currentmarker}{\pgfqpoint{0.000000in}{-0.048611in}}{\pgfqpoint{0.000000in}{0.000000in}}{%
\pgfpathmoveto{\pgfqpoint{0.000000in}{0.000000in}}%
\pgfpathlineto{\pgfqpoint{0.000000in}{-0.048611in}}%
\pgfusepath{stroke,fill}%
}%
\begin{pgfscope}%
\pgfsys@transformshift{4.663636in}{0.330000in}%
\pgfsys@useobject{currentmarker}{}%
\end{pgfscope}%
\end{pgfscope}%
\begin{pgfscope}%
\pgfsetbuttcap%
\pgfsetroundjoin%
\definecolor{currentfill}{rgb}{0.000000,0.000000,0.000000}%
\pgfsetfillcolor{currentfill}%
\pgfsetlinewidth{0.803000pt}%
\definecolor{currentstroke}{rgb}{0.000000,0.000000,0.000000}%
\pgfsetstrokecolor{currentstroke}%
\pgfsetdash{}{0pt}%
\pgfsys@defobject{currentmarker}{\pgfqpoint{0.000000in}{0.000000in}}{\pgfqpoint{0.000000in}{0.048611in}}{%
\pgfpathmoveto{\pgfqpoint{0.000000in}{0.000000in}}%
\pgfpathlineto{\pgfqpoint{0.000000in}{0.048611in}}%
\pgfusepath{stroke,fill}%
}%
\begin{pgfscope}%
\pgfsys@transformshift{4.663636in}{2.640000in}%
\pgfsys@useobject{currentmarker}{}%
\end{pgfscope}%
\end{pgfscope}%
\begin{pgfscope}%
\definecolor{textcolor}{rgb}{0.000000,0.000000,0.000000}%
\pgfsetstrokecolor{textcolor}%
\pgfsetfillcolor{textcolor}%
\pgftext[x=4.663636in,y=0.232778in,,top]{\color{textcolor}\rmfamily\fontsize{10.000000}{12.000000}\selectfont \(\displaystyle {15}\)}%
\end{pgfscope}%
\begin{pgfscope}%
\pgfpathrectangle{\pgfqpoint{1.000000in}{0.330000in}}{\pgfqpoint{6.200000in}{2.310000in}}%
\pgfusepath{clip}%
\pgfsetbuttcap%
\pgfsetroundjoin%
\pgfsetlinewidth{0.803000pt}%
\definecolor{currentstroke}{rgb}{0.000000,0.000000,0.000000}%
\pgfsetstrokecolor{currentstroke}%
\pgfsetdash{{0.800000pt}{1.320000pt}}{0.000000pt}%
\pgfpathmoveto{\pgfqpoint{5.790909in}{0.330000in}}%
\pgfpathlineto{\pgfqpoint{5.790909in}{2.640000in}}%
\pgfusepath{stroke}%
\end{pgfscope}%
\begin{pgfscope}%
\pgfsetbuttcap%
\pgfsetroundjoin%
\definecolor{currentfill}{rgb}{0.000000,0.000000,0.000000}%
\pgfsetfillcolor{currentfill}%
\pgfsetlinewidth{0.803000pt}%
\definecolor{currentstroke}{rgb}{0.000000,0.000000,0.000000}%
\pgfsetstrokecolor{currentstroke}%
\pgfsetdash{}{0pt}%
\pgfsys@defobject{currentmarker}{\pgfqpoint{0.000000in}{-0.048611in}}{\pgfqpoint{0.000000in}{0.000000in}}{%
\pgfpathmoveto{\pgfqpoint{0.000000in}{0.000000in}}%
\pgfpathlineto{\pgfqpoint{0.000000in}{-0.048611in}}%
\pgfusepath{stroke,fill}%
}%
\begin{pgfscope}%
\pgfsys@transformshift{5.790909in}{0.330000in}%
\pgfsys@useobject{currentmarker}{}%
\end{pgfscope}%
\end{pgfscope}%
\begin{pgfscope}%
\pgfsetbuttcap%
\pgfsetroundjoin%
\definecolor{currentfill}{rgb}{0.000000,0.000000,0.000000}%
\pgfsetfillcolor{currentfill}%
\pgfsetlinewidth{0.803000pt}%
\definecolor{currentstroke}{rgb}{0.000000,0.000000,0.000000}%
\pgfsetstrokecolor{currentstroke}%
\pgfsetdash{}{0pt}%
\pgfsys@defobject{currentmarker}{\pgfqpoint{0.000000in}{0.000000in}}{\pgfqpoint{0.000000in}{0.048611in}}{%
\pgfpathmoveto{\pgfqpoint{0.000000in}{0.000000in}}%
\pgfpathlineto{\pgfqpoint{0.000000in}{0.048611in}}%
\pgfusepath{stroke,fill}%
}%
\begin{pgfscope}%
\pgfsys@transformshift{5.790909in}{2.640000in}%
\pgfsys@useobject{currentmarker}{}%
\end{pgfscope}%
\end{pgfscope}%
\begin{pgfscope}%
\definecolor{textcolor}{rgb}{0.000000,0.000000,0.000000}%
\pgfsetstrokecolor{textcolor}%
\pgfsetfillcolor{textcolor}%
\pgftext[x=5.790909in,y=0.232778in,,top]{\color{textcolor}\rmfamily\fontsize{10.000000}{12.000000}\selectfont \(\displaystyle {20}\)}%
\end{pgfscope}%
\begin{pgfscope}%
\pgfpathrectangle{\pgfqpoint{1.000000in}{0.330000in}}{\pgfqpoint{6.200000in}{2.310000in}}%
\pgfusepath{clip}%
\pgfsetbuttcap%
\pgfsetroundjoin%
\pgfsetlinewidth{0.803000pt}%
\definecolor{currentstroke}{rgb}{0.000000,0.000000,0.000000}%
\pgfsetstrokecolor{currentstroke}%
\pgfsetdash{{0.800000pt}{1.320000pt}}{0.000000pt}%
\pgfpathmoveto{\pgfqpoint{6.918182in}{0.330000in}}%
\pgfpathlineto{\pgfqpoint{6.918182in}{2.640000in}}%
\pgfusepath{stroke}%
\end{pgfscope}%
\begin{pgfscope}%
\pgfsetbuttcap%
\pgfsetroundjoin%
\definecolor{currentfill}{rgb}{0.000000,0.000000,0.000000}%
\pgfsetfillcolor{currentfill}%
\pgfsetlinewidth{0.803000pt}%
\definecolor{currentstroke}{rgb}{0.000000,0.000000,0.000000}%
\pgfsetstrokecolor{currentstroke}%
\pgfsetdash{}{0pt}%
\pgfsys@defobject{currentmarker}{\pgfqpoint{0.000000in}{-0.048611in}}{\pgfqpoint{0.000000in}{0.000000in}}{%
\pgfpathmoveto{\pgfqpoint{0.000000in}{0.000000in}}%
\pgfpathlineto{\pgfqpoint{0.000000in}{-0.048611in}}%
\pgfusepath{stroke,fill}%
}%
\begin{pgfscope}%
\pgfsys@transformshift{6.918182in}{0.330000in}%
\pgfsys@useobject{currentmarker}{}%
\end{pgfscope}%
\end{pgfscope}%
\begin{pgfscope}%
\pgfsetbuttcap%
\pgfsetroundjoin%
\definecolor{currentfill}{rgb}{0.000000,0.000000,0.000000}%
\pgfsetfillcolor{currentfill}%
\pgfsetlinewidth{0.803000pt}%
\definecolor{currentstroke}{rgb}{0.000000,0.000000,0.000000}%
\pgfsetstrokecolor{currentstroke}%
\pgfsetdash{}{0pt}%
\pgfsys@defobject{currentmarker}{\pgfqpoint{0.000000in}{0.000000in}}{\pgfqpoint{0.000000in}{0.048611in}}{%
\pgfpathmoveto{\pgfqpoint{0.000000in}{0.000000in}}%
\pgfpathlineto{\pgfqpoint{0.000000in}{0.048611in}}%
\pgfusepath{stroke,fill}%
}%
\begin{pgfscope}%
\pgfsys@transformshift{6.918182in}{2.640000in}%
\pgfsys@useobject{currentmarker}{}%
\end{pgfscope}%
\end{pgfscope}%
\begin{pgfscope}%
\definecolor{textcolor}{rgb}{0.000000,0.000000,0.000000}%
\pgfsetstrokecolor{textcolor}%
\pgfsetfillcolor{textcolor}%
\pgftext[x=6.918182in,y=0.232778in,,top]{\color{textcolor}\rmfamily\fontsize{10.000000}{12.000000}\selectfont \(\displaystyle {25}\)}%
\end{pgfscope}%
\begin{pgfscope}%
\pgfpathrectangle{\pgfqpoint{1.000000in}{0.330000in}}{\pgfqpoint{6.200000in}{2.310000in}}%
\pgfusepath{clip}%
\pgfsetbuttcap%
\pgfsetroundjoin%
\pgfsetlinewidth{0.803000pt}%
\definecolor{currentstroke}{rgb}{0.000000,0.000000,0.000000}%
\pgfsetstrokecolor{currentstroke}%
\pgfsetdash{{0.800000pt}{1.320000pt}}{0.000000pt}%
\pgfpathmoveto{\pgfqpoint{1.000000in}{0.756983in}}%
\pgfpathlineto{\pgfqpoint{7.200000in}{0.756983in}}%
\pgfusepath{stroke}%
\end{pgfscope}%
\begin{pgfscope}%
\pgfsetbuttcap%
\pgfsetroundjoin%
\definecolor{currentfill}{rgb}{0.000000,0.000000,0.000000}%
\pgfsetfillcolor{currentfill}%
\pgfsetlinewidth{0.803000pt}%
\definecolor{currentstroke}{rgb}{0.000000,0.000000,0.000000}%
\pgfsetstrokecolor{currentstroke}%
\pgfsetdash{}{0pt}%
\pgfsys@defobject{currentmarker}{\pgfqpoint{-0.048611in}{0.000000in}}{\pgfqpoint{-0.000000in}{0.000000in}}{%
\pgfpathmoveto{\pgfqpoint{-0.000000in}{0.000000in}}%
\pgfpathlineto{\pgfqpoint{-0.048611in}{0.000000in}}%
\pgfusepath{stroke,fill}%
}%
\begin{pgfscope}%
\pgfsys@transformshift{1.000000in}{0.756983in}%
\pgfsys@useobject{currentmarker}{}%
\end{pgfscope}%
\end{pgfscope}%
\begin{pgfscope}%
\pgfsetbuttcap%
\pgfsetroundjoin%
\definecolor{currentfill}{rgb}{0.000000,0.000000,0.000000}%
\pgfsetfillcolor{currentfill}%
\pgfsetlinewidth{0.803000pt}%
\definecolor{currentstroke}{rgb}{0.000000,0.000000,0.000000}%
\pgfsetstrokecolor{currentstroke}%
\pgfsetdash{}{0pt}%
\pgfsys@defobject{currentmarker}{\pgfqpoint{0.000000in}{0.000000in}}{\pgfqpoint{0.048611in}{0.000000in}}{%
\pgfpathmoveto{\pgfqpoint{0.000000in}{0.000000in}}%
\pgfpathlineto{\pgfqpoint{0.048611in}{0.000000in}}%
\pgfusepath{stroke,fill}%
}%
\begin{pgfscope}%
\pgfsys@transformshift{7.200000in}{0.756983in}%
\pgfsys@useobject{currentmarker}{}%
\end{pgfscope}%
\end{pgfscope}%
\begin{pgfscope}%
\definecolor{textcolor}{rgb}{0.000000,0.000000,0.000000}%
\pgfsetstrokecolor{textcolor}%
\pgfsetfillcolor{textcolor}%
\pgftext[x=0.655863in, y=0.708758in, left, base]{\color{textcolor}\rmfamily\fontsize{10.000000}{12.000000}\selectfont \(\displaystyle {\ensuremath{-}20}\)}%
\end{pgfscope}%
\begin{pgfscope}%
\pgfpathrectangle{\pgfqpoint{1.000000in}{0.330000in}}{\pgfqpoint{6.200000in}{2.310000in}}%
\pgfusepath{clip}%
\pgfsetbuttcap%
\pgfsetroundjoin%
\pgfsetlinewidth{0.803000pt}%
\definecolor{currentstroke}{rgb}{0.000000,0.000000,0.000000}%
\pgfsetstrokecolor{currentstroke}%
\pgfsetdash{{0.800000pt}{1.320000pt}}{0.000000pt}%
\pgfpathmoveto{\pgfqpoint{1.000000in}{1.329248in}}%
\pgfpathlineto{\pgfqpoint{7.200000in}{1.329248in}}%
\pgfusepath{stroke}%
\end{pgfscope}%
\begin{pgfscope}%
\pgfsetbuttcap%
\pgfsetroundjoin%
\definecolor{currentfill}{rgb}{0.000000,0.000000,0.000000}%
\pgfsetfillcolor{currentfill}%
\pgfsetlinewidth{0.803000pt}%
\definecolor{currentstroke}{rgb}{0.000000,0.000000,0.000000}%
\pgfsetstrokecolor{currentstroke}%
\pgfsetdash{}{0pt}%
\pgfsys@defobject{currentmarker}{\pgfqpoint{-0.048611in}{0.000000in}}{\pgfqpoint{-0.000000in}{0.000000in}}{%
\pgfpathmoveto{\pgfqpoint{-0.000000in}{0.000000in}}%
\pgfpathlineto{\pgfqpoint{-0.048611in}{0.000000in}}%
\pgfusepath{stroke,fill}%
}%
\begin{pgfscope}%
\pgfsys@transformshift{1.000000in}{1.329248in}%
\pgfsys@useobject{currentmarker}{}%
\end{pgfscope}%
\end{pgfscope}%
\begin{pgfscope}%
\pgfsetbuttcap%
\pgfsetroundjoin%
\definecolor{currentfill}{rgb}{0.000000,0.000000,0.000000}%
\pgfsetfillcolor{currentfill}%
\pgfsetlinewidth{0.803000pt}%
\definecolor{currentstroke}{rgb}{0.000000,0.000000,0.000000}%
\pgfsetstrokecolor{currentstroke}%
\pgfsetdash{}{0pt}%
\pgfsys@defobject{currentmarker}{\pgfqpoint{0.000000in}{0.000000in}}{\pgfqpoint{0.048611in}{0.000000in}}{%
\pgfpathmoveto{\pgfqpoint{0.000000in}{0.000000in}}%
\pgfpathlineto{\pgfqpoint{0.048611in}{0.000000in}}%
\pgfusepath{stroke,fill}%
}%
\begin{pgfscope}%
\pgfsys@transformshift{7.200000in}{1.329248in}%
\pgfsys@useobject{currentmarker}{}%
\end{pgfscope}%
\end{pgfscope}%
\begin{pgfscope}%
\definecolor{textcolor}{rgb}{0.000000,0.000000,0.000000}%
\pgfsetstrokecolor{textcolor}%
\pgfsetfillcolor{textcolor}%
\pgftext[x=0.833333in, y=1.281023in, left, base]{\color{textcolor}\rmfamily\fontsize{10.000000}{12.000000}\selectfont \(\displaystyle {0}\)}%
\end{pgfscope}%
\begin{pgfscope}%
\pgfpathrectangle{\pgfqpoint{1.000000in}{0.330000in}}{\pgfqpoint{6.200000in}{2.310000in}}%
\pgfusepath{clip}%
\pgfsetbuttcap%
\pgfsetroundjoin%
\pgfsetlinewidth{0.803000pt}%
\definecolor{currentstroke}{rgb}{0.000000,0.000000,0.000000}%
\pgfsetstrokecolor{currentstroke}%
\pgfsetdash{{0.800000pt}{1.320000pt}}{0.000000pt}%
\pgfpathmoveto{\pgfqpoint{1.000000in}{1.901513in}}%
\pgfpathlineto{\pgfqpoint{7.200000in}{1.901513in}}%
\pgfusepath{stroke}%
\end{pgfscope}%
\begin{pgfscope}%
\pgfsetbuttcap%
\pgfsetroundjoin%
\definecolor{currentfill}{rgb}{0.000000,0.000000,0.000000}%
\pgfsetfillcolor{currentfill}%
\pgfsetlinewidth{0.803000pt}%
\definecolor{currentstroke}{rgb}{0.000000,0.000000,0.000000}%
\pgfsetstrokecolor{currentstroke}%
\pgfsetdash{}{0pt}%
\pgfsys@defobject{currentmarker}{\pgfqpoint{-0.048611in}{0.000000in}}{\pgfqpoint{-0.000000in}{0.000000in}}{%
\pgfpathmoveto{\pgfqpoint{-0.000000in}{0.000000in}}%
\pgfpathlineto{\pgfqpoint{-0.048611in}{0.000000in}}%
\pgfusepath{stroke,fill}%
}%
\begin{pgfscope}%
\pgfsys@transformshift{1.000000in}{1.901513in}%
\pgfsys@useobject{currentmarker}{}%
\end{pgfscope}%
\end{pgfscope}%
\begin{pgfscope}%
\pgfsetbuttcap%
\pgfsetroundjoin%
\definecolor{currentfill}{rgb}{0.000000,0.000000,0.000000}%
\pgfsetfillcolor{currentfill}%
\pgfsetlinewidth{0.803000pt}%
\definecolor{currentstroke}{rgb}{0.000000,0.000000,0.000000}%
\pgfsetstrokecolor{currentstroke}%
\pgfsetdash{}{0pt}%
\pgfsys@defobject{currentmarker}{\pgfqpoint{0.000000in}{0.000000in}}{\pgfqpoint{0.048611in}{0.000000in}}{%
\pgfpathmoveto{\pgfqpoint{0.000000in}{0.000000in}}%
\pgfpathlineto{\pgfqpoint{0.048611in}{0.000000in}}%
\pgfusepath{stroke,fill}%
}%
\begin{pgfscope}%
\pgfsys@transformshift{7.200000in}{1.901513in}%
\pgfsys@useobject{currentmarker}{}%
\end{pgfscope}%
\end{pgfscope}%
\begin{pgfscope}%
\definecolor{textcolor}{rgb}{0.000000,0.000000,0.000000}%
\pgfsetstrokecolor{textcolor}%
\pgfsetfillcolor{textcolor}%
\pgftext[x=0.763888in, y=1.853288in, left, base]{\color{textcolor}\rmfamily\fontsize{10.000000}{12.000000}\selectfont \(\displaystyle {20}\)}%
\end{pgfscope}%
\begin{pgfscope}%
\pgfpathrectangle{\pgfqpoint{1.000000in}{0.330000in}}{\pgfqpoint{6.200000in}{2.310000in}}%
\pgfusepath{clip}%
\pgfsetbuttcap%
\pgfsetroundjoin%
\pgfsetlinewidth{0.803000pt}%
\definecolor{currentstroke}{rgb}{0.000000,0.000000,0.000000}%
\pgfsetstrokecolor{currentstroke}%
\pgfsetdash{{0.800000pt}{1.320000pt}}{0.000000pt}%
\pgfpathmoveto{\pgfqpoint{1.000000in}{2.473778in}}%
\pgfpathlineto{\pgfqpoint{7.200000in}{2.473778in}}%
\pgfusepath{stroke}%
\end{pgfscope}%
\begin{pgfscope}%
\pgfsetbuttcap%
\pgfsetroundjoin%
\definecolor{currentfill}{rgb}{0.000000,0.000000,0.000000}%
\pgfsetfillcolor{currentfill}%
\pgfsetlinewidth{0.803000pt}%
\definecolor{currentstroke}{rgb}{0.000000,0.000000,0.000000}%
\pgfsetstrokecolor{currentstroke}%
\pgfsetdash{}{0pt}%
\pgfsys@defobject{currentmarker}{\pgfqpoint{-0.048611in}{0.000000in}}{\pgfqpoint{-0.000000in}{0.000000in}}{%
\pgfpathmoveto{\pgfqpoint{-0.000000in}{0.000000in}}%
\pgfpathlineto{\pgfqpoint{-0.048611in}{0.000000in}}%
\pgfusepath{stroke,fill}%
}%
\begin{pgfscope}%
\pgfsys@transformshift{1.000000in}{2.473778in}%
\pgfsys@useobject{currentmarker}{}%
\end{pgfscope}%
\end{pgfscope}%
\begin{pgfscope}%
\pgfsetbuttcap%
\pgfsetroundjoin%
\definecolor{currentfill}{rgb}{0.000000,0.000000,0.000000}%
\pgfsetfillcolor{currentfill}%
\pgfsetlinewidth{0.803000pt}%
\definecolor{currentstroke}{rgb}{0.000000,0.000000,0.000000}%
\pgfsetstrokecolor{currentstroke}%
\pgfsetdash{}{0pt}%
\pgfsys@defobject{currentmarker}{\pgfqpoint{0.000000in}{0.000000in}}{\pgfqpoint{0.048611in}{0.000000in}}{%
\pgfpathmoveto{\pgfqpoint{0.000000in}{0.000000in}}%
\pgfpathlineto{\pgfqpoint{0.048611in}{0.000000in}}%
\pgfusepath{stroke,fill}%
}%
\begin{pgfscope}%
\pgfsys@transformshift{7.200000in}{2.473778in}%
\pgfsys@useobject{currentmarker}{}%
\end{pgfscope}%
\end{pgfscope}%
\begin{pgfscope}%
\definecolor{textcolor}{rgb}{0.000000,0.000000,0.000000}%
\pgfsetstrokecolor{textcolor}%
\pgfsetfillcolor{textcolor}%
\pgftext[x=0.763888in, y=2.425553in, left, base]{\color{textcolor}\rmfamily\fontsize{10.000000}{12.000000}\selectfont \(\displaystyle {40}\)}%
\end{pgfscope}%
\begin{pgfscope}%
\pgfpathrectangle{\pgfqpoint{1.000000in}{0.330000in}}{\pgfqpoint{6.200000in}{2.310000in}}%
\pgfusepath{clip}%
\pgfsetrectcap%
\pgfsetroundjoin%
\pgfsetlinewidth{1.505625pt}%
\definecolor{currentstroke}{rgb}{0.121569,0.466667,0.705882}%
\pgfsetstrokecolor{currentstroke}%
\pgfsetdash{}{0pt}%
\pgfpathmoveto{\pgfqpoint{1.281818in}{1.329248in}}%
\pgfpathlineto{\pgfqpoint{1.281818in}{1.329248in}}%
\pgfpathlineto{\pgfqpoint{1.488485in}{1.364408in}}%
\pgfpathlineto{\pgfqpoint{1.695152in}{1.397262in}}%
\pgfpathlineto{\pgfqpoint{1.901818in}{1.427809in}}%
\pgfpathlineto{\pgfqpoint{2.108485in}{1.456050in}}%
\pgfpathlineto{\pgfqpoint{2.315152in}{1.481984in}}%
\pgfpathlineto{\pgfqpoint{2.409091in}{1.493010in}}%
\pgfpathlineto{\pgfqpoint{2.615758in}{1.478869in}}%
\pgfpathlineto{\pgfqpoint{2.822424in}{1.462422in}}%
\pgfpathlineto{\pgfqpoint{3.029091in}{1.443668in}}%
\pgfpathlineto{\pgfqpoint{3.235758in}{1.422607in}}%
\pgfpathlineto{\pgfqpoint{3.442424in}{1.399241in}}%
\pgfpathlineto{\pgfqpoint{3.649091in}{1.373568in}}%
\pgfpathlineto{\pgfqpoint{3.855758in}{1.345588in}}%
\pgfpathlineto{\pgfqpoint{4.062424in}{1.315302in}}%
\pgfpathlineto{\pgfqpoint{4.269091in}{1.282709in}}%
\pgfpathlineto{\pgfqpoint{4.475758in}{1.247810in}}%
\pgfpathlineto{\pgfqpoint{4.663636in}{1.214088in}}%
\pgfpathlineto{\pgfqpoint{4.870303in}{1.237869in}}%
\pgfpathlineto{\pgfqpoint{5.074159in}{1.259072in}}%
\pgfpathlineto{\pgfqpoint{5.280826in}{1.278271in}}%
\pgfpathlineto{\pgfqpoint{5.487492in}{1.295164in}}%
\pgfpathlineto{\pgfqpoint{5.694159in}{1.309751in}}%
\pgfpathlineto{\pgfqpoint{5.903636in}{1.322182in}}%
\pgfpathlineto{\pgfqpoint{6.110303in}{1.332124in}}%
\pgfpathlineto{\pgfqpoint{6.316970in}{1.339760in}}%
\pgfpathlineto{\pgfqpoint{6.523636in}{1.345089in}}%
\pgfpathlineto{\pgfqpoint{6.730303in}{1.348143in}}%
\pgfpathlineto{\pgfqpoint{6.918182in}{1.350192in}}%
\pgfpathlineto{\pgfqpoint{6.918182in}{1.329248in}}%
\pgfpathlineto{\pgfqpoint{6.918182in}{1.329248in}}%
\pgfusepath{stroke}%
\end{pgfscope}%
\begin{pgfscope}%
\pgfpathrectangle{\pgfqpoint{1.000000in}{0.330000in}}{\pgfqpoint{6.200000in}{2.310000in}}%
\pgfusepath{clip}%
\pgfsetrectcap%
\pgfsetroundjoin%
\pgfsetlinewidth{1.505625pt}%
\definecolor{currentstroke}{rgb}{1.000000,0.498039,0.054902}%
\pgfsetstrokecolor{currentstroke}%
\pgfsetdash{}{0pt}%
\pgfpathmoveto{\pgfqpoint{1.281818in}{1.329248in}}%
\pgfpathlineto{\pgfqpoint{1.281818in}{1.329248in}}%
\pgfpathlineto{\pgfqpoint{1.432121in}{1.372942in}}%
\pgfpathlineto{\pgfqpoint{1.582424in}{1.414546in}}%
\pgfpathlineto{\pgfqpoint{1.732727in}{1.454057in}}%
\pgfpathlineto{\pgfqpoint{1.883030in}{1.491478in}}%
\pgfpathlineto{\pgfqpoint{2.033333in}{1.526807in}}%
\pgfpathlineto{\pgfqpoint{2.183636in}{1.560045in}}%
\pgfpathlineto{\pgfqpoint{2.333939in}{1.591191in}}%
\pgfpathlineto{\pgfqpoint{2.409091in}{1.605980in}}%
\pgfpathlineto{\pgfqpoint{2.559394in}{1.588209in}}%
\pgfpathlineto{\pgfqpoint{2.709697in}{1.568346in}}%
\pgfpathlineto{\pgfqpoint{2.860000in}{1.546391in}}%
\pgfpathlineto{\pgfqpoint{3.010303in}{1.522345in}}%
\pgfpathlineto{\pgfqpoint{3.160606in}{1.496208in}}%
\pgfpathlineto{\pgfqpoint{3.310909in}{1.467980in}}%
\pgfpathlineto{\pgfqpoint{3.461212in}{1.437660in}}%
\pgfpathlineto{\pgfqpoint{3.611515in}{1.405249in}}%
\pgfpathlineto{\pgfqpoint{3.761818in}{1.370747in}}%
\pgfpathlineto{\pgfqpoint{3.912121in}{1.334153in}}%
\pgfpathlineto{\pgfqpoint{4.062424in}{1.295468in}}%
\pgfpathlineto{\pgfqpoint{4.212727in}{1.254691in}}%
\pgfpathlineto{\pgfqpoint{4.363030in}{1.211824in}}%
\pgfpathlineto{\pgfqpoint{4.513333in}{1.166864in}}%
\pgfpathlineto{\pgfqpoint{4.663636in}{1.119825in}}%
\pgfpathlineto{\pgfqpoint{4.738788in}{1.143952in}}%
\pgfpathlineto{\pgfqpoint{4.813939in}{1.165491in}}%
\pgfpathlineto{\pgfqpoint{4.889091in}{1.184420in}}%
\pgfpathlineto{\pgfqpoint{4.945455in}{1.196890in}}%
\pgfpathlineto{\pgfqpoint{5.055371in}{1.217941in}}%
\pgfpathlineto{\pgfqpoint{5.205674in}{1.244489in}}%
\pgfpathlineto{\pgfqpoint{5.355977in}{1.268946in}}%
\pgfpathlineto{\pgfqpoint{5.506280in}{1.291312in}}%
\pgfpathlineto{\pgfqpoint{5.656583in}{1.311586in}}%
\pgfpathlineto{\pgfqpoint{5.809697in}{1.330073in}}%
\pgfpathlineto{\pgfqpoint{5.903636in}{1.339523in}}%
\pgfpathlineto{\pgfqpoint{5.997576in}{1.346914in}}%
\pgfpathlineto{\pgfqpoint{6.091515in}{1.352246in}}%
\pgfpathlineto{\pgfqpoint{6.185455in}{1.355519in}}%
\pgfpathlineto{\pgfqpoint{6.279394in}{1.356734in}}%
\pgfpathlineto{\pgfqpoint{6.373333in}{1.355889in}}%
\pgfpathlineto{\pgfqpoint{6.467273in}{1.352986in}}%
\pgfpathlineto{\pgfqpoint{6.561212in}{1.348024in}}%
\pgfpathlineto{\pgfqpoint{6.655152in}{1.341003in}}%
\pgfpathlineto{\pgfqpoint{6.767879in}{1.330100in}}%
\pgfpathlineto{\pgfqpoint{6.899394in}{1.315031in}}%
\pgfpathlineto{\pgfqpoint{6.918182in}{1.312680in}}%
\pgfpathlineto{\pgfqpoint{6.918182in}{1.329248in}}%
\pgfpathlineto{\pgfqpoint{6.918182in}{1.329248in}}%
\pgfusepath{stroke}%
\end{pgfscope}%
\begin{pgfscope}%
\pgfpathrectangle{\pgfqpoint{1.000000in}{0.330000in}}{\pgfqpoint{6.200000in}{2.310000in}}%
\pgfusepath{clip}%
\pgfsetrectcap%
\pgfsetroundjoin%
\pgfsetlinewidth{1.505625pt}%
\definecolor{currentstroke}{rgb}{0.172549,0.627451,0.172549}%
\pgfsetstrokecolor{currentstroke}%
\pgfsetdash{}{0pt}%
\pgfpathmoveto{\pgfqpoint{1.281818in}{1.329248in}}%
\pgfpathlineto{\pgfqpoint{1.281818in}{1.329248in}}%
\pgfpathlineto{\pgfqpoint{1.469697in}{1.370467in}}%
\pgfpathlineto{\pgfqpoint{1.657576in}{1.409235in}}%
\pgfpathlineto{\pgfqpoint{1.845455in}{1.445552in}}%
\pgfpathlineto{\pgfqpoint{2.033333in}{1.479418in}}%
\pgfpathlineto{\pgfqpoint{2.221212in}{1.510834in}}%
\pgfpathlineto{\pgfqpoint{2.409091in}{1.539799in}}%
\pgfpathlineto{\pgfqpoint{2.596970in}{1.523393in}}%
\pgfpathlineto{\pgfqpoint{2.784848in}{1.504537in}}%
\pgfpathlineto{\pgfqpoint{2.972727in}{1.483229in}}%
\pgfpathlineto{\pgfqpoint{3.160606in}{1.459471in}}%
\pgfpathlineto{\pgfqpoint{3.348485in}{1.433262in}}%
\pgfpathlineto{\pgfqpoint{3.536364in}{1.404602in}}%
\pgfpathlineto{\pgfqpoint{3.724242in}{1.373492in}}%
\pgfpathlineto{\pgfqpoint{3.912121in}{1.339931in}}%
\pgfpathlineto{\pgfqpoint{4.100000in}{1.303919in}}%
\pgfpathlineto{\pgfqpoint{4.287879in}{1.265456in}}%
\pgfpathlineto{\pgfqpoint{4.475758in}{1.224542in}}%
\pgfpathlineto{\pgfqpoint{4.663636in}{1.181186in}}%
\pgfpathlineto{\pgfqpoint{4.851515in}{1.209104in}}%
\pgfpathlineto{\pgfqpoint{5.036583in}{1.234215in}}%
\pgfpathlineto{\pgfqpoint{5.224462in}{1.257269in}}%
\pgfpathlineto{\pgfqpoint{5.412341in}{1.277871in}}%
\pgfpathlineto{\pgfqpoint{5.600220in}{1.296023in}}%
\pgfpathlineto{\pgfqpoint{5.769310in}{1.310264in}}%
\pgfpathlineto{\pgfqpoint{5.941212in}{1.322708in}}%
\pgfpathlineto{\pgfqpoint{6.129091in}{1.333961in}}%
\pgfpathlineto{\pgfqpoint{6.316970in}{1.342763in}}%
\pgfpathlineto{\pgfqpoint{6.504848in}{1.349115in}}%
\pgfpathlineto{\pgfqpoint{6.692727in}{1.353015in}}%
\pgfpathlineto{\pgfqpoint{6.918182in}{1.356176in}}%
\pgfpathlineto{\pgfqpoint{6.918182in}{1.329248in}}%
\pgfpathlineto{\pgfqpoint{6.918182in}{1.329248in}}%
\pgfusepath{stroke}%
\end{pgfscope}%
\begin{pgfscope}%
\pgfpathrectangle{\pgfqpoint{1.000000in}{0.330000in}}{\pgfqpoint{6.200000in}{2.310000in}}%
\pgfusepath{clip}%
\pgfsetrectcap%
\pgfsetroundjoin%
\pgfsetlinewidth{1.505625pt}%
\definecolor{currentstroke}{rgb}{0.839216,0.152941,0.156863}%
\pgfsetstrokecolor{currentstroke}%
\pgfsetdash{}{0pt}%
\pgfpathmoveto{\pgfqpoint{1.281818in}{1.329248in}}%
\pgfpathlineto{\pgfqpoint{1.281818in}{1.329248in}}%
\pgfpathlineto{\pgfqpoint{1.432121in}{1.373476in}}%
\pgfpathlineto{\pgfqpoint{1.582424in}{1.415613in}}%
\pgfpathlineto{\pgfqpoint{1.732727in}{1.455658in}}%
\pgfpathlineto{\pgfqpoint{1.883030in}{1.493613in}}%
\pgfpathlineto{\pgfqpoint{2.033333in}{1.529475in}}%
\pgfpathlineto{\pgfqpoint{2.183636in}{1.563247in}}%
\pgfpathlineto{\pgfqpoint{2.333939in}{1.594927in}}%
\pgfpathlineto{\pgfqpoint{2.409091in}{1.609983in}}%
\pgfpathlineto{\pgfqpoint{2.559394in}{1.592745in}}%
\pgfpathlineto{\pgfqpoint{2.709697in}{1.573415in}}%
\pgfpathlineto{\pgfqpoint{2.860000in}{1.551994in}}%
\pgfpathlineto{\pgfqpoint{3.010303in}{1.528482in}}%
\pgfpathlineto{\pgfqpoint{3.160606in}{1.502879in}}%
\pgfpathlineto{\pgfqpoint{3.310909in}{1.475184in}}%
\pgfpathlineto{\pgfqpoint{3.461212in}{1.445398in}}%
\pgfpathlineto{\pgfqpoint{3.611515in}{1.413521in}}%
\pgfpathlineto{\pgfqpoint{3.761818in}{1.379552in}}%
\pgfpathlineto{\pgfqpoint{3.912121in}{1.343492in}}%
\pgfpathlineto{\pgfqpoint{4.062424in}{1.305340in}}%
\pgfpathlineto{\pgfqpoint{4.212727in}{1.265097in}}%
\pgfpathlineto{\pgfqpoint{4.363030in}{1.222763in}}%
\pgfpathlineto{\pgfqpoint{4.513333in}{1.178338in}}%
\pgfpathlineto{\pgfqpoint{4.663636in}{1.131832in}}%
\pgfpathlineto{\pgfqpoint{4.813939in}{1.161873in}}%
\pgfpathlineto{\pgfqpoint{4.980220in}{1.192679in}}%
\pgfpathlineto{\pgfqpoint{5.130523in}{1.218315in}}%
\pgfpathlineto{\pgfqpoint{5.280826in}{1.241860in}}%
\pgfpathlineto{\pgfqpoint{5.431129in}{1.263313in}}%
\pgfpathlineto{\pgfqpoint{5.581432in}{1.282675in}}%
\pgfpathlineto{\pgfqpoint{5.731735in}{1.299945in}}%
\pgfpathlineto{\pgfqpoint{5.884848in}{1.315390in}}%
\pgfpathlineto{\pgfqpoint{6.035152in}{1.328438in}}%
\pgfpathlineto{\pgfqpoint{6.185455in}{1.339396in}}%
\pgfpathlineto{\pgfqpoint{6.335758in}{1.348262in}}%
\pgfpathlineto{\pgfqpoint{6.486061in}{1.355037in}}%
\pgfpathlineto{\pgfqpoint{6.636364in}{1.359720in}}%
\pgfpathlineto{\pgfqpoint{6.899394in}{1.364801in}}%
\pgfpathlineto{\pgfqpoint{6.918182in}{1.365153in}}%
\pgfpathlineto{\pgfqpoint{6.918182in}{1.329248in}}%
\pgfpathlineto{\pgfqpoint{6.918182in}{1.329248in}}%
\pgfusepath{stroke}%
\end{pgfscope}%
\begin{pgfscope}%
\pgfpathrectangle{\pgfqpoint{1.000000in}{0.330000in}}{\pgfqpoint{6.200000in}{2.310000in}}%
\pgfusepath{clip}%
\pgfsetrectcap%
\pgfsetroundjoin%
\pgfsetlinewidth{1.505625pt}%
\definecolor{currentstroke}{rgb}{0.580392,0.403922,0.741176}%
\pgfsetstrokecolor{currentstroke}%
\pgfsetdash{}{0pt}%
\pgfpathmoveto{\pgfqpoint{1.281818in}{1.329248in}}%
\pgfpathlineto{\pgfqpoint{1.281818in}{1.329248in}}%
\pgfpathlineto{\pgfqpoint{1.432121in}{1.372622in}}%
\pgfpathlineto{\pgfqpoint{1.582424in}{1.413905in}}%
\pgfpathlineto{\pgfqpoint{1.732727in}{1.453097in}}%
\pgfpathlineto{\pgfqpoint{1.883030in}{1.490197in}}%
\pgfpathlineto{\pgfqpoint{2.033333in}{1.525206in}}%
\pgfpathlineto{\pgfqpoint{2.183636in}{1.558124in}}%
\pgfpathlineto{\pgfqpoint{2.333939in}{1.588950in}}%
\pgfpathlineto{\pgfqpoint{2.409091in}{1.603579in}}%
\pgfpathlineto{\pgfqpoint{2.559394in}{1.585487in}}%
\pgfpathlineto{\pgfqpoint{2.709697in}{1.565304in}}%
\pgfpathlineto{\pgfqpoint{2.860000in}{1.543029in}}%
\pgfpathlineto{\pgfqpoint{3.010303in}{1.518663in}}%
\pgfpathlineto{\pgfqpoint{3.160606in}{1.492206in}}%
\pgfpathlineto{\pgfqpoint{3.310909in}{1.463658in}}%
\pgfpathlineto{\pgfqpoint{3.461212in}{1.433018in}}%
\pgfpathlineto{\pgfqpoint{3.611515in}{1.400286in}}%
\pgfpathlineto{\pgfqpoint{3.761818in}{1.365464in}}%
\pgfpathlineto{\pgfqpoint{3.912121in}{1.328550in}}%
\pgfpathlineto{\pgfqpoint{4.062424in}{1.289544in}}%
\pgfpathlineto{\pgfqpoint{4.212727in}{1.248448in}}%
\pgfpathlineto{\pgfqpoint{4.363030in}{1.205260in}}%
\pgfpathlineto{\pgfqpoint{4.513333in}{1.159980in}}%
\pgfpathlineto{\pgfqpoint{4.663636in}{1.112621in}}%
\pgfpathlineto{\pgfqpoint{4.720000in}{1.135054in}}%
\pgfpathlineto{\pgfqpoint{4.776364in}{1.155339in}}%
\pgfpathlineto{\pgfqpoint{4.832727in}{1.173462in}}%
\pgfpathlineto{\pgfqpoint{4.889091in}{1.189406in}}%
\pgfpathlineto{\pgfqpoint{4.945455in}{1.203158in}}%
\pgfpathlineto{\pgfqpoint{5.036583in}{1.221639in}}%
\pgfpathlineto{\pgfqpoint{5.186886in}{1.249623in}}%
\pgfpathlineto{\pgfqpoint{5.337189in}{1.275516in}}%
\pgfpathlineto{\pgfqpoint{5.487492in}{1.299317in}}%
\pgfpathlineto{\pgfqpoint{5.637795in}{1.321027in}}%
\pgfpathlineto{\pgfqpoint{5.790909in}{1.340994in}}%
\pgfpathlineto{\pgfqpoint{5.866061in}{1.349417in}}%
\pgfpathlineto{\pgfqpoint{5.941212in}{1.356045in}}%
\pgfpathlineto{\pgfqpoint{6.016364in}{1.360879in}}%
\pgfpathlineto{\pgfqpoint{6.091515in}{1.363918in}}%
\pgfpathlineto{\pgfqpoint{6.166667in}{1.365162in}}%
\pgfpathlineto{\pgfqpoint{6.241818in}{1.364612in}}%
\pgfpathlineto{\pgfqpoint{6.316970in}{1.362268in}}%
\pgfpathlineto{\pgfqpoint{6.392121in}{1.358129in}}%
\pgfpathlineto{\pgfqpoint{6.467273in}{1.352195in}}%
\pgfpathlineto{\pgfqpoint{6.542424in}{1.344467in}}%
\pgfpathlineto{\pgfqpoint{6.617576in}{1.334945in}}%
\pgfpathlineto{\pgfqpoint{6.711515in}{1.320529in}}%
\pgfpathlineto{\pgfqpoint{6.805455in}{1.303842in}}%
\pgfpathlineto{\pgfqpoint{6.899394in}{1.285169in}}%
\pgfpathlineto{\pgfqpoint{6.918182in}{1.281196in}}%
\pgfpathlineto{\pgfqpoint{6.918182in}{1.329248in}}%
\pgfpathlineto{\pgfqpoint{6.918182in}{1.329248in}}%
\pgfusepath{stroke}%
\end{pgfscope}%
\begin{pgfscope}%
\pgfpathrectangle{\pgfqpoint{1.000000in}{0.330000in}}{\pgfqpoint{6.200000in}{2.310000in}}%
\pgfusepath{clip}%
\pgfsetrectcap%
\pgfsetroundjoin%
\pgfsetlinewidth{1.505625pt}%
\definecolor{currentstroke}{rgb}{0.549020,0.337255,0.294118}%
\pgfsetstrokecolor{currentstroke}%
\pgfsetdash{}{0pt}%
\pgfpathmoveto{\pgfqpoint{1.281818in}{1.329248in}}%
\pgfpathlineto{\pgfqpoint{1.281818in}{1.329248in}}%
\pgfpathlineto{\pgfqpoint{1.394545in}{1.432043in}}%
\pgfpathlineto{\pgfqpoint{1.488485in}{1.514920in}}%
\pgfpathlineto{\pgfqpoint{1.582424in}{1.595219in}}%
\pgfpathlineto{\pgfqpoint{1.676364in}{1.672898in}}%
\pgfpathlineto{\pgfqpoint{1.770303in}{1.747915in}}%
\pgfpathlineto{\pgfqpoint{1.864242in}{1.820226in}}%
\pgfpathlineto{\pgfqpoint{1.958182in}{1.889790in}}%
\pgfpathlineto{\pgfqpoint{2.052121in}{1.956564in}}%
\pgfpathlineto{\pgfqpoint{2.146061in}{2.020506in}}%
\pgfpathlineto{\pgfqpoint{2.240000in}{2.081574in}}%
\pgfpathlineto{\pgfqpoint{2.333939in}{2.139725in}}%
\pgfpathlineto{\pgfqpoint{2.409091in}{2.184116in}}%
\pgfpathlineto{\pgfqpoint{2.484242in}{2.165551in}}%
\pgfpathlineto{\pgfqpoint{2.559394in}{2.145048in}}%
\pgfpathlineto{\pgfqpoint{2.634545in}{2.122586in}}%
\pgfpathlineto{\pgfqpoint{2.709697in}{2.098143in}}%
\pgfpathlineto{\pgfqpoint{2.784848in}{2.071697in}}%
\pgfpathlineto{\pgfqpoint{2.860000in}{2.043228in}}%
\pgfpathlineto{\pgfqpoint{2.935152in}{2.012712in}}%
\pgfpathlineto{\pgfqpoint{3.010303in}{1.980129in}}%
\pgfpathlineto{\pgfqpoint{3.085455in}{1.945457in}}%
\pgfpathlineto{\pgfqpoint{3.160606in}{1.908673in}}%
\pgfpathlineto{\pgfqpoint{3.235758in}{1.869757in}}%
\pgfpathlineto{\pgfqpoint{3.310909in}{1.828686in}}%
\pgfpathlineto{\pgfqpoint{3.386061in}{1.785439in}}%
\pgfpathlineto{\pgfqpoint{3.461212in}{1.739995in}}%
\pgfpathlineto{\pgfqpoint{3.536364in}{1.692330in}}%
\pgfpathlineto{\pgfqpoint{3.611515in}{1.642425in}}%
\pgfpathlineto{\pgfqpoint{3.686667in}{1.590256in}}%
\pgfpathlineto{\pgfqpoint{3.761818in}{1.535803in}}%
\pgfpathlineto{\pgfqpoint{3.836970in}{1.479043in}}%
\pgfpathlineto{\pgfqpoint{3.912121in}{1.419956in}}%
\pgfpathlineto{\pgfqpoint{3.987273in}{1.358518in}}%
\pgfpathlineto{\pgfqpoint{4.062424in}{1.294709in}}%
\pgfpathlineto{\pgfqpoint{4.137576in}{1.228507in}}%
\pgfpathlineto{\pgfqpoint{4.212727in}{1.159890in}}%
\pgfpathlineto{\pgfqpoint{4.287879in}{1.088836in}}%
\pgfpathlineto{\pgfqpoint{4.381818in}{0.996559in}}%
\pgfpathlineto{\pgfqpoint{4.475758in}{0.900399in}}%
\pgfpathlineto{\pgfqpoint{4.569697in}{0.800313in}}%
\pgfpathlineto{\pgfqpoint{4.663636in}{0.696312in}}%
\pgfpathlineto{\pgfqpoint{4.738788in}{0.741078in}}%
\pgfpathlineto{\pgfqpoint{4.813939in}{0.783256in}}%
\pgfpathlineto{\pgfqpoint{4.889091in}{0.822824in}}%
\pgfpathlineto{\pgfqpoint{4.945455in}{0.850773in}}%
\pgfpathlineto{\pgfqpoint{5.055371in}{0.902011in}}%
\pgfpathlineto{\pgfqpoint{5.224462in}{0.978168in}}%
\pgfpathlineto{\pgfqpoint{5.393553in}{1.051679in}}%
\pgfpathlineto{\pgfqpoint{5.562644in}{1.122542in}}%
\pgfpathlineto{\pgfqpoint{5.731735in}{1.190759in}}%
\pgfpathlineto{\pgfqpoint{5.866061in}{1.242701in}}%
\pgfpathlineto{\pgfqpoint{5.960000in}{1.276714in}}%
\pgfpathlineto{\pgfqpoint{6.053939in}{1.308668in}}%
\pgfpathlineto{\pgfqpoint{6.147879in}{1.338564in}}%
\pgfpathlineto{\pgfqpoint{6.241818in}{1.366401in}}%
\pgfpathlineto{\pgfqpoint{6.335758in}{1.392179in}}%
\pgfpathlineto{\pgfqpoint{6.429697in}{1.415898in}}%
\pgfpathlineto{\pgfqpoint{6.523636in}{1.437558in}}%
\pgfpathlineto{\pgfqpoint{6.617576in}{1.457160in}}%
\pgfpathlineto{\pgfqpoint{6.711515in}{1.474713in}}%
\pgfpathlineto{\pgfqpoint{6.843030in}{1.496805in}}%
\pgfpathlineto{\pgfqpoint{6.918182in}{1.508336in}}%
\pgfpathlineto{\pgfqpoint{6.918182in}{1.329248in}}%
\pgfpathlineto{\pgfqpoint{6.918182in}{1.329248in}}%
\pgfusepath{stroke}%
\end{pgfscope}%
\begin{pgfscope}%
\pgfpathrectangle{\pgfqpoint{1.000000in}{0.330000in}}{\pgfqpoint{6.200000in}{2.310000in}}%
\pgfusepath{clip}%
\pgfsetrectcap%
\pgfsetroundjoin%
\pgfsetlinewidth{1.505625pt}%
\definecolor{currentstroke}{rgb}{0.890196,0.466667,0.760784}%
\pgfsetstrokecolor{currentstroke}%
\pgfsetdash{}{0pt}%
\pgfpathmoveto{\pgfqpoint{1.281818in}{1.329248in}}%
\pgfpathlineto{\pgfqpoint{1.281818in}{1.329248in}}%
\pgfpathlineto{\pgfqpoint{1.375758in}{1.449978in}}%
\pgfpathlineto{\pgfqpoint{1.469697in}{1.567156in}}%
\pgfpathlineto{\pgfqpoint{1.563636in}{1.680713in}}%
\pgfpathlineto{\pgfqpoint{1.657576in}{1.790581in}}%
\pgfpathlineto{\pgfqpoint{1.751515in}{1.896692in}}%
\pgfpathlineto{\pgfqpoint{1.845455in}{1.998980in}}%
\pgfpathlineto{\pgfqpoint{1.939394in}{2.097375in}}%
\pgfpathlineto{\pgfqpoint{2.014545in}{2.173243in}}%
\pgfpathlineto{\pgfqpoint{2.089697in}{2.246543in}}%
\pgfpathlineto{\pgfqpoint{2.164848in}{2.317238in}}%
\pgfpathlineto{\pgfqpoint{2.240000in}{2.385295in}}%
\pgfpathlineto{\pgfqpoint{2.315152in}{2.450679in}}%
\pgfpathlineto{\pgfqpoint{2.390303in}{2.513355in}}%
\pgfpathlineto{\pgfqpoint{2.409091in}{2.528596in}}%
\pgfpathlineto{\pgfqpoint{2.484242in}{2.503906in}}%
\pgfpathlineto{\pgfqpoint{2.559394in}{2.476430in}}%
\pgfpathlineto{\pgfqpoint{2.634545in}{2.446132in}}%
\pgfpathlineto{\pgfqpoint{2.709697in}{2.412979in}}%
\pgfpathlineto{\pgfqpoint{2.784848in}{2.376936in}}%
\pgfpathlineto{\pgfqpoint{2.860000in}{2.337968in}}%
\pgfpathlineto{\pgfqpoint{2.935152in}{2.296040in}}%
\pgfpathlineto{\pgfqpoint{3.010303in}{2.251117in}}%
\pgfpathlineto{\pgfqpoint{3.085455in}{2.203165in}}%
\pgfpathlineto{\pgfqpoint{3.160606in}{2.152150in}}%
\pgfpathlineto{\pgfqpoint{3.235758in}{2.098035in}}%
\pgfpathlineto{\pgfqpoint{3.310909in}{2.040787in}}%
\pgfpathlineto{\pgfqpoint{3.386061in}{1.980371in}}%
\pgfpathlineto{\pgfqpoint{3.461212in}{1.916753in}}%
\pgfpathlineto{\pgfqpoint{3.536364in}{1.849896in}}%
\pgfpathlineto{\pgfqpoint{3.611515in}{1.779767in}}%
\pgfpathlineto{\pgfqpoint{3.686667in}{1.706331in}}%
\pgfpathlineto{\pgfqpoint{3.761818in}{1.629554in}}%
\pgfpathlineto{\pgfqpoint{3.836970in}{1.549400in}}%
\pgfpathlineto{\pgfqpoint{3.912121in}{1.465834in}}%
\pgfpathlineto{\pgfqpoint{3.987273in}{1.378823in}}%
\pgfpathlineto{\pgfqpoint{4.062424in}{1.288331in}}%
\pgfpathlineto{\pgfqpoint{4.137576in}{1.194323in}}%
\pgfpathlineto{\pgfqpoint{4.212727in}{1.096765in}}%
\pgfpathlineto{\pgfqpoint{4.287879in}{0.995623in}}%
\pgfpathlineto{\pgfqpoint{4.363030in}{0.890861in}}%
\pgfpathlineto{\pgfqpoint{4.438182in}{0.782444in}}%
\pgfpathlineto{\pgfqpoint{4.513333in}{0.670338in}}%
\pgfpathlineto{\pgfqpoint{4.588485in}{0.554509in}}%
\pgfpathlineto{\pgfqpoint{4.663636in}{0.435000in}}%
\pgfpathlineto{\pgfqpoint{4.738788in}{0.497457in}}%
\pgfpathlineto{\pgfqpoint{4.795152in}{0.541790in}}%
\pgfpathlineto{\pgfqpoint{4.851515in}{0.583954in}}%
\pgfpathlineto{\pgfqpoint{4.907879in}{0.623937in}}%
\pgfpathlineto{\pgfqpoint{4.980220in}{0.672095in}}%
\pgfpathlineto{\pgfqpoint{5.149310in}{0.778613in}}%
\pgfpathlineto{\pgfqpoint{5.318401in}{0.882484in}}%
\pgfpathlineto{\pgfqpoint{5.487492in}{0.983708in}}%
\pgfpathlineto{\pgfqpoint{5.656583in}{1.082285in}}%
\pgfpathlineto{\pgfqpoint{5.809697in}{1.169239in}}%
\pgfpathlineto{\pgfqpoint{5.903636in}{1.220204in}}%
\pgfpathlineto{\pgfqpoint{5.997576in}{1.268365in}}%
\pgfpathlineto{\pgfqpoint{6.091515in}{1.313722in}}%
\pgfpathlineto{\pgfqpoint{6.185455in}{1.356276in}}%
\pgfpathlineto{\pgfqpoint{6.279394in}{1.396025in}}%
\pgfpathlineto{\pgfqpoint{6.373333in}{1.432970in}}%
\pgfpathlineto{\pgfqpoint{6.467273in}{1.467112in}}%
\pgfpathlineto{\pgfqpoint{6.561212in}{1.498449in}}%
\pgfpathlineto{\pgfqpoint{6.655152in}{1.526983in}}%
\pgfpathlineto{\pgfqpoint{6.749091in}{1.552843in}}%
\pgfpathlineto{\pgfqpoint{6.843030in}{1.576640in}}%
\pgfpathlineto{\pgfqpoint{6.918182in}{1.594247in}}%
\pgfpathlineto{\pgfqpoint{6.918182in}{1.329248in}}%
\pgfpathlineto{\pgfqpoint{6.918182in}{1.329248in}}%
\pgfusepath{stroke}%
\end{pgfscope}%
\begin{pgfscope}%
\pgfpathrectangle{\pgfqpoint{1.000000in}{0.330000in}}{\pgfqpoint{6.200000in}{2.310000in}}%
\pgfusepath{clip}%
\pgfsetrectcap%
\pgfsetroundjoin%
\pgfsetlinewidth{1.505625pt}%
\definecolor{currentstroke}{rgb}{0.498039,0.498039,0.498039}%
\pgfsetstrokecolor{currentstroke}%
\pgfsetdash{}{0pt}%
\pgfpathmoveto{\pgfqpoint{1.281818in}{1.329248in}}%
\pgfpathlineto{\pgfqpoint{1.281818in}{1.329248in}}%
\pgfpathlineto{\pgfqpoint{1.394545in}{1.402890in}}%
\pgfpathlineto{\pgfqpoint{1.507273in}{1.473923in}}%
\pgfpathlineto{\pgfqpoint{1.620000in}{1.542310in}}%
\pgfpathlineto{\pgfqpoint{1.732727in}{1.608013in}}%
\pgfpathlineto{\pgfqpoint{1.845455in}{1.670997in}}%
\pgfpathlineto{\pgfqpoint{1.958182in}{1.731224in}}%
\pgfpathlineto{\pgfqpoint{2.070909in}{1.788659in}}%
\pgfpathlineto{\pgfqpoint{2.183636in}{1.843264in}}%
\pgfpathlineto{\pgfqpoint{2.296364in}{1.895002in}}%
\pgfpathlineto{\pgfqpoint{2.409091in}{1.943839in}}%
\pgfpathlineto{\pgfqpoint{2.503030in}{1.925065in}}%
\pgfpathlineto{\pgfqpoint{2.596970in}{1.904229in}}%
\pgfpathlineto{\pgfqpoint{2.690909in}{1.881309in}}%
\pgfpathlineto{\pgfqpoint{2.784848in}{1.856285in}}%
\pgfpathlineto{\pgfqpoint{2.878788in}{1.829134in}}%
\pgfpathlineto{\pgfqpoint{2.972727in}{1.799837in}}%
\pgfpathlineto{\pgfqpoint{3.066667in}{1.768370in}}%
\pgfpathlineto{\pgfqpoint{3.160606in}{1.734715in}}%
\pgfpathlineto{\pgfqpoint{3.254545in}{1.698848in}}%
\pgfpathlineto{\pgfqpoint{3.348485in}{1.660749in}}%
\pgfpathlineto{\pgfqpoint{3.442424in}{1.620397in}}%
\pgfpathlineto{\pgfqpoint{3.536364in}{1.577771in}}%
\pgfpathlineto{\pgfqpoint{3.630303in}{1.532849in}}%
\pgfpathlineto{\pgfqpoint{3.724242in}{1.485610in}}%
\pgfpathlineto{\pgfqpoint{3.818182in}{1.436034in}}%
\pgfpathlineto{\pgfqpoint{3.912121in}{1.384098in}}%
\pgfpathlineto{\pgfqpoint{4.006061in}{1.329781in}}%
\pgfpathlineto{\pgfqpoint{4.100000in}{1.273063in}}%
\pgfpathlineto{\pgfqpoint{4.193939in}{1.213923in}}%
\pgfpathlineto{\pgfqpoint{4.287879in}{1.152338in}}%
\pgfpathlineto{\pgfqpoint{4.381818in}{1.088288in}}%
\pgfpathlineto{\pgfqpoint{4.475758in}{1.021752in}}%
\pgfpathlineto{\pgfqpoint{4.569697in}{0.952708in}}%
\pgfpathlineto{\pgfqpoint{4.663636in}{0.881169in}}%
\pgfpathlineto{\pgfqpoint{4.757576in}{0.921627in}}%
\pgfpathlineto{\pgfqpoint{4.851515in}{0.959513in}}%
\pgfpathlineto{\pgfqpoint{4.945455in}{0.994807in}}%
\pgfpathlineto{\pgfqpoint{5.074159in}{1.039725in}}%
\pgfpathlineto{\pgfqpoint{5.224462in}{1.089726in}}%
\pgfpathlineto{\pgfqpoint{5.374765in}{1.137288in}}%
\pgfpathlineto{\pgfqpoint{5.525068in}{1.182410in}}%
\pgfpathlineto{\pgfqpoint{5.675371in}{1.225092in}}%
\pgfpathlineto{\pgfqpoint{5.828485in}{1.266023in}}%
\pgfpathlineto{\pgfqpoint{5.941212in}{1.293849in}}%
\pgfpathlineto{\pgfqpoint{6.053939in}{1.319408in}}%
\pgfpathlineto{\pgfqpoint{6.166667in}{1.342700in}}%
\pgfpathlineto{\pgfqpoint{6.279394in}{1.363726in}}%
\pgfpathlineto{\pgfqpoint{6.392121in}{1.382486in}}%
\pgfpathlineto{\pgfqpoint{6.504848in}{1.398979in}}%
\pgfpathlineto{\pgfqpoint{6.617576in}{1.413205in}}%
\pgfpathlineto{\pgfqpoint{6.730303in}{1.425228in}}%
\pgfpathlineto{\pgfqpoint{6.899394in}{1.441090in}}%
\pgfpathlineto{\pgfqpoint{6.918182in}{1.442729in}}%
\pgfpathlineto{\pgfqpoint{6.918182in}{1.329248in}}%
\pgfpathlineto{\pgfqpoint{6.918182in}{1.329248in}}%
\pgfusepath{stroke}%
\end{pgfscope}%
\begin{pgfscope}%
\pgfpathrectangle{\pgfqpoint{1.000000in}{0.330000in}}{\pgfqpoint{6.200000in}{2.310000in}}%
\pgfusepath{clip}%
\pgfsetrectcap%
\pgfsetroundjoin%
\pgfsetlinewidth{1.505625pt}%
\definecolor{currentstroke}{rgb}{0.737255,0.741176,0.133333}%
\pgfsetstrokecolor{currentstroke}%
\pgfsetdash{}{0pt}%
\pgfpathmoveto{\pgfqpoint{1.281818in}{1.329248in}}%
\pgfpathlineto{\pgfqpoint{1.281818in}{1.329248in}}%
\pgfpathlineto{\pgfqpoint{1.394545in}{1.432444in}}%
\pgfpathlineto{\pgfqpoint{1.488485in}{1.515654in}}%
\pgfpathlineto{\pgfqpoint{1.582424in}{1.596287in}}%
\pgfpathlineto{\pgfqpoint{1.676364in}{1.674299in}}%
\pgfpathlineto{\pgfqpoint{1.770303in}{1.749649in}}%
\pgfpathlineto{\pgfqpoint{1.864242in}{1.822294in}}%
\pgfpathlineto{\pgfqpoint{1.958182in}{1.892192in}}%
\pgfpathlineto{\pgfqpoint{2.052121in}{1.959299in}}%
\pgfpathlineto{\pgfqpoint{2.146061in}{2.023575in}}%
\pgfpathlineto{\pgfqpoint{2.240000in}{2.084976in}}%
\pgfpathlineto{\pgfqpoint{2.333939in}{2.143460in}}%
\pgfpathlineto{\pgfqpoint{2.409091in}{2.188119in}}%
\pgfpathlineto{\pgfqpoint{2.484242in}{2.169820in}}%
\pgfpathlineto{\pgfqpoint{2.559394in}{2.149584in}}%
\pgfpathlineto{\pgfqpoint{2.634545in}{2.127388in}}%
\pgfpathlineto{\pgfqpoint{2.709697in}{2.103212in}}%
\pgfpathlineto{\pgfqpoint{2.784848in}{2.077034in}}%
\pgfpathlineto{\pgfqpoint{2.860000in}{2.048831in}}%
\pgfpathlineto{\pgfqpoint{2.935152in}{2.018582in}}%
\pgfpathlineto{\pgfqpoint{3.010303in}{1.986266in}}%
\pgfpathlineto{\pgfqpoint{3.085455in}{1.951860in}}%
\pgfpathlineto{\pgfqpoint{3.160606in}{1.915344in}}%
\pgfpathlineto{\pgfqpoint{3.235758in}{1.876694in}}%
\pgfpathlineto{\pgfqpoint{3.310909in}{1.835890in}}%
\pgfpathlineto{\pgfqpoint{3.386061in}{1.792910in}}%
\pgfpathlineto{\pgfqpoint{3.461212in}{1.747732in}}%
\pgfpathlineto{\pgfqpoint{3.536364in}{1.700335in}}%
\pgfpathlineto{\pgfqpoint{3.611515in}{1.650696in}}%
\pgfpathlineto{\pgfqpoint{3.686667in}{1.598794in}}%
\pgfpathlineto{\pgfqpoint{3.761818in}{1.544608in}}%
\pgfpathlineto{\pgfqpoint{3.836970in}{1.488115in}}%
\pgfpathlineto{\pgfqpoint{3.912121in}{1.429294in}}%
\pgfpathlineto{\pgfqpoint{3.987273in}{1.368124in}}%
\pgfpathlineto{\pgfqpoint{4.062424in}{1.304582in}}%
\pgfpathlineto{\pgfqpoint{4.137576in}{1.238646in}}%
\pgfpathlineto{\pgfqpoint{4.212727in}{1.170296in}}%
\pgfpathlineto{\pgfqpoint{4.287879in}{1.099509in}}%
\pgfpathlineto{\pgfqpoint{4.363030in}{1.026264in}}%
\pgfpathlineto{\pgfqpoint{4.456970in}{0.931217in}}%
\pgfpathlineto{\pgfqpoint{4.550909in}{0.832253in}}%
\pgfpathlineto{\pgfqpoint{4.626061in}{0.750233in}}%
\pgfpathlineto{\pgfqpoint{4.663636in}{0.708319in}}%
\pgfpathlineto{\pgfqpoint{4.832727in}{0.788406in}}%
\pgfpathlineto{\pgfqpoint{4.999007in}{0.864588in}}%
\pgfpathlineto{\pgfqpoint{5.168098in}{0.939425in}}%
\pgfpathlineto{\pgfqpoint{5.337189in}{1.011616in}}%
\pgfpathlineto{\pgfqpoint{5.506280in}{1.081159in}}%
\pgfpathlineto{\pgfqpoint{5.675371in}{1.148055in}}%
\pgfpathlineto{\pgfqpoint{5.828485in}{1.206348in}}%
\pgfpathlineto{\pgfqpoint{5.997576in}{1.268201in}}%
\pgfpathlineto{\pgfqpoint{6.166667in}{1.327407in}}%
\pgfpathlineto{\pgfqpoint{6.316970in}{1.377813in}}%
\pgfpathlineto{\pgfqpoint{6.467273in}{1.426127in}}%
\pgfpathlineto{\pgfqpoint{6.617576in}{1.472350in}}%
\pgfpathlineto{\pgfqpoint{6.843030in}{1.538765in}}%
\pgfpathlineto{\pgfqpoint{6.918182in}{1.560809in}}%
\pgfpathlineto{\pgfqpoint{6.918182in}{1.329248in}}%
\pgfpathlineto{\pgfqpoint{6.918182in}{1.329248in}}%
\pgfusepath{stroke}%
\end{pgfscope}%
\begin{pgfscope}%
\pgfpathrectangle{\pgfqpoint{1.000000in}{0.330000in}}{\pgfqpoint{6.200000in}{2.310000in}}%
\pgfusepath{clip}%
\pgfsetrectcap%
\pgfsetroundjoin%
\pgfsetlinewidth{1.505625pt}%
\definecolor{currentstroke}{rgb}{0.090196,0.745098,0.811765}%
\pgfsetstrokecolor{currentstroke}%
\pgfsetdash{}{0pt}%
\pgfpathmoveto{\pgfqpoint{1.281818in}{1.329248in}}%
\pgfpathlineto{\pgfqpoint{1.281818in}{1.329248in}}%
\pgfpathlineto{\pgfqpoint{1.394545in}{1.403091in}}%
\pgfpathlineto{\pgfqpoint{1.507273in}{1.474323in}}%
\pgfpathlineto{\pgfqpoint{1.620000in}{1.542910in}}%
\pgfpathlineto{\pgfqpoint{1.732727in}{1.608813in}}%
\pgfpathlineto{\pgfqpoint{1.845455in}{1.671997in}}%
\pgfpathlineto{\pgfqpoint{1.958182in}{1.732425in}}%
\pgfpathlineto{\pgfqpoint{2.070909in}{1.790059in}}%
\pgfpathlineto{\pgfqpoint{2.183636in}{1.844865in}}%
\pgfpathlineto{\pgfqpoint{2.296364in}{1.896804in}}%
\pgfpathlineto{\pgfqpoint{2.409091in}{1.945840in}}%
\pgfpathlineto{\pgfqpoint{2.503030in}{1.927233in}}%
\pgfpathlineto{\pgfqpoint{2.596970in}{1.906564in}}%
\pgfpathlineto{\pgfqpoint{2.690909in}{1.883811in}}%
\pgfpathlineto{\pgfqpoint{2.784848in}{1.858953in}}%
\pgfpathlineto{\pgfqpoint{2.878788in}{1.831969in}}%
\pgfpathlineto{\pgfqpoint{2.972727in}{1.802838in}}%
\pgfpathlineto{\pgfqpoint{3.066667in}{1.771539in}}%
\pgfpathlineto{\pgfqpoint{3.160606in}{1.738050in}}%
\pgfpathlineto{\pgfqpoint{3.254545in}{1.702350in}}%
\pgfpathlineto{\pgfqpoint{3.348485in}{1.664418in}}%
\pgfpathlineto{\pgfqpoint{3.442424in}{1.624233in}}%
\pgfpathlineto{\pgfqpoint{3.536364in}{1.581773in}}%
\pgfpathlineto{\pgfqpoint{3.630303in}{1.537018in}}%
\pgfpathlineto{\pgfqpoint{3.724242in}{1.489946in}}%
\pgfpathlineto{\pgfqpoint{3.818182in}{1.440536in}}%
\pgfpathlineto{\pgfqpoint{3.912121in}{1.388767in}}%
\pgfpathlineto{\pgfqpoint{4.006061in}{1.334617in}}%
\pgfpathlineto{\pgfqpoint{4.100000in}{1.278066in}}%
\pgfpathlineto{\pgfqpoint{4.193939in}{1.219092in}}%
\pgfpathlineto{\pgfqpoint{4.287879in}{1.157674in}}%
\pgfpathlineto{\pgfqpoint{4.381818in}{1.093791in}}%
\pgfpathlineto{\pgfqpoint{4.475758in}{1.027422in}}%
\pgfpathlineto{\pgfqpoint{4.569697in}{0.958544in}}%
\pgfpathlineto{\pgfqpoint{4.663636in}{0.887173in}}%
\pgfpathlineto{\pgfqpoint{4.813939in}{0.942859in}}%
\pgfpathlineto{\pgfqpoint{4.945455in}{0.989584in}}%
\pgfpathlineto{\pgfqpoint{5.092947in}{1.039772in}}%
\pgfpathlineto{\pgfqpoint{5.243250in}{1.088490in}}%
\pgfpathlineto{\pgfqpoint{5.393553in}{1.134768in}}%
\pgfpathlineto{\pgfqpoint{5.543856in}{1.178606in}}%
\pgfpathlineto{\pgfqpoint{5.694159in}{1.220004in}}%
\pgfpathlineto{\pgfqpoint{5.847273in}{1.259668in}}%
\pgfpathlineto{\pgfqpoint{5.997576in}{1.296141in}}%
\pgfpathlineto{\pgfqpoint{6.147879in}{1.330174in}}%
\pgfpathlineto{\pgfqpoint{6.298182in}{1.361766in}}%
\pgfpathlineto{\pgfqpoint{6.448485in}{1.390919in}}%
\pgfpathlineto{\pgfqpoint{6.598788in}{1.417632in}}%
\pgfpathlineto{\pgfqpoint{6.767879in}{1.445048in}}%
\pgfpathlineto{\pgfqpoint{6.918182in}{1.468965in}}%
\pgfpathlineto{\pgfqpoint{6.918182in}{1.329248in}}%
\pgfpathlineto{\pgfqpoint{6.918182in}{1.329248in}}%
\pgfusepath{stroke}%
\end{pgfscope}%
\begin{pgfscope}%
\pgfpathrectangle{\pgfqpoint{1.000000in}{0.330000in}}{\pgfqpoint{6.200000in}{2.310000in}}%
\pgfusepath{clip}%
\pgfsetrectcap%
\pgfsetroundjoin%
\pgfsetlinewidth{1.505625pt}%
\definecolor{currentstroke}{rgb}{0.121569,0.466667,0.705882}%
\pgfsetstrokecolor{currentstroke}%
\pgfsetdash{}{0pt}%
\pgfpathmoveto{\pgfqpoint{1.281818in}{1.329248in}}%
\pgfpathlineto{\pgfqpoint{1.281818in}{1.329248in}}%
\pgfpathlineto{\pgfqpoint{1.432121in}{1.380581in}}%
\pgfpathlineto{\pgfqpoint{1.582424in}{1.429473in}}%
\pgfpathlineto{\pgfqpoint{1.732727in}{1.475926in}}%
\pgfpathlineto{\pgfqpoint{1.883030in}{1.519939in}}%
\pgfpathlineto{\pgfqpoint{2.033333in}{1.561512in}}%
\pgfpathlineto{\pgfqpoint{2.183636in}{1.600646in}}%
\pgfpathlineto{\pgfqpoint{2.333939in}{1.637339in}}%
\pgfpathlineto{\pgfqpoint{2.409091in}{1.654771in}}%
\pgfpathlineto{\pgfqpoint{2.559394in}{1.634393in}}%
\pgfpathlineto{\pgfqpoint{2.709697in}{1.611575in}}%
\pgfpathlineto{\pgfqpoint{2.860000in}{1.586317in}}%
\pgfpathlineto{\pgfqpoint{3.010303in}{1.558620in}}%
\pgfpathlineto{\pgfqpoint{3.160606in}{1.528482in}}%
\pgfpathlineto{\pgfqpoint{3.310909in}{1.495905in}}%
\pgfpathlineto{\pgfqpoint{3.461212in}{1.460887in}}%
\pgfpathlineto{\pgfqpoint{3.611515in}{1.423430in}}%
\pgfpathlineto{\pgfqpoint{3.761818in}{1.383533in}}%
\pgfpathlineto{\pgfqpoint{3.912121in}{1.341196in}}%
\pgfpathlineto{\pgfqpoint{4.062424in}{1.296419in}}%
\pgfpathlineto{\pgfqpoint{4.212727in}{1.249203in}}%
\pgfpathlineto{\pgfqpoint{4.363030in}{1.199546in}}%
\pgfpathlineto{\pgfqpoint{4.513333in}{1.147449in}}%
\pgfpathlineto{\pgfqpoint{4.663636in}{1.092926in}}%
\pgfpathlineto{\pgfqpoint{4.757576in}{1.120484in}}%
\pgfpathlineto{\pgfqpoint{4.851515in}{1.145471in}}%
\pgfpathlineto{\pgfqpoint{4.945455in}{1.167866in}}%
\pgfpathlineto{\pgfqpoint{5.074159in}{1.195110in}}%
\pgfpathlineto{\pgfqpoint{5.224462in}{1.224473in}}%
\pgfpathlineto{\pgfqpoint{5.374765in}{1.251395in}}%
\pgfpathlineto{\pgfqpoint{5.525068in}{1.275878in}}%
\pgfpathlineto{\pgfqpoint{5.675371in}{1.297921in}}%
\pgfpathlineto{\pgfqpoint{5.809697in}{1.315549in}}%
\pgfpathlineto{\pgfqpoint{5.922424in}{1.328273in}}%
\pgfpathlineto{\pgfqpoint{6.035152in}{1.338731in}}%
\pgfpathlineto{\pgfqpoint{6.147879in}{1.346922in}}%
\pgfpathlineto{\pgfqpoint{6.260606in}{1.352847in}}%
\pgfpathlineto{\pgfqpoint{6.373333in}{1.356505in}}%
\pgfpathlineto{\pgfqpoint{6.486061in}{1.357896in}}%
\pgfpathlineto{\pgfqpoint{6.598788in}{1.357021in}}%
\pgfpathlineto{\pgfqpoint{6.711515in}{1.353891in}}%
\pgfpathlineto{\pgfqpoint{6.880606in}{1.346758in}}%
\pgfpathlineto{\pgfqpoint{6.918182in}{1.344900in}}%
\pgfpathlineto{\pgfqpoint{6.918182in}{1.329248in}}%
\pgfpathlineto{\pgfqpoint{6.918182in}{1.329248in}}%
\pgfusepath{stroke}%
\end{pgfscope}%
\begin{pgfscope}%
\pgfpathrectangle{\pgfqpoint{1.000000in}{0.330000in}}{\pgfqpoint{6.200000in}{2.310000in}}%
\pgfusepath{clip}%
\pgfsetrectcap%
\pgfsetroundjoin%
\pgfsetlinewidth{1.505625pt}%
\definecolor{currentstroke}{rgb}{1.000000,0.498039,0.054902}%
\pgfsetstrokecolor{currentstroke}%
\pgfsetdash{}{0pt}%
\pgfpathmoveto{\pgfqpoint{1.281818in}{1.329248in}}%
\pgfpathlineto{\pgfqpoint{1.281818in}{1.329248in}}%
\pgfpathlineto{\pgfqpoint{1.375758in}{1.450512in}}%
\pgfpathlineto{\pgfqpoint{1.469697in}{1.568223in}}%
\pgfpathlineto{\pgfqpoint{1.563636in}{1.682313in}}%
\pgfpathlineto{\pgfqpoint{1.657576in}{1.792715in}}%
\pgfpathlineto{\pgfqpoint{1.751515in}{1.899361in}}%
\pgfpathlineto{\pgfqpoint{1.845455in}{2.002182in}}%
\pgfpathlineto{\pgfqpoint{1.939394in}{2.101111in}}%
\pgfpathlineto{\pgfqpoint{2.014545in}{2.177406in}}%
\pgfpathlineto{\pgfqpoint{2.089697in}{2.251132in}}%
\pgfpathlineto{\pgfqpoint{2.164848in}{2.322254in}}%
\pgfpathlineto{\pgfqpoint{2.240000in}{2.390738in}}%
\pgfpathlineto{\pgfqpoint{2.315152in}{2.456549in}}%
\pgfpathlineto{\pgfqpoint{2.390303in}{2.519652in}}%
\pgfpathlineto{\pgfqpoint{2.409091in}{2.535000in}}%
\pgfpathlineto{\pgfqpoint{2.484242in}{2.510737in}}%
\pgfpathlineto{\pgfqpoint{2.559394in}{2.483687in}}%
\pgfpathlineto{\pgfqpoint{2.634545in}{2.453817in}}%
\pgfpathlineto{\pgfqpoint{2.709697in}{2.421091in}}%
\pgfpathlineto{\pgfqpoint{2.784848in}{2.385474in}}%
\pgfpathlineto{\pgfqpoint{2.860000in}{2.346933in}}%
\pgfpathlineto{\pgfqpoint{2.935152in}{2.305432in}}%
\pgfpathlineto{\pgfqpoint{3.010303in}{2.260936in}}%
\pgfpathlineto{\pgfqpoint{3.085455in}{2.213411in}}%
\pgfpathlineto{\pgfqpoint{3.160606in}{2.162822in}}%
\pgfpathlineto{\pgfqpoint{3.235758in}{2.109135in}}%
\pgfpathlineto{\pgfqpoint{3.310909in}{2.052314in}}%
\pgfpathlineto{\pgfqpoint{3.386061in}{1.992325in}}%
\pgfpathlineto{\pgfqpoint{3.461212in}{1.929133in}}%
\pgfpathlineto{\pgfqpoint{3.536364in}{1.862703in}}%
\pgfpathlineto{\pgfqpoint{3.611515in}{1.793002in}}%
\pgfpathlineto{\pgfqpoint{3.686667in}{1.719993in}}%
\pgfpathlineto{\pgfqpoint{3.761818in}{1.643642in}}%
\pgfpathlineto{\pgfqpoint{3.836970in}{1.563915in}}%
\pgfpathlineto{\pgfqpoint{3.912121in}{1.480776in}}%
\pgfpathlineto{\pgfqpoint{3.987273in}{1.394192in}}%
\pgfpathlineto{\pgfqpoint{4.062424in}{1.304126in}}%
\pgfpathlineto{\pgfqpoint{4.137576in}{1.210546in}}%
\pgfpathlineto{\pgfqpoint{4.212727in}{1.113415in}}%
\pgfpathlineto{\pgfqpoint{4.287879in}{1.012699in}}%
\pgfpathlineto{\pgfqpoint{4.363030in}{0.908364in}}%
\pgfpathlineto{\pgfqpoint{4.438182in}{0.800374in}}%
\pgfpathlineto{\pgfqpoint{4.513333in}{0.688696in}}%
\pgfpathlineto{\pgfqpoint{4.588485in}{0.573293in}}%
\pgfpathlineto{\pgfqpoint{4.663636in}{0.454211in}}%
\pgfpathlineto{\pgfqpoint{4.832727in}{0.562160in}}%
\pgfpathlineto{\pgfqpoint{4.999007in}{0.665743in}}%
\pgfpathlineto{\pgfqpoint{5.168098in}{0.768442in}}%
\pgfpathlineto{\pgfqpoint{5.337189in}{0.868495in}}%
\pgfpathlineto{\pgfqpoint{5.506280in}{0.965901in}}%
\pgfpathlineto{\pgfqpoint{5.675371in}{1.060660in}}%
\pgfpathlineto{\pgfqpoint{5.847273in}{1.154282in}}%
\pgfpathlineto{\pgfqpoint{6.016364in}{1.243704in}}%
\pgfpathlineto{\pgfqpoint{6.185455in}{1.330478in}}%
\pgfpathlineto{\pgfqpoint{6.354545in}{1.414606in}}%
\pgfpathlineto{\pgfqpoint{6.523636in}{1.496087in}}%
\pgfpathlineto{\pgfqpoint{6.692727in}{1.574921in}}%
\pgfpathlineto{\pgfqpoint{6.918182in}{1.678203in}}%
\pgfpathlineto{\pgfqpoint{6.918182in}{1.329248in}}%
\pgfpathlineto{\pgfqpoint{6.918182in}{1.329248in}}%
\pgfusepath{stroke}%
\end{pgfscope}%
\begin{pgfscope}%
\pgfpathrectangle{\pgfqpoint{1.000000in}{0.330000in}}{\pgfqpoint{6.200000in}{2.310000in}}%
\pgfusepath{clip}%
\pgfsetrectcap%
\pgfsetroundjoin%
\pgfsetlinewidth{1.505625pt}%
\definecolor{currentstroke}{rgb}{0.172549,0.627451,0.172549}%
\pgfsetstrokecolor{currentstroke}%
\pgfsetdash{}{0pt}%
\pgfpathmoveto{\pgfqpoint{1.281818in}{1.329248in}}%
\pgfpathlineto{\pgfqpoint{1.281818in}{1.329248in}}%
\pgfpathlineto{\pgfqpoint{1.432121in}{1.380847in}}%
\pgfpathlineto{\pgfqpoint{1.582424in}{1.430007in}}%
\pgfpathlineto{\pgfqpoint{1.732727in}{1.476727in}}%
\pgfpathlineto{\pgfqpoint{1.883030in}{1.521007in}}%
\pgfpathlineto{\pgfqpoint{2.033333in}{1.562847in}}%
\pgfpathlineto{\pgfqpoint{2.183636in}{1.602247in}}%
\pgfpathlineto{\pgfqpoint{2.333939in}{1.639207in}}%
\pgfpathlineto{\pgfqpoint{2.409091in}{1.656772in}}%
\pgfpathlineto{\pgfqpoint{2.559394in}{1.636661in}}%
\pgfpathlineto{\pgfqpoint{2.709697in}{1.614110in}}%
\pgfpathlineto{\pgfqpoint{2.860000in}{1.589119in}}%
\pgfpathlineto{\pgfqpoint{3.010303in}{1.561688in}}%
\pgfpathlineto{\pgfqpoint{3.160606in}{1.531817in}}%
\pgfpathlineto{\pgfqpoint{3.310909in}{1.499507in}}%
\pgfpathlineto{\pgfqpoint{3.461212in}{1.464756in}}%
\pgfpathlineto{\pgfqpoint{3.611515in}{1.427566in}}%
\pgfpathlineto{\pgfqpoint{3.761818in}{1.387936in}}%
\pgfpathlineto{\pgfqpoint{3.912121in}{1.345866in}}%
\pgfpathlineto{\pgfqpoint{4.062424in}{1.301355in}}%
\pgfpathlineto{\pgfqpoint{4.212727in}{1.254406in}}%
\pgfpathlineto{\pgfqpoint{4.363030in}{1.205016in}}%
\pgfpathlineto{\pgfqpoint{4.513333in}{1.153186in}}%
\pgfpathlineto{\pgfqpoint{4.663636in}{1.098929in}}%
\pgfpathlineto{\pgfqpoint{4.813939in}{1.133977in}}%
\pgfpathlineto{\pgfqpoint{4.945455in}{1.162642in}}%
\pgfpathlineto{\pgfqpoint{5.092947in}{1.192578in}}%
\pgfpathlineto{\pgfqpoint{5.243250in}{1.220656in}}%
\pgfpathlineto{\pgfqpoint{5.393553in}{1.246295in}}%
\pgfpathlineto{\pgfqpoint{5.543856in}{1.269494in}}%
\pgfpathlineto{\pgfqpoint{5.694159in}{1.290253in}}%
\pgfpathlineto{\pgfqpoint{5.847273in}{1.308893in}}%
\pgfpathlineto{\pgfqpoint{5.997576in}{1.324726in}}%
\pgfpathlineto{\pgfqpoint{6.147879in}{1.338120in}}%
\pgfpathlineto{\pgfqpoint{6.298182in}{1.349074in}}%
\pgfpathlineto{\pgfqpoint{6.448485in}{1.357588in}}%
\pgfpathlineto{\pgfqpoint{6.598788in}{1.363662in}}%
\pgfpathlineto{\pgfqpoint{6.767879in}{1.367858in}}%
\pgfpathlineto{\pgfqpoint{6.918182in}{1.371137in}}%
\pgfpathlineto{\pgfqpoint{6.918182in}{1.329248in}}%
\pgfpathlineto{\pgfqpoint{6.918182in}{1.329248in}}%
\pgfusepath{stroke}%
\end{pgfscope}%
\begin{pgfscope}%
\pgfsetroundcap%
\pgfsetroundjoin%
\pgfsetlinewidth{1.003750pt}%
\definecolor{currentstroke}{rgb}{0.000000,0.000000,0.000000}%
\pgfsetstrokecolor{currentstroke}%
\pgfsetdash{}{0pt}%
\pgfpathmoveto{\pgfqpoint{2.842287in}{2.535000in}}%
\pgfpathquadraticcurveto{\pgfqpoint{2.639583in}{2.535000in}}{\pgfqpoint{2.436880in}{2.535000in}}%
\pgfusepath{stroke}%
\end{pgfscope}%
\begin{pgfscope}%
\pgfsetbuttcap%
\pgfsetmiterjoin%
\definecolor{currentfill}{rgb}{0.800000,0.800000,0.800000}%
\pgfsetfillcolor{currentfill}%
\pgfsetlinewidth{1.003750pt}%
\definecolor{currentstroke}{rgb}{0.000000,0.000000,0.000000}%
\pgfsetstrokecolor{currentstroke}%
\pgfsetdash{}{0pt}%
\pgfpathmoveto{\pgfqpoint{2.900039in}{2.438549in}}%
\pgfpathcurveto{\pgfqpoint{2.934761in}{2.403827in}}{\pgfqpoint{3.797980in}{2.403827in}}{\pgfqpoint{3.832702in}{2.438549in}}%
\pgfpathcurveto{\pgfqpoint{3.867424in}{2.473272in}}{\pgfqpoint{3.867424in}{2.596728in}}{\pgfqpoint{3.832702in}{2.631451in}}%
\pgfpathcurveto{\pgfqpoint{3.797980in}{2.666173in}}{\pgfqpoint{2.934761in}{2.666173in}}{\pgfqpoint{2.900039in}{2.631451in}}%
\pgfpathcurveto{\pgfqpoint{2.865317in}{2.596728in}}{\pgfqpoint{2.865317in}{2.473272in}}{\pgfqpoint{2.900039in}{2.438549in}}%
\pgfpathclose%
\pgfusepath{stroke,fill}%
\end{pgfscope}%
\begin{pgfscope}%
\definecolor{textcolor}{rgb}{0.000000,0.000000,0.000000}%
\pgfsetstrokecolor{textcolor}%
\pgfsetfillcolor{textcolor}%
\pgftext[x=3.797980in,y=2.535000in,right,]{\color{textcolor}\rmfamily\fontsize{10.000000}{12.000000}\selectfont \(\displaystyle M_u =\) 42.1 kft}%
\end{pgfscope}%
\begin{pgfscope}%
\pgfsetbuttcap%
\pgfsetmiterjoin%
\definecolor{currentfill}{rgb}{0.800000,0.800000,0.800000}%
\pgfsetfillcolor{currentfill}%
\pgfsetlinewidth{1.003750pt}%
\definecolor{currentstroke}{rgb}{0.000000,0.000000,0.000000}%
\pgfsetstrokecolor{currentstroke}%
\pgfsetdash{}{0pt}%
\pgfpathmoveto{\pgfqpoint{0.965278in}{0.358599in}}%
\pgfpathcurveto{\pgfqpoint{1.000000in}{0.323877in}}{\pgfqpoint{2.720682in}{0.323877in}}{\pgfqpoint{2.755404in}{0.358599in}}%
\pgfpathcurveto{\pgfqpoint{2.790127in}{0.393321in}}{\pgfqpoint{2.790127in}{0.668784in}}{\pgfqpoint{2.755404in}{0.703506in}}%
\pgfpathcurveto{\pgfqpoint{2.720682in}{0.738228in}}{\pgfqpoint{1.000000in}{0.738228in}}{\pgfqpoint{0.965278in}{0.703506in}}%
\pgfpathcurveto{\pgfqpoint{0.930556in}{0.668784in}}{\pgfqpoint{0.930556in}{0.393321in}}{\pgfqpoint{0.965278in}{0.358599in}}%
\pgfpathclose%
\pgfusepath{stroke,fill}%
\end{pgfscope}%
\begin{pgfscope}%
\definecolor{textcolor}{rgb}{0.000000,0.000000,0.000000}%
\pgfsetstrokecolor{textcolor}%
\pgfsetfillcolor{textcolor}%
\pgftext[x=1.000000in, y=0.580049in, left, base]{\color{textcolor}\rmfamily\fontsize{10.000000}{12.000000}\selectfont Max combo: 1.2D + 1.6L0}%
\end{pgfscope}%
\begin{pgfscope}%
\definecolor{textcolor}{rgb}{0.000000,0.000000,0.000000}%
\pgfsetstrokecolor{textcolor}%
\pgfsetfillcolor{textcolor}%
\pgftext[x=1.000000in, y=0.428043in, left, base]{\color{textcolor}\rmfamily\fontsize{10.000000}{12.000000}\selectfont ASCE7-16 Sec. 2.3.1 (LC 2)}%
\end{pgfscope}%
\end{pgfpicture}%
\makeatother%
\endgroup%

\end{center}
\caption{Moment Demand Envelope}
\end{figure}
L\textsubscript{p}, the limiting laterally unbraced length for the limit state of yielding, is calculated per AISC/ANSI 360-16 Eq. F2-5 as follows:
\begin{flalign*}
L_p = 1.76\cdot r_y \cdot \sqrt{\frac{E}{F_y}}  = 1.76\cdot 1.33 {\color{darkBlue}{\mathbf{ \; in}}} \cdot \sqrt{\frac{29000 {\color{darkBlue}{\mathbf{ \; ksi}}}}{50 {\color{darkBlue}{\mathbf{ \; ksi}}}}}  = \mathbf{4.7 {\color{darkBlue}{\mathbf{ \; ft}}}}
\end{flalign*}
r\textsubscript{ts}, a coefficient used in the calculation of L\textsubscript{r} and C\textsubscript{b}, is calculated per AISC/ANSI 360-16 Eq. F2-7 as follows:
\begin{flalign*}
r_{{ts}} = \sqrt{\frac{\sqrt{I_y \cdot C_w}}{S_x}}  = \sqrt{\frac{\sqrt{11.4 {\color{darkBlue}{\mathbf{ \; {\color{darkBlue}{\mathbf{ \; in}}}^{4}}}} \cdot 275 {\color{darkBlue}{\mathbf{ \; {\color{darkBlue}{\mathbf{ \; in}}}^{6}}}}}}{23.2 {\color{darkBlue}{\mathbf{ \; {\color{darkBlue}{\mathbf{ \; in}}}^{3}}}}}}  = \mathbf{1.6 {\color{darkBlue}{\mathbf{ \; in}}}}
\end{flalign*}
L\textsubscript{r}, the limiting unbraced length for the limit state of inelastic lateral-torsional buckling, is calculated per AISC/ANSI 360-16 Eq. F2-6 as follows:
\begin{flalign*}
L_r &= 1.95\cdot r_{ts} \cdot \frac{E}{0.7\cdot F_y} \sqrt{\frac{J \cdot c}{S_x \cdot h_0}+\sqrt{{\left(\frac{J \cdot c}{S_x \cdot h_0}\right)}^2+6.76{\left(\frac{0.7\cdot F_y}{E}\right)}^2}} \\ &= 1.95\cdot 1.6 {\color{darkBlue}{\mathbf{ \; in}}} \cdot \frac{29000 {\color{darkBlue}{\mathbf{ \; ksi}}}}{0.7\cdot 50 {\color{darkBlue}{\mathbf{ \; ksi}}}} \sqrt{\frac{0.24 {\color{darkBlue}{\mathbf{ \; {\color{darkBlue}{\mathbf{ \; in}}}^{4}}}} \cdot 1}{23.2 {\color{darkBlue}{\mathbf{ \; {\color{darkBlue}{\mathbf{ \; in}}}^{3}}}} \cdot 9.8 {\color{darkBlue}{\mathbf{ \; in}}}}+\sqrt{{\left(\frac{0.24 {\color{darkBlue}{\mathbf{ \; {\color{darkBlue}{\mathbf{ \; in}}}^{4}}}} \cdot 1}{23.2 {\color{darkBlue}{\mathbf{ \; {\color{darkBlue}{\mathbf{ \; in}}}^{3}}}} \cdot 9.8 {\color{darkBlue}{\mathbf{ \; in}}}}\right)}^2+6.76{\left(\frac{0.7\cdot 50 {\color{darkBlue}{\mathbf{ \; ksi}}}}{29000 {\color{darkBlue}{\mathbf{ \; ksi}}}}\right)}^2}} \\ &= \mathbf{13.8 {\color{darkBlue}{\mathbf{ \; ft}}}}
\end{flalign*}
\textlambda\textsubscript{web}, the web width-to-thickness ratio, is calculated per {AISC/ANSI 360-16 Table B4.1b} as follows:
\begin{flalign*}
\lambda_{{web}} = \frac{h}{t_w}  = \frac{8.88 {\color{darkBlue}{\mathbf{ \; in}}}}{0.24 {\color{darkBlue}{\mathbf{ \; in}}}}  = \mathbf{37.0 }
\end{flalign*}
\textlambda\textsubscript{P-web}, the limiting width-to-thickness ratio for compact/noncompact web, is calculated per {AISC/ANSI 360-16 Table B4.1b} as follows:
\begin{flalign*}
\lambda_{P-web} = 3.76\cdot \sqrt{\frac{E}{F_y}} = 3.76\cdot \sqrt{\frac{29000 {\color{darkBlue}{\mathbf{ \; ksi}}}}{50 {\color{darkBlue}{\mathbf{ \; ksi}}}}} = \mathbf{90.6}
\end{flalign*}
\textlambda\textsubscript{R-web}, the limiting width-to-thickness ratio for noncompact/slender web, is calculated per {AISC/ANSI 360-16 Table B4.1b} as follows:
\begin{flalign*}
\lambda_{R-web} = 5.7\cdot \sqrt{\frac{E}{F_y}} = 5.7\cdot \sqrt{\frac{29000 {\color{darkBlue}{\mathbf{ \; ksi}}}}{50 {\color{darkBlue}{\mathbf{ \; ksi}}}}} = \mathbf{137.3}
\end{flalign*}
\textlambda\textsubscript{web} $<$ \textlambda\textsubscript{P-web} \textrightarrow \; \textbf{Compact Web}
\\\\
\textlambda\textsubscript{flange}, the flange width-to-thickness ratio, is calculated per {AISC/ANSI 360-16 Table B4.1b} as follows:
\begin{flalign*}
\lambda_{{flange}} = \frac{b}{t}  = \frac{2.88 {\color{darkBlue}{\mathbf{ \; in}}}}{0.36 {\color{darkBlue}{\mathbf{ \; in}}}}  = \mathbf{8.0 }
\end{flalign*}
\textlambda\textsubscript{P-flange}, the limiting width-to-thickness ratio for compact/noncompact flange, is calculated per {AISC/ANSI 360-16 Table B4.1b} as follows:
\begin{flalign*}
\lambda_{P-flange} = 0.38\cdot \sqrt{\frac{E}{F_y}} = 0.38\cdot \sqrt{\frac{29000 {\color{darkBlue}{\mathbf{ \; ksi}}}}{50 {\color{darkBlue}{\mathbf{ \; ksi}}}}} = \mathbf{9.2}
\end{flalign*}
\textlambda\textsubscript{R-flange}, the limiting width-to-thickness ratio for noncompact/slender flange, is calculated per {AISC/ANSI 360-16 Table B4.1b} as follows:
\begin{flalign*}
\lambda_{R-flange} = 1\cdot \sqrt{\frac{E}{F_y}} = 1\cdot \sqrt{\frac{29000 {\color{darkBlue}{\mathbf{ \; ksi}}}}{50 {\color{darkBlue}{\mathbf{ \; ksi}}}}} = \mathbf{24.1}
\end{flalign*}
\textlambda\textsubscript{flange} $<$ \textlambda\textsubscript{P-flange} \textrightarrow \; \textbf{Compact Flange}
\\\\
Since \(\mathbf{{L_b} > {L_r}}\) and the beam's flanges are \textbf{compact}, controlling limit state for flexure is \textbf{inelastic LTB (not to exceed capacity based on yielding)}.
\\\\
M\textsubscript{p}, the plastic bending moment, is calculated per AISC/ANSI 360-16 Eq. F2-1 as follows:
\begin{flalign*}
M_p = F_y \cdot Z_x  = 50 {\color{darkBlue}{\mathbf{ \; ksi}}} \cdot 26 {\color{darkBlue}{\mathbf{ \; {\color{darkBlue}{\mathbf{ \; in}}}^{3}}}}  = \mathbf{108.3 {\color{darkBlue}{\mathbf{ \; kft}}}}
\end{flalign*}
C\textsubscript{b}, the lateral-torsional buckling modification factor in the critical unbraced span for the critical load combination, is calculated per AISC/ANSI 360- 16 Sec. F1 as follows:
\\
\begin{flalign*}
C_b &= \frac{12.5\cdot M_{max}}{2.5\cdot M_{max}+3\cdot M_A+4\cdot M_B+3\cdot M_C} \\ &= \frac{12.5\cdot 42.1 {\color{darkBlue}{\mathbf{ \; kft}}}}{2.5\cdot 42.1 {\color{darkBlue}{\mathbf{ \; kft}}}+3\cdot 32.8 {\color{darkBlue}{\mathbf{ \; kft}}}+4\cdot 33.0 {\color{darkBlue}{\mathbf{ \; kft}}}+3\cdot 8.2 {\color{darkBlue}{\mathbf{ \; kft}}}} \\ &= \mathbf{1.5 }
\end{flalign*}
\\
For brevity, the C\textsubscript{b} calculation is not shown for each span. The following figure illustrates the value of C\textsubscript{b} for each span.
\begin{figure}[H]
\begin{center}
%% Creator: Matplotlib, PGF backend
%%
%% To include the figure in your LaTeX document, write
%%   \input{<filename>.pgf}
%%
%% Make sure the required packages are loaded in your preamble
%%   \usepackage{pgf}
%%
%% Figures using additional raster images can only be included by \input if
%% they are in the same directory as the main LaTeX file. For loading figures
%% from other directories you can use the `import` package
%%   \usepackage{import}
%%
%% and then include the figures with
%%   \import{<path to file>}{<filename>.pgf}
%%
%% Matplotlib used the following preamble
%%
\begingroup%
\makeatletter%
\begin{pgfpicture}%
\pgfpathrectangle{\pgfpointorigin}{\pgfqpoint{8.000000in}{1.000000in}}%
\pgfusepath{use as bounding box, clip}%
\begin{pgfscope}%
\pgfpathrectangle{\pgfqpoint{1.000000in}{0.110000in}}{\pgfqpoint{6.200000in}{0.770000in}}%
\pgfusepath{clip}%
\pgfsetrectcap%
\pgfsetroundjoin%
\pgfsetlinewidth{0.752812pt}%
\definecolor{currentstroke}{rgb}{0.000000,0.000000,0.000000}%
\pgfsetstrokecolor{currentstroke}%
\pgfsetdash{}{0pt}%
\pgfpathmoveto{\pgfqpoint{1.281818in}{0.880000in}}%
\pgfpathlineto{\pgfqpoint{4.663636in}{0.880000in}}%
\pgfpathlineto{\pgfqpoint{6.918182in}{0.880000in}}%
\pgfusepath{stroke}%
\end{pgfscope}%
\begin{pgfscope}%
\pgfpathrectangle{\pgfqpoint{1.000000in}{0.110000in}}{\pgfqpoint{6.200000in}{0.770000in}}%
\pgfusepath{clip}%
\pgfsetrectcap%
\pgfsetroundjoin%
\pgfsetlinewidth{0.752812pt}%
\definecolor{currentstroke}{rgb}{0.000000,0.000000,0.000000}%
\pgfsetstrokecolor{currentstroke}%
\pgfsetdash{}{0pt}%
\pgfpathmoveto{\pgfqpoint{1.281818in}{0.852824in}}%
\pgfpathlineto{\pgfqpoint{4.663636in}{0.852824in}}%
\pgfpathlineto{\pgfqpoint{6.918182in}{0.852824in}}%
\pgfusepath{stroke}%
\end{pgfscope}%
\begin{pgfscope}%
\pgfpathrectangle{\pgfqpoint{1.000000in}{0.110000in}}{\pgfqpoint{6.200000in}{0.770000in}}%
\pgfusepath{clip}%
\pgfsetrectcap%
\pgfsetroundjoin%
\pgfsetlinewidth{0.752812pt}%
\definecolor{currentstroke}{rgb}{0.000000,0.000000,0.000000}%
\pgfsetstrokecolor{currentstroke}%
\pgfsetdash{}{0pt}%
\pgfpathmoveto{\pgfqpoint{1.281818in}{0.110000in}}%
\pgfpathlineto{\pgfqpoint{4.663636in}{0.110000in}}%
\pgfpathlineto{\pgfqpoint{6.918182in}{0.110000in}}%
\pgfusepath{stroke}%
\end{pgfscope}%
\begin{pgfscope}%
\pgfpathrectangle{\pgfqpoint{1.000000in}{0.110000in}}{\pgfqpoint{6.200000in}{0.770000in}}%
\pgfusepath{clip}%
\pgfsetrectcap%
\pgfsetroundjoin%
\pgfsetlinewidth{0.752812pt}%
\definecolor{currentstroke}{rgb}{0.000000,0.000000,0.000000}%
\pgfsetstrokecolor{currentstroke}%
\pgfsetdash{}{0pt}%
\pgfpathmoveto{\pgfqpoint{1.281818in}{0.137176in}}%
\pgfpathlineto{\pgfqpoint{4.663636in}{0.137176in}}%
\pgfpathlineto{\pgfqpoint{6.918182in}{0.137176in}}%
\pgfusepath{stroke}%
\end{pgfscope}%
\begin{pgfscope}%
\pgfpathrectangle{\pgfqpoint{1.000000in}{0.110000in}}{\pgfqpoint{6.200000in}{0.770000in}}%
\pgfusepath{clip}%
\pgfsetbuttcap%
\pgfsetroundjoin%
\pgfsetlinewidth{1.505625pt}%
\definecolor{currentstroke}{rgb}{1.000000,0.000000,0.000000}%
\pgfsetstrokecolor{currentstroke}%
\pgfsetdash{{1.500000pt}{2.475000pt}}{0.000000pt}%
\pgfpathmoveto{\pgfqpoint{1.281818in}{0.110000in}}%
\pgfpathlineto{\pgfqpoint{1.281818in}{0.880000in}}%
\pgfusepath{stroke}%
\end{pgfscope}%
\begin{pgfscope}%
\pgfpathrectangle{\pgfqpoint{1.000000in}{0.110000in}}{\pgfqpoint{6.200000in}{0.770000in}}%
\pgfusepath{clip}%
\pgfsetbuttcap%
\pgfsetroundjoin%
\pgfsetlinewidth{1.505625pt}%
\definecolor{currentstroke}{rgb}{1.000000,0.000000,0.000000}%
\pgfsetstrokecolor{currentstroke}%
\pgfsetdash{{1.500000pt}{2.475000pt}}{0.000000pt}%
\pgfpathmoveto{\pgfqpoint{4.663636in}{0.110000in}}%
\pgfpathlineto{\pgfqpoint{4.663636in}{0.880000in}}%
\pgfusepath{stroke}%
\end{pgfscope}%
\begin{pgfscope}%
\pgfpathrectangle{\pgfqpoint{1.000000in}{0.110000in}}{\pgfqpoint{6.200000in}{0.770000in}}%
\pgfusepath{clip}%
\pgfsetbuttcap%
\pgfsetroundjoin%
\pgfsetlinewidth{1.505625pt}%
\definecolor{currentstroke}{rgb}{1.000000,0.000000,0.000000}%
\pgfsetstrokecolor{currentstroke}%
\pgfsetdash{{1.500000pt}{2.475000pt}}{0.000000pt}%
\pgfpathmoveto{\pgfqpoint{6.918182in}{0.110000in}}%
\pgfpathlineto{\pgfqpoint{6.918182in}{0.880000in}}%
\pgfusepath{stroke}%
\end{pgfscope}%
\begin{pgfscope}%
\pgfsetbuttcap%
\pgfsetmiterjoin%
\definecolor{currentfill}{rgb}{0.800000,0.800000,0.800000}%
\pgfsetfillcolor{currentfill}%
\pgfsetlinewidth{1.003750pt}%
\definecolor{currentstroke}{rgb}{0.000000,0.000000,0.000000}%
\pgfsetstrokecolor{currentstroke}%
\pgfsetdash{}{0pt}%
\pgfpathmoveto{\pgfqpoint{2.629941in}{0.396662in}}%
\pgfpathcurveto{\pgfqpoint{2.664663in}{0.361940in}}{\pgfqpoint{3.280791in}{0.361940in}}{\pgfqpoint{3.315513in}{0.396662in}}%
\pgfpathcurveto{\pgfqpoint{3.350236in}{0.431384in}}{\pgfqpoint{3.350236in}{0.554841in}}{\pgfqpoint{3.315513in}{0.589563in}}%
\pgfpathcurveto{\pgfqpoint{3.280791in}{0.624286in}}{\pgfqpoint{2.664663in}{0.624286in}}{\pgfqpoint{2.629941in}{0.589563in}}%
\pgfpathcurveto{\pgfqpoint{2.595219in}{0.554841in}}{\pgfqpoint{2.595219in}{0.431384in}}{\pgfqpoint{2.629941in}{0.396662in}}%
\pgfpathclose%
\pgfusepath{stroke,fill}%
\end{pgfscope}%
\begin{pgfscope}%
\definecolor{textcolor}{rgb}{0.000000,0.000000,0.000000}%
\pgfsetstrokecolor{textcolor}%
\pgfsetfillcolor{textcolor}%
\pgftext[x=2.972727in,y=0.493113in,,]{\color{textcolor}\rmfamily\fontsize{10.000000}{12.000000}\selectfont C\textsubscript{b} = 1.46}%
\end{pgfscope}%
\begin{pgfscope}%
\pgfsetbuttcap%
\pgfsetmiterjoin%
\definecolor{currentfill}{rgb}{0.800000,0.800000,0.800000}%
\pgfsetfillcolor{currentfill}%
\pgfsetlinewidth{1.003750pt}%
\definecolor{currentstroke}{rgb}{0.000000,0.000000,0.000000}%
\pgfsetstrokecolor{currentstroke}%
\pgfsetdash{}{0pt}%
\pgfpathmoveto{\pgfqpoint{5.448123in}{0.396662in}}%
\pgfpathcurveto{\pgfqpoint{5.482845in}{0.361940in}}{\pgfqpoint{6.098973in}{0.361940in}}{\pgfqpoint{6.133695in}{0.396662in}}%
\pgfpathcurveto{\pgfqpoint{6.168417in}{0.431384in}}{\pgfqpoint{6.168417in}{0.554841in}}{\pgfqpoint{6.133695in}{0.589563in}}%
\pgfpathcurveto{\pgfqpoint{6.098973in}{0.624286in}}{\pgfqpoint{5.482845in}{0.624286in}}{\pgfqpoint{5.448123in}{0.589563in}}%
\pgfpathcurveto{\pgfqpoint{5.413401in}{0.554841in}}{\pgfqpoint{5.413401in}{0.431384in}}{\pgfqpoint{5.448123in}{0.396662in}}%
\pgfpathclose%
\pgfusepath{stroke,fill}%
\end{pgfscope}%
\begin{pgfscope}%
\definecolor{textcolor}{rgb}{0.000000,0.000000,0.000000}%
\pgfsetstrokecolor{textcolor}%
\pgfsetfillcolor{textcolor}%
\pgftext[x=5.790909in,y=0.493113in,,]{\color{textcolor}\rmfamily\fontsize{10.000000}{12.000000}\selectfont C\textsubscript{b} = 2.26}%
\end{pgfscope}%
\end{pgfpicture}%
\makeatother%
\endgroup%

\end{center}
\caption{C\textsubscript{b} Along Member}
\end{figure}
F\textsubscript{cr}, the buckling stress for the critical section in the critical unbraced span, is calculated per AISC/ANSI 360- 16 Eq. F2-4 as follows:
\begin{flalign*}
F_{cr} & = \cfrac{C_b \cdot \pi^2 \cdot {{E}}} {\left(\cfrac{L_b}{r_{ts}}\right)^2} \cdot \sqrt{1 + 0.078 \cdot \cfrac{{J} \cdot {c}}{{S_x} \cdot h_0} \cdot \left(\cfrac{L_b}{r_{ts}}\right)^2} \\ & = \cfrac{1.46  \cdot \pi^2 \cdot {{29000 {\color{darkBlue}{\mathbf{ \; ksi}}}}}} {\left(\cfrac{15 {\color{darkBlue}{\mathbf{ \; ft}}}}{1.6 {\color{darkBlue}{\mathbf{ \; in}}}}\right)^2} \cdot \sqrt{1 + 0.078 \cdot \cfrac{{0.239 {\color{darkBlue}{\mathbf{ \; {\color{darkBlue}{\mathbf{ \; in}}}^{4}}}}} \cdot {1}}{{23.2 {\color{darkBlue}{\mathbf{ \; {\color{darkBlue}{\mathbf{ \; in}}}^{3}}}}} \cdot 9.8 {\color{darkBlue}{\mathbf{ \; in}}}} \cdot \left(\cfrac{15 {\color{darkBlue}{\mathbf{ \; ft}}}}{1.6 {\color{darkBlue}{\mathbf{ \; in}}}}\right)^2} = \mathbf{45.1 {\color{darkBlue}{\mathbf{ \; ksi}}}}
\end{flalign*}
\\
\textphi\textsubscript{b}, the resistance factor for bending, is determined per AISC/ANSI 360-16 {\S}F1a as \textbf{0.9}.
\\\\
\textphi\textsubscript{b}M\textsubscript{n}, the design flexural strength, is calculated per AISC/ANSI 360-16 Eq. F2-3 as follows:
\begin{flalign*}
{\phi_b}{M_{n}} & = {\phi_b} \cdot {F_{cr}} \cdot {S_{x}} < {\phi_b} \cdot {M_p} \\ & = {0.9} \cdot {45.1 {\color{darkBlue}{\mathbf{ \; ksi}}}} \cdot {23.2 {\color{darkBlue}{\mathbf{ \; {\color{darkBlue}{\mathbf{ \; in}}}^{3}}}}} < {0.9} \cdot {108.3 {\color{darkBlue}{\mathbf{ \; kft}}}} = \mathbf{78.5 {\color{darkBlue}{\mathbf{ \; kft}}}}
\end{flalign*}
\vspace{-20pt}
{\setlength{\mathindent}{0cm}
\begin{flalign*}
\mathbf{|M_u| = 42.1 {\color{darkBlue}{\mathbf{ \; kft}}}  \;  < \phi_b \cdot M_n = 78.5 {\color{darkBlue}{\mathbf{ \; kft}}}  \;  (DCR = 0.54 - OK)}
\end{flalign*}
\textbf{(inelastic LTB controls)}
%	-------------------------------- SHEAR CHECK ---------------------------------
\section{Shear Check}
\begin{figure}[H]
\begin{center}
%% Creator: Matplotlib, PGF backend
%%
%% To include the figure in your LaTeX document, write
%%   \input{<filename>.pgf}
%%
%% Make sure the required packages are loaded in your preamble
%%   \usepackage{pgf}
%%
%% Figures using additional raster images can only be included by \input if
%% they are in the same directory as the main LaTeX file. For loading figures
%% from other directories you can use the `import` package
%%   \usepackage{import}
%%
%% and then include the figures with
%%   \import{<path to file>}{<filename>.pgf}
%%
%% Matplotlib used the following preamble
%%
\begingroup%
\makeatletter%
\begin{pgfpicture}%
\pgfpathrectangle{\pgfpointorigin}{\pgfqpoint{8.000000in}{3.000000in}}%
\pgfusepath{use as bounding box, clip}%
\begin{pgfscope}%
\pgfsetbuttcap%
\pgfsetmiterjoin%
\definecolor{currentfill}{rgb}{1.000000,1.000000,1.000000}%
\pgfsetfillcolor{currentfill}%
\pgfsetlinewidth{0.000000pt}%
\definecolor{currentstroke}{rgb}{1.000000,1.000000,1.000000}%
\pgfsetstrokecolor{currentstroke}%
\pgfsetdash{}{0pt}%
\pgfpathmoveto{\pgfqpoint{0.000000in}{0.000000in}}%
\pgfpathlineto{\pgfqpoint{8.000000in}{0.000000in}}%
\pgfpathlineto{\pgfqpoint{8.000000in}{3.000000in}}%
\pgfpathlineto{\pgfqpoint{0.000000in}{3.000000in}}%
\pgfpathclose%
\pgfusepath{fill}%
\end{pgfscope}%
\begin{pgfscope}%
\pgfsetbuttcap%
\pgfsetmiterjoin%
\definecolor{currentfill}{rgb}{1.000000,1.000000,1.000000}%
\pgfsetfillcolor{currentfill}%
\pgfsetlinewidth{0.000000pt}%
\definecolor{currentstroke}{rgb}{0.000000,0.000000,0.000000}%
\pgfsetstrokecolor{currentstroke}%
\pgfsetstrokeopacity{0.000000}%
\pgfsetdash{}{0pt}%
\pgfpathmoveto{\pgfqpoint{1.000000in}{0.330000in}}%
\pgfpathlineto{\pgfqpoint{7.200000in}{0.330000in}}%
\pgfpathlineto{\pgfqpoint{7.200000in}{2.640000in}}%
\pgfpathlineto{\pgfqpoint{1.000000in}{2.640000in}}%
\pgfpathclose%
\pgfusepath{fill}%
\end{pgfscope}%
\begin{pgfscope}%
\pgfpathrectangle{\pgfqpoint{1.000000in}{0.330000in}}{\pgfqpoint{6.200000in}{2.310000in}}%
\pgfusepath{clip}%
\pgfsetbuttcap%
\pgfsetroundjoin%
\pgfsetlinewidth{0.803000pt}%
\definecolor{currentstroke}{rgb}{0.000000,0.000000,0.000000}%
\pgfsetstrokecolor{currentstroke}%
\pgfsetdash{{0.800000pt}{1.320000pt}}{0.000000pt}%
\pgfpathmoveto{\pgfqpoint{1.281818in}{0.330000in}}%
\pgfpathlineto{\pgfqpoint{1.281818in}{2.640000in}}%
\pgfusepath{stroke}%
\end{pgfscope}%
\begin{pgfscope}%
\pgfsetbuttcap%
\pgfsetroundjoin%
\definecolor{currentfill}{rgb}{0.000000,0.000000,0.000000}%
\pgfsetfillcolor{currentfill}%
\pgfsetlinewidth{0.803000pt}%
\definecolor{currentstroke}{rgb}{0.000000,0.000000,0.000000}%
\pgfsetstrokecolor{currentstroke}%
\pgfsetdash{}{0pt}%
\pgfsys@defobject{currentmarker}{\pgfqpoint{0.000000in}{-0.048611in}}{\pgfqpoint{0.000000in}{0.000000in}}{%
\pgfpathmoveto{\pgfqpoint{0.000000in}{0.000000in}}%
\pgfpathlineto{\pgfqpoint{0.000000in}{-0.048611in}}%
\pgfusepath{stroke,fill}%
}%
\begin{pgfscope}%
\pgfsys@transformshift{1.281818in}{0.330000in}%
\pgfsys@useobject{currentmarker}{}%
\end{pgfscope}%
\end{pgfscope}%
\begin{pgfscope}%
\pgfsetbuttcap%
\pgfsetroundjoin%
\definecolor{currentfill}{rgb}{0.000000,0.000000,0.000000}%
\pgfsetfillcolor{currentfill}%
\pgfsetlinewidth{0.803000pt}%
\definecolor{currentstroke}{rgb}{0.000000,0.000000,0.000000}%
\pgfsetstrokecolor{currentstroke}%
\pgfsetdash{}{0pt}%
\pgfsys@defobject{currentmarker}{\pgfqpoint{0.000000in}{0.000000in}}{\pgfqpoint{0.000000in}{0.048611in}}{%
\pgfpathmoveto{\pgfqpoint{0.000000in}{0.000000in}}%
\pgfpathlineto{\pgfqpoint{0.000000in}{0.048611in}}%
\pgfusepath{stroke,fill}%
}%
\begin{pgfscope}%
\pgfsys@transformshift{1.281818in}{2.640000in}%
\pgfsys@useobject{currentmarker}{}%
\end{pgfscope}%
\end{pgfscope}%
\begin{pgfscope}%
\definecolor{textcolor}{rgb}{0.000000,0.000000,0.000000}%
\pgfsetstrokecolor{textcolor}%
\pgfsetfillcolor{textcolor}%
\pgftext[x=1.281818in,y=0.232778in,,top]{\color{textcolor}\rmfamily\fontsize{10.000000}{12.000000}\selectfont \(\displaystyle {0}\)}%
\end{pgfscope}%
\begin{pgfscope}%
\pgfpathrectangle{\pgfqpoint{1.000000in}{0.330000in}}{\pgfqpoint{6.200000in}{2.310000in}}%
\pgfusepath{clip}%
\pgfsetbuttcap%
\pgfsetroundjoin%
\pgfsetlinewidth{0.803000pt}%
\definecolor{currentstroke}{rgb}{0.000000,0.000000,0.000000}%
\pgfsetstrokecolor{currentstroke}%
\pgfsetdash{{0.800000pt}{1.320000pt}}{0.000000pt}%
\pgfpathmoveto{\pgfqpoint{2.409091in}{0.330000in}}%
\pgfpathlineto{\pgfqpoint{2.409091in}{2.640000in}}%
\pgfusepath{stroke}%
\end{pgfscope}%
\begin{pgfscope}%
\pgfsetbuttcap%
\pgfsetroundjoin%
\definecolor{currentfill}{rgb}{0.000000,0.000000,0.000000}%
\pgfsetfillcolor{currentfill}%
\pgfsetlinewidth{0.803000pt}%
\definecolor{currentstroke}{rgb}{0.000000,0.000000,0.000000}%
\pgfsetstrokecolor{currentstroke}%
\pgfsetdash{}{0pt}%
\pgfsys@defobject{currentmarker}{\pgfqpoint{0.000000in}{-0.048611in}}{\pgfqpoint{0.000000in}{0.000000in}}{%
\pgfpathmoveto{\pgfqpoint{0.000000in}{0.000000in}}%
\pgfpathlineto{\pgfqpoint{0.000000in}{-0.048611in}}%
\pgfusepath{stroke,fill}%
}%
\begin{pgfscope}%
\pgfsys@transformshift{2.409091in}{0.330000in}%
\pgfsys@useobject{currentmarker}{}%
\end{pgfscope}%
\end{pgfscope}%
\begin{pgfscope}%
\pgfsetbuttcap%
\pgfsetroundjoin%
\definecolor{currentfill}{rgb}{0.000000,0.000000,0.000000}%
\pgfsetfillcolor{currentfill}%
\pgfsetlinewidth{0.803000pt}%
\definecolor{currentstroke}{rgb}{0.000000,0.000000,0.000000}%
\pgfsetstrokecolor{currentstroke}%
\pgfsetdash{}{0pt}%
\pgfsys@defobject{currentmarker}{\pgfqpoint{0.000000in}{0.000000in}}{\pgfqpoint{0.000000in}{0.048611in}}{%
\pgfpathmoveto{\pgfqpoint{0.000000in}{0.000000in}}%
\pgfpathlineto{\pgfqpoint{0.000000in}{0.048611in}}%
\pgfusepath{stroke,fill}%
}%
\begin{pgfscope}%
\pgfsys@transformshift{2.409091in}{2.640000in}%
\pgfsys@useobject{currentmarker}{}%
\end{pgfscope}%
\end{pgfscope}%
\begin{pgfscope}%
\definecolor{textcolor}{rgb}{0.000000,0.000000,0.000000}%
\pgfsetstrokecolor{textcolor}%
\pgfsetfillcolor{textcolor}%
\pgftext[x=2.409091in,y=0.232778in,,top]{\color{textcolor}\rmfamily\fontsize{10.000000}{12.000000}\selectfont \(\displaystyle {5}\)}%
\end{pgfscope}%
\begin{pgfscope}%
\pgfpathrectangle{\pgfqpoint{1.000000in}{0.330000in}}{\pgfqpoint{6.200000in}{2.310000in}}%
\pgfusepath{clip}%
\pgfsetbuttcap%
\pgfsetroundjoin%
\pgfsetlinewidth{0.803000pt}%
\definecolor{currentstroke}{rgb}{0.000000,0.000000,0.000000}%
\pgfsetstrokecolor{currentstroke}%
\pgfsetdash{{0.800000pt}{1.320000pt}}{0.000000pt}%
\pgfpathmoveto{\pgfqpoint{3.536364in}{0.330000in}}%
\pgfpathlineto{\pgfqpoint{3.536364in}{2.640000in}}%
\pgfusepath{stroke}%
\end{pgfscope}%
\begin{pgfscope}%
\pgfsetbuttcap%
\pgfsetroundjoin%
\definecolor{currentfill}{rgb}{0.000000,0.000000,0.000000}%
\pgfsetfillcolor{currentfill}%
\pgfsetlinewidth{0.803000pt}%
\definecolor{currentstroke}{rgb}{0.000000,0.000000,0.000000}%
\pgfsetstrokecolor{currentstroke}%
\pgfsetdash{}{0pt}%
\pgfsys@defobject{currentmarker}{\pgfqpoint{0.000000in}{-0.048611in}}{\pgfqpoint{0.000000in}{0.000000in}}{%
\pgfpathmoveto{\pgfqpoint{0.000000in}{0.000000in}}%
\pgfpathlineto{\pgfqpoint{0.000000in}{-0.048611in}}%
\pgfusepath{stroke,fill}%
}%
\begin{pgfscope}%
\pgfsys@transformshift{3.536364in}{0.330000in}%
\pgfsys@useobject{currentmarker}{}%
\end{pgfscope}%
\end{pgfscope}%
\begin{pgfscope}%
\pgfsetbuttcap%
\pgfsetroundjoin%
\definecolor{currentfill}{rgb}{0.000000,0.000000,0.000000}%
\pgfsetfillcolor{currentfill}%
\pgfsetlinewidth{0.803000pt}%
\definecolor{currentstroke}{rgb}{0.000000,0.000000,0.000000}%
\pgfsetstrokecolor{currentstroke}%
\pgfsetdash{}{0pt}%
\pgfsys@defobject{currentmarker}{\pgfqpoint{0.000000in}{0.000000in}}{\pgfqpoint{0.000000in}{0.048611in}}{%
\pgfpathmoveto{\pgfqpoint{0.000000in}{0.000000in}}%
\pgfpathlineto{\pgfqpoint{0.000000in}{0.048611in}}%
\pgfusepath{stroke,fill}%
}%
\begin{pgfscope}%
\pgfsys@transformshift{3.536364in}{2.640000in}%
\pgfsys@useobject{currentmarker}{}%
\end{pgfscope}%
\end{pgfscope}%
\begin{pgfscope}%
\definecolor{textcolor}{rgb}{0.000000,0.000000,0.000000}%
\pgfsetstrokecolor{textcolor}%
\pgfsetfillcolor{textcolor}%
\pgftext[x=3.536364in,y=0.232778in,,top]{\color{textcolor}\rmfamily\fontsize{10.000000}{12.000000}\selectfont \(\displaystyle {10}\)}%
\end{pgfscope}%
\begin{pgfscope}%
\pgfpathrectangle{\pgfqpoint{1.000000in}{0.330000in}}{\pgfqpoint{6.200000in}{2.310000in}}%
\pgfusepath{clip}%
\pgfsetbuttcap%
\pgfsetroundjoin%
\pgfsetlinewidth{0.803000pt}%
\definecolor{currentstroke}{rgb}{0.000000,0.000000,0.000000}%
\pgfsetstrokecolor{currentstroke}%
\pgfsetdash{{0.800000pt}{1.320000pt}}{0.000000pt}%
\pgfpathmoveto{\pgfqpoint{4.663636in}{0.330000in}}%
\pgfpathlineto{\pgfqpoint{4.663636in}{2.640000in}}%
\pgfusepath{stroke}%
\end{pgfscope}%
\begin{pgfscope}%
\pgfsetbuttcap%
\pgfsetroundjoin%
\definecolor{currentfill}{rgb}{0.000000,0.000000,0.000000}%
\pgfsetfillcolor{currentfill}%
\pgfsetlinewidth{0.803000pt}%
\definecolor{currentstroke}{rgb}{0.000000,0.000000,0.000000}%
\pgfsetstrokecolor{currentstroke}%
\pgfsetdash{}{0pt}%
\pgfsys@defobject{currentmarker}{\pgfqpoint{0.000000in}{-0.048611in}}{\pgfqpoint{0.000000in}{0.000000in}}{%
\pgfpathmoveto{\pgfqpoint{0.000000in}{0.000000in}}%
\pgfpathlineto{\pgfqpoint{0.000000in}{-0.048611in}}%
\pgfusepath{stroke,fill}%
}%
\begin{pgfscope}%
\pgfsys@transformshift{4.663636in}{0.330000in}%
\pgfsys@useobject{currentmarker}{}%
\end{pgfscope}%
\end{pgfscope}%
\begin{pgfscope}%
\pgfsetbuttcap%
\pgfsetroundjoin%
\definecolor{currentfill}{rgb}{0.000000,0.000000,0.000000}%
\pgfsetfillcolor{currentfill}%
\pgfsetlinewidth{0.803000pt}%
\definecolor{currentstroke}{rgb}{0.000000,0.000000,0.000000}%
\pgfsetstrokecolor{currentstroke}%
\pgfsetdash{}{0pt}%
\pgfsys@defobject{currentmarker}{\pgfqpoint{0.000000in}{0.000000in}}{\pgfqpoint{0.000000in}{0.048611in}}{%
\pgfpathmoveto{\pgfqpoint{0.000000in}{0.000000in}}%
\pgfpathlineto{\pgfqpoint{0.000000in}{0.048611in}}%
\pgfusepath{stroke,fill}%
}%
\begin{pgfscope}%
\pgfsys@transformshift{4.663636in}{2.640000in}%
\pgfsys@useobject{currentmarker}{}%
\end{pgfscope}%
\end{pgfscope}%
\begin{pgfscope}%
\definecolor{textcolor}{rgb}{0.000000,0.000000,0.000000}%
\pgfsetstrokecolor{textcolor}%
\pgfsetfillcolor{textcolor}%
\pgftext[x=4.663636in,y=0.232778in,,top]{\color{textcolor}\rmfamily\fontsize{10.000000}{12.000000}\selectfont \(\displaystyle {15}\)}%
\end{pgfscope}%
\begin{pgfscope}%
\pgfpathrectangle{\pgfqpoint{1.000000in}{0.330000in}}{\pgfqpoint{6.200000in}{2.310000in}}%
\pgfusepath{clip}%
\pgfsetbuttcap%
\pgfsetroundjoin%
\pgfsetlinewidth{0.803000pt}%
\definecolor{currentstroke}{rgb}{0.000000,0.000000,0.000000}%
\pgfsetstrokecolor{currentstroke}%
\pgfsetdash{{0.800000pt}{1.320000pt}}{0.000000pt}%
\pgfpathmoveto{\pgfqpoint{5.790909in}{0.330000in}}%
\pgfpathlineto{\pgfqpoint{5.790909in}{2.640000in}}%
\pgfusepath{stroke}%
\end{pgfscope}%
\begin{pgfscope}%
\pgfsetbuttcap%
\pgfsetroundjoin%
\definecolor{currentfill}{rgb}{0.000000,0.000000,0.000000}%
\pgfsetfillcolor{currentfill}%
\pgfsetlinewidth{0.803000pt}%
\definecolor{currentstroke}{rgb}{0.000000,0.000000,0.000000}%
\pgfsetstrokecolor{currentstroke}%
\pgfsetdash{}{0pt}%
\pgfsys@defobject{currentmarker}{\pgfqpoint{0.000000in}{-0.048611in}}{\pgfqpoint{0.000000in}{0.000000in}}{%
\pgfpathmoveto{\pgfqpoint{0.000000in}{0.000000in}}%
\pgfpathlineto{\pgfqpoint{0.000000in}{-0.048611in}}%
\pgfusepath{stroke,fill}%
}%
\begin{pgfscope}%
\pgfsys@transformshift{5.790909in}{0.330000in}%
\pgfsys@useobject{currentmarker}{}%
\end{pgfscope}%
\end{pgfscope}%
\begin{pgfscope}%
\pgfsetbuttcap%
\pgfsetroundjoin%
\definecolor{currentfill}{rgb}{0.000000,0.000000,0.000000}%
\pgfsetfillcolor{currentfill}%
\pgfsetlinewidth{0.803000pt}%
\definecolor{currentstroke}{rgb}{0.000000,0.000000,0.000000}%
\pgfsetstrokecolor{currentstroke}%
\pgfsetdash{}{0pt}%
\pgfsys@defobject{currentmarker}{\pgfqpoint{0.000000in}{0.000000in}}{\pgfqpoint{0.000000in}{0.048611in}}{%
\pgfpathmoveto{\pgfqpoint{0.000000in}{0.000000in}}%
\pgfpathlineto{\pgfqpoint{0.000000in}{0.048611in}}%
\pgfusepath{stroke,fill}%
}%
\begin{pgfscope}%
\pgfsys@transformshift{5.790909in}{2.640000in}%
\pgfsys@useobject{currentmarker}{}%
\end{pgfscope}%
\end{pgfscope}%
\begin{pgfscope}%
\definecolor{textcolor}{rgb}{0.000000,0.000000,0.000000}%
\pgfsetstrokecolor{textcolor}%
\pgfsetfillcolor{textcolor}%
\pgftext[x=5.790909in,y=0.232778in,,top]{\color{textcolor}\rmfamily\fontsize{10.000000}{12.000000}\selectfont \(\displaystyle {20}\)}%
\end{pgfscope}%
\begin{pgfscope}%
\pgfpathrectangle{\pgfqpoint{1.000000in}{0.330000in}}{\pgfqpoint{6.200000in}{2.310000in}}%
\pgfusepath{clip}%
\pgfsetbuttcap%
\pgfsetroundjoin%
\pgfsetlinewidth{0.803000pt}%
\definecolor{currentstroke}{rgb}{0.000000,0.000000,0.000000}%
\pgfsetstrokecolor{currentstroke}%
\pgfsetdash{{0.800000pt}{1.320000pt}}{0.000000pt}%
\pgfpathmoveto{\pgfqpoint{6.918182in}{0.330000in}}%
\pgfpathlineto{\pgfqpoint{6.918182in}{2.640000in}}%
\pgfusepath{stroke}%
\end{pgfscope}%
\begin{pgfscope}%
\pgfsetbuttcap%
\pgfsetroundjoin%
\definecolor{currentfill}{rgb}{0.000000,0.000000,0.000000}%
\pgfsetfillcolor{currentfill}%
\pgfsetlinewidth{0.803000pt}%
\definecolor{currentstroke}{rgb}{0.000000,0.000000,0.000000}%
\pgfsetstrokecolor{currentstroke}%
\pgfsetdash{}{0pt}%
\pgfsys@defobject{currentmarker}{\pgfqpoint{0.000000in}{-0.048611in}}{\pgfqpoint{0.000000in}{0.000000in}}{%
\pgfpathmoveto{\pgfqpoint{0.000000in}{0.000000in}}%
\pgfpathlineto{\pgfqpoint{0.000000in}{-0.048611in}}%
\pgfusepath{stroke,fill}%
}%
\begin{pgfscope}%
\pgfsys@transformshift{6.918182in}{0.330000in}%
\pgfsys@useobject{currentmarker}{}%
\end{pgfscope}%
\end{pgfscope}%
\begin{pgfscope}%
\pgfsetbuttcap%
\pgfsetroundjoin%
\definecolor{currentfill}{rgb}{0.000000,0.000000,0.000000}%
\pgfsetfillcolor{currentfill}%
\pgfsetlinewidth{0.803000pt}%
\definecolor{currentstroke}{rgb}{0.000000,0.000000,0.000000}%
\pgfsetstrokecolor{currentstroke}%
\pgfsetdash{}{0pt}%
\pgfsys@defobject{currentmarker}{\pgfqpoint{0.000000in}{0.000000in}}{\pgfqpoint{0.000000in}{0.048611in}}{%
\pgfpathmoveto{\pgfqpoint{0.000000in}{0.000000in}}%
\pgfpathlineto{\pgfqpoint{0.000000in}{0.048611in}}%
\pgfusepath{stroke,fill}%
}%
\begin{pgfscope}%
\pgfsys@transformshift{6.918182in}{2.640000in}%
\pgfsys@useobject{currentmarker}{}%
\end{pgfscope}%
\end{pgfscope}%
\begin{pgfscope}%
\definecolor{textcolor}{rgb}{0.000000,0.000000,0.000000}%
\pgfsetstrokecolor{textcolor}%
\pgfsetfillcolor{textcolor}%
\pgftext[x=6.918182in,y=0.232778in,,top]{\color{textcolor}\rmfamily\fontsize{10.000000}{12.000000}\selectfont \(\displaystyle {25}\)}%
\end{pgfscope}%
\begin{pgfscope}%
\pgfpathrectangle{\pgfqpoint{1.000000in}{0.330000in}}{\pgfqpoint{6.200000in}{2.310000in}}%
\pgfusepath{clip}%
\pgfsetbuttcap%
\pgfsetroundjoin%
\pgfsetlinewidth{0.803000pt}%
\definecolor{currentstroke}{rgb}{0.000000,0.000000,0.000000}%
\pgfsetstrokecolor{currentstroke}%
\pgfsetdash{{0.800000pt}{1.320000pt}}{0.000000pt}%
\pgfpathmoveto{\pgfqpoint{1.000000in}{0.669652in}}%
\pgfpathlineto{\pgfqpoint{7.200000in}{0.669652in}}%
\pgfusepath{stroke}%
\end{pgfscope}%
\begin{pgfscope}%
\pgfsetbuttcap%
\pgfsetroundjoin%
\definecolor{currentfill}{rgb}{0.000000,0.000000,0.000000}%
\pgfsetfillcolor{currentfill}%
\pgfsetlinewidth{0.803000pt}%
\definecolor{currentstroke}{rgb}{0.000000,0.000000,0.000000}%
\pgfsetstrokecolor{currentstroke}%
\pgfsetdash{}{0pt}%
\pgfsys@defobject{currentmarker}{\pgfqpoint{-0.048611in}{0.000000in}}{\pgfqpoint{-0.000000in}{0.000000in}}{%
\pgfpathmoveto{\pgfqpoint{-0.000000in}{0.000000in}}%
\pgfpathlineto{\pgfqpoint{-0.048611in}{0.000000in}}%
\pgfusepath{stroke,fill}%
}%
\begin{pgfscope}%
\pgfsys@transformshift{1.000000in}{0.669652in}%
\pgfsys@useobject{currentmarker}{}%
\end{pgfscope}%
\end{pgfscope}%
\begin{pgfscope}%
\pgfsetbuttcap%
\pgfsetroundjoin%
\definecolor{currentfill}{rgb}{0.000000,0.000000,0.000000}%
\pgfsetfillcolor{currentfill}%
\pgfsetlinewidth{0.803000pt}%
\definecolor{currentstroke}{rgb}{0.000000,0.000000,0.000000}%
\pgfsetstrokecolor{currentstroke}%
\pgfsetdash{}{0pt}%
\pgfsys@defobject{currentmarker}{\pgfqpoint{0.000000in}{0.000000in}}{\pgfqpoint{0.048611in}{0.000000in}}{%
\pgfpathmoveto{\pgfqpoint{0.000000in}{0.000000in}}%
\pgfpathlineto{\pgfqpoint{0.048611in}{0.000000in}}%
\pgfusepath{stroke,fill}%
}%
\begin{pgfscope}%
\pgfsys@transformshift{7.200000in}{0.669652in}%
\pgfsys@useobject{currentmarker}{}%
\end{pgfscope}%
\end{pgfscope}%
\begin{pgfscope}%
\definecolor{textcolor}{rgb}{0.000000,0.000000,0.000000}%
\pgfsetstrokecolor{textcolor}%
\pgfsetfillcolor{textcolor}%
\pgftext[x=0.655863in, y=0.621427in, left, base]{\color{textcolor}\rmfamily\fontsize{10.000000}{12.000000}\selectfont \(\displaystyle {\ensuremath{-}10}\)}%
\end{pgfscope}%
\begin{pgfscope}%
\pgfpathrectangle{\pgfqpoint{1.000000in}{0.330000in}}{\pgfqpoint{6.200000in}{2.310000in}}%
\pgfusepath{clip}%
\pgfsetbuttcap%
\pgfsetroundjoin%
\pgfsetlinewidth{0.803000pt}%
\definecolor{currentstroke}{rgb}{0.000000,0.000000,0.000000}%
\pgfsetstrokecolor{currentstroke}%
\pgfsetdash{{0.800000pt}{1.320000pt}}{0.000000pt}%
\pgfpathmoveto{\pgfqpoint{1.000000in}{1.128902in}}%
\pgfpathlineto{\pgfqpoint{7.200000in}{1.128902in}}%
\pgfusepath{stroke}%
\end{pgfscope}%
\begin{pgfscope}%
\pgfsetbuttcap%
\pgfsetroundjoin%
\definecolor{currentfill}{rgb}{0.000000,0.000000,0.000000}%
\pgfsetfillcolor{currentfill}%
\pgfsetlinewidth{0.803000pt}%
\definecolor{currentstroke}{rgb}{0.000000,0.000000,0.000000}%
\pgfsetstrokecolor{currentstroke}%
\pgfsetdash{}{0pt}%
\pgfsys@defobject{currentmarker}{\pgfqpoint{-0.048611in}{0.000000in}}{\pgfqpoint{-0.000000in}{0.000000in}}{%
\pgfpathmoveto{\pgfqpoint{-0.000000in}{0.000000in}}%
\pgfpathlineto{\pgfqpoint{-0.048611in}{0.000000in}}%
\pgfusepath{stroke,fill}%
}%
\begin{pgfscope}%
\pgfsys@transformshift{1.000000in}{1.128902in}%
\pgfsys@useobject{currentmarker}{}%
\end{pgfscope}%
\end{pgfscope}%
\begin{pgfscope}%
\pgfsetbuttcap%
\pgfsetroundjoin%
\definecolor{currentfill}{rgb}{0.000000,0.000000,0.000000}%
\pgfsetfillcolor{currentfill}%
\pgfsetlinewidth{0.803000pt}%
\definecolor{currentstroke}{rgb}{0.000000,0.000000,0.000000}%
\pgfsetstrokecolor{currentstroke}%
\pgfsetdash{}{0pt}%
\pgfsys@defobject{currentmarker}{\pgfqpoint{0.000000in}{0.000000in}}{\pgfqpoint{0.048611in}{0.000000in}}{%
\pgfpathmoveto{\pgfqpoint{0.000000in}{0.000000in}}%
\pgfpathlineto{\pgfqpoint{0.048611in}{0.000000in}}%
\pgfusepath{stroke,fill}%
}%
\begin{pgfscope}%
\pgfsys@transformshift{7.200000in}{1.128902in}%
\pgfsys@useobject{currentmarker}{}%
\end{pgfscope}%
\end{pgfscope}%
\begin{pgfscope}%
\definecolor{textcolor}{rgb}{0.000000,0.000000,0.000000}%
\pgfsetstrokecolor{textcolor}%
\pgfsetfillcolor{textcolor}%
\pgftext[x=0.725308in, y=1.080677in, left, base]{\color{textcolor}\rmfamily\fontsize{10.000000}{12.000000}\selectfont \(\displaystyle {\ensuremath{-}5}\)}%
\end{pgfscope}%
\begin{pgfscope}%
\pgfpathrectangle{\pgfqpoint{1.000000in}{0.330000in}}{\pgfqpoint{6.200000in}{2.310000in}}%
\pgfusepath{clip}%
\pgfsetbuttcap%
\pgfsetroundjoin%
\pgfsetlinewidth{0.803000pt}%
\definecolor{currentstroke}{rgb}{0.000000,0.000000,0.000000}%
\pgfsetstrokecolor{currentstroke}%
\pgfsetdash{{0.800000pt}{1.320000pt}}{0.000000pt}%
\pgfpathmoveto{\pgfqpoint{1.000000in}{1.588152in}}%
\pgfpathlineto{\pgfqpoint{7.200000in}{1.588152in}}%
\pgfusepath{stroke}%
\end{pgfscope}%
\begin{pgfscope}%
\pgfsetbuttcap%
\pgfsetroundjoin%
\definecolor{currentfill}{rgb}{0.000000,0.000000,0.000000}%
\pgfsetfillcolor{currentfill}%
\pgfsetlinewidth{0.803000pt}%
\definecolor{currentstroke}{rgb}{0.000000,0.000000,0.000000}%
\pgfsetstrokecolor{currentstroke}%
\pgfsetdash{}{0pt}%
\pgfsys@defobject{currentmarker}{\pgfqpoint{-0.048611in}{0.000000in}}{\pgfqpoint{-0.000000in}{0.000000in}}{%
\pgfpathmoveto{\pgfqpoint{-0.000000in}{0.000000in}}%
\pgfpathlineto{\pgfqpoint{-0.048611in}{0.000000in}}%
\pgfusepath{stroke,fill}%
}%
\begin{pgfscope}%
\pgfsys@transformshift{1.000000in}{1.588152in}%
\pgfsys@useobject{currentmarker}{}%
\end{pgfscope}%
\end{pgfscope}%
\begin{pgfscope}%
\pgfsetbuttcap%
\pgfsetroundjoin%
\definecolor{currentfill}{rgb}{0.000000,0.000000,0.000000}%
\pgfsetfillcolor{currentfill}%
\pgfsetlinewidth{0.803000pt}%
\definecolor{currentstroke}{rgb}{0.000000,0.000000,0.000000}%
\pgfsetstrokecolor{currentstroke}%
\pgfsetdash{}{0pt}%
\pgfsys@defobject{currentmarker}{\pgfqpoint{0.000000in}{0.000000in}}{\pgfqpoint{0.048611in}{0.000000in}}{%
\pgfpathmoveto{\pgfqpoint{0.000000in}{0.000000in}}%
\pgfpathlineto{\pgfqpoint{0.048611in}{0.000000in}}%
\pgfusepath{stroke,fill}%
}%
\begin{pgfscope}%
\pgfsys@transformshift{7.200000in}{1.588152in}%
\pgfsys@useobject{currentmarker}{}%
\end{pgfscope}%
\end{pgfscope}%
\begin{pgfscope}%
\definecolor{textcolor}{rgb}{0.000000,0.000000,0.000000}%
\pgfsetstrokecolor{textcolor}%
\pgfsetfillcolor{textcolor}%
\pgftext[x=0.833333in, y=1.539926in, left, base]{\color{textcolor}\rmfamily\fontsize{10.000000}{12.000000}\selectfont \(\displaystyle {0}\)}%
\end{pgfscope}%
\begin{pgfscope}%
\pgfpathrectangle{\pgfqpoint{1.000000in}{0.330000in}}{\pgfqpoint{6.200000in}{2.310000in}}%
\pgfusepath{clip}%
\pgfsetbuttcap%
\pgfsetroundjoin%
\pgfsetlinewidth{0.803000pt}%
\definecolor{currentstroke}{rgb}{0.000000,0.000000,0.000000}%
\pgfsetstrokecolor{currentstroke}%
\pgfsetdash{{0.800000pt}{1.320000pt}}{0.000000pt}%
\pgfpathmoveto{\pgfqpoint{1.000000in}{2.047401in}}%
\pgfpathlineto{\pgfqpoint{7.200000in}{2.047401in}}%
\pgfusepath{stroke}%
\end{pgfscope}%
\begin{pgfscope}%
\pgfsetbuttcap%
\pgfsetroundjoin%
\definecolor{currentfill}{rgb}{0.000000,0.000000,0.000000}%
\pgfsetfillcolor{currentfill}%
\pgfsetlinewidth{0.803000pt}%
\definecolor{currentstroke}{rgb}{0.000000,0.000000,0.000000}%
\pgfsetstrokecolor{currentstroke}%
\pgfsetdash{}{0pt}%
\pgfsys@defobject{currentmarker}{\pgfqpoint{-0.048611in}{0.000000in}}{\pgfqpoint{-0.000000in}{0.000000in}}{%
\pgfpathmoveto{\pgfqpoint{-0.000000in}{0.000000in}}%
\pgfpathlineto{\pgfqpoint{-0.048611in}{0.000000in}}%
\pgfusepath{stroke,fill}%
}%
\begin{pgfscope}%
\pgfsys@transformshift{1.000000in}{2.047401in}%
\pgfsys@useobject{currentmarker}{}%
\end{pgfscope}%
\end{pgfscope}%
\begin{pgfscope}%
\pgfsetbuttcap%
\pgfsetroundjoin%
\definecolor{currentfill}{rgb}{0.000000,0.000000,0.000000}%
\pgfsetfillcolor{currentfill}%
\pgfsetlinewidth{0.803000pt}%
\definecolor{currentstroke}{rgb}{0.000000,0.000000,0.000000}%
\pgfsetstrokecolor{currentstroke}%
\pgfsetdash{}{0pt}%
\pgfsys@defobject{currentmarker}{\pgfqpoint{0.000000in}{0.000000in}}{\pgfqpoint{0.048611in}{0.000000in}}{%
\pgfpathmoveto{\pgfqpoint{0.000000in}{0.000000in}}%
\pgfpathlineto{\pgfqpoint{0.048611in}{0.000000in}}%
\pgfusepath{stroke,fill}%
}%
\begin{pgfscope}%
\pgfsys@transformshift{7.200000in}{2.047401in}%
\pgfsys@useobject{currentmarker}{}%
\end{pgfscope}%
\end{pgfscope}%
\begin{pgfscope}%
\definecolor{textcolor}{rgb}{0.000000,0.000000,0.000000}%
\pgfsetstrokecolor{textcolor}%
\pgfsetfillcolor{textcolor}%
\pgftext[x=0.833333in, y=1.999176in, left, base]{\color{textcolor}\rmfamily\fontsize{10.000000}{12.000000}\selectfont \(\displaystyle {5}\)}%
\end{pgfscope}%
\begin{pgfscope}%
\pgfpathrectangle{\pgfqpoint{1.000000in}{0.330000in}}{\pgfqpoint{6.200000in}{2.310000in}}%
\pgfusepath{clip}%
\pgfsetbuttcap%
\pgfsetroundjoin%
\pgfsetlinewidth{0.803000pt}%
\definecolor{currentstroke}{rgb}{0.000000,0.000000,0.000000}%
\pgfsetstrokecolor{currentstroke}%
\pgfsetdash{{0.800000pt}{1.320000pt}}{0.000000pt}%
\pgfpathmoveto{\pgfqpoint{1.000000in}{2.506651in}}%
\pgfpathlineto{\pgfqpoint{7.200000in}{2.506651in}}%
\pgfusepath{stroke}%
\end{pgfscope}%
\begin{pgfscope}%
\pgfsetbuttcap%
\pgfsetroundjoin%
\definecolor{currentfill}{rgb}{0.000000,0.000000,0.000000}%
\pgfsetfillcolor{currentfill}%
\pgfsetlinewidth{0.803000pt}%
\definecolor{currentstroke}{rgb}{0.000000,0.000000,0.000000}%
\pgfsetstrokecolor{currentstroke}%
\pgfsetdash{}{0pt}%
\pgfsys@defobject{currentmarker}{\pgfqpoint{-0.048611in}{0.000000in}}{\pgfqpoint{-0.000000in}{0.000000in}}{%
\pgfpathmoveto{\pgfqpoint{-0.000000in}{0.000000in}}%
\pgfpathlineto{\pgfqpoint{-0.048611in}{0.000000in}}%
\pgfusepath{stroke,fill}%
}%
\begin{pgfscope}%
\pgfsys@transformshift{1.000000in}{2.506651in}%
\pgfsys@useobject{currentmarker}{}%
\end{pgfscope}%
\end{pgfscope}%
\begin{pgfscope}%
\pgfsetbuttcap%
\pgfsetroundjoin%
\definecolor{currentfill}{rgb}{0.000000,0.000000,0.000000}%
\pgfsetfillcolor{currentfill}%
\pgfsetlinewidth{0.803000pt}%
\definecolor{currentstroke}{rgb}{0.000000,0.000000,0.000000}%
\pgfsetstrokecolor{currentstroke}%
\pgfsetdash{}{0pt}%
\pgfsys@defobject{currentmarker}{\pgfqpoint{0.000000in}{0.000000in}}{\pgfqpoint{0.048611in}{0.000000in}}{%
\pgfpathmoveto{\pgfqpoint{0.000000in}{0.000000in}}%
\pgfpathlineto{\pgfqpoint{0.048611in}{0.000000in}}%
\pgfusepath{stroke,fill}%
}%
\begin{pgfscope}%
\pgfsys@transformshift{7.200000in}{2.506651in}%
\pgfsys@useobject{currentmarker}{}%
\end{pgfscope}%
\end{pgfscope}%
\begin{pgfscope}%
\definecolor{textcolor}{rgb}{0.000000,0.000000,0.000000}%
\pgfsetstrokecolor{textcolor}%
\pgfsetfillcolor{textcolor}%
\pgftext[x=0.763888in, y=2.458425in, left, base]{\color{textcolor}\rmfamily\fontsize{10.000000}{12.000000}\selectfont \(\displaystyle {10}\)}%
\end{pgfscope}%
\begin{pgfscope}%
\pgfpathrectangle{\pgfqpoint{1.000000in}{0.330000in}}{\pgfqpoint{6.200000in}{2.310000in}}%
\pgfusepath{clip}%
\pgfsetrectcap%
\pgfsetroundjoin%
\pgfsetlinewidth{1.505625pt}%
\definecolor{currentstroke}{rgb}{0.121569,0.466667,0.705882}%
\pgfsetstrokecolor{currentstroke}%
\pgfsetdash{}{0pt}%
\pgfpathmoveto{\pgfqpoint{1.281818in}{1.588152in}}%
\pgfpathlineto{\pgfqpoint{1.281818in}{1.715194in}}%
\pgfpathlineto{\pgfqpoint{2.390303in}{1.671872in}}%
\pgfpathlineto{\pgfqpoint{2.409091in}{1.542548in}}%
\pgfpathlineto{\pgfqpoint{4.626061in}{1.455904in}}%
\pgfpathlineto{\pgfqpoint{4.663636in}{1.675345in}}%
\pgfpathlineto{\pgfqpoint{6.711515in}{1.596044in}}%
\pgfpathlineto{\pgfqpoint{6.918182in}{1.596044in}}%
\pgfpathlineto{\pgfqpoint{6.918182in}{1.588152in}}%
\pgfpathlineto{\pgfqpoint{6.918182in}{1.588152in}}%
\pgfusepath{stroke}%
\end{pgfscope}%
\begin{pgfscope}%
\pgfpathrectangle{\pgfqpoint{1.000000in}{0.330000in}}{\pgfqpoint{6.200000in}{2.310000in}}%
\pgfusepath{clip}%
\pgfsetrectcap%
\pgfsetroundjoin%
\pgfsetlinewidth{1.505625pt}%
\definecolor{currentstroke}{rgb}{1.000000,0.498039,0.054902}%
\pgfsetstrokecolor{currentstroke}%
\pgfsetdash{}{0pt}%
\pgfpathmoveto{\pgfqpoint{1.281818in}{1.588152in}}%
\pgfpathlineto{\pgfqpoint{1.281818in}{1.803369in}}%
\pgfpathlineto{\pgfqpoint{2.390303in}{1.729103in}}%
\pgfpathlineto{\pgfqpoint{2.409091in}{1.507405in}}%
\pgfpathlineto{\pgfqpoint{4.626061in}{1.358873in}}%
\pgfpathlineto{\pgfqpoint{4.663636in}{1.831866in}}%
\pgfpathlineto{\pgfqpoint{4.907879in}{1.750396in}}%
\pgfpathlineto{\pgfqpoint{4.980220in}{1.725843in}}%
\pgfpathlineto{\pgfqpoint{5.790909in}{1.671528in}}%
\pgfpathlineto{\pgfqpoint{6.692727in}{1.519259in}}%
\pgfpathlineto{\pgfqpoint{6.918182in}{1.496296in}}%
\pgfpathlineto{\pgfqpoint{6.918182in}{1.588152in}}%
\pgfpathlineto{\pgfqpoint{6.918182in}{1.588152in}}%
\pgfusepath{stroke}%
\end{pgfscope}%
\begin{pgfscope}%
\pgfpathrectangle{\pgfqpoint{1.000000in}{0.330000in}}{\pgfqpoint{6.200000in}{2.310000in}}%
\pgfusepath{clip}%
\pgfsetrectcap%
\pgfsetroundjoin%
\pgfsetlinewidth{1.505625pt}%
\definecolor{currentstroke}{rgb}{0.172549,0.627451,0.172549}%
\pgfsetstrokecolor{currentstroke}%
\pgfsetdash{}{0pt}%
\pgfpathmoveto{\pgfqpoint{1.281818in}{1.588152in}}%
\pgfpathlineto{\pgfqpoint{1.281818in}{1.751492in}}%
\pgfpathlineto{\pgfqpoint{2.390303in}{1.695792in}}%
\pgfpathlineto{\pgfqpoint{2.409091in}{1.529518in}}%
\pgfpathlineto{\pgfqpoint{4.626061in}{1.418120in}}%
\pgfpathlineto{\pgfqpoint{4.663636in}{1.700257in}}%
\pgfpathlineto{\pgfqpoint{6.711515in}{1.598299in}}%
\pgfpathlineto{\pgfqpoint{6.918182in}{1.598299in}}%
\pgfpathlineto{\pgfqpoint{6.918182in}{1.588152in}}%
\pgfpathlineto{\pgfqpoint{6.918182in}{1.588152in}}%
\pgfusepath{stroke}%
\end{pgfscope}%
\begin{pgfscope}%
\pgfpathrectangle{\pgfqpoint{1.000000in}{0.330000in}}{\pgfqpoint{6.200000in}{2.310000in}}%
\pgfusepath{clip}%
\pgfsetrectcap%
\pgfsetroundjoin%
\pgfsetlinewidth{1.505625pt}%
\definecolor{currentstroke}{rgb}{0.839216,0.152941,0.156863}%
\pgfsetstrokecolor{currentstroke}%
\pgfsetdash{}{0pt}%
\pgfpathmoveto{\pgfqpoint{1.281818in}{1.588152in}}%
\pgfpathlineto{\pgfqpoint{1.281818in}{1.805938in}}%
\pgfpathlineto{\pgfqpoint{2.390303in}{1.731673in}}%
\pgfpathlineto{\pgfqpoint{2.409091in}{1.509974in}}%
\pgfpathlineto{\pgfqpoint{4.626061in}{1.361442in}}%
\pgfpathlineto{\pgfqpoint{4.663636in}{1.737626in}}%
\pgfpathlineto{\pgfqpoint{6.692727in}{1.601682in}}%
\pgfpathlineto{\pgfqpoint{6.918182in}{1.601682in}}%
\pgfpathlineto{\pgfqpoint{6.918182in}{1.588152in}}%
\pgfpathlineto{\pgfqpoint{6.918182in}{1.588152in}}%
\pgfusepath{stroke}%
\end{pgfscope}%
\begin{pgfscope}%
\pgfpathrectangle{\pgfqpoint{1.000000in}{0.330000in}}{\pgfqpoint{6.200000in}{2.310000in}}%
\pgfusepath{clip}%
\pgfsetrectcap%
\pgfsetroundjoin%
\pgfsetlinewidth{1.505625pt}%
\definecolor{currentstroke}{rgb}{0.580392,0.403922,0.741176}%
\pgfsetstrokecolor{currentstroke}%
\pgfsetdash{}{0pt}%
\pgfpathmoveto{\pgfqpoint{1.281818in}{1.588152in}}%
\pgfpathlineto{\pgfqpoint{1.281818in}{1.801827in}}%
\pgfpathlineto{\pgfqpoint{2.390303in}{1.727561in}}%
\pgfpathlineto{\pgfqpoint{2.409091in}{1.505863in}}%
\pgfpathlineto{\pgfqpoint{4.626061in}{1.357331in}}%
\pgfpathlineto{\pgfqpoint{4.663636in}{1.888410in}}%
\pgfpathlineto{\pgfqpoint{4.870303in}{1.786650in}}%
\pgfpathlineto{\pgfqpoint{4.980220in}{1.731499in}}%
\pgfpathlineto{\pgfqpoint{5.790909in}{1.677184in}}%
\pgfpathlineto{\pgfqpoint{6.692727in}{1.469805in}}%
\pgfpathlineto{\pgfqpoint{6.918182in}{1.433065in}}%
\pgfpathlineto{\pgfqpoint{6.918182in}{1.588152in}}%
\pgfpathlineto{\pgfqpoint{6.918182in}{1.588152in}}%
\pgfusepath{stroke}%
\end{pgfscope}%
\begin{pgfscope}%
\pgfpathrectangle{\pgfqpoint{1.000000in}{0.330000in}}{\pgfqpoint{6.200000in}{2.310000in}}%
\pgfusepath{clip}%
\pgfsetrectcap%
\pgfsetroundjoin%
\pgfsetlinewidth{1.505625pt}%
\definecolor{currentstroke}{rgb}{0.549020,0.337255,0.294118}%
\pgfsetstrokecolor{currentstroke}%
\pgfsetdash{}{0pt}%
\pgfpathmoveto{\pgfqpoint{1.281818in}{1.588152in}}%
\pgfpathlineto{\pgfqpoint{1.281818in}{2.259032in}}%
\pgfpathlineto{\pgfqpoint{1.526061in}{2.208133in}}%
\pgfpathlineto{\pgfqpoint{1.770303in}{2.155026in}}%
\pgfpathlineto{\pgfqpoint{2.014545in}{2.099711in}}%
\pgfpathlineto{\pgfqpoint{2.258788in}{2.042188in}}%
\pgfpathlineto{\pgfqpoint{2.390303in}{2.010299in}}%
\pgfpathlineto{\pgfqpoint{2.409091in}{1.417852in}}%
\pgfpathlineto{\pgfqpoint{2.653333in}{1.356763in}}%
\pgfpathlineto{\pgfqpoint{2.897576in}{1.293466in}}%
\pgfpathlineto{\pgfqpoint{3.141818in}{1.227960in}}%
\pgfpathlineto{\pgfqpoint{3.386061in}{1.160247in}}%
\pgfpathlineto{\pgfqpoint{3.630303in}{1.090327in}}%
\pgfpathlineto{\pgfqpoint{3.874545in}{1.018198in}}%
\pgfpathlineto{\pgfqpoint{4.118788in}{0.943862in}}%
\pgfpathlineto{\pgfqpoint{4.363030in}{0.867317in}}%
\pgfpathlineto{\pgfqpoint{4.607273in}{0.788565in}}%
\pgfpathlineto{\pgfqpoint{4.626061in}{0.782416in}}%
\pgfpathlineto{\pgfqpoint{4.663636in}{2.030622in}}%
\pgfpathlineto{\pgfqpoint{4.907879in}{1.949153in}}%
\pgfpathlineto{\pgfqpoint{4.980220in}{1.924599in}}%
\pgfpathlineto{\pgfqpoint{5.790909in}{1.870285in}}%
\pgfpathlineto{\pgfqpoint{6.692727in}{1.718015in}}%
\pgfpathlineto{\pgfqpoint{6.918182in}{1.695053in}}%
\pgfpathlineto{\pgfqpoint{6.918182in}{1.588152in}}%
\pgfpathlineto{\pgfqpoint{6.918182in}{1.588152in}}%
\pgfusepath{stroke}%
\end{pgfscope}%
\begin{pgfscope}%
\pgfpathrectangle{\pgfqpoint{1.000000in}{0.330000in}}{\pgfqpoint{6.200000in}{2.310000in}}%
\pgfusepath{clip}%
\pgfsetrectcap%
\pgfsetroundjoin%
\pgfsetlinewidth{1.505625pt}%
\definecolor{currentstroke}{rgb}{0.890196,0.466667,0.760784}%
\pgfsetstrokecolor{currentstroke}%
\pgfsetdash{}{0pt}%
\pgfpathmoveto{\pgfqpoint{1.281818in}{1.588152in}}%
\pgfpathlineto{\pgfqpoint{1.281818in}{2.530889in}}%
\pgfpathlineto{\pgfqpoint{1.488485in}{2.470516in}}%
\pgfpathlineto{\pgfqpoint{1.695152in}{2.407615in}}%
\pgfpathlineto{\pgfqpoint{1.901818in}{2.342184in}}%
\pgfpathlineto{\pgfqpoint{2.108485in}{2.274224in}}%
\pgfpathlineto{\pgfqpoint{2.315152in}{2.203735in}}%
\pgfpathlineto{\pgfqpoint{2.390303in}{2.177475in}}%
\pgfpathlineto{\pgfqpoint{2.409091in}{1.362579in}}%
\pgfpathlineto{\pgfqpoint{2.615758in}{1.288411in}}%
\pgfpathlineto{\pgfqpoint{2.822424in}{1.211714in}}%
\pgfpathlineto{\pgfqpoint{3.029091in}{1.132487in}}%
\pgfpathlineto{\pgfqpoint{3.235758in}{1.050732in}}%
\pgfpathlineto{\pgfqpoint{3.442424in}{0.966447in}}%
\pgfpathlineto{\pgfqpoint{3.649091in}{0.879633in}}%
\pgfpathlineto{\pgfqpoint{3.855758in}{0.790289in}}%
\pgfpathlineto{\pgfqpoint{4.062424in}{0.698417in}}%
\pgfpathlineto{\pgfqpoint{4.269091in}{0.604015in}}%
\pgfpathlineto{\pgfqpoint{4.475758in}{0.507084in}}%
\pgfpathlineto{\pgfqpoint{4.626061in}{0.435000in}}%
\pgfpathlineto{\pgfqpoint{4.663636in}{2.206420in}}%
\pgfpathlineto{\pgfqpoint{4.870303in}{2.104660in}}%
\pgfpathlineto{\pgfqpoint{4.980220in}{2.049509in}}%
\pgfpathlineto{\pgfqpoint{5.790909in}{1.995195in}}%
\pgfpathlineto{\pgfqpoint{6.692727in}{1.787815in}}%
\pgfpathlineto{\pgfqpoint{6.918182in}{1.751075in}}%
\pgfpathlineto{\pgfqpoint{6.918182in}{1.588152in}}%
\pgfpathlineto{\pgfqpoint{6.918182in}{1.588152in}}%
\pgfusepath{stroke}%
\end{pgfscope}%
\begin{pgfscope}%
\pgfpathrectangle{\pgfqpoint{1.000000in}{0.330000in}}{\pgfqpoint{6.200000in}{2.310000in}}%
\pgfusepath{clip}%
\pgfsetrectcap%
\pgfsetroundjoin%
\pgfsetlinewidth{1.505625pt}%
\definecolor{currentstroke}{rgb}{0.498039,0.498039,0.498039}%
\pgfsetstrokecolor{currentstroke}%
\pgfsetdash{}{0pt}%
\pgfpathmoveto{\pgfqpoint{1.281818in}{1.588152in}}%
\pgfpathlineto{\pgfqpoint{1.281818in}{2.068783in}}%
\pgfpathlineto{\pgfqpoint{1.638788in}{2.015271in}}%
\pgfpathlineto{\pgfqpoint{1.995758in}{1.959401in}}%
\pgfpathlineto{\pgfqpoint{2.352727in}{1.901173in}}%
\pgfpathlineto{\pgfqpoint{2.390303in}{1.894906in}}%
\pgfpathlineto{\pgfqpoint{2.409091in}{1.450884in}}%
\pgfpathlineto{\pgfqpoint{2.766061in}{1.389925in}}%
\pgfpathlineto{\pgfqpoint{3.123030in}{1.326608in}}%
\pgfpathlineto{\pgfqpoint{3.480000in}{1.260933in}}%
\pgfpathlineto{\pgfqpoint{3.836970in}{1.192900in}}%
\pgfpathlineto{\pgfqpoint{4.193939in}{1.122509in}}%
\pgfpathlineto{\pgfqpoint{4.550909in}{1.049760in}}%
\pgfpathlineto{\pgfqpoint{4.626061in}{1.034144in}}%
\pgfpathlineto{\pgfqpoint{4.663636in}{1.909037in}}%
\pgfpathlineto{\pgfqpoint{4.980220in}{1.841885in}}%
\pgfpathlineto{\pgfqpoint{5.809697in}{1.776093in}}%
\pgfpathlineto{\pgfqpoint{6.692727in}{1.662103in}}%
\pgfpathlineto{\pgfqpoint{6.918182in}{1.650622in}}%
\pgfpathlineto{\pgfqpoint{6.918182in}{1.588152in}}%
\pgfpathlineto{\pgfqpoint{6.918182in}{1.588152in}}%
\pgfusepath{stroke}%
\end{pgfscope}%
\begin{pgfscope}%
\pgfpathrectangle{\pgfqpoint{1.000000in}{0.330000in}}{\pgfqpoint{6.200000in}{2.310000in}}%
\pgfusepath{clip}%
\pgfsetrectcap%
\pgfsetroundjoin%
\pgfsetlinewidth{1.505625pt}%
\definecolor{currentstroke}{rgb}{0.737255,0.741176,0.133333}%
\pgfsetstrokecolor{currentstroke}%
\pgfsetdash{}{0pt}%
\pgfpathmoveto{\pgfqpoint{1.281818in}{1.588152in}}%
\pgfpathlineto{\pgfqpoint{1.281818in}{2.261602in}}%
\pgfpathlineto{\pgfqpoint{1.526061in}{2.210703in}}%
\pgfpathlineto{\pgfqpoint{1.770303in}{2.157595in}}%
\pgfpathlineto{\pgfqpoint{2.014545in}{2.102280in}}%
\pgfpathlineto{\pgfqpoint{2.258788in}{2.044757in}}%
\pgfpathlineto{\pgfqpoint{2.390303in}{2.012869in}}%
\pgfpathlineto{\pgfqpoint{2.409091in}{1.420422in}}%
\pgfpathlineto{\pgfqpoint{2.653333in}{1.359332in}}%
\pgfpathlineto{\pgfqpoint{2.897576in}{1.296035in}}%
\pgfpathlineto{\pgfqpoint{3.141818in}{1.230530in}}%
\pgfpathlineto{\pgfqpoint{3.386061in}{1.162817in}}%
\pgfpathlineto{\pgfqpoint{3.630303in}{1.092896in}}%
\pgfpathlineto{\pgfqpoint{3.874545in}{1.020768in}}%
\pgfpathlineto{\pgfqpoint{4.118788in}{0.946431in}}%
\pgfpathlineto{\pgfqpoint{4.363030in}{0.869887in}}%
\pgfpathlineto{\pgfqpoint{4.607273in}{0.791135in}}%
\pgfpathlineto{\pgfqpoint{4.626061in}{0.784985in}}%
\pgfpathlineto{\pgfqpoint{4.663636in}{1.936383in}}%
\pgfpathlineto{\pgfqpoint{6.692727in}{1.800438in}}%
\pgfpathlineto{\pgfqpoint{6.918182in}{1.800438in}}%
\pgfpathlineto{\pgfqpoint{6.918182in}{1.588152in}}%
\pgfpathlineto{\pgfqpoint{6.918182in}{1.588152in}}%
\pgfusepath{stroke}%
\end{pgfscope}%
\begin{pgfscope}%
\pgfpathrectangle{\pgfqpoint{1.000000in}{0.330000in}}{\pgfqpoint{6.200000in}{2.310000in}}%
\pgfusepath{clip}%
\pgfsetrectcap%
\pgfsetroundjoin%
\pgfsetlinewidth{1.505625pt}%
\definecolor{currentstroke}{rgb}{0.090196,0.745098,0.811765}%
\pgfsetstrokecolor{currentstroke}%
\pgfsetdash{}{0pt}%
\pgfpathmoveto{\pgfqpoint{1.281818in}{1.588152in}}%
\pgfpathlineto{\pgfqpoint{1.281818in}{2.070068in}}%
\pgfpathlineto{\pgfqpoint{1.638788in}{2.016556in}}%
\pgfpathlineto{\pgfqpoint{1.995758in}{1.960686in}}%
\pgfpathlineto{\pgfqpoint{2.352727in}{1.902457in}}%
\pgfpathlineto{\pgfqpoint{2.390303in}{1.896191in}}%
\pgfpathlineto{\pgfqpoint{2.409091in}{1.452168in}}%
\pgfpathlineto{\pgfqpoint{2.766061in}{1.391210in}}%
\pgfpathlineto{\pgfqpoint{3.123030in}{1.327893in}}%
\pgfpathlineto{\pgfqpoint{3.480000in}{1.262218in}}%
\pgfpathlineto{\pgfqpoint{3.836970in}{1.194185in}}%
\pgfpathlineto{\pgfqpoint{4.193939in}{1.123794in}}%
\pgfpathlineto{\pgfqpoint{4.550909in}{1.051045in}}%
\pgfpathlineto{\pgfqpoint{4.626061in}{1.035429in}}%
\pgfpathlineto{\pgfqpoint{4.663636in}{1.861917in}}%
\pgfpathlineto{\pgfqpoint{6.692727in}{1.703315in}}%
\pgfpathlineto{\pgfqpoint{6.918182in}{1.703315in}}%
\pgfpathlineto{\pgfqpoint{6.918182in}{1.588152in}}%
\pgfpathlineto{\pgfqpoint{6.918182in}{1.588152in}}%
\pgfusepath{stroke}%
\end{pgfscope}%
\begin{pgfscope}%
\pgfpathrectangle{\pgfqpoint{1.000000in}{0.330000in}}{\pgfqpoint{6.200000in}{2.310000in}}%
\pgfusepath{clip}%
\pgfsetrectcap%
\pgfsetroundjoin%
\pgfsetlinewidth{1.505625pt}%
\definecolor{currentstroke}{rgb}{0.121569,0.466667,0.705882}%
\pgfsetstrokecolor{currentstroke}%
\pgfsetdash{}{0pt}%
\pgfpathmoveto{\pgfqpoint{1.281818in}{1.588152in}}%
\pgfpathlineto{\pgfqpoint{1.281818in}{1.840952in}}%
\pgfpathlineto{\pgfqpoint{2.390303in}{1.754308in}}%
\pgfpathlineto{\pgfqpoint{2.409091in}{1.495660in}}%
\pgfpathlineto{\pgfqpoint{4.626061in}{1.322373in}}%
\pgfpathlineto{\pgfqpoint{4.663636in}{1.809658in}}%
\pgfpathlineto{\pgfqpoint{4.980220in}{1.742507in}}%
\pgfpathlineto{\pgfqpoint{5.809697in}{1.676715in}}%
\pgfpathlineto{\pgfqpoint{6.692727in}{1.562725in}}%
\pgfpathlineto{\pgfqpoint{6.918182in}{1.551244in}}%
\pgfpathlineto{\pgfqpoint{6.918182in}{1.588152in}}%
\pgfpathlineto{\pgfqpoint{6.918182in}{1.588152in}}%
\pgfusepath{stroke}%
\end{pgfscope}%
\begin{pgfscope}%
\pgfpathrectangle{\pgfqpoint{1.000000in}{0.330000in}}{\pgfqpoint{6.200000in}{2.310000in}}%
\pgfusepath{clip}%
\pgfsetrectcap%
\pgfsetroundjoin%
\pgfsetlinewidth{1.505625pt}%
\definecolor{currentstroke}{rgb}{1.000000,0.498039,0.054902}%
\pgfsetstrokecolor{currentstroke}%
\pgfsetdash{}{0pt}%
\pgfpathmoveto{\pgfqpoint{1.281818in}{1.588152in}}%
\pgfpathlineto{\pgfqpoint{1.281818in}{2.535000in}}%
\pgfpathlineto{\pgfqpoint{1.488485in}{2.474628in}}%
\pgfpathlineto{\pgfqpoint{1.695152in}{2.411726in}}%
\pgfpathlineto{\pgfqpoint{1.901818in}{2.346295in}}%
\pgfpathlineto{\pgfqpoint{2.108485in}{2.278335in}}%
\pgfpathlineto{\pgfqpoint{2.315152in}{2.207846in}}%
\pgfpathlineto{\pgfqpoint{2.390303in}{2.181587in}}%
\pgfpathlineto{\pgfqpoint{2.409091in}{1.366690in}}%
\pgfpathlineto{\pgfqpoint{2.615758in}{1.292522in}}%
\pgfpathlineto{\pgfqpoint{2.822424in}{1.215825in}}%
\pgfpathlineto{\pgfqpoint{3.029091in}{1.136598in}}%
\pgfpathlineto{\pgfqpoint{3.235758in}{1.054843in}}%
\pgfpathlineto{\pgfqpoint{3.442424in}{0.970558in}}%
\pgfpathlineto{\pgfqpoint{3.649091in}{0.883744in}}%
\pgfpathlineto{\pgfqpoint{3.855758in}{0.794400in}}%
\pgfpathlineto{\pgfqpoint{4.062424in}{0.702528in}}%
\pgfpathlineto{\pgfqpoint{4.269091in}{0.608126in}}%
\pgfpathlineto{\pgfqpoint{4.475758in}{0.511195in}}%
\pgfpathlineto{\pgfqpoint{4.626061in}{0.439111in}}%
\pgfpathlineto{\pgfqpoint{4.663636in}{2.055636in}}%
\pgfpathlineto{\pgfqpoint{6.692727in}{1.919692in}}%
\pgfpathlineto{\pgfqpoint{6.918182in}{1.919692in}}%
\pgfpathlineto{\pgfqpoint{6.918182in}{1.588152in}}%
\pgfpathlineto{\pgfqpoint{6.918182in}{1.588152in}}%
\pgfusepath{stroke}%
\end{pgfscope}%
\begin{pgfscope}%
\pgfpathrectangle{\pgfqpoint{1.000000in}{0.330000in}}{\pgfqpoint{6.200000in}{2.310000in}}%
\pgfusepath{clip}%
\pgfsetrectcap%
\pgfsetroundjoin%
\pgfsetlinewidth{1.505625pt}%
\definecolor{currentstroke}{rgb}{0.172549,0.627451,0.172549}%
\pgfsetstrokecolor{currentstroke}%
\pgfsetdash{}{0pt}%
\pgfpathmoveto{\pgfqpoint{1.281818in}{1.588152in}}%
\pgfpathlineto{\pgfqpoint{1.281818in}{1.842236in}}%
\pgfpathlineto{\pgfqpoint{2.390303in}{1.755593in}}%
\pgfpathlineto{\pgfqpoint{2.409091in}{1.496944in}}%
\pgfpathlineto{\pgfqpoint{4.626061in}{1.323657in}}%
\pgfpathlineto{\pgfqpoint{4.663636in}{1.762539in}}%
\pgfpathlineto{\pgfqpoint{6.692727in}{1.603937in}}%
\pgfpathlineto{\pgfqpoint{6.918182in}{1.603937in}}%
\pgfpathlineto{\pgfqpoint{6.918182in}{1.588152in}}%
\pgfpathlineto{\pgfqpoint{6.918182in}{1.588152in}}%
\pgfusepath{stroke}%
\end{pgfscope}%
\begin{pgfscope}%
\pgfsetroundcap%
\pgfsetroundjoin%
\pgfsetlinewidth{1.003750pt}%
\definecolor{currentstroke}{rgb}{0.000000,0.000000,0.000000}%
\pgfsetstrokecolor{currentstroke}%
\pgfsetdash{}{0pt}%
\pgfpathmoveto{\pgfqpoint{4.204719in}{0.435000in}}%
\pgfpathquadraticcurveto{\pgfqpoint{4.401490in}{0.435000in}}{\pgfqpoint{4.598261in}{0.435000in}}%
\pgfusepath{stroke}%
\end{pgfscope}%
\begin{pgfscope}%
\pgfsetbuttcap%
\pgfsetmiterjoin%
\definecolor{currentfill}{rgb}{0.800000,0.800000,0.800000}%
\pgfsetfillcolor{currentfill}%
\pgfsetlinewidth{1.003750pt}%
\definecolor{currentstroke}{rgb}{0.000000,0.000000,0.000000}%
\pgfsetstrokecolor{currentstroke}%
\pgfsetdash{}{0pt}%
\pgfpathmoveto{\pgfqpoint{3.202449in}{0.338549in}}%
\pgfpathcurveto{\pgfqpoint{3.237172in}{0.303827in}}{\pgfqpoint{4.112253in}{0.303827in}}{\pgfqpoint{4.146975in}{0.338549in}}%
\pgfpathcurveto{\pgfqpoint{4.181698in}{0.373272in}}{\pgfqpoint{4.181698in}{0.496728in}}{\pgfqpoint{4.146975in}{0.531451in}}%
\pgfpathcurveto{\pgfqpoint{4.112253in}{0.566173in}}{\pgfqpoint{3.237172in}{0.566173in}}{\pgfqpoint{3.202449in}{0.531451in}}%
\pgfpathcurveto{\pgfqpoint{3.167727in}{0.496728in}}{\pgfqpoint{3.167727in}{0.373272in}}{\pgfqpoint{3.202449in}{0.338549in}}%
\pgfpathclose%
\pgfusepath{stroke,fill}%
\end{pgfscope}%
\begin{pgfscope}%
\definecolor{textcolor}{rgb}{0.000000,0.000000,0.000000}%
\pgfsetstrokecolor{textcolor}%
\pgfsetfillcolor{textcolor}%
\pgftext[x=3.237172in,y=0.435000in,left,]{\color{textcolor}\rmfamily\fontsize{10.000000}{12.000000}\selectfont \(\displaystyle V_u =\) -12.6 kip}%
\end{pgfscope}%
\begin{pgfscope}%
\pgfsetbuttcap%
\pgfsetmiterjoin%
\definecolor{currentfill}{rgb}{0.800000,0.800000,0.800000}%
\pgfsetfillcolor{currentfill}%
\pgfsetlinewidth{1.003750pt}%
\definecolor{currentstroke}{rgb}{0.000000,0.000000,0.000000}%
\pgfsetstrokecolor{currentstroke}%
\pgfsetdash{}{0pt}%
\pgfpathmoveto{\pgfqpoint{0.965278in}{0.358599in}}%
\pgfpathcurveto{\pgfqpoint{1.000000in}{0.323877in}}{\pgfqpoint{3.162428in}{0.323877in}}{\pgfqpoint{3.197150in}{0.358599in}}%
\pgfpathcurveto{\pgfqpoint{3.231872in}{0.393321in}}{\pgfqpoint{3.231872in}{0.668784in}}{\pgfqpoint{3.197150in}{0.703506in}}%
\pgfpathcurveto{\pgfqpoint{3.162428in}{0.738228in}}{\pgfqpoint{1.000000in}{0.738228in}}{\pgfqpoint{0.965278in}{0.703506in}}%
\pgfpathcurveto{\pgfqpoint{0.930556in}{0.668784in}}{\pgfqpoint{0.930556in}{0.393321in}}{\pgfqpoint{0.965278in}{0.358599in}}%
\pgfpathclose%
\pgfusepath{stroke,fill}%
\end{pgfscope}%
\begin{pgfscope}%
\definecolor{textcolor}{rgb}{0.000000,0.000000,0.000000}%
\pgfsetstrokecolor{textcolor}%
\pgfsetfillcolor{textcolor}%
\pgftext[x=1.000000in, y=0.580049in, left, base]{\color{textcolor}\rmfamily\fontsize{10.000000}{12.000000}\selectfont Max combo: 1.2D + 1.6L0 + 1.6L1}%
\end{pgfscope}%
\begin{pgfscope}%
\definecolor{textcolor}{rgb}{0.000000,0.000000,0.000000}%
\pgfsetstrokecolor{textcolor}%
\pgfsetfillcolor{textcolor}%
\pgftext[x=1.000000in, y=0.428043in, left, base]{\color{textcolor}\rmfamily\fontsize{10.000000}{12.000000}\selectfont ASCE7-16 Sec. 2.3.1 (LC 2)}%
\end{pgfscope}%
\end{pgfpicture}%
\makeatother%
\endgroup%

\end{center}
\caption{Shear Demand Envelope}
\end{figure}
C\textsubscript{v1}, the web shear strength coefficient, is calculated per AISC/ANSI 360-16 Eq. G2-2 as follows, based on the ratio of the clear distance between flanges to web thickness:
\begin{flalign*}
\frac{h}{t_w} = \frac{8.9 {\color{darkBlue}{\mathbf{ \; in}}}}{0.2 {\color{darkBlue}{\mathbf{ \; in}}}} = \mathbf{37.0 } <= 2.24\sqrt{\frac{E}{F_y}} = 2.24\sqrt{\frac{29000 {\color{darkBlue}{\mathbf{ \; ksi}}}}{50 {\color{darkBlue}{\mathbf{ \; ksi}}}}} = \mathbf{53.9 } \rightarrow C_{v1} = \mathbf{1.0}
\end{flalign*}
\textphi\textsubscript{v}, the resistance factor for shear, is calculated per AISC/ANSI 360-16 {\S}G2.1.a as follows:
\begin{flalign*}
\frac{h}{t_w} = \frac{8.9 {\color{darkBlue}{\mathbf{ \; in}}}}{0.24 {\color{darkBlue}{\mathbf{ \; in}}}} = \mathbf{37.0 } <= 2.24\cdot \sqrt{\frac{E}{F_y}} = 2.24\cdot \sqrt{\frac{29000 {\color{darkBlue}{\mathbf{ \; ksi}}}}{50 {\color{darkBlue}{\mathbf{ \; ksi}}}}} = \mathbf{53.9} \rightarrow \phi_v = \mathbf{1.0}
\end{flalign*}
\textphi\textsubscript{v}V\textsubscript{n}, the design shear strength, is calculated per AISC/ANSI 360-16 Eq. G2-1 as follows:
\begin{flalign*}
\phi_v V_n = 0.6\cdot F_y \cdot A_w \cdot C_{v1}  = 0.6\cdot 50 {\color{darkBlue}{\mathbf{ \; ksi}}} \cdot 2.45 {\color{darkBlue}{\mathbf{ \; {\color{darkBlue}{\mathbf{ \; in}}}^{2}}}} \cdot 1.0  = \mathbf{73.4 {\color{darkBlue}{\mathbf{ \; kip}}}}
\end{flalign*}
\vspace{-26pt}
{\setlength{\mathindent}{0cm}
\begin{flalign*}
\mathbf{|V_u| = 12.6 {\color{darkBlue}{\mathbf{ \; kip}}}  \;  < \phi_v \cdot V_n = 73.4 {\color{darkBlue}{\mathbf{ \; kip}}}  \;  (DCR = 0.17 - OK)}
\end{flalign*}
%	----------------------------- DEFLECTION CHECK -------------------------------
\section{Deflection Check}
\begin{figure}[H]
\begin{center}
%% Creator: Matplotlib, PGF backend
%%
%% To include the figure in your LaTeX document, write
%%   \input{<filename>.pgf}
%%
%% Make sure the required packages are loaded in your preamble
%%   \usepackage{pgf}
%%
%% Figures using additional raster images can only be included by \input if
%% they are in the same directory as the main LaTeX file. For loading figures
%% from other directories you can use the `import` package
%%   \usepackage{import}
%%
%% and then include the figures with
%%   \import{<path to file>}{<filename>.pgf}
%%
%% Matplotlib used the following preamble
%%
\begingroup%
\makeatletter%
\begin{pgfpicture}%
\pgfpathrectangle{\pgfpointorigin}{\pgfqpoint{8.000000in}{3.000000in}}%
\pgfusepath{use as bounding box, clip}%
\begin{pgfscope}%
\pgfsetbuttcap%
\pgfsetmiterjoin%
\definecolor{currentfill}{rgb}{1.000000,1.000000,1.000000}%
\pgfsetfillcolor{currentfill}%
\pgfsetlinewidth{0.000000pt}%
\definecolor{currentstroke}{rgb}{1.000000,1.000000,1.000000}%
\pgfsetstrokecolor{currentstroke}%
\pgfsetdash{}{0pt}%
\pgfpathmoveto{\pgfqpoint{0.000000in}{0.000000in}}%
\pgfpathlineto{\pgfqpoint{8.000000in}{0.000000in}}%
\pgfpathlineto{\pgfqpoint{8.000000in}{3.000000in}}%
\pgfpathlineto{\pgfqpoint{0.000000in}{3.000000in}}%
\pgfpathclose%
\pgfusepath{fill}%
\end{pgfscope}%
\begin{pgfscope}%
\pgfsetbuttcap%
\pgfsetmiterjoin%
\definecolor{currentfill}{rgb}{1.000000,1.000000,1.000000}%
\pgfsetfillcolor{currentfill}%
\pgfsetlinewidth{0.000000pt}%
\definecolor{currentstroke}{rgb}{0.000000,0.000000,0.000000}%
\pgfsetstrokecolor{currentstroke}%
\pgfsetstrokeopacity{0.000000}%
\pgfsetdash{}{0pt}%
\pgfpathmoveto{\pgfqpoint{1.000000in}{0.330000in}}%
\pgfpathlineto{\pgfqpoint{7.200000in}{0.330000in}}%
\pgfpathlineto{\pgfqpoint{7.200000in}{2.640000in}}%
\pgfpathlineto{\pgfqpoint{1.000000in}{2.640000in}}%
\pgfpathclose%
\pgfusepath{fill}%
\end{pgfscope}%
\begin{pgfscope}%
\pgfpathrectangle{\pgfqpoint{1.000000in}{0.330000in}}{\pgfqpoint{6.200000in}{2.310000in}}%
\pgfusepath{clip}%
\pgfsetbuttcap%
\pgfsetroundjoin%
\pgfsetlinewidth{0.803000pt}%
\definecolor{currentstroke}{rgb}{0.000000,0.000000,0.000000}%
\pgfsetstrokecolor{currentstroke}%
\pgfsetdash{{0.800000pt}{1.320000pt}}{0.000000pt}%
\pgfpathmoveto{\pgfqpoint{1.281818in}{0.330000in}}%
\pgfpathlineto{\pgfqpoint{1.281818in}{2.640000in}}%
\pgfusepath{stroke}%
\end{pgfscope}%
\begin{pgfscope}%
\pgfsetbuttcap%
\pgfsetroundjoin%
\definecolor{currentfill}{rgb}{0.000000,0.000000,0.000000}%
\pgfsetfillcolor{currentfill}%
\pgfsetlinewidth{0.803000pt}%
\definecolor{currentstroke}{rgb}{0.000000,0.000000,0.000000}%
\pgfsetstrokecolor{currentstroke}%
\pgfsetdash{}{0pt}%
\pgfsys@defobject{currentmarker}{\pgfqpoint{0.000000in}{-0.048611in}}{\pgfqpoint{0.000000in}{0.000000in}}{%
\pgfpathmoveto{\pgfqpoint{0.000000in}{0.000000in}}%
\pgfpathlineto{\pgfqpoint{0.000000in}{-0.048611in}}%
\pgfusepath{stroke,fill}%
}%
\begin{pgfscope}%
\pgfsys@transformshift{1.281818in}{0.330000in}%
\pgfsys@useobject{currentmarker}{}%
\end{pgfscope}%
\end{pgfscope}%
\begin{pgfscope}%
\pgfsetbuttcap%
\pgfsetroundjoin%
\definecolor{currentfill}{rgb}{0.000000,0.000000,0.000000}%
\pgfsetfillcolor{currentfill}%
\pgfsetlinewidth{0.803000pt}%
\definecolor{currentstroke}{rgb}{0.000000,0.000000,0.000000}%
\pgfsetstrokecolor{currentstroke}%
\pgfsetdash{}{0pt}%
\pgfsys@defobject{currentmarker}{\pgfqpoint{0.000000in}{0.000000in}}{\pgfqpoint{0.000000in}{0.048611in}}{%
\pgfpathmoveto{\pgfqpoint{0.000000in}{0.000000in}}%
\pgfpathlineto{\pgfqpoint{0.000000in}{0.048611in}}%
\pgfusepath{stroke,fill}%
}%
\begin{pgfscope}%
\pgfsys@transformshift{1.281818in}{2.640000in}%
\pgfsys@useobject{currentmarker}{}%
\end{pgfscope}%
\end{pgfscope}%
\begin{pgfscope}%
\definecolor{textcolor}{rgb}{0.000000,0.000000,0.000000}%
\pgfsetstrokecolor{textcolor}%
\pgfsetfillcolor{textcolor}%
\pgftext[x=1.281818in,y=0.232778in,,top]{\color{textcolor}\rmfamily\fontsize{10.000000}{12.000000}\selectfont \(\displaystyle {0}\)}%
\end{pgfscope}%
\begin{pgfscope}%
\pgfpathrectangle{\pgfqpoint{1.000000in}{0.330000in}}{\pgfqpoint{6.200000in}{2.310000in}}%
\pgfusepath{clip}%
\pgfsetbuttcap%
\pgfsetroundjoin%
\pgfsetlinewidth{0.803000pt}%
\definecolor{currentstroke}{rgb}{0.000000,0.000000,0.000000}%
\pgfsetstrokecolor{currentstroke}%
\pgfsetdash{{0.800000pt}{1.320000pt}}{0.000000pt}%
\pgfpathmoveto{\pgfqpoint{2.409091in}{0.330000in}}%
\pgfpathlineto{\pgfqpoint{2.409091in}{2.640000in}}%
\pgfusepath{stroke}%
\end{pgfscope}%
\begin{pgfscope}%
\pgfsetbuttcap%
\pgfsetroundjoin%
\definecolor{currentfill}{rgb}{0.000000,0.000000,0.000000}%
\pgfsetfillcolor{currentfill}%
\pgfsetlinewidth{0.803000pt}%
\definecolor{currentstroke}{rgb}{0.000000,0.000000,0.000000}%
\pgfsetstrokecolor{currentstroke}%
\pgfsetdash{}{0pt}%
\pgfsys@defobject{currentmarker}{\pgfqpoint{0.000000in}{-0.048611in}}{\pgfqpoint{0.000000in}{0.000000in}}{%
\pgfpathmoveto{\pgfqpoint{0.000000in}{0.000000in}}%
\pgfpathlineto{\pgfqpoint{0.000000in}{-0.048611in}}%
\pgfusepath{stroke,fill}%
}%
\begin{pgfscope}%
\pgfsys@transformshift{2.409091in}{0.330000in}%
\pgfsys@useobject{currentmarker}{}%
\end{pgfscope}%
\end{pgfscope}%
\begin{pgfscope}%
\pgfsetbuttcap%
\pgfsetroundjoin%
\definecolor{currentfill}{rgb}{0.000000,0.000000,0.000000}%
\pgfsetfillcolor{currentfill}%
\pgfsetlinewidth{0.803000pt}%
\definecolor{currentstroke}{rgb}{0.000000,0.000000,0.000000}%
\pgfsetstrokecolor{currentstroke}%
\pgfsetdash{}{0pt}%
\pgfsys@defobject{currentmarker}{\pgfqpoint{0.000000in}{0.000000in}}{\pgfqpoint{0.000000in}{0.048611in}}{%
\pgfpathmoveto{\pgfqpoint{0.000000in}{0.000000in}}%
\pgfpathlineto{\pgfqpoint{0.000000in}{0.048611in}}%
\pgfusepath{stroke,fill}%
}%
\begin{pgfscope}%
\pgfsys@transformshift{2.409091in}{2.640000in}%
\pgfsys@useobject{currentmarker}{}%
\end{pgfscope}%
\end{pgfscope}%
\begin{pgfscope}%
\definecolor{textcolor}{rgb}{0.000000,0.000000,0.000000}%
\pgfsetstrokecolor{textcolor}%
\pgfsetfillcolor{textcolor}%
\pgftext[x=2.409091in,y=0.232778in,,top]{\color{textcolor}\rmfamily\fontsize{10.000000}{12.000000}\selectfont \(\displaystyle {5}\)}%
\end{pgfscope}%
\begin{pgfscope}%
\pgfpathrectangle{\pgfqpoint{1.000000in}{0.330000in}}{\pgfqpoint{6.200000in}{2.310000in}}%
\pgfusepath{clip}%
\pgfsetbuttcap%
\pgfsetroundjoin%
\pgfsetlinewidth{0.803000pt}%
\definecolor{currentstroke}{rgb}{0.000000,0.000000,0.000000}%
\pgfsetstrokecolor{currentstroke}%
\pgfsetdash{{0.800000pt}{1.320000pt}}{0.000000pt}%
\pgfpathmoveto{\pgfqpoint{3.536364in}{0.330000in}}%
\pgfpathlineto{\pgfqpoint{3.536364in}{2.640000in}}%
\pgfusepath{stroke}%
\end{pgfscope}%
\begin{pgfscope}%
\pgfsetbuttcap%
\pgfsetroundjoin%
\definecolor{currentfill}{rgb}{0.000000,0.000000,0.000000}%
\pgfsetfillcolor{currentfill}%
\pgfsetlinewidth{0.803000pt}%
\definecolor{currentstroke}{rgb}{0.000000,0.000000,0.000000}%
\pgfsetstrokecolor{currentstroke}%
\pgfsetdash{}{0pt}%
\pgfsys@defobject{currentmarker}{\pgfqpoint{0.000000in}{-0.048611in}}{\pgfqpoint{0.000000in}{0.000000in}}{%
\pgfpathmoveto{\pgfqpoint{0.000000in}{0.000000in}}%
\pgfpathlineto{\pgfqpoint{0.000000in}{-0.048611in}}%
\pgfusepath{stroke,fill}%
}%
\begin{pgfscope}%
\pgfsys@transformshift{3.536364in}{0.330000in}%
\pgfsys@useobject{currentmarker}{}%
\end{pgfscope}%
\end{pgfscope}%
\begin{pgfscope}%
\pgfsetbuttcap%
\pgfsetroundjoin%
\definecolor{currentfill}{rgb}{0.000000,0.000000,0.000000}%
\pgfsetfillcolor{currentfill}%
\pgfsetlinewidth{0.803000pt}%
\definecolor{currentstroke}{rgb}{0.000000,0.000000,0.000000}%
\pgfsetstrokecolor{currentstroke}%
\pgfsetdash{}{0pt}%
\pgfsys@defobject{currentmarker}{\pgfqpoint{0.000000in}{0.000000in}}{\pgfqpoint{0.000000in}{0.048611in}}{%
\pgfpathmoveto{\pgfqpoint{0.000000in}{0.000000in}}%
\pgfpathlineto{\pgfqpoint{0.000000in}{0.048611in}}%
\pgfusepath{stroke,fill}%
}%
\begin{pgfscope}%
\pgfsys@transformshift{3.536364in}{2.640000in}%
\pgfsys@useobject{currentmarker}{}%
\end{pgfscope}%
\end{pgfscope}%
\begin{pgfscope}%
\definecolor{textcolor}{rgb}{0.000000,0.000000,0.000000}%
\pgfsetstrokecolor{textcolor}%
\pgfsetfillcolor{textcolor}%
\pgftext[x=3.536364in,y=0.232778in,,top]{\color{textcolor}\rmfamily\fontsize{10.000000}{12.000000}\selectfont \(\displaystyle {10}\)}%
\end{pgfscope}%
\begin{pgfscope}%
\pgfpathrectangle{\pgfqpoint{1.000000in}{0.330000in}}{\pgfqpoint{6.200000in}{2.310000in}}%
\pgfusepath{clip}%
\pgfsetbuttcap%
\pgfsetroundjoin%
\pgfsetlinewidth{0.803000pt}%
\definecolor{currentstroke}{rgb}{0.000000,0.000000,0.000000}%
\pgfsetstrokecolor{currentstroke}%
\pgfsetdash{{0.800000pt}{1.320000pt}}{0.000000pt}%
\pgfpathmoveto{\pgfqpoint{4.663636in}{0.330000in}}%
\pgfpathlineto{\pgfqpoint{4.663636in}{2.640000in}}%
\pgfusepath{stroke}%
\end{pgfscope}%
\begin{pgfscope}%
\pgfsetbuttcap%
\pgfsetroundjoin%
\definecolor{currentfill}{rgb}{0.000000,0.000000,0.000000}%
\pgfsetfillcolor{currentfill}%
\pgfsetlinewidth{0.803000pt}%
\definecolor{currentstroke}{rgb}{0.000000,0.000000,0.000000}%
\pgfsetstrokecolor{currentstroke}%
\pgfsetdash{}{0pt}%
\pgfsys@defobject{currentmarker}{\pgfqpoint{0.000000in}{-0.048611in}}{\pgfqpoint{0.000000in}{0.000000in}}{%
\pgfpathmoveto{\pgfqpoint{0.000000in}{0.000000in}}%
\pgfpathlineto{\pgfqpoint{0.000000in}{-0.048611in}}%
\pgfusepath{stroke,fill}%
}%
\begin{pgfscope}%
\pgfsys@transformshift{4.663636in}{0.330000in}%
\pgfsys@useobject{currentmarker}{}%
\end{pgfscope}%
\end{pgfscope}%
\begin{pgfscope}%
\pgfsetbuttcap%
\pgfsetroundjoin%
\definecolor{currentfill}{rgb}{0.000000,0.000000,0.000000}%
\pgfsetfillcolor{currentfill}%
\pgfsetlinewidth{0.803000pt}%
\definecolor{currentstroke}{rgb}{0.000000,0.000000,0.000000}%
\pgfsetstrokecolor{currentstroke}%
\pgfsetdash{}{0pt}%
\pgfsys@defobject{currentmarker}{\pgfqpoint{0.000000in}{0.000000in}}{\pgfqpoint{0.000000in}{0.048611in}}{%
\pgfpathmoveto{\pgfqpoint{0.000000in}{0.000000in}}%
\pgfpathlineto{\pgfqpoint{0.000000in}{0.048611in}}%
\pgfusepath{stroke,fill}%
}%
\begin{pgfscope}%
\pgfsys@transformshift{4.663636in}{2.640000in}%
\pgfsys@useobject{currentmarker}{}%
\end{pgfscope}%
\end{pgfscope}%
\begin{pgfscope}%
\definecolor{textcolor}{rgb}{0.000000,0.000000,0.000000}%
\pgfsetstrokecolor{textcolor}%
\pgfsetfillcolor{textcolor}%
\pgftext[x=4.663636in,y=0.232778in,,top]{\color{textcolor}\rmfamily\fontsize{10.000000}{12.000000}\selectfont \(\displaystyle {15}\)}%
\end{pgfscope}%
\begin{pgfscope}%
\pgfpathrectangle{\pgfqpoint{1.000000in}{0.330000in}}{\pgfqpoint{6.200000in}{2.310000in}}%
\pgfusepath{clip}%
\pgfsetbuttcap%
\pgfsetroundjoin%
\pgfsetlinewidth{0.803000pt}%
\definecolor{currentstroke}{rgb}{0.000000,0.000000,0.000000}%
\pgfsetstrokecolor{currentstroke}%
\pgfsetdash{{0.800000pt}{1.320000pt}}{0.000000pt}%
\pgfpathmoveto{\pgfqpoint{5.790909in}{0.330000in}}%
\pgfpathlineto{\pgfqpoint{5.790909in}{2.640000in}}%
\pgfusepath{stroke}%
\end{pgfscope}%
\begin{pgfscope}%
\pgfsetbuttcap%
\pgfsetroundjoin%
\definecolor{currentfill}{rgb}{0.000000,0.000000,0.000000}%
\pgfsetfillcolor{currentfill}%
\pgfsetlinewidth{0.803000pt}%
\definecolor{currentstroke}{rgb}{0.000000,0.000000,0.000000}%
\pgfsetstrokecolor{currentstroke}%
\pgfsetdash{}{0pt}%
\pgfsys@defobject{currentmarker}{\pgfqpoint{0.000000in}{-0.048611in}}{\pgfqpoint{0.000000in}{0.000000in}}{%
\pgfpathmoveto{\pgfqpoint{0.000000in}{0.000000in}}%
\pgfpathlineto{\pgfqpoint{0.000000in}{-0.048611in}}%
\pgfusepath{stroke,fill}%
}%
\begin{pgfscope}%
\pgfsys@transformshift{5.790909in}{0.330000in}%
\pgfsys@useobject{currentmarker}{}%
\end{pgfscope}%
\end{pgfscope}%
\begin{pgfscope}%
\pgfsetbuttcap%
\pgfsetroundjoin%
\definecolor{currentfill}{rgb}{0.000000,0.000000,0.000000}%
\pgfsetfillcolor{currentfill}%
\pgfsetlinewidth{0.803000pt}%
\definecolor{currentstroke}{rgb}{0.000000,0.000000,0.000000}%
\pgfsetstrokecolor{currentstroke}%
\pgfsetdash{}{0pt}%
\pgfsys@defobject{currentmarker}{\pgfqpoint{0.000000in}{0.000000in}}{\pgfqpoint{0.000000in}{0.048611in}}{%
\pgfpathmoveto{\pgfqpoint{0.000000in}{0.000000in}}%
\pgfpathlineto{\pgfqpoint{0.000000in}{0.048611in}}%
\pgfusepath{stroke,fill}%
}%
\begin{pgfscope}%
\pgfsys@transformshift{5.790909in}{2.640000in}%
\pgfsys@useobject{currentmarker}{}%
\end{pgfscope}%
\end{pgfscope}%
\begin{pgfscope}%
\definecolor{textcolor}{rgb}{0.000000,0.000000,0.000000}%
\pgfsetstrokecolor{textcolor}%
\pgfsetfillcolor{textcolor}%
\pgftext[x=5.790909in,y=0.232778in,,top]{\color{textcolor}\rmfamily\fontsize{10.000000}{12.000000}\selectfont \(\displaystyle {20}\)}%
\end{pgfscope}%
\begin{pgfscope}%
\pgfpathrectangle{\pgfqpoint{1.000000in}{0.330000in}}{\pgfqpoint{6.200000in}{2.310000in}}%
\pgfusepath{clip}%
\pgfsetbuttcap%
\pgfsetroundjoin%
\pgfsetlinewidth{0.803000pt}%
\definecolor{currentstroke}{rgb}{0.000000,0.000000,0.000000}%
\pgfsetstrokecolor{currentstroke}%
\pgfsetdash{{0.800000pt}{1.320000pt}}{0.000000pt}%
\pgfpathmoveto{\pgfqpoint{6.918182in}{0.330000in}}%
\pgfpathlineto{\pgfqpoint{6.918182in}{2.640000in}}%
\pgfusepath{stroke}%
\end{pgfscope}%
\begin{pgfscope}%
\pgfsetbuttcap%
\pgfsetroundjoin%
\definecolor{currentfill}{rgb}{0.000000,0.000000,0.000000}%
\pgfsetfillcolor{currentfill}%
\pgfsetlinewidth{0.803000pt}%
\definecolor{currentstroke}{rgb}{0.000000,0.000000,0.000000}%
\pgfsetstrokecolor{currentstroke}%
\pgfsetdash{}{0pt}%
\pgfsys@defobject{currentmarker}{\pgfqpoint{0.000000in}{-0.048611in}}{\pgfqpoint{0.000000in}{0.000000in}}{%
\pgfpathmoveto{\pgfqpoint{0.000000in}{0.000000in}}%
\pgfpathlineto{\pgfqpoint{0.000000in}{-0.048611in}}%
\pgfusepath{stroke,fill}%
}%
\begin{pgfscope}%
\pgfsys@transformshift{6.918182in}{0.330000in}%
\pgfsys@useobject{currentmarker}{}%
\end{pgfscope}%
\end{pgfscope}%
\begin{pgfscope}%
\pgfsetbuttcap%
\pgfsetroundjoin%
\definecolor{currentfill}{rgb}{0.000000,0.000000,0.000000}%
\pgfsetfillcolor{currentfill}%
\pgfsetlinewidth{0.803000pt}%
\definecolor{currentstroke}{rgb}{0.000000,0.000000,0.000000}%
\pgfsetstrokecolor{currentstroke}%
\pgfsetdash{}{0pt}%
\pgfsys@defobject{currentmarker}{\pgfqpoint{0.000000in}{0.000000in}}{\pgfqpoint{0.000000in}{0.048611in}}{%
\pgfpathmoveto{\pgfqpoint{0.000000in}{0.000000in}}%
\pgfpathlineto{\pgfqpoint{0.000000in}{0.048611in}}%
\pgfusepath{stroke,fill}%
}%
\begin{pgfscope}%
\pgfsys@transformshift{6.918182in}{2.640000in}%
\pgfsys@useobject{currentmarker}{}%
\end{pgfscope}%
\end{pgfscope}%
\begin{pgfscope}%
\definecolor{textcolor}{rgb}{0.000000,0.000000,0.000000}%
\pgfsetstrokecolor{textcolor}%
\pgfsetfillcolor{textcolor}%
\pgftext[x=6.918182in,y=0.232778in,,top]{\color{textcolor}\rmfamily\fontsize{10.000000}{12.000000}\selectfont \(\displaystyle {25}\)}%
\end{pgfscope}%
\begin{pgfscope}%
\pgfpathrectangle{\pgfqpoint{1.000000in}{0.330000in}}{\pgfqpoint{6.200000in}{2.310000in}}%
\pgfusepath{clip}%
\pgfsetbuttcap%
\pgfsetroundjoin%
\pgfsetlinewidth{0.803000pt}%
\definecolor{currentstroke}{rgb}{0.000000,0.000000,0.000000}%
\pgfsetstrokecolor{currentstroke}%
\pgfsetdash{{0.800000pt}{1.320000pt}}{0.000000pt}%
\pgfpathmoveto{\pgfqpoint{1.000000in}{0.534060in}}%
\pgfpathlineto{\pgfqpoint{7.200000in}{0.534060in}}%
\pgfusepath{stroke}%
\end{pgfscope}%
\begin{pgfscope}%
\pgfsetbuttcap%
\pgfsetroundjoin%
\definecolor{currentfill}{rgb}{0.000000,0.000000,0.000000}%
\pgfsetfillcolor{currentfill}%
\pgfsetlinewidth{0.803000pt}%
\definecolor{currentstroke}{rgb}{0.000000,0.000000,0.000000}%
\pgfsetstrokecolor{currentstroke}%
\pgfsetdash{}{0pt}%
\pgfsys@defobject{currentmarker}{\pgfqpoint{-0.048611in}{0.000000in}}{\pgfqpoint{-0.000000in}{0.000000in}}{%
\pgfpathmoveto{\pgfqpoint{-0.000000in}{0.000000in}}%
\pgfpathlineto{\pgfqpoint{-0.048611in}{0.000000in}}%
\pgfusepath{stroke,fill}%
}%
\begin{pgfscope}%
\pgfsys@transformshift{1.000000in}{0.534060in}%
\pgfsys@useobject{currentmarker}{}%
\end{pgfscope}%
\end{pgfscope}%
\begin{pgfscope}%
\pgfsetbuttcap%
\pgfsetroundjoin%
\definecolor{currentfill}{rgb}{0.000000,0.000000,0.000000}%
\pgfsetfillcolor{currentfill}%
\pgfsetlinewidth{0.803000pt}%
\definecolor{currentstroke}{rgb}{0.000000,0.000000,0.000000}%
\pgfsetstrokecolor{currentstroke}%
\pgfsetdash{}{0pt}%
\pgfsys@defobject{currentmarker}{\pgfqpoint{0.000000in}{0.000000in}}{\pgfqpoint{0.048611in}{0.000000in}}{%
\pgfpathmoveto{\pgfqpoint{0.000000in}{0.000000in}}%
\pgfpathlineto{\pgfqpoint{0.048611in}{0.000000in}}%
\pgfusepath{stroke,fill}%
}%
\begin{pgfscope}%
\pgfsys@transformshift{7.200000in}{0.534060in}%
\pgfsys@useobject{currentmarker}{}%
\end{pgfscope}%
\end{pgfscope}%
\begin{pgfscope}%
\definecolor{textcolor}{rgb}{0.000000,0.000000,0.000000}%
\pgfsetstrokecolor{textcolor}%
\pgfsetfillcolor{textcolor}%
\pgftext[x=0.547838in, y=0.485834in, left, base]{\color{textcolor}\rmfamily\fontsize{10.000000}{12.000000}\selectfont \(\displaystyle {\ensuremath{-}0.25}\)}%
\end{pgfscope}%
\begin{pgfscope}%
\pgfpathrectangle{\pgfqpoint{1.000000in}{0.330000in}}{\pgfqpoint{6.200000in}{2.310000in}}%
\pgfusepath{clip}%
\pgfsetbuttcap%
\pgfsetroundjoin%
\pgfsetlinewidth{0.803000pt}%
\definecolor{currentstroke}{rgb}{0.000000,0.000000,0.000000}%
\pgfsetstrokecolor{currentstroke}%
\pgfsetdash{{0.800000pt}{1.320000pt}}{0.000000pt}%
\pgfpathmoveto{\pgfqpoint{1.000000in}{0.883780in}}%
\pgfpathlineto{\pgfqpoint{7.200000in}{0.883780in}}%
\pgfusepath{stroke}%
\end{pgfscope}%
\begin{pgfscope}%
\pgfsetbuttcap%
\pgfsetroundjoin%
\definecolor{currentfill}{rgb}{0.000000,0.000000,0.000000}%
\pgfsetfillcolor{currentfill}%
\pgfsetlinewidth{0.803000pt}%
\definecolor{currentstroke}{rgb}{0.000000,0.000000,0.000000}%
\pgfsetstrokecolor{currentstroke}%
\pgfsetdash{}{0pt}%
\pgfsys@defobject{currentmarker}{\pgfqpoint{-0.048611in}{0.000000in}}{\pgfqpoint{-0.000000in}{0.000000in}}{%
\pgfpathmoveto{\pgfqpoint{-0.000000in}{0.000000in}}%
\pgfpathlineto{\pgfqpoint{-0.048611in}{0.000000in}}%
\pgfusepath{stroke,fill}%
}%
\begin{pgfscope}%
\pgfsys@transformshift{1.000000in}{0.883780in}%
\pgfsys@useobject{currentmarker}{}%
\end{pgfscope}%
\end{pgfscope}%
\begin{pgfscope}%
\pgfsetbuttcap%
\pgfsetroundjoin%
\definecolor{currentfill}{rgb}{0.000000,0.000000,0.000000}%
\pgfsetfillcolor{currentfill}%
\pgfsetlinewidth{0.803000pt}%
\definecolor{currentstroke}{rgb}{0.000000,0.000000,0.000000}%
\pgfsetstrokecolor{currentstroke}%
\pgfsetdash{}{0pt}%
\pgfsys@defobject{currentmarker}{\pgfqpoint{0.000000in}{0.000000in}}{\pgfqpoint{0.048611in}{0.000000in}}{%
\pgfpathmoveto{\pgfqpoint{0.000000in}{0.000000in}}%
\pgfpathlineto{\pgfqpoint{0.048611in}{0.000000in}}%
\pgfusepath{stroke,fill}%
}%
\begin{pgfscope}%
\pgfsys@transformshift{7.200000in}{0.883780in}%
\pgfsys@useobject{currentmarker}{}%
\end{pgfscope}%
\end{pgfscope}%
\begin{pgfscope}%
\definecolor{textcolor}{rgb}{0.000000,0.000000,0.000000}%
\pgfsetstrokecolor{textcolor}%
\pgfsetfillcolor{textcolor}%
\pgftext[x=0.547838in, y=0.835555in, left, base]{\color{textcolor}\rmfamily\fontsize{10.000000}{12.000000}\selectfont \(\displaystyle {\ensuremath{-}0.20}\)}%
\end{pgfscope}%
\begin{pgfscope}%
\pgfpathrectangle{\pgfqpoint{1.000000in}{0.330000in}}{\pgfqpoint{6.200000in}{2.310000in}}%
\pgfusepath{clip}%
\pgfsetbuttcap%
\pgfsetroundjoin%
\pgfsetlinewidth{0.803000pt}%
\definecolor{currentstroke}{rgb}{0.000000,0.000000,0.000000}%
\pgfsetstrokecolor{currentstroke}%
\pgfsetdash{{0.800000pt}{1.320000pt}}{0.000000pt}%
\pgfpathmoveto{\pgfqpoint{1.000000in}{1.233500in}}%
\pgfpathlineto{\pgfqpoint{7.200000in}{1.233500in}}%
\pgfusepath{stroke}%
\end{pgfscope}%
\begin{pgfscope}%
\pgfsetbuttcap%
\pgfsetroundjoin%
\definecolor{currentfill}{rgb}{0.000000,0.000000,0.000000}%
\pgfsetfillcolor{currentfill}%
\pgfsetlinewidth{0.803000pt}%
\definecolor{currentstroke}{rgb}{0.000000,0.000000,0.000000}%
\pgfsetstrokecolor{currentstroke}%
\pgfsetdash{}{0pt}%
\pgfsys@defobject{currentmarker}{\pgfqpoint{-0.048611in}{0.000000in}}{\pgfqpoint{-0.000000in}{0.000000in}}{%
\pgfpathmoveto{\pgfqpoint{-0.000000in}{0.000000in}}%
\pgfpathlineto{\pgfqpoint{-0.048611in}{0.000000in}}%
\pgfusepath{stroke,fill}%
}%
\begin{pgfscope}%
\pgfsys@transformshift{1.000000in}{1.233500in}%
\pgfsys@useobject{currentmarker}{}%
\end{pgfscope}%
\end{pgfscope}%
\begin{pgfscope}%
\pgfsetbuttcap%
\pgfsetroundjoin%
\definecolor{currentfill}{rgb}{0.000000,0.000000,0.000000}%
\pgfsetfillcolor{currentfill}%
\pgfsetlinewidth{0.803000pt}%
\definecolor{currentstroke}{rgb}{0.000000,0.000000,0.000000}%
\pgfsetstrokecolor{currentstroke}%
\pgfsetdash{}{0pt}%
\pgfsys@defobject{currentmarker}{\pgfqpoint{0.000000in}{0.000000in}}{\pgfqpoint{0.048611in}{0.000000in}}{%
\pgfpathmoveto{\pgfqpoint{0.000000in}{0.000000in}}%
\pgfpathlineto{\pgfqpoint{0.048611in}{0.000000in}}%
\pgfusepath{stroke,fill}%
}%
\begin{pgfscope}%
\pgfsys@transformshift{7.200000in}{1.233500in}%
\pgfsys@useobject{currentmarker}{}%
\end{pgfscope}%
\end{pgfscope}%
\begin{pgfscope}%
\definecolor{textcolor}{rgb}{0.000000,0.000000,0.000000}%
\pgfsetstrokecolor{textcolor}%
\pgfsetfillcolor{textcolor}%
\pgftext[x=0.547838in, y=1.185275in, left, base]{\color{textcolor}\rmfamily\fontsize{10.000000}{12.000000}\selectfont \(\displaystyle {\ensuremath{-}0.15}\)}%
\end{pgfscope}%
\begin{pgfscope}%
\pgfpathrectangle{\pgfqpoint{1.000000in}{0.330000in}}{\pgfqpoint{6.200000in}{2.310000in}}%
\pgfusepath{clip}%
\pgfsetbuttcap%
\pgfsetroundjoin%
\pgfsetlinewidth{0.803000pt}%
\definecolor{currentstroke}{rgb}{0.000000,0.000000,0.000000}%
\pgfsetstrokecolor{currentstroke}%
\pgfsetdash{{0.800000pt}{1.320000pt}}{0.000000pt}%
\pgfpathmoveto{\pgfqpoint{1.000000in}{1.583220in}}%
\pgfpathlineto{\pgfqpoint{7.200000in}{1.583220in}}%
\pgfusepath{stroke}%
\end{pgfscope}%
\begin{pgfscope}%
\pgfsetbuttcap%
\pgfsetroundjoin%
\definecolor{currentfill}{rgb}{0.000000,0.000000,0.000000}%
\pgfsetfillcolor{currentfill}%
\pgfsetlinewidth{0.803000pt}%
\definecolor{currentstroke}{rgb}{0.000000,0.000000,0.000000}%
\pgfsetstrokecolor{currentstroke}%
\pgfsetdash{}{0pt}%
\pgfsys@defobject{currentmarker}{\pgfqpoint{-0.048611in}{0.000000in}}{\pgfqpoint{-0.000000in}{0.000000in}}{%
\pgfpathmoveto{\pgfqpoint{-0.000000in}{0.000000in}}%
\pgfpathlineto{\pgfqpoint{-0.048611in}{0.000000in}}%
\pgfusepath{stroke,fill}%
}%
\begin{pgfscope}%
\pgfsys@transformshift{1.000000in}{1.583220in}%
\pgfsys@useobject{currentmarker}{}%
\end{pgfscope}%
\end{pgfscope}%
\begin{pgfscope}%
\pgfsetbuttcap%
\pgfsetroundjoin%
\definecolor{currentfill}{rgb}{0.000000,0.000000,0.000000}%
\pgfsetfillcolor{currentfill}%
\pgfsetlinewidth{0.803000pt}%
\definecolor{currentstroke}{rgb}{0.000000,0.000000,0.000000}%
\pgfsetstrokecolor{currentstroke}%
\pgfsetdash{}{0pt}%
\pgfsys@defobject{currentmarker}{\pgfqpoint{0.000000in}{0.000000in}}{\pgfqpoint{0.048611in}{0.000000in}}{%
\pgfpathmoveto{\pgfqpoint{0.000000in}{0.000000in}}%
\pgfpathlineto{\pgfqpoint{0.048611in}{0.000000in}}%
\pgfusepath{stroke,fill}%
}%
\begin{pgfscope}%
\pgfsys@transformshift{7.200000in}{1.583220in}%
\pgfsys@useobject{currentmarker}{}%
\end{pgfscope}%
\end{pgfscope}%
\begin{pgfscope}%
\definecolor{textcolor}{rgb}{0.000000,0.000000,0.000000}%
\pgfsetstrokecolor{textcolor}%
\pgfsetfillcolor{textcolor}%
\pgftext[x=0.547838in, y=1.534995in, left, base]{\color{textcolor}\rmfamily\fontsize{10.000000}{12.000000}\selectfont \(\displaystyle {\ensuremath{-}0.10}\)}%
\end{pgfscope}%
\begin{pgfscope}%
\pgfpathrectangle{\pgfqpoint{1.000000in}{0.330000in}}{\pgfqpoint{6.200000in}{2.310000in}}%
\pgfusepath{clip}%
\pgfsetbuttcap%
\pgfsetroundjoin%
\pgfsetlinewidth{0.803000pt}%
\definecolor{currentstroke}{rgb}{0.000000,0.000000,0.000000}%
\pgfsetstrokecolor{currentstroke}%
\pgfsetdash{{0.800000pt}{1.320000pt}}{0.000000pt}%
\pgfpathmoveto{\pgfqpoint{1.000000in}{1.932940in}}%
\pgfpathlineto{\pgfqpoint{7.200000in}{1.932940in}}%
\pgfusepath{stroke}%
\end{pgfscope}%
\begin{pgfscope}%
\pgfsetbuttcap%
\pgfsetroundjoin%
\definecolor{currentfill}{rgb}{0.000000,0.000000,0.000000}%
\pgfsetfillcolor{currentfill}%
\pgfsetlinewidth{0.803000pt}%
\definecolor{currentstroke}{rgb}{0.000000,0.000000,0.000000}%
\pgfsetstrokecolor{currentstroke}%
\pgfsetdash{}{0pt}%
\pgfsys@defobject{currentmarker}{\pgfqpoint{-0.048611in}{0.000000in}}{\pgfqpoint{-0.000000in}{0.000000in}}{%
\pgfpathmoveto{\pgfqpoint{-0.000000in}{0.000000in}}%
\pgfpathlineto{\pgfqpoint{-0.048611in}{0.000000in}}%
\pgfusepath{stroke,fill}%
}%
\begin{pgfscope}%
\pgfsys@transformshift{1.000000in}{1.932940in}%
\pgfsys@useobject{currentmarker}{}%
\end{pgfscope}%
\end{pgfscope}%
\begin{pgfscope}%
\pgfsetbuttcap%
\pgfsetroundjoin%
\definecolor{currentfill}{rgb}{0.000000,0.000000,0.000000}%
\pgfsetfillcolor{currentfill}%
\pgfsetlinewidth{0.803000pt}%
\definecolor{currentstroke}{rgb}{0.000000,0.000000,0.000000}%
\pgfsetstrokecolor{currentstroke}%
\pgfsetdash{}{0pt}%
\pgfsys@defobject{currentmarker}{\pgfqpoint{0.000000in}{0.000000in}}{\pgfqpoint{0.048611in}{0.000000in}}{%
\pgfpathmoveto{\pgfqpoint{0.000000in}{0.000000in}}%
\pgfpathlineto{\pgfqpoint{0.048611in}{0.000000in}}%
\pgfusepath{stroke,fill}%
}%
\begin{pgfscope}%
\pgfsys@transformshift{7.200000in}{1.932940in}%
\pgfsys@useobject{currentmarker}{}%
\end{pgfscope}%
\end{pgfscope}%
\begin{pgfscope}%
\definecolor{textcolor}{rgb}{0.000000,0.000000,0.000000}%
\pgfsetstrokecolor{textcolor}%
\pgfsetfillcolor{textcolor}%
\pgftext[x=0.547838in, y=1.884715in, left, base]{\color{textcolor}\rmfamily\fontsize{10.000000}{12.000000}\selectfont \(\displaystyle {\ensuremath{-}0.05}\)}%
\end{pgfscope}%
\begin{pgfscope}%
\pgfpathrectangle{\pgfqpoint{1.000000in}{0.330000in}}{\pgfqpoint{6.200000in}{2.310000in}}%
\pgfusepath{clip}%
\pgfsetbuttcap%
\pgfsetroundjoin%
\pgfsetlinewidth{0.803000pt}%
\definecolor{currentstroke}{rgb}{0.000000,0.000000,0.000000}%
\pgfsetstrokecolor{currentstroke}%
\pgfsetdash{{0.800000pt}{1.320000pt}}{0.000000pt}%
\pgfpathmoveto{\pgfqpoint{1.000000in}{2.282660in}}%
\pgfpathlineto{\pgfqpoint{7.200000in}{2.282660in}}%
\pgfusepath{stroke}%
\end{pgfscope}%
\begin{pgfscope}%
\pgfsetbuttcap%
\pgfsetroundjoin%
\definecolor{currentfill}{rgb}{0.000000,0.000000,0.000000}%
\pgfsetfillcolor{currentfill}%
\pgfsetlinewidth{0.803000pt}%
\definecolor{currentstroke}{rgb}{0.000000,0.000000,0.000000}%
\pgfsetstrokecolor{currentstroke}%
\pgfsetdash{}{0pt}%
\pgfsys@defobject{currentmarker}{\pgfqpoint{-0.048611in}{0.000000in}}{\pgfqpoint{-0.000000in}{0.000000in}}{%
\pgfpathmoveto{\pgfqpoint{-0.000000in}{0.000000in}}%
\pgfpathlineto{\pgfqpoint{-0.048611in}{0.000000in}}%
\pgfusepath{stroke,fill}%
}%
\begin{pgfscope}%
\pgfsys@transformshift{1.000000in}{2.282660in}%
\pgfsys@useobject{currentmarker}{}%
\end{pgfscope}%
\end{pgfscope}%
\begin{pgfscope}%
\pgfsetbuttcap%
\pgfsetroundjoin%
\definecolor{currentfill}{rgb}{0.000000,0.000000,0.000000}%
\pgfsetfillcolor{currentfill}%
\pgfsetlinewidth{0.803000pt}%
\definecolor{currentstroke}{rgb}{0.000000,0.000000,0.000000}%
\pgfsetstrokecolor{currentstroke}%
\pgfsetdash{}{0pt}%
\pgfsys@defobject{currentmarker}{\pgfqpoint{0.000000in}{0.000000in}}{\pgfqpoint{0.048611in}{0.000000in}}{%
\pgfpathmoveto{\pgfqpoint{0.000000in}{0.000000in}}%
\pgfpathlineto{\pgfqpoint{0.048611in}{0.000000in}}%
\pgfusepath{stroke,fill}%
}%
\begin{pgfscope}%
\pgfsys@transformshift{7.200000in}{2.282660in}%
\pgfsys@useobject{currentmarker}{}%
\end{pgfscope}%
\end{pgfscope}%
\begin{pgfscope}%
\definecolor{textcolor}{rgb}{0.000000,0.000000,0.000000}%
\pgfsetstrokecolor{textcolor}%
\pgfsetfillcolor{textcolor}%
\pgftext[x=0.655863in, y=2.234435in, left, base]{\color{textcolor}\rmfamily\fontsize{10.000000}{12.000000}\selectfont \(\displaystyle {0.00}\)}%
\end{pgfscope}%
\begin{pgfscope}%
\pgfpathrectangle{\pgfqpoint{1.000000in}{0.330000in}}{\pgfqpoint{6.200000in}{2.310000in}}%
\pgfusepath{clip}%
\pgfsetbuttcap%
\pgfsetroundjoin%
\pgfsetlinewidth{0.803000pt}%
\definecolor{currentstroke}{rgb}{0.000000,0.000000,0.000000}%
\pgfsetstrokecolor{currentstroke}%
\pgfsetdash{{0.800000pt}{1.320000pt}}{0.000000pt}%
\pgfpathmoveto{\pgfqpoint{1.000000in}{2.632380in}}%
\pgfpathlineto{\pgfqpoint{7.200000in}{2.632380in}}%
\pgfusepath{stroke}%
\end{pgfscope}%
\begin{pgfscope}%
\pgfsetbuttcap%
\pgfsetroundjoin%
\definecolor{currentfill}{rgb}{0.000000,0.000000,0.000000}%
\pgfsetfillcolor{currentfill}%
\pgfsetlinewidth{0.803000pt}%
\definecolor{currentstroke}{rgb}{0.000000,0.000000,0.000000}%
\pgfsetstrokecolor{currentstroke}%
\pgfsetdash{}{0pt}%
\pgfsys@defobject{currentmarker}{\pgfqpoint{-0.048611in}{0.000000in}}{\pgfqpoint{-0.000000in}{0.000000in}}{%
\pgfpathmoveto{\pgfqpoint{-0.000000in}{0.000000in}}%
\pgfpathlineto{\pgfqpoint{-0.048611in}{0.000000in}}%
\pgfusepath{stroke,fill}%
}%
\begin{pgfscope}%
\pgfsys@transformshift{1.000000in}{2.632380in}%
\pgfsys@useobject{currentmarker}{}%
\end{pgfscope}%
\end{pgfscope}%
\begin{pgfscope}%
\pgfsetbuttcap%
\pgfsetroundjoin%
\definecolor{currentfill}{rgb}{0.000000,0.000000,0.000000}%
\pgfsetfillcolor{currentfill}%
\pgfsetlinewidth{0.803000pt}%
\definecolor{currentstroke}{rgb}{0.000000,0.000000,0.000000}%
\pgfsetstrokecolor{currentstroke}%
\pgfsetdash{}{0pt}%
\pgfsys@defobject{currentmarker}{\pgfqpoint{0.000000in}{0.000000in}}{\pgfqpoint{0.048611in}{0.000000in}}{%
\pgfpathmoveto{\pgfqpoint{0.000000in}{0.000000in}}%
\pgfpathlineto{\pgfqpoint{0.048611in}{0.000000in}}%
\pgfusepath{stroke,fill}%
}%
\begin{pgfscope}%
\pgfsys@transformshift{7.200000in}{2.632380in}%
\pgfsys@useobject{currentmarker}{}%
\end{pgfscope}%
\end{pgfscope}%
\begin{pgfscope}%
\definecolor{textcolor}{rgb}{0.000000,0.000000,0.000000}%
\pgfsetstrokecolor{textcolor}%
\pgfsetfillcolor{textcolor}%
\pgftext[x=0.655863in, y=2.584155in, left, base]{\color{textcolor}\rmfamily\fontsize{10.000000}{12.000000}\selectfont \(\displaystyle {0.05}\)}%
\end{pgfscope}%
\begin{pgfscope}%
\pgfpathrectangle{\pgfqpoint{1.000000in}{0.330000in}}{\pgfqpoint{6.200000in}{2.310000in}}%
\pgfusepath{clip}%
\pgfsetrectcap%
\pgfsetroundjoin%
\pgfsetlinewidth{1.505625pt}%
\definecolor{currentstroke}{rgb}{0.121569,0.466667,0.705882}%
\pgfsetstrokecolor{currentstroke}%
\pgfsetdash{}{0pt}%
\pgfpathmoveto{\pgfqpoint{1.281818in}{2.282660in}}%
\pgfpathlineto{\pgfqpoint{1.450909in}{1.954312in}}%
\pgfpathlineto{\pgfqpoint{1.544848in}{1.775955in}}%
\pgfpathlineto{\pgfqpoint{1.638788in}{1.602604in}}%
\pgfpathlineto{\pgfqpoint{1.713939in}{1.468604in}}%
\pgfpathlineto{\pgfqpoint{1.789091in}{1.339629in}}%
\pgfpathlineto{\pgfqpoint{1.845455in}{1.246660in}}%
\pgfpathlineto{\pgfqpoint{1.901818in}{1.157268in}}%
\pgfpathlineto{\pgfqpoint{1.958182in}{1.071766in}}%
\pgfpathlineto{\pgfqpoint{2.014545in}{0.990454in}}%
\pgfpathlineto{\pgfqpoint{2.070909in}{0.913629in}}%
\pgfpathlineto{\pgfqpoint{2.127273in}{0.841580in}}%
\pgfpathlineto{\pgfqpoint{2.183636in}{0.774586in}}%
\pgfpathlineto{\pgfqpoint{2.221212in}{0.732867in}}%
\pgfpathlineto{\pgfqpoint{2.258788in}{0.693596in}}%
\pgfpathlineto{\pgfqpoint{2.296364in}{0.656850in}}%
\pgfpathlineto{\pgfqpoint{2.333939in}{0.622704in}}%
\pgfpathlineto{\pgfqpoint{2.371515in}{0.591234in}}%
\pgfpathlineto{\pgfqpoint{2.409091in}{0.562512in}}%
\pgfpathlineto{\pgfqpoint{2.446667in}{0.539655in}}%
\pgfpathlineto{\pgfqpoint{2.484242in}{0.519589in}}%
\pgfpathlineto{\pgfqpoint{2.521818in}{0.502284in}}%
\pgfpathlineto{\pgfqpoint{2.559394in}{0.487708in}}%
\pgfpathlineto{\pgfqpoint{2.596970in}{0.475830in}}%
\pgfpathlineto{\pgfqpoint{2.634545in}{0.466613in}}%
\pgfpathlineto{\pgfqpoint{2.672121in}{0.460021in}}%
\pgfpathlineto{\pgfqpoint{2.709697in}{0.456018in}}%
\pgfpathlineto{\pgfqpoint{2.747273in}{0.454562in}}%
\pgfpathlineto{\pgfqpoint{2.784848in}{0.455612in}}%
\pgfpathlineto{\pgfqpoint{2.822424in}{0.459127in}}%
\pgfpathlineto{\pgfqpoint{2.860000in}{0.465061in}}%
\pgfpathlineto{\pgfqpoint{2.897576in}{0.473368in}}%
\pgfpathlineto{\pgfqpoint{2.935152in}{0.484000in}}%
\pgfpathlineto{\pgfqpoint{2.972727in}{0.496908in}}%
\pgfpathlineto{\pgfqpoint{3.010303in}{0.512040in}}%
\pgfpathlineto{\pgfqpoint{3.047879in}{0.529343in}}%
\pgfpathlineto{\pgfqpoint{3.085455in}{0.548763in}}%
\pgfpathlineto{\pgfqpoint{3.123030in}{0.570243in}}%
\pgfpathlineto{\pgfqpoint{3.160606in}{0.593725in}}%
\pgfpathlineto{\pgfqpoint{3.198182in}{0.619149in}}%
\pgfpathlineto{\pgfqpoint{3.235758in}{0.646454in}}%
\pgfpathlineto{\pgfqpoint{3.292121in}{0.690797in}}%
\pgfpathlineto{\pgfqpoint{3.348485in}{0.739007in}}%
\pgfpathlineto{\pgfqpoint{3.404848in}{0.790852in}}%
\pgfpathlineto{\pgfqpoint{3.461212in}{0.846093in}}%
\pgfpathlineto{\pgfqpoint{3.517576in}{0.904481in}}%
\pgfpathlineto{\pgfqpoint{3.573939in}{0.965758in}}%
\pgfpathlineto{\pgfqpoint{3.649091in}{1.051491in}}%
\pgfpathlineto{\pgfqpoint{3.724242in}{1.141223in}}%
\pgfpathlineto{\pgfqpoint{3.799394in}{1.234264in}}%
\pgfpathlineto{\pgfqpoint{3.893333in}{1.354119in}}%
\pgfpathlineto{\pgfqpoint{4.024848in}{1.525886in}}%
\pgfpathlineto{\pgfqpoint{4.212727in}{1.771405in}}%
\pgfpathlineto{\pgfqpoint{4.306667in}{1.890510in}}%
\pgfpathlineto{\pgfqpoint{4.381818in}{1.982408in}}%
\pgfpathlineto{\pgfqpoint{4.456970in}{2.070253in}}%
\pgfpathlineto{\pgfqpoint{4.513333in}{2.132878in}}%
\pgfpathlineto{\pgfqpoint{4.569697in}{2.192228in}}%
\pgfpathlineto{\pgfqpoint{4.626061in}{2.247861in}}%
\pgfpathlineto{\pgfqpoint{4.663636in}{2.282660in}}%
\pgfpathlineto{\pgfqpoint{4.701212in}{2.308852in}}%
\pgfpathlineto{\pgfqpoint{4.738788in}{2.333126in}}%
\pgfpathlineto{\pgfqpoint{4.776364in}{2.355554in}}%
\pgfpathlineto{\pgfqpoint{4.813939in}{2.376206in}}%
\pgfpathlineto{\pgfqpoint{4.870303in}{2.403999in}}%
\pgfpathlineto{\pgfqpoint{4.926667in}{2.428164in}}%
\pgfpathlineto{\pgfqpoint{4.980220in}{2.447952in}}%
\pgfpathlineto{\pgfqpoint{5.036583in}{2.465563in}}%
\pgfpathlineto{\pgfqpoint{5.092947in}{2.480000in}}%
\pgfpathlineto{\pgfqpoint{5.149310in}{2.491451in}}%
\pgfpathlineto{\pgfqpoint{5.205674in}{2.500099in}}%
\pgfpathlineto{\pgfqpoint{5.262038in}{2.506127in}}%
\pgfpathlineto{\pgfqpoint{5.318401in}{2.509717in}}%
\pgfpathlineto{\pgfqpoint{5.374765in}{2.511047in}}%
\pgfpathlineto{\pgfqpoint{5.431129in}{2.510295in}}%
\pgfpathlineto{\pgfqpoint{5.487492in}{2.507635in}}%
\pgfpathlineto{\pgfqpoint{5.562644in}{2.501421in}}%
\pgfpathlineto{\pgfqpoint{5.637795in}{2.492530in}}%
\pgfpathlineto{\pgfqpoint{5.712947in}{2.481362in}}%
\pgfpathlineto{\pgfqpoint{5.809697in}{2.464267in}}%
\pgfpathlineto{\pgfqpoint{5.922424in}{2.441628in}}%
\pgfpathlineto{\pgfqpoint{6.091515in}{2.404865in}}%
\pgfpathlineto{\pgfqpoint{6.298182in}{2.360292in}}%
\pgfpathlineto{\pgfqpoint{6.410909in}{2.338201in}}%
\pgfpathlineto{\pgfqpoint{6.504848in}{2.321823in}}%
\pgfpathlineto{\pgfqpoint{6.598788in}{2.307785in}}%
\pgfpathlineto{\pgfqpoint{6.673939in}{2.298521in}}%
\pgfpathlineto{\pgfqpoint{6.749091in}{2.291205in}}%
\pgfpathlineto{\pgfqpoint{6.824242in}{2.285985in}}%
\pgfpathlineto{\pgfqpoint{6.899394in}{2.283027in}}%
\pgfpathlineto{\pgfqpoint{6.918182in}{2.282660in}}%
\pgfpathlineto{\pgfqpoint{6.918182in}{2.282660in}}%
\pgfusepath{stroke}%
\end{pgfscope}%
\begin{pgfscope}%
\pgfpathrectangle{\pgfqpoint{1.000000in}{0.330000in}}{\pgfqpoint{6.200000in}{2.310000in}}%
\pgfusepath{clip}%
\pgfsetrectcap%
\pgfsetroundjoin%
\pgfsetlinewidth{1.505625pt}%
\definecolor{currentstroke}{rgb}{1.000000,0.498039,0.054902}%
\pgfsetstrokecolor{currentstroke}%
\pgfsetdash{}{0pt}%
\pgfpathmoveto{\pgfqpoint{1.281818in}{2.282660in}}%
\pgfpathlineto{\pgfqpoint{1.450909in}{1.951540in}}%
\pgfpathlineto{\pgfqpoint{1.544848in}{1.771658in}}%
\pgfpathlineto{\pgfqpoint{1.638788in}{1.596802in}}%
\pgfpathlineto{\pgfqpoint{1.713939in}{1.461619in}}%
\pgfpathlineto{\pgfqpoint{1.789091in}{1.331480in}}%
\pgfpathlineto{\pgfqpoint{1.845455in}{1.237654in}}%
\pgfpathlineto{\pgfqpoint{1.901818in}{1.147421in}}%
\pgfpathlineto{\pgfqpoint{1.958182in}{1.061094in}}%
\pgfpathlineto{\pgfqpoint{2.014545in}{0.978977in}}%
\pgfpathlineto{\pgfqpoint{2.070909in}{0.901367in}}%
\pgfpathlineto{\pgfqpoint{2.127273in}{0.828554in}}%
\pgfpathlineto{\pgfqpoint{2.183636in}{0.760819in}}%
\pgfpathlineto{\pgfqpoint{2.221212in}{0.718620in}}%
\pgfpathlineto{\pgfqpoint{2.258788in}{0.678880in}}%
\pgfpathlineto{\pgfqpoint{2.296364in}{0.641677in}}%
\pgfpathlineto{\pgfqpoint{2.333939in}{0.607087in}}%
\pgfpathlineto{\pgfqpoint{2.371515in}{0.575185in}}%
\pgfpathlineto{\pgfqpoint{2.409091in}{0.546045in}}%
\pgfpathlineto{\pgfqpoint{2.446667in}{0.522782in}}%
\pgfpathlineto{\pgfqpoint{2.484242in}{0.502326in}}%
\pgfpathlineto{\pgfqpoint{2.521818in}{0.484645in}}%
\pgfpathlineto{\pgfqpoint{2.559394in}{0.469709in}}%
\pgfpathlineto{\pgfqpoint{2.596970in}{0.457484in}}%
\pgfpathlineto{\pgfqpoint{2.634545in}{0.447938in}}%
\pgfpathlineto{\pgfqpoint{2.672121in}{0.441034in}}%
\pgfpathlineto{\pgfqpoint{2.709697in}{0.436734in}}%
\pgfpathlineto{\pgfqpoint{2.747273in}{0.435000in}}%
\pgfpathlineto{\pgfqpoint{2.784848in}{0.435790in}}%
\pgfpathlineto{\pgfqpoint{2.822424in}{0.439062in}}%
\pgfpathlineto{\pgfqpoint{2.860000in}{0.444773in}}%
\pgfpathlineto{\pgfqpoint{2.897576in}{0.452875in}}%
\pgfpathlineto{\pgfqpoint{2.935152in}{0.463323in}}%
\pgfpathlineto{\pgfqpoint{2.972727in}{0.476066in}}%
\pgfpathlineto{\pgfqpoint{3.010303in}{0.491054in}}%
\pgfpathlineto{\pgfqpoint{3.047879in}{0.508234in}}%
\pgfpathlineto{\pgfqpoint{3.085455in}{0.527553in}}%
\pgfpathlineto{\pgfqpoint{3.123030in}{0.548953in}}%
\pgfpathlineto{\pgfqpoint{3.160606in}{0.572378in}}%
\pgfpathlineto{\pgfqpoint{3.198182in}{0.597768in}}%
\pgfpathlineto{\pgfqpoint{3.235758in}{0.625061in}}%
\pgfpathlineto{\pgfqpoint{3.292121in}{0.669433in}}%
\pgfpathlineto{\pgfqpoint{3.348485in}{0.717726in}}%
\pgfpathlineto{\pgfqpoint{3.404848in}{0.769711in}}%
\pgfpathlineto{\pgfqpoint{3.461212in}{0.825151in}}%
\pgfpathlineto{\pgfqpoint{3.517576in}{0.883797in}}%
\pgfpathlineto{\pgfqpoint{3.573939in}{0.945393in}}%
\pgfpathlineto{\pgfqpoint{3.630303in}{1.009673in}}%
\pgfpathlineto{\pgfqpoint{3.705455in}{1.099078in}}%
\pgfpathlineto{\pgfqpoint{3.780606in}{1.192086in}}%
\pgfpathlineto{\pgfqpoint{3.874545in}{1.312329in}}%
\pgfpathlineto{\pgfqpoint{3.987273in}{1.460535in}}%
\pgfpathlineto{\pgfqpoint{4.250303in}{1.808312in}}%
\pgfpathlineto{\pgfqpoint{4.344242in}{1.927847in}}%
\pgfpathlineto{\pgfqpoint{4.419394in}{2.019720in}}%
\pgfpathlineto{\pgfqpoint{4.475758in}{2.085794in}}%
\pgfpathlineto{\pgfqpoint{4.532121in}{2.148973in}}%
\pgfpathlineto{\pgfqpoint{4.588485in}{2.208819in}}%
\pgfpathlineto{\pgfqpoint{4.626061in}{2.246646in}}%
\pgfpathlineto{\pgfqpoint{4.663636in}{2.282660in}}%
\pgfpathlineto{\pgfqpoint{4.701212in}{2.310615in}}%
\pgfpathlineto{\pgfqpoint{4.738788in}{2.336623in}}%
\pgfpathlineto{\pgfqpoint{4.776364in}{2.360740in}}%
\pgfpathlineto{\pgfqpoint{4.813939in}{2.383024in}}%
\pgfpathlineto{\pgfqpoint{4.851515in}{2.403531in}}%
\pgfpathlineto{\pgfqpoint{4.889091in}{2.422316in}}%
\pgfpathlineto{\pgfqpoint{4.945455in}{2.447388in}}%
\pgfpathlineto{\pgfqpoint{4.999007in}{2.467907in}}%
\pgfpathlineto{\pgfqpoint{5.055371in}{2.486203in}}%
\pgfpathlineto{\pgfqpoint{5.111735in}{2.501293in}}%
\pgfpathlineto{\pgfqpoint{5.168098in}{2.513358in}}%
\pgfpathlineto{\pgfqpoint{5.224462in}{2.522577in}}%
\pgfpathlineto{\pgfqpoint{5.280826in}{2.529126in}}%
\pgfpathlineto{\pgfqpoint{5.337189in}{2.533179in}}%
\pgfpathlineto{\pgfqpoint{5.393553in}{2.534910in}}%
\pgfpathlineto{\pgfqpoint{5.449916in}{2.534488in}}%
\pgfpathlineto{\pgfqpoint{5.506280in}{2.532084in}}%
\pgfpathlineto{\pgfqpoint{5.562644in}{2.527864in}}%
\pgfpathlineto{\pgfqpoint{5.637795in}{2.519697in}}%
\pgfpathlineto{\pgfqpoint{5.712947in}{2.508982in}}%
\pgfpathlineto{\pgfqpoint{5.790909in}{2.495579in}}%
\pgfpathlineto{\pgfqpoint{5.884848in}{2.476922in}}%
\pgfpathlineto{\pgfqpoint{5.997576in}{2.451871in}}%
\pgfpathlineto{\pgfqpoint{6.166667in}{2.411361in}}%
\pgfpathlineto{\pgfqpoint{6.354545in}{2.366509in}}%
\pgfpathlineto{\pgfqpoint{6.467273in}{2.341770in}}%
\pgfpathlineto{\pgfqpoint{6.561212in}{2.323372in}}%
\pgfpathlineto{\pgfqpoint{6.636364in}{2.310548in}}%
\pgfpathlineto{\pgfqpoint{6.711515in}{2.299739in}}%
\pgfpathlineto{\pgfqpoint{6.786667in}{2.291214in}}%
\pgfpathlineto{\pgfqpoint{6.861818in}{2.285259in}}%
\pgfpathlineto{\pgfqpoint{6.918182in}{2.282660in}}%
\pgfpathlineto{\pgfqpoint{6.918182in}{2.282660in}}%
\pgfusepath{stroke}%
\end{pgfscope}%
\begin{pgfscope}%
\pgfpathrectangle{\pgfqpoint{1.000000in}{0.330000in}}{\pgfqpoint{6.200000in}{2.310000in}}%
\pgfusepath{clip}%
\pgfsetrectcap%
\pgfsetroundjoin%
\pgfsetlinewidth{1.505625pt}%
\definecolor{currentstroke}{rgb}{0.172549,0.627451,0.172549}%
\pgfsetstrokecolor{currentstroke}%
\pgfsetdash{}{0pt}%
\pgfpathmoveto{\pgfqpoint{1.281818in}{2.282660in}}%
\pgfpathlineto{\pgfqpoint{2.183636in}{2.296427in}}%
\pgfpathlineto{\pgfqpoint{2.672121in}{2.301647in}}%
\pgfpathlineto{\pgfqpoint{3.085455in}{2.303871in}}%
\pgfpathlineto{\pgfqpoint{3.442424in}{2.303675in}}%
\pgfpathlineto{\pgfqpoint{3.780606in}{2.301306in}}%
\pgfpathlineto{\pgfqpoint{4.100000in}{2.296812in}}%
\pgfpathlineto{\pgfqpoint{4.400606in}{2.290323in}}%
\pgfpathlineto{\pgfqpoint{4.663636in}{2.282660in}}%
\pgfpathlineto{\pgfqpoint{4.926667in}{2.271388in}}%
\pgfpathlineto{\pgfqpoint{5.149310in}{2.264449in}}%
\pgfpathlineto{\pgfqpoint{5.393553in}{2.258770in}}%
\pgfpathlineto{\pgfqpoint{5.619007in}{2.255649in}}%
\pgfpathlineto{\pgfqpoint{5.828485in}{2.254913in}}%
\pgfpathlineto{\pgfqpoint{6.016364in}{2.256589in}}%
\pgfpathlineto{\pgfqpoint{6.204242in}{2.260579in}}%
\pgfpathlineto{\pgfqpoint{6.429697in}{2.267680in}}%
\pgfpathlineto{\pgfqpoint{6.843030in}{2.281196in}}%
\pgfpathlineto{\pgfqpoint{6.918182in}{2.282660in}}%
\pgfpathlineto{\pgfqpoint{6.918182in}{2.282660in}}%
\pgfusepath{stroke}%
\end{pgfscope}%
\begin{pgfscope}%
\pgfpathrectangle{\pgfqpoint{1.000000in}{0.330000in}}{\pgfqpoint{6.200000in}{2.310000in}}%
\pgfusepath{clip}%
\pgfsetrectcap%
\pgfsetroundjoin%
\pgfsetlinewidth{1.505625pt}%
\definecolor{currentstroke}{rgb}{0.839216,0.152941,0.156863}%
\pgfsetstrokecolor{currentstroke}%
\pgfsetdash{}{0pt}%
\pgfpathmoveto{\pgfqpoint{1.281818in}{2.282660in}}%
\pgfpathlineto{\pgfqpoint{1.488485in}{2.155717in}}%
\pgfpathlineto{\pgfqpoint{1.601212in}{2.088770in}}%
\pgfpathlineto{\pgfqpoint{1.695152in}{2.035082in}}%
\pgfpathlineto{\pgfqpoint{1.789091in}{1.983848in}}%
\pgfpathlineto{\pgfqpoint{1.864242in}{1.944962in}}%
\pgfpathlineto{\pgfqpoint{1.939394in}{1.908210in}}%
\pgfpathlineto{\pgfqpoint{2.014545in}{1.873831in}}%
\pgfpathlineto{\pgfqpoint{2.089697in}{1.842055in}}%
\pgfpathlineto{\pgfqpoint{2.164848in}{1.813110in}}%
\pgfpathlineto{\pgfqpoint{2.240000in}{1.787212in}}%
\pgfpathlineto{\pgfqpoint{2.296364in}{1.769917in}}%
\pgfpathlineto{\pgfqpoint{2.352727in}{1.754542in}}%
\pgfpathlineto{\pgfqpoint{2.409091in}{1.741171in}}%
\pgfpathlineto{\pgfqpoint{2.465455in}{1.731612in}}%
\pgfpathlineto{\pgfqpoint{2.521818in}{1.724094in}}%
\pgfpathlineto{\pgfqpoint{2.578182in}{1.718571in}}%
\pgfpathlineto{\pgfqpoint{2.634545in}{1.714991in}}%
\pgfpathlineto{\pgfqpoint{2.690909in}{1.713303in}}%
\pgfpathlineto{\pgfqpoint{2.747273in}{1.713453in}}%
\pgfpathlineto{\pgfqpoint{2.803636in}{1.715385in}}%
\pgfpathlineto{\pgfqpoint{2.860000in}{1.719039in}}%
\pgfpathlineto{\pgfqpoint{2.935152in}{1.726487in}}%
\pgfpathlineto{\pgfqpoint{3.010303in}{1.736739in}}%
\pgfpathlineto{\pgfqpoint{3.085455in}{1.749640in}}%
\pgfpathlineto{\pgfqpoint{3.160606in}{1.765026in}}%
\pgfpathlineto{\pgfqpoint{3.235758in}{1.782727in}}%
\pgfpathlineto{\pgfqpoint{3.310909in}{1.802565in}}%
\pgfpathlineto{\pgfqpoint{3.386061in}{1.824358in}}%
\pgfpathlineto{\pgfqpoint{3.480000in}{1.854056in}}%
\pgfpathlineto{\pgfqpoint{3.573939in}{1.886121in}}%
\pgfpathlineto{\pgfqpoint{3.686667in}{1.927154in}}%
\pgfpathlineto{\pgfqpoint{3.818182in}{1.977623in}}%
\pgfpathlineto{\pgfqpoint{4.043636in}{2.067237in}}%
\pgfpathlineto{\pgfqpoint{4.212727in}{2.133648in}}%
\pgfpathlineto{\pgfqpoint{4.325455in}{2.175903in}}%
\pgfpathlineto{\pgfqpoint{4.419394in}{2.209101in}}%
\pgfpathlineto{\pgfqpoint{4.513333in}{2.239879in}}%
\pgfpathlineto{\pgfqpoint{4.588485in}{2.262376in}}%
\pgfpathlineto{\pgfqpoint{4.663636in}{2.282660in}}%
\pgfpathlineto{\pgfqpoint{4.738788in}{2.296513in}}%
\pgfpathlineto{\pgfqpoint{4.813939in}{2.308044in}}%
\pgfpathlineto{\pgfqpoint{4.889091in}{2.317448in}}%
\pgfpathlineto{\pgfqpoint{4.980220in}{2.326261in}}%
\pgfpathlineto{\pgfqpoint{5.074159in}{2.332692in}}%
\pgfpathlineto{\pgfqpoint{5.168098in}{2.336754in}}%
\pgfpathlineto{\pgfqpoint{5.262038in}{2.338764in}}%
\pgfpathlineto{\pgfqpoint{5.374765in}{2.338884in}}%
\pgfpathlineto{\pgfqpoint{5.506280in}{2.336471in}}%
\pgfpathlineto{\pgfqpoint{5.656583in}{2.331209in}}%
\pgfpathlineto{\pgfqpoint{5.847273in}{2.322111in}}%
\pgfpathlineto{\pgfqpoint{6.410909in}{2.293658in}}%
\pgfpathlineto{\pgfqpoint{6.598788in}{2.287298in}}%
\pgfpathlineto{\pgfqpoint{6.767879in}{2.283857in}}%
\pgfpathlineto{\pgfqpoint{6.918182in}{2.282660in}}%
\pgfpathlineto{\pgfqpoint{6.918182in}{2.282660in}}%
\pgfusepath{stroke}%
\end{pgfscope}%
\begin{pgfscope}%
\pgfpathrectangle{\pgfqpoint{1.000000in}{0.330000in}}{\pgfqpoint{6.200000in}{2.310000in}}%
\pgfusepath{clip}%
\pgfsetrectcap%
\pgfsetroundjoin%
\pgfsetlinewidth{1.505625pt}%
\definecolor{currentstroke}{rgb}{0.580392,0.403922,0.741176}%
\pgfsetstrokecolor{currentstroke}%
\pgfsetdash{}{0pt}%
\pgfpathmoveto{\pgfqpoint{1.281818in}{2.282660in}}%
\pgfpathlineto{\pgfqpoint{1.488485in}{2.162152in}}%
\pgfpathlineto{\pgfqpoint{1.601212in}{2.098625in}}%
\pgfpathlineto{\pgfqpoint{1.713939in}{2.037784in}}%
\pgfpathlineto{\pgfqpoint{1.807879in}{1.989746in}}%
\pgfpathlineto{\pgfqpoint{1.883030in}{1.953397in}}%
\pgfpathlineto{\pgfqpoint{1.958182in}{1.919154in}}%
\pgfpathlineto{\pgfqpoint{2.033333in}{1.887243in}}%
\pgfpathlineto{\pgfqpoint{2.108485in}{1.857884in}}%
\pgfpathlineto{\pgfqpoint{2.183636in}{1.831291in}}%
\pgfpathlineto{\pgfqpoint{2.258788in}{1.807671in}}%
\pgfpathlineto{\pgfqpoint{2.315152in}{1.792028in}}%
\pgfpathlineto{\pgfqpoint{2.371515in}{1.778253in}}%
\pgfpathlineto{\pgfqpoint{2.409091in}{1.770147in}}%
\pgfpathlineto{\pgfqpoint{2.465455in}{1.761332in}}%
\pgfpathlineto{\pgfqpoint{2.521818in}{1.754473in}}%
\pgfpathlineto{\pgfqpoint{2.578182in}{1.749520in}}%
\pgfpathlineto{\pgfqpoint{2.634545in}{1.746425in}}%
\pgfpathlineto{\pgfqpoint{2.690909in}{1.745137in}}%
\pgfpathlineto{\pgfqpoint{2.747273in}{1.745600in}}%
\pgfpathlineto{\pgfqpoint{2.822424in}{1.748848in}}%
\pgfpathlineto{\pgfqpoint{2.897576in}{1.754973in}}%
\pgfpathlineto{\pgfqpoint{2.972727in}{1.763832in}}%
\pgfpathlineto{\pgfqpoint{3.047879in}{1.775273in}}%
\pgfpathlineto{\pgfqpoint{3.123030in}{1.789138in}}%
\pgfpathlineto{\pgfqpoint{3.198182in}{1.805263in}}%
\pgfpathlineto{\pgfqpoint{3.273333in}{1.823477in}}%
\pgfpathlineto{\pgfqpoint{3.367273in}{1.848910in}}%
\pgfpathlineto{\pgfqpoint{3.461212in}{1.876968in}}%
\pgfpathlineto{\pgfqpoint{3.555152in}{1.907273in}}%
\pgfpathlineto{\pgfqpoint{3.667879in}{1.946049in}}%
\pgfpathlineto{\pgfqpoint{3.799394in}{1.993708in}}%
\pgfpathlineto{\pgfqpoint{4.287879in}{2.173549in}}%
\pgfpathlineto{\pgfqpoint{4.381818in}{2.204842in}}%
\pgfpathlineto{\pgfqpoint{4.475758in}{2.233829in}}%
\pgfpathlineto{\pgfqpoint{4.550909in}{2.254987in}}%
\pgfpathlineto{\pgfqpoint{4.626061in}{2.274027in}}%
\pgfpathlineto{\pgfqpoint{4.663636in}{2.282660in}}%
\pgfpathlineto{\pgfqpoint{4.738788in}{2.293465in}}%
\pgfpathlineto{\pgfqpoint{4.813939in}{2.302151in}}%
\pgfpathlineto{\pgfqpoint{4.907879in}{2.310421in}}%
\pgfpathlineto{\pgfqpoint{4.999007in}{2.316078in}}%
\pgfpathlineto{\pgfqpoint{5.092947in}{2.319661in}}%
\pgfpathlineto{\pgfqpoint{5.186886in}{2.321216in}}%
\pgfpathlineto{\pgfqpoint{5.299613in}{2.320855in}}%
\pgfpathlineto{\pgfqpoint{5.431129in}{2.318023in}}%
\pgfpathlineto{\pgfqpoint{5.600220in}{2.311750in}}%
\pgfpathlineto{\pgfqpoint{6.223030in}{2.285396in}}%
\pgfpathlineto{\pgfqpoint{6.392121in}{2.281851in}}%
\pgfpathlineto{\pgfqpoint{6.561212in}{2.280482in}}%
\pgfpathlineto{\pgfqpoint{6.767879in}{2.281256in}}%
\pgfpathlineto{\pgfqpoint{6.918182in}{2.282660in}}%
\pgfpathlineto{\pgfqpoint{6.918182in}{2.282660in}}%
\pgfusepath{stroke}%
\end{pgfscope}%
\begin{pgfscope}%
\pgfpathrectangle{\pgfqpoint{1.000000in}{0.330000in}}{\pgfqpoint{6.200000in}{2.310000in}}%
\pgfusepath{clip}%
\pgfsetrectcap%
\pgfsetroundjoin%
\pgfsetlinewidth{1.505625pt}%
\definecolor{currentstroke}{rgb}{0.549020,0.337255,0.294118}%
\pgfsetstrokecolor{currentstroke}%
\pgfsetdash{}{0pt}%
\pgfpathmoveto{\pgfqpoint{1.281818in}{2.282660in}}%
\pgfpathlineto{\pgfqpoint{1.488485in}{2.171306in}}%
\pgfpathlineto{\pgfqpoint{1.620000in}{2.103011in}}%
\pgfpathlineto{\pgfqpoint{1.732727in}{2.047226in}}%
\pgfpathlineto{\pgfqpoint{1.826667in}{2.003269in}}%
\pgfpathlineto{\pgfqpoint{1.920606in}{1.962068in}}%
\pgfpathlineto{\pgfqpoint{1.995758in}{1.931371in}}%
\pgfpathlineto{\pgfqpoint{2.070909in}{1.902908in}}%
\pgfpathlineto{\pgfqpoint{2.146061in}{1.876878in}}%
\pgfpathlineto{\pgfqpoint{2.221212in}{1.853475in}}%
\pgfpathlineto{\pgfqpoint{2.296364in}{1.832885in}}%
\pgfpathlineto{\pgfqpoint{2.352727in}{1.819398in}}%
\pgfpathlineto{\pgfqpoint{2.409091in}{1.807670in}}%
\pgfpathlineto{\pgfqpoint{2.465455in}{1.799285in}}%
\pgfpathlineto{\pgfqpoint{2.521818in}{1.792690in}}%
\pgfpathlineto{\pgfqpoint{2.578182in}{1.787845in}}%
\pgfpathlineto{\pgfqpoint{2.634545in}{1.784705in}}%
\pgfpathlineto{\pgfqpoint{2.709697in}{1.783092in}}%
\pgfpathlineto{\pgfqpoint{2.784848in}{1.784315in}}%
\pgfpathlineto{\pgfqpoint{2.860000in}{1.788256in}}%
\pgfpathlineto{\pgfqpoint{2.935152in}{1.794789in}}%
\pgfpathlineto{\pgfqpoint{3.010303in}{1.803782in}}%
\pgfpathlineto{\pgfqpoint{3.085455in}{1.815099in}}%
\pgfpathlineto{\pgfqpoint{3.160606in}{1.828595in}}%
\pgfpathlineto{\pgfqpoint{3.235758in}{1.844122in}}%
\pgfpathlineto{\pgfqpoint{3.329697in}{1.866149in}}%
\pgfpathlineto{\pgfqpoint{3.423636in}{1.890790in}}%
\pgfpathlineto{\pgfqpoint{3.517576in}{1.917712in}}%
\pgfpathlineto{\pgfqpoint{3.630303in}{1.952545in}}%
\pgfpathlineto{\pgfqpoint{3.761818in}{1.995871in}}%
\pgfpathlineto{\pgfqpoint{3.930909in}{2.054214in}}%
\pgfpathlineto{\pgfqpoint{4.231515in}{2.158254in}}%
\pgfpathlineto{\pgfqpoint{4.344242in}{2.194984in}}%
\pgfpathlineto{\pgfqpoint{4.438182in}{2.223720in}}%
\pgfpathlineto{\pgfqpoint{4.532121in}{2.250234in}}%
\pgfpathlineto{\pgfqpoint{4.607273in}{2.269508in}}%
\pgfpathlineto{\pgfqpoint{4.663636in}{2.282660in}}%
\pgfpathlineto{\pgfqpoint{4.738788in}{2.294811in}}%
\pgfpathlineto{\pgfqpoint{4.813939in}{2.304927in}}%
\pgfpathlineto{\pgfqpoint{4.907879in}{2.314965in}}%
\pgfpathlineto{\pgfqpoint{4.999007in}{2.322215in}}%
\pgfpathlineto{\pgfqpoint{5.092947in}{2.327417in}}%
\pgfpathlineto{\pgfqpoint{5.186886in}{2.330599in}}%
\pgfpathlineto{\pgfqpoint{5.299613in}{2.332134in}}%
\pgfpathlineto{\pgfqpoint{5.431129in}{2.331334in}}%
\pgfpathlineto{\pgfqpoint{5.581432in}{2.327798in}}%
\pgfpathlineto{\pgfqpoint{5.769310in}{2.320729in}}%
\pgfpathlineto{\pgfqpoint{6.129091in}{2.304003in}}%
\pgfpathlineto{\pgfqpoint{6.392121in}{2.292979in}}%
\pgfpathlineto{\pgfqpoint{6.580000in}{2.287186in}}%
\pgfpathlineto{\pgfqpoint{6.767879in}{2.283710in}}%
\pgfpathlineto{\pgfqpoint{6.918182in}{2.282660in}}%
\pgfpathlineto{\pgfqpoint{6.918182in}{2.282660in}}%
\pgfusepath{stroke}%
\end{pgfscope}%
\begin{pgfscope}%
\pgfpathrectangle{\pgfqpoint{1.000000in}{0.330000in}}{\pgfqpoint{6.200000in}{2.310000in}}%
\pgfusepath{clip}%
\pgfsetrectcap%
\pgfsetroundjoin%
\pgfsetlinewidth{1.505625pt}%
\definecolor{currentstroke}{rgb}{0.890196,0.466667,0.760784}%
\pgfsetstrokecolor{currentstroke}%
\pgfsetdash{}{0pt}%
\pgfpathmoveto{\pgfqpoint{1.281818in}{2.282660in}}%
\pgfpathlineto{\pgfqpoint{1.450909in}{2.045683in}}%
\pgfpathlineto{\pgfqpoint{1.563636in}{1.891570in}}%
\pgfpathlineto{\pgfqpoint{1.657576in}{1.767246in}}%
\pgfpathlineto{\pgfqpoint{1.732727in}{1.671293in}}%
\pgfpathlineto{\pgfqpoint{1.807879in}{1.579068in}}%
\pgfpathlineto{\pgfqpoint{1.883030in}{1.491118in}}%
\pgfpathlineto{\pgfqpoint{1.939394in}{1.428280in}}%
\pgfpathlineto{\pgfqpoint{1.995758in}{1.368363in}}%
\pgfpathlineto{\pgfqpoint{2.052121in}{1.311578in}}%
\pgfpathlineto{\pgfqpoint{2.108485in}{1.258132in}}%
\pgfpathlineto{\pgfqpoint{2.164848in}{1.208225in}}%
\pgfpathlineto{\pgfqpoint{2.221212in}{1.162052in}}%
\pgfpathlineto{\pgfqpoint{2.277576in}{1.119803in}}%
\pgfpathlineto{\pgfqpoint{2.315152in}{1.093908in}}%
\pgfpathlineto{\pgfqpoint{2.352727in}{1.069892in}}%
\pgfpathlineto{\pgfqpoint{2.390303in}{1.047806in}}%
\pgfpathlineto{\pgfqpoint{2.409091in}{1.037503in}}%
\pgfpathlineto{\pgfqpoint{2.446667in}{1.020436in}}%
\pgfpathlineto{\pgfqpoint{2.484242in}{1.005359in}}%
\pgfpathlineto{\pgfqpoint{2.521818in}{0.992254in}}%
\pgfpathlineto{\pgfqpoint{2.559394in}{0.981100in}}%
\pgfpathlineto{\pgfqpoint{2.596970in}{0.971879in}}%
\pgfpathlineto{\pgfqpoint{2.634545in}{0.964568in}}%
\pgfpathlineto{\pgfqpoint{2.672121in}{0.959145in}}%
\pgfpathlineto{\pgfqpoint{2.709697in}{0.955586in}}%
\pgfpathlineto{\pgfqpoint{2.747273in}{0.953866in}}%
\pgfpathlineto{\pgfqpoint{2.784848in}{0.953958in}}%
\pgfpathlineto{\pgfqpoint{2.822424in}{0.955834in}}%
\pgfpathlineto{\pgfqpoint{2.860000in}{0.959465in}}%
\pgfpathlineto{\pgfqpoint{2.897576in}{0.964822in}}%
\pgfpathlineto{\pgfqpoint{2.935152in}{0.971872in}}%
\pgfpathlineto{\pgfqpoint{2.972727in}{0.980582in}}%
\pgfpathlineto{\pgfqpoint{3.010303in}{0.990918in}}%
\pgfpathlineto{\pgfqpoint{3.047879in}{1.002845in}}%
\pgfpathlineto{\pgfqpoint{3.104242in}{1.023635in}}%
\pgfpathlineto{\pgfqpoint{3.160606in}{1.047790in}}%
\pgfpathlineto{\pgfqpoint{3.216970in}{1.075172in}}%
\pgfpathlineto{\pgfqpoint{3.273333in}{1.105637in}}%
\pgfpathlineto{\pgfqpoint{3.329697in}{1.139032in}}%
\pgfpathlineto{\pgfqpoint{3.386061in}{1.175200in}}%
\pgfpathlineto{\pgfqpoint{3.442424in}{1.213975in}}%
\pgfpathlineto{\pgfqpoint{3.498788in}{1.255183in}}%
\pgfpathlineto{\pgfqpoint{3.555152in}{1.298644in}}%
\pgfpathlineto{\pgfqpoint{3.630303in}{1.359772in}}%
\pgfpathlineto{\pgfqpoint{3.705455in}{1.424104in}}%
\pgfpathlineto{\pgfqpoint{3.780606in}{1.491152in}}%
\pgfpathlineto{\pgfqpoint{3.874545in}{1.577994in}}%
\pgfpathlineto{\pgfqpoint{4.006061in}{1.703321in}}%
\pgfpathlineto{\pgfqpoint{4.269091in}{1.955280in}}%
\pgfpathlineto{\pgfqpoint{4.363030in}{2.041550in}}%
\pgfpathlineto{\pgfqpoint{4.438182in}{2.107666in}}%
\pgfpathlineto{\pgfqpoint{4.513333in}{2.170405in}}%
\pgfpathlineto{\pgfqpoint{4.569697in}{2.214783in}}%
\pgfpathlineto{\pgfqpoint{4.626061in}{2.256498in}}%
\pgfpathlineto{\pgfqpoint{4.663636in}{2.282660in}}%
\pgfpathlineto{\pgfqpoint{4.720000in}{2.311913in}}%
\pgfpathlineto{\pgfqpoint{4.776364in}{2.338101in}}%
\pgfpathlineto{\pgfqpoint{4.832727in}{2.361389in}}%
\pgfpathlineto{\pgfqpoint{4.889091in}{2.381931in}}%
\pgfpathlineto{\pgfqpoint{4.945455in}{2.399874in}}%
\pgfpathlineto{\pgfqpoint{5.017795in}{2.419284in}}%
\pgfpathlineto{\pgfqpoint{5.074159in}{2.431641in}}%
\pgfpathlineto{\pgfqpoint{5.130523in}{2.441698in}}%
\pgfpathlineto{\pgfqpoint{5.186886in}{2.449577in}}%
\pgfpathlineto{\pgfqpoint{5.243250in}{2.455405in}}%
\pgfpathlineto{\pgfqpoint{5.299613in}{2.459307in}}%
\pgfpathlineto{\pgfqpoint{5.355977in}{2.461407in}}%
\pgfpathlineto{\pgfqpoint{5.431129in}{2.461621in}}%
\pgfpathlineto{\pgfqpoint{5.506280in}{2.459152in}}%
\pgfpathlineto{\pgfqpoint{5.581432in}{2.454296in}}%
\pgfpathlineto{\pgfqpoint{5.656583in}{2.447349in}}%
\pgfpathlineto{\pgfqpoint{5.750523in}{2.436178in}}%
\pgfpathlineto{\pgfqpoint{5.847273in}{2.422377in}}%
\pgfpathlineto{\pgfqpoint{5.978788in}{2.401164in}}%
\pgfpathlineto{\pgfqpoint{6.429697in}{2.325765in}}%
\pgfpathlineto{\pgfqpoint{6.542424in}{2.310393in}}%
\pgfpathlineto{\pgfqpoint{6.636364in}{2.299693in}}%
\pgfpathlineto{\pgfqpoint{6.730303in}{2.291291in}}%
\pgfpathlineto{\pgfqpoint{6.824242in}{2.285513in}}%
\pgfpathlineto{\pgfqpoint{6.899394in}{2.282983in}}%
\pgfpathlineto{\pgfqpoint{6.918182in}{2.282660in}}%
\pgfpathlineto{\pgfqpoint{6.918182in}{2.282660in}}%
\pgfusepath{stroke}%
\end{pgfscope}%
\begin{pgfscope}%
\pgfpathrectangle{\pgfqpoint{1.000000in}{0.330000in}}{\pgfqpoint{6.200000in}{2.310000in}}%
\pgfusepath{clip}%
\pgfsetrectcap%
\pgfsetroundjoin%
\pgfsetlinewidth{1.505625pt}%
\definecolor{currentstroke}{rgb}{0.498039,0.498039,0.498039}%
\pgfsetstrokecolor{currentstroke}%
\pgfsetdash{}{0pt}%
\pgfpathmoveto{\pgfqpoint{1.281818in}{2.282660in}}%
\pgfpathlineto{\pgfqpoint{1.450909in}{2.013556in}}%
\pgfpathlineto{\pgfqpoint{1.563636in}{1.838605in}}%
\pgfpathlineto{\pgfqpoint{1.657576in}{1.697530in}}%
\pgfpathlineto{\pgfqpoint{1.732727in}{1.588701in}}%
\pgfpathlineto{\pgfqpoint{1.807879in}{1.484159in}}%
\pgfpathlineto{\pgfqpoint{1.864242in}{1.408946in}}%
\pgfpathlineto{\pgfqpoint{1.920606in}{1.336757in}}%
\pgfpathlineto{\pgfqpoint{1.976970in}{1.267847in}}%
\pgfpathlineto{\pgfqpoint{2.033333in}{1.202462in}}%
\pgfpathlineto{\pgfqpoint{2.089697in}{1.140843in}}%
\pgfpathlineto{\pgfqpoint{2.146061in}{1.083225in}}%
\pgfpathlineto{\pgfqpoint{2.202424in}{1.029837in}}%
\pgfpathlineto{\pgfqpoint{2.240000in}{0.996705in}}%
\pgfpathlineto{\pgfqpoint{2.277576in}{0.965617in}}%
\pgfpathlineto{\pgfqpoint{2.315152in}{0.936635in}}%
\pgfpathlineto{\pgfqpoint{2.352727in}{0.909822in}}%
\pgfpathlineto{\pgfqpoint{2.390303in}{0.885239in}}%
\pgfpathlineto{\pgfqpoint{2.409091in}{0.873802in}}%
\pgfpathlineto{\pgfqpoint{2.446667in}{0.855211in}}%
\pgfpathlineto{\pgfqpoint{2.484242in}{0.838914in}}%
\pgfpathlineto{\pgfqpoint{2.521818in}{0.824885in}}%
\pgfpathlineto{\pgfqpoint{2.559394in}{0.813098in}}%
\pgfpathlineto{\pgfqpoint{2.596970in}{0.803525in}}%
\pgfpathlineto{\pgfqpoint{2.634545in}{0.796136in}}%
\pgfpathlineto{\pgfqpoint{2.672121in}{0.790900in}}%
\pgfpathlineto{\pgfqpoint{2.709697in}{0.787786in}}%
\pgfpathlineto{\pgfqpoint{2.747273in}{0.786760in}}%
\pgfpathlineto{\pgfqpoint{2.784848in}{0.787788in}}%
\pgfpathlineto{\pgfqpoint{2.822424in}{0.790834in}}%
\pgfpathlineto{\pgfqpoint{2.860000in}{0.795860in}}%
\pgfpathlineto{\pgfqpoint{2.897576in}{0.802828in}}%
\pgfpathlineto{\pgfqpoint{2.935152in}{0.811697in}}%
\pgfpathlineto{\pgfqpoint{2.972727in}{0.822427in}}%
\pgfpathlineto{\pgfqpoint{3.010303in}{0.834975in}}%
\pgfpathlineto{\pgfqpoint{3.047879in}{0.849297in}}%
\pgfpathlineto{\pgfqpoint{3.085455in}{0.865347in}}%
\pgfpathlineto{\pgfqpoint{3.123030in}{0.883078in}}%
\pgfpathlineto{\pgfqpoint{3.179394in}{0.912722in}}%
\pgfpathlineto{\pgfqpoint{3.235758in}{0.945871in}}%
\pgfpathlineto{\pgfqpoint{3.292121in}{0.982349in}}%
\pgfpathlineto{\pgfqpoint{3.348485in}{1.021975in}}%
\pgfpathlineto{\pgfqpoint{3.404848in}{1.064556in}}%
\pgfpathlineto{\pgfqpoint{3.461212in}{1.109896in}}%
\pgfpathlineto{\pgfqpoint{3.517576in}{1.157789in}}%
\pgfpathlineto{\pgfqpoint{3.573939in}{1.208023in}}%
\pgfpathlineto{\pgfqpoint{3.649091in}{1.278263in}}%
\pgfpathlineto{\pgfqpoint{3.724242in}{1.351730in}}%
\pgfpathlineto{\pgfqpoint{3.818182in}{1.447239in}}%
\pgfpathlineto{\pgfqpoint{3.930909in}{1.565702in}}%
\pgfpathlineto{\pgfqpoint{4.287879in}{1.944482in}}%
\pgfpathlineto{\pgfqpoint{4.363030in}{2.020053in}}%
\pgfpathlineto{\pgfqpoint{4.438182in}{2.092474in}}%
\pgfpathlineto{\pgfqpoint{4.494545in}{2.144241in}}%
\pgfpathlineto{\pgfqpoint{4.550909in}{2.193433in}}%
\pgfpathlineto{\pgfqpoint{4.607273in}{2.239695in}}%
\pgfpathlineto{\pgfqpoint{4.663636in}{2.282660in}}%
\pgfpathlineto{\pgfqpoint{4.720000in}{2.313911in}}%
\pgfpathlineto{\pgfqpoint{4.776364in}{2.341694in}}%
\pgfpathlineto{\pgfqpoint{4.832727in}{2.366204in}}%
\pgfpathlineto{\pgfqpoint{4.889091in}{2.387629in}}%
\pgfpathlineto{\pgfqpoint{4.945455in}{2.406146in}}%
\pgfpathlineto{\pgfqpoint{5.017795in}{2.425898in}}%
\pgfpathlineto{\pgfqpoint{5.074159in}{2.438284in}}%
\pgfpathlineto{\pgfqpoint{5.130523in}{2.448195in}}%
\pgfpathlineto{\pgfqpoint{5.186886in}{2.455787in}}%
\pgfpathlineto{\pgfqpoint{5.243250in}{2.461212in}}%
\pgfpathlineto{\pgfqpoint{5.299613in}{2.464619in}}%
\pgfpathlineto{\pgfqpoint{5.355977in}{2.466158in}}%
\pgfpathlineto{\pgfqpoint{5.431129in}{2.465555in}}%
\pgfpathlineto{\pgfqpoint{5.506280in}{2.462232in}}%
\pgfpathlineto{\pgfqpoint{5.581432in}{2.456525in}}%
\pgfpathlineto{\pgfqpoint{5.656583in}{2.448764in}}%
\pgfpathlineto{\pgfqpoint{5.750523in}{2.436667in}}%
\pgfpathlineto{\pgfqpoint{5.866061in}{2.419027in}}%
\pgfpathlineto{\pgfqpoint{6.016364in}{2.393393in}}%
\pgfpathlineto{\pgfqpoint{6.316970in}{2.341305in}}%
\pgfpathlineto{\pgfqpoint{6.429697in}{2.323983in}}%
\pgfpathlineto{\pgfqpoint{6.542424in}{2.308970in}}%
\pgfpathlineto{\pgfqpoint{6.636364in}{2.298662in}}%
\pgfpathlineto{\pgfqpoint{6.730303in}{2.290687in}}%
\pgfpathlineto{\pgfqpoint{6.824242in}{2.285272in}}%
\pgfpathlineto{\pgfqpoint{6.899394in}{2.282946in}}%
\pgfpathlineto{\pgfqpoint{6.918182in}{2.282660in}}%
\pgfpathlineto{\pgfqpoint{6.918182in}{2.282660in}}%
\pgfusepath{stroke}%
\end{pgfscope}%
\begin{pgfscope}%
\pgfpathrectangle{\pgfqpoint{1.000000in}{0.330000in}}{\pgfqpoint{6.200000in}{2.310000in}}%
\pgfusepath{clip}%
\pgfsetrectcap%
\pgfsetroundjoin%
\pgfsetlinewidth{1.505625pt}%
\definecolor{currentstroke}{rgb}{0.737255,0.741176,0.133333}%
\pgfsetstrokecolor{currentstroke}%
\pgfsetdash{}{0pt}%
\pgfpathmoveto{\pgfqpoint{1.281818in}{2.282660in}}%
\pgfpathlineto{\pgfqpoint{1.450909in}{2.011477in}}%
\pgfpathlineto{\pgfqpoint{1.563636in}{1.835155in}}%
\pgfpathlineto{\pgfqpoint{1.657576in}{1.692955in}}%
\pgfpathlineto{\pgfqpoint{1.732727in}{1.583242in}}%
\pgfpathlineto{\pgfqpoint{1.807879in}{1.477832in}}%
\pgfpathlineto{\pgfqpoint{1.864242in}{1.401980in}}%
\pgfpathlineto{\pgfqpoint{1.920606in}{1.329164in}}%
\pgfpathlineto{\pgfqpoint{1.976970in}{1.259640in}}%
\pgfpathlineto{\pgfqpoint{2.033333in}{1.193656in}}%
\pgfpathlineto{\pgfqpoint{2.089697in}{1.131453in}}%
\pgfpathlineto{\pgfqpoint{2.146061in}{1.073268in}}%
\pgfpathlineto{\pgfqpoint{2.202424in}{1.019330in}}%
\pgfpathlineto{\pgfqpoint{2.240000in}{0.985843in}}%
\pgfpathlineto{\pgfqpoint{2.277576in}{0.954407in}}%
\pgfpathlineto{\pgfqpoint{2.315152in}{0.925087in}}%
\pgfpathlineto{\pgfqpoint{2.352727in}{0.897946in}}%
\pgfpathlineto{\pgfqpoint{2.390303in}{0.873044in}}%
\pgfpathlineto{\pgfqpoint{2.409091in}{0.861451in}}%
\pgfpathlineto{\pgfqpoint{2.446667in}{0.842556in}}%
\pgfpathlineto{\pgfqpoint{2.484242in}{0.825967in}}%
\pgfpathlineto{\pgfqpoint{2.521818in}{0.811656in}}%
\pgfpathlineto{\pgfqpoint{2.559394in}{0.799598in}}%
\pgfpathlineto{\pgfqpoint{2.596970in}{0.789766in}}%
\pgfpathlineto{\pgfqpoint{2.634545in}{0.782130in}}%
\pgfpathlineto{\pgfqpoint{2.672121in}{0.776660in}}%
\pgfpathlineto{\pgfqpoint{2.709697in}{0.773324in}}%
\pgfpathlineto{\pgfqpoint{2.747273in}{0.772089in}}%
\pgfpathlineto{\pgfqpoint{2.784848in}{0.772921in}}%
\pgfpathlineto{\pgfqpoint{2.822424in}{0.775785in}}%
\pgfpathlineto{\pgfqpoint{2.860000in}{0.780644in}}%
\pgfpathlineto{\pgfqpoint{2.897576in}{0.787458in}}%
\pgfpathlineto{\pgfqpoint{2.935152in}{0.796189in}}%
\pgfpathlineto{\pgfqpoint{2.972727in}{0.806796in}}%
\pgfpathlineto{\pgfqpoint{3.010303in}{0.819236in}}%
\pgfpathlineto{\pgfqpoint{3.047879in}{0.833465in}}%
\pgfpathlineto{\pgfqpoint{3.085455in}{0.849439in}}%
\pgfpathlineto{\pgfqpoint{3.123030in}{0.867111in}}%
\pgfpathlineto{\pgfqpoint{3.179394in}{0.896696in}}%
\pgfpathlineto{\pgfqpoint{3.235758in}{0.929827in}}%
\pgfpathlineto{\pgfqpoint{3.292121in}{0.966327in}}%
\pgfpathlineto{\pgfqpoint{3.348485in}{1.006014in}}%
\pgfpathlineto{\pgfqpoint{3.404848in}{1.048701in}}%
\pgfpathlineto{\pgfqpoint{3.461212in}{1.094189in}}%
\pgfpathlineto{\pgfqpoint{3.517576in}{1.142276in}}%
\pgfpathlineto{\pgfqpoint{3.573939in}{1.192749in}}%
\pgfpathlineto{\pgfqpoint{3.649091in}{1.263381in}}%
\pgfpathlineto{\pgfqpoint{3.724242in}{1.337328in}}%
\pgfpathlineto{\pgfqpoint{3.818182in}{1.433561in}}%
\pgfpathlineto{\pgfqpoint{3.930909in}{1.553085in}}%
\pgfpathlineto{\pgfqpoint{4.306667in}{1.956171in}}%
\pgfpathlineto{\pgfqpoint{4.381818in}{2.032381in}}%
\pgfpathlineto{\pgfqpoint{4.456970in}{2.105357in}}%
\pgfpathlineto{\pgfqpoint{4.513333in}{2.157480in}}%
\pgfpathlineto{\pgfqpoint{4.569697in}{2.206977in}}%
\pgfpathlineto{\pgfqpoint{4.626061in}{2.253491in}}%
\pgfpathlineto{\pgfqpoint{4.663636in}{2.282660in}}%
\pgfpathlineto{\pgfqpoint{4.720000in}{2.315887in}}%
\pgfpathlineto{\pgfqpoint{4.776364in}{2.345583in}}%
\pgfpathlineto{\pgfqpoint{4.832727in}{2.371912in}}%
\pgfpathlineto{\pgfqpoint{4.889091in}{2.395031in}}%
\pgfpathlineto{\pgfqpoint{4.945455in}{2.415100in}}%
\pgfpathlineto{\pgfqpoint{4.999007in}{2.431484in}}%
\pgfpathlineto{\pgfqpoint{5.055371in}{2.446051in}}%
\pgfpathlineto{\pgfqpoint{5.111735in}{2.458021in}}%
\pgfpathlineto{\pgfqpoint{5.168098in}{2.467547in}}%
\pgfpathlineto{\pgfqpoint{5.224462in}{2.474775in}}%
\pgfpathlineto{\pgfqpoint{5.280826in}{2.479853in}}%
\pgfpathlineto{\pgfqpoint{5.337189in}{2.482925in}}%
\pgfpathlineto{\pgfqpoint{5.393553in}{2.484135in}}%
\pgfpathlineto{\pgfqpoint{5.468704in}{2.483094in}}%
\pgfpathlineto{\pgfqpoint{5.543856in}{2.479321in}}%
\pgfpathlineto{\pgfqpoint{5.619007in}{2.473139in}}%
\pgfpathlineto{\pgfqpoint{5.694159in}{2.464864in}}%
\pgfpathlineto{\pgfqpoint{5.790909in}{2.451633in}}%
\pgfpathlineto{\pgfqpoint{5.884848in}{2.436575in}}%
\pgfpathlineto{\pgfqpoint{6.016364in}{2.412946in}}%
\pgfpathlineto{\pgfqpoint{6.467273in}{2.328929in}}%
\pgfpathlineto{\pgfqpoint{6.561212in}{2.314446in}}%
\pgfpathlineto{\pgfqpoint{6.655152in}{2.302110in}}%
\pgfpathlineto{\pgfqpoint{6.749091in}{2.292368in}}%
\pgfpathlineto{\pgfqpoint{6.824242in}{2.286705in}}%
\pgfpathlineto{\pgfqpoint{6.899394in}{2.283181in}}%
\pgfpathlineto{\pgfqpoint{6.918182in}{2.282660in}}%
\pgfpathlineto{\pgfqpoint{6.918182in}{2.282660in}}%
\pgfusepath{stroke}%
\end{pgfscope}%
\begin{pgfscope}%
\pgfpathrectangle{\pgfqpoint{1.000000in}{0.330000in}}{\pgfqpoint{6.200000in}{2.310000in}}%
\pgfusepath{clip}%
\pgfsetrectcap%
\pgfsetroundjoin%
\pgfsetlinewidth{1.505625pt}%
\definecolor{currentstroke}{rgb}{0.090196,0.745098,0.811765}%
\pgfsetstrokecolor{currentstroke}%
\pgfsetdash{}{0pt}%
\pgfpathmoveto{\pgfqpoint{1.281818in}{2.282660in}}%
\pgfpathlineto{\pgfqpoint{1.488485in}{2.174690in}}%
\pgfpathlineto{\pgfqpoint{1.620000in}{2.108514in}}%
\pgfpathlineto{\pgfqpoint{1.732727in}{2.054505in}}%
\pgfpathlineto{\pgfqpoint{1.826667in}{2.011990in}}%
\pgfpathlineto{\pgfqpoint{1.920606in}{1.972192in}}%
\pgfpathlineto{\pgfqpoint{1.995758in}{1.942581in}}%
\pgfpathlineto{\pgfqpoint{2.070909in}{1.915170in}}%
\pgfpathlineto{\pgfqpoint{2.146061in}{1.890154in}}%
\pgfpathlineto{\pgfqpoint{2.221212in}{1.867722in}}%
\pgfpathlineto{\pgfqpoint{2.296364in}{1.848058in}}%
\pgfpathlineto{\pgfqpoint{2.371515in}{1.831339in}}%
\pgfpathlineto{\pgfqpoint{2.427879in}{1.821345in}}%
\pgfpathlineto{\pgfqpoint{2.484242in}{1.814153in}}%
\pgfpathlineto{\pgfqpoint{2.540606in}{1.808704in}}%
\pgfpathlineto{\pgfqpoint{2.596970in}{1.804956in}}%
\pgfpathlineto{\pgfqpoint{2.672121in}{1.802524in}}%
\pgfpathlineto{\pgfqpoint{2.747273in}{1.802918in}}%
\pgfpathlineto{\pgfqpoint{2.822424in}{1.806018in}}%
\pgfpathlineto{\pgfqpoint{2.897576in}{1.811699in}}%
\pgfpathlineto{\pgfqpoint{2.972727in}{1.819828in}}%
\pgfpathlineto{\pgfqpoint{3.047879in}{1.830268in}}%
\pgfpathlineto{\pgfqpoint{3.123030in}{1.842873in}}%
\pgfpathlineto{\pgfqpoint{3.198182in}{1.857496in}}%
\pgfpathlineto{\pgfqpoint{3.292121in}{1.878371in}}%
\pgfpathlineto{\pgfqpoint{3.386061in}{1.901834in}}%
\pgfpathlineto{\pgfqpoint{3.480000in}{1.927554in}}%
\pgfpathlineto{\pgfqpoint{3.592727in}{1.960907in}}%
\pgfpathlineto{\pgfqpoint{3.724242in}{2.002455in}}%
\pgfpathlineto{\pgfqpoint{3.912121in}{2.064726in}}%
\pgfpathlineto{\pgfqpoint{4.175152in}{2.151954in}}%
\pgfpathlineto{\pgfqpoint{4.306667in}{2.192918in}}%
\pgfpathlineto{\pgfqpoint{4.400606in}{2.220128in}}%
\pgfpathlineto{\pgfqpoint{4.494545in}{2.245075in}}%
\pgfpathlineto{\pgfqpoint{4.569697in}{2.263066in}}%
\pgfpathlineto{\pgfqpoint{4.626061in}{2.275238in}}%
\pgfpathlineto{\pgfqpoint{4.663636in}{2.282660in}}%
\pgfpathlineto{\pgfqpoint{4.738788in}{2.291314in}}%
\pgfpathlineto{\pgfqpoint{4.832727in}{2.299548in}}%
\pgfpathlineto{\pgfqpoint{4.926667in}{2.305379in}}%
\pgfpathlineto{\pgfqpoint{5.036583in}{2.309650in}}%
\pgfpathlineto{\pgfqpoint{5.149310in}{2.311341in}}%
\pgfpathlineto{\pgfqpoint{5.262038in}{2.310773in}}%
\pgfpathlineto{\pgfqpoint{5.393553in}{2.307921in}}%
\pgfpathlineto{\pgfqpoint{5.562644in}{2.301922in}}%
\pgfpathlineto{\pgfqpoint{6.110303in}{2.280531in}}%
\pgfpathlineto{\pgfqpoint{6.279394in}{2.277440in}}%
\pgfpathlineto{\pgfqpoint{6.448485in}{2.276703in}}%
\pgfpathlineto{\pgfqpoint{6.636364in}{2.278259in}}%
\pgfpathlineto{\pgfqpoint{6.918182in}{2.282660in}}%
\pgfpathlineto{\pgfqpoint{6.918182in}{2.282660in}}%
\pgfusepath{stroke}%
\end{pgfscope}%
\begin{pgfscope}%
\pgfpathrectangle{\pgfqpoint{1.000000in}{0.330000in}}{\pgfqpoint{6.200000in}{2.310000in}}%
\pgfusepath{clip}%
\pgfsetrectcap%
\pgfsetroundjoin%
\pgfsetlinewidth{1.505625pt}%
\definecolor{currentstroke}{rgb}{0.121569,0.466667,0.705882}%
\pgfsetstrokecolor{currentstroke}%
\pgfsetdash{}{0pt}%
\pgfpathmoveto{\pgfqpoint{1.281818in}{2.282660in}}%
\pgfpathlineto{\pgfqpoint{1.450909in}{2.001883in}}%
\pgfpathlineto{\pgfqpoint{1.563636in}{1.819328in}}%
\pgfpathlineto{\pgfqpoint{1.657576in}{1.672105in}}%
\pgfpathlineto{\pgfqpoint{1.732727in}{1.558521in}}%
\pgfpathlineto{\pgfqpoint{1.807879in}{1.449397in}}%
\pgfpathlineto{\pgfqpoint{1.864242in}{1.370876in}}%
\pgfpathlineto{\pgfqpoint{1.920606in}{1.295502in}}%
\pgfpathlineto{\pgfqpoint{1.976970in}{1.223540in}}%
\pgfpathlineto{\pgfqpoint{2.033333in}{1.155246in}}%
\pgfpathlineto{\pgfqpoint{2.089697in}{1.090871in}}%
\pgfpathlineto{\pgfqpoint{2.146061in}{1.030661in}}%
\pgfpathlineto{\pgfqpoint{2.202424in}{0.974853in}}%
\pgfpathlineto{\pgfqpoint{2.240000in}{0.940209in}}%
\pgfpathlineto{\pgfqpoint{2.277576in}{0.907692in}}%
\pgfpathlineto{\pgfqpoint{2.315152in}{0.877369in}}%
\pgfpathlineto{\pgfqpoint{2.352727in}{0.849303in}}%
\pgfpathlineto{\pgfqpoint{2.390303in}{0.823559in}}%
\pgfpathlineto{\pgfqpoint{2.409091in}{0.811577in}}%
\pgfpathlineto{\pgfqpoint{2.446667in}{0.792074in}}%
\pgfpathlineto{\pgfqpoint{2.484242in}{0.774961in}}%
\pgfpathlineto{\pgfqpoint{2.521818in}{0.760209in}}%
\pgfpathlineto{\pgfqpoint{2.559394in}{0.747792in}}%
\pgfpathlineto{\pgfqpoint{2.596970in}{0.737681in}}%
\pgfpathlineto{\pgfqpoint{2.634545in}{0.729845in}}%
\pgfpathlineto{\pgfqpoint{2.672121in}{0.724252in}}%
\pgfpathlineto{\pgfqpoint{2.709697in}{0.720869in}}%
\pgfpathlineto{\pgfqpoint{2.747273in}{0.719662in}}%
\pgfpathlineto{\pgfqpoint{2.784848in}{0.720595in}}%
\pgfpathlineto{\pgfqpoint{2.822424in}{0.723631in}}%
\pgfpathlineto{\pgfqpoint{2.860000in}{0.728731in}}%
\pgfpathlineto{\pgfqpoint{2.897576in}{0.735855in}}%
\pgfpathlineto{\pgfqpoint{2.935152in}{0.744963in}}%
\pgfpathlineto{\pgfqpoint{2.972727in}{0.756010in}}%
\pgfpathlineto{\pgfqpoint{3.010303in}{0.768954in}}%
\pgfpathlineto{\pgfqpoint{3.047879in}{0.783748in}}%
\pgfpathlineto{\pgfqpoint{3.085455in}{0.800345in}}%
\pgfpathlineto{\pgfqpoint{3.123030in}{0.818698in}}%
\pgfpathlineto{\pgfqpoint{3.179394in}{0.849408in}}%
\pgfpathlineto{\pgfqpoint{3.235758in}{0.883780in}}%
\pgfpathlineto{\pgfqpoint{3.292121in}{0.921633in}}%
\pgfpathlineto{\pgfqpoint{3.348485in}{0.962777in}}%
\pgfpathlineto{\pgfqpoint{3.404848in}{1.007016in}}%
\pgfpathlineto{\pgfqpoint{3.461212in}{1.054147in}}%
\pgfpathlineto{\pgfqpoint{3.517576in}{1.103956in}}%
\pgfpathlineto{\pgfqpoint{3.573939in}{1.156226in}}%
\pgfpathlineto{\pgfqpoint{3.649091in}{1.229353in}}%
\pgfpathlineto{\pgfqpoint{3.724242in}{1.305890in}}%
\pgfpathlineto{\pgfqpoint{3.818182in}{1.405466in}}%
\pgfpathlineto{\pgfqpoint{3.930909in}{1.529099in}}%
\pgfpathlineto{\pgfqpoint{4.306667in}{1.945704in}}%
\pgfpathlineto{\pgfqpoint{4.381818in}{2.024407in}}%
\pgfpathlineto{\pgfqpoint{4.456970in}{2.099745in}}%
\pgfpathlineto{\pgfqpoint{4.513333in}{2.153539in}}%
\pgfpathlineto{\pgfqpoint{4.569697in}{2.204609in}}%
\pgfpathlineto{\pgfqpoint{4.626061in}{2.252584in}}%
\pgfpathlineto{\pgfqpoint{4.663636in}{2.282660in}}%
\pgfpathlineto{\pgfqpoint{4.720000in}{2.316864in}}%
\pgfpathlineto{\pgfqpoint{4.776364in}{2.347416in}}%
\pgfpathlineto{\pgfqpoint{4.832727in}{2.374484in}}%
\pgfpathlineto{\pgfqpoint{4.889091in}{2.398235in}}%
\pgfpathlineto{\pgfqpoint{4.945455in}{2.418835in}}%
\pgfpathlineto{\pgfqpoint{4.999007in}{2.435637in}}%
\pgfpathlineto{\pgfqpoint{5.055371in}{2.450559in}}%
\pgfpathlineto{\pgfqpoint{5.111735in}{2.462804in}}%
\pgfpathlineto{\pgfqpoint{5.168098in}{2.472529in}}%
\pgfpathlineto{\pgfqpoint{5.224462in}{2.479889in}}%
\pgfpathlineto{\pgfqpoint{5.280826in}{2.485037in}}%
\pgfpathlineto{\pgfqpoint{5.337189in}{2.488123in}}%
\pgfpathlineto{\pgfqpoint{5.393553in}{2.489296in}}%
\pgfpathlineto{\pgfqpoint{5.449916in}{2.488702in}}%
\pgfpathlineto{\pgfqpoint{5.525068in}{2.485410in}}%
\pgfpathlineto{\pgfqpoint{5.600220in}{2.479570in}}%
\pgfpathlineto{\pgfqpoint{5.675371in}{2.471510in}}%
\pgfpathlineto{\pgfqpoint{5.769310in}{2.458806in}}%
\pgfpathlineto{\pgfqpoint{5.866061in}{2.443275in}}%
\pgfpathlineto{\pgfqpoint{5.978788in}{2.422911in}}%
\pgfpathlineto{\pgfqpoint{6.166667in}{2.386152in}}%
\pgfpathlineto{\pgfqpoint{6.373333in}{2.346242in}}%
\pgfpathlineto{\pgfqpoint{6.486061in}{2.326635in}}%
\pgfpathlineto{\pgfqpoint{6.580000in}{2.312267in}}%
\pgfpathlineto{\pgfqpoint{6.673939in}{2.300202in}}%
\pgfpathlineto{\pgfqpoint{6.749091in}{2.292503in}}%
\pgfpathlineto{\pgfqpoint{6.824242in}{2.286755in}}%
\pgfpathlineto{\pgfqpoint{6.899394in}{2.283185in}}%
\pgfpathlineto{\pgfqpoint{6.918182in}{2.282660in}}%
\pgfpathlineto{\pgfqpoint{6.918182in}{2.282660in}}%
\pgfusepath{stroke}%
\end{pgfscope}%
\begin{pgfscope}%
\pgfpathrectangle{\pgfqpoint{1.000000in}{0.330000in}}{\pgfqpoint{6.200000in}{2.310000in}}%
\pgfusepath{clip}%
\pgfsetrectcap%
\pgfsetroundjoin%
\pgfsetlinewidth{1.505625pt}%
\definecolor{currentstroke}{rgb}{1.000000,0.498039,0.054902}%
\pgfsetstrokecolor{currentstroke}%
\pgfsetdash{}{0pt}%
\pgfpathmoveto{\pgfqpoint{1.281818in}{2.282660in}}%
\pgfpathlineto{\pgfqpoint{1.450909in}{2.003962in}}%
\pgfpathlineto{\pgfqpoint{1.563636in}{1.822777in}}%
\pgfpathlineto{\pgfqpoint{1.657576in}{1.676680in}}%
\pgfpathlineto{\pgfqpoint{1.732727in}{1.563980in}}%
\pgfpathlineto{\pgfqpoint{1.807879in}{1.455724in}}%
\pgfpathlineto{\pgfqpoint{1.864242in}{1.377842in}}%
\pgfpathlineto{\pgfqpoint{1.920606in}{1.303095in}}%
\pgfpathlineto{\pgfqpoint{1.976970in}{1.231746in}}%
\pgfpathlineto{\pgfqpoint{2.033333in}{1.164051in}}%
\pgfpathlineto{\pgfqpoint{2.089697in}{1.100260in}}%
\pgfpathlineto{\pgfqpoint{2.146061in}{1.040617in}}%
\pgfpathlineto{\pgfqpoint{2.202424in}{0.985360in}}%
\pgfpathlineto{\pgfqpoint{2.240000in}{0.951071in}}%
\pgfpathlineto{\pgfqpoint{2.277576in}{0.918902in}}%
\pgfpathlineto{\pgfqpoint{2.315152in}{0.888917in}}%
\pgfpathlineto{\pgfqpoint{2.352727in}{0.861180in}}%
\pgfpathlineto{\pgfqpoint{2.390303in}{0.835754in}}%
\pgfpathlineto{\pgfqpoint{2.409091in}{0.823928in}}%
\pgfpathlineto{\pgfqpoint{2.446667in}{0.804729in}}%
\pgfpathlineto{\pgfqpoint{2.484242in}{0.787908in}}%
\pgfpathlineto{\pgfqpoint{2.521818in}{0.773438in}}%
\pgfpathlineto{\pgfqpoint{2.559394in}{0.761292in}}%
\pgfpathlineto{\pgfqpoint{2.596970in}{0.751440in}}%
\pgfpathlineto{\pgfqpoint{2.634545in}{0.743850in}}%
\pgfpathlineto{\pgfqpoint{2.672121in}{0.738492in}}%
\pgfpathlineto{\pgfqpoint{2.709697in}{0.735331in}}%
\pgfpathlineto{\pgfqpoint{2.747273in}{0.734333in}}%
\pgfpathlineto{\pgfqpoint{2.784848in}{0.735462in}}%
\pgfpathlineto{\pgfqpoint{2.822424in}{0.738679in}}%
\pgfpathlineto{\pgfqpoint{2.860000in}{0.743947in}}%
\pgfpathlineto{\pgfqpoint{2.897576in}{0.751225in}}%
\pgfpathlineto{\pgfqpoint{2.935152in}{0.760471in}}%
\pgfpathlineto{\pgfqpoint{2.972727in}{0.771642in}}%
\pgfpathlineto{\pgfqpoint{3.010303in}{0.784693in}}%
\pgfpathlineto{\pgfqpoint{3.047879in}{0.799579in}}%
\pgfpathlineto{\pgfqpoint{3.085455in}{0.816253in}}%
\pgfpathlineto{\pgfqpoint{3.123030in}{0.834665in}}%
\pgfpathlineto{\pgfqpoint{3.179394in}{0.865433in}}%
\pgfpathlineto{\pgfqpoint{3.235758in}{0.899824in}}%
\pgfpathlineto{\pgfqpoint{3.292121in}{0.937656in}}%
\pgfpathlineto{\pgfqpoint{3.348485in}{0.978737in}}%
\pgfpathlineto{\pgfqpoint{3.404848in}{1.022872in}}%
\pgfpathlineto{\pgfqpoint{3.461212in}{1.069853in}}%
\pgfpathlineto{\pgfqpoint{3.517576in}{1.119469in}}%
\pgfpathlineto{\pgfqpoint{3.573939in}{1.171500in}}%
\pgfpathlineto{\pgfqpoint{3.649091in}{1.244234in}}%
\pgfpathlineto{\pgfqpoint{3.724242in}{1.320292in}}%
\pgfpathlineto{\pgfqpoint{3.818182in}{1.419144in}}%
\pgfpathlineto{\pgfqpoint{3.930909in}{1.541715in}}%
\pgfpathlineto{\pgfqpoint{4.269091in}{1.913435in}}%
\pgfpathlineto{\pgfqpoint{4.344242in}{1.992218in}}%
\pgfpathlineto{\pgfqpoint{4.419394in}{2.067933in}}%
\pgfpathlineto{\pgfqpoint{4.475758in}{2.122209in}}%
\pgfpathlineto{\pgfqpoint{4.532121in}{2.173939in}}%
\pgfpathlineto{\pgfqpoint{4.588485in}{2.222755in}}%
\pgfpathlineto{\pgfqpoint{4.626061in}{2.253495in}}%
\pgfpathlineto{\pgfqpoint{4.663636in}{2.282660in}}%
\pgfpathlineto{\pgfqpoint{4.720000in}{2.314888in}}%
\pgfpathlineto{\pgfqpoint{4.776364in}{2.343526in}}%
\pgfpathlineto{\pgfqpoint{4.832727in}{2.368777in}}%
\pgfpathlineto{\pgfqpoint{4.889091in}{2.390833in}}%
\pgfpathlineto{\pgfqpoint{4.945455in}{2.409881in}}%
\pgfpathlineto{\pgfqpoint{4.999007in}{2.425347in}}%
\pgfpathlineto{\pgfqpoint{5.055371in}{2.438946in}}%
\pgfpathlineto{\pgfqpoint{5.111735in}{2.449939in}}%
\pgfpathlineto{\pgfqpoint{5.168098in}{2.458487in}}%
\pgfpathlineto{\pgfqpoint{5.224462in}{2.464749in}}%
\pgfpathlineto{\pgfqpoint{5.280826in}{2.468882in}}%
\pgfpathlineto{\pgfqpoint{5.337189in}{2.471041in}}%
\pgfpathlineto{\pgfqpoint{5.393553in}{2.471378in}}%
\pgfpathlineto{\pgfqpoint{5.468704in}{2.469254in}}%
\pgfpathlineto{\pgfqpoint{5.543856in}{2.464511in}}%
\pgfpathlineto{\pgfqpoint{5.619007in}{2.457492in}}%
\pgfpathlineto{\pgfqpoint{5.712947in}{2.446034in}}%
\pgfpathlineto{\pgfqpoint{5.809697in}{2.431752in}}%
\pgfpathlineto{\pgfqpoint{5.941212in}{2.409638in}}%
\pgfpathlineto{\pgfqpoint{6.392121in}{2.330616in}}%
\pgfpathlineto{\pgfqpoint{6.504848in}{2.314370in}}%
\pgfpathlineto{\pgfqpoint{6.598788in}{2.302948in}}%
\pgfpathlineto{\pgfqpoint{6.692727in}{2.293807in}}%
\pgfpathlineto{\pgfqpoint{6.786667in}{2.287203in}}%
\pgfpathlineto{\pgfqpoint{6.861818in}{2.283899in}}%
\pgfpathlineto{\pgfqpoint{6.918182in}{2.282660in}}%
\pgfpathlineto{\pgfqpoint{6.918182in}{2.282660in}}%
\pgfusepath{stroke}%
\end{pgfscope}%
\begin{pgfscope}%
\pgfpathrectangle{\pgfqpoint{1.000000in}{0.330000in}}{\pgfqpoint{6.200000in}{2.310000in}}%
\pgfusepath{clip}%
\pgfsetrectcap%
\pgfsetroundjoin%
\pgfsetlinewidth{1.505625pt}%
\definecolor{currentstroke}{rgb}{0.172549,0.627451,0.172549}%
\pgfsetstrokecolor{currentstroke}%
\pgfsetdash{}{0pt}%
\pgfpathmoveto{\pgfqpoint{1.281818in}{2.282660in}}%
\pgfpathlineto{\pgfqpoint{1.526061in}{2.203968in}}%
\pgfpathlineto{\pgfqpoint{1.676364in}{2.157912in}}%
\pgfpathlineto{\pgfqpoint{1.789091in}{2.125391in}}%
\pgfpathlineto{\pgfqpoint{1.901818in}{2.095104in}}%
\pgfpathlineto{\pgfqpoint{1.995758in}{2.071886in}}%
\pgfpathlineto{\pgfqpoint{2.089697in}{2.050763in}}%
\pgfpathlineto{\pgfqpoint{2.183636in}{2.031966in}}%
\pgfpathlineto{\pgfqpoint{2.277576in}{2.015719in}}%
\pgfpathlineto{\pgfqpoint{2.352727in}{2.004703in}}%
\pgfpathlineto{\pgfqpoint{2.427879in}{1.995868in}}%
\pgfpathlineto{\pgfqpoint{2.503030in}{1.989879in}}%
\pgfpathlineto{\pgfqpoint{2.578182in}{1.985771in}}%
\pgfpathlineto{\pgfqpoint{2.653333in}{1.983482in}}%
\pgfpathlineto{\pgfqpoint{2.747273in}{1.983078in}}%
\pgfpathlineto{\pgfqpoint{2.841212in}{1.985278in}}%
\pgfpathlineto{\pgfqpoint{2.935152in}{1.989937in}}%
\pgfpathlineto{\pgfqpoint{3.029091in}{1.996903in}}%
\pgfpathlineto{\pgfqpoint{3.123030in}{2.006014in}}%
\pgfpathlineto{\pgfqpoint{3.216970in}{2.017099in}}%
\pgfpathlineto{\pgfqpoint{3.329697in}{2.032754in}}%
\pgfpathlineto{\pgfqpoint{3.442424in}{2.050666in}}%
\pgfpathlineto{\pgfqpoint{3.573939in}{2.073955in}}%
\pgfpathlineto{\pgfqpoint{3.724242in}{2.103015in}}%
\pgfpathlineto{\pgfqpoint{3.930909in}{2.145592in}}%
\pgfpathlineto{\pgfqpoint{4.269091in}{2.215498in}}%
\pgfpathlineto{\pgfqpoint{4.419394in}{2.243945in}}%
\pgfpathlineto{\pgfqpoint{4.532121in}{2.263205in}}%
\pgfpathlineto{\pgfqpoint{4.626061in}{2.277478in}}%
\pgfpathlineto{\pgfqpoint{4.682424in}{2.284603in}}%
\pgfpathlineto{\pgfqpoint{4.776364in}{2.293132in}}%
\pgfpathlineto{\pgfqpoint{4.889091in}{2.300969in}}%
\pgfpathlineto{\pgfqpoint{4.999007in}{2.306393in}}%
\pgfpathlineto{\pgfqpoint{5.130523in}{2.310414in}}%
\pgfpathlineto{\pgfqpoint{5.262038in}{2.312189in}}%
\pgfpathlineto{\pgfqpoint{5.412341in}{2.312022in}}%
\pgfpathlineto{\pgfqpoint{5.600220in}{2.309385in}}%
\pgfpathlineto{\pgfqpoint{5.847273in}{2.303424in}}%
\pgfpathlineto{\pgfqpoint{6.467273in}{2.287309in}}%
\pgfpathlineto{\pgfqpoint{6.711515in}{2.283764in}}%
\pgfpathlineto{\pgfqpoint{6.918182in}{2.282660in}}%
\pgfpathlineto{\pgfqpoint{6.918182in}{2.282660in}}%
\pgfusepath{stroke}%
\end{pgfscope}%
\begin{pgfscope}%
\pgfpathrectangle{\pgfqpoint{1.000000in}{0.330000in}}{\pgfqpoint{6.200000in}{2.310000in}}%
\pgfusepath{clip}%
\pgfsetrectcap%
\pgfsetroundjoin%
\pgfsetlinewidth{1.505625pt}%
\definecolor{currentstroke}{rgb}{0.839216,0.152941,0.156863}%
\pgfsetstrokecolor{currentstroke}%
\pgfsetdash{}{0pt}%
\pgfpathmoveto{\pgfqpoint{1.281818in}{2.282660in}}%
\pgfpathlineto{\pgfqpoint{1.450909in}{2.042911in}}%
\pgfpathlineto{\pgfqpoint{1.563636in}{1.886970in}}%
\pgfpathlineto{\pgfqpoint{1.657576in}{1.761146in}}%
\pgfpathlineto{\pgfqpoint{1.732727in}{1.664014in}}%
\pgfpathlineto{\pgfqpoint{1.807879in}{1.570632in}}%
\pgfpathlineto{\pgfqpoint{1.883030in}{1.481550in}}%
\pgfpathlineto{\pgfqpoint{1.939394in}{1.417882in}}%
\pgfpathlineto{\pgfqpoint{1.995758in}{1.357153in}}%
\pgfpathlineto{\pgfqpoint{2.052121in}{1.299575in}}%
\pgfpathlineto{\pgfqpoint{2.108485in}{1.245358in}}%
\pgfpathlineto{\pgfqpoint{2.164848in}{1.194702in}}%
\pgfpathlineto{\pgfqpoint{2.221212in}{1.147805in}}%
\pgfpathlineto{\pgfqpoint{2.277576in}{1.104857in}}%
\pgfpathlineto{\pgfqpoint{2.315152in}{1.078512in}}%
\pgfpathlineto{\pgfqpoint{2.352727in}{1.054057in}}%
\pgfpathlineto{\pgfqpoint{2.390303in}{1.031546in}}%
\pgfpathlineto{\pgfqpoint{2.409091in}{1.021035in}}%
\pgfpathlineto{\pgfqpoint{2.446667in}{1.003564in}}%
\pgfpathlineto{\pgfqpoint{2.484242in}{0.988096in}}%
\pgfpathlineto{\pgfqpoint{2.521818in}{0.974615in}}%
\pgfpathlineto{\pgfqpoint{2.559394in}{0.963100in}}%
\pgfpathlineto{\pgfqpoint{2.596970in}{0.953534in}}%
\pgfpathlineto{\pgfqpoint{2.634545in}{0.945894in}}%
\pgfpathlineto{\pgfqpoint{2.672121in}{0.940158in}}%
\pgfpathlineto{\pgfqpoint{2.709697in}{0.936303in}}%
\pgfpathlineto{\pgfqpoint{2.747273in}{0.934304in}}%
\pgfpathlineto{\pgfqpoint{2.784848in}{0.934135in}}%
\pgfpathlineto{\pgfqpoint{2.822424in}{0.935769in}}%
\pgfpathlineto{\pgfqpoint{2.860000in}{0.939177in}}%
\pgfpathlineto{\pgfqpoint{2.897576in}{0.944329in}}%
\pgfpathlineto{\pgfqpoint{2.935152in}{0.951194in}}%
\pgfpathlineto{\pgfqpoint{2.972727in}{0.959740in}}%
\pgfpathlineto{\pgfqpoint{3.010303in}{0.969932in}}%
\pgfpathlineto{\pgfqpoint{3.047879in}{0.981736in}}%
\pgfpathlineto{\pgfqpoint{3.104242in}{1.002382in}}%
\pgfpathlineto{\pgfqpoint{3.160606in}{1.026443in}}%
\pgfpathlineto{\pgfqpoint{3.216970in}{1.053782in}}%
\pgfpathlineto{\pgfqpoint{3.273333in}{1.084257in}}%
\pgfpathlineto{\pgfqpoint{3.329697in}{1.117718in}}%
\pgfpathlineto{\pgfqpoint{3.386061in}{1.154007in}}%
\pgfpathlineto{\pgfqpoint{3.442424in}{1.192960in}}%
\pgfpathlineto{\pgfqpoint{3.498788in}{1.234406in}}%
\pgfpathlineto{\pgfqpoint{3.555152in}{1.278166in}}%
\pgfpathlineto{\pgfqpoint{3.630303in}{1.339788in}}%
\pgfpathlineto{\pgfqpoint{3.705455in}{1.404731in}}%
\pgfpathlineto{\pgfqpoint{3.780606in}{1.472506in}}%
\pgfpathlineto{\pgfqpoint{3.874545in}{1.560429in}}%
\pgfpathlineto{\pgfqpoint{3.987273in}{1.669240in}}%
\pgfpathlineto{\pgfqpoint{4.306667in}{1.980252in}}%
\pgfpathlineto{\pgfqpoint{4.400606in}{2.067320in}}%
\pgfpathlineto{\pgfqpoint{4.475758in}{2.133834in}}%
\pgfpathlineto{\pgfqpoint{4.532121in}{2.181399in}}%
\pgfpathlineto{\pgfqpoint{4.588485in}{2.226612in}}%
\pgfpathlineto{\pgfqpoint{4.626061in}{2.255283in}}%
\pgfpathlineto{\pgfqpoint{4.663636in}{2.282660in}}%
\pgfpathlineto{\pgfqpoint{4.720000in}{2.314548in}}%
\pgfpathlineto{\pgfqpoint{4.776364in}{2.343287in}}%
\pgfpathlineto{\pgfqpoint{4.832727in}{2.368999in}}%
\pgfpathlineto{\pgfqpoint{4.889091in}{2.391801in}}%
\pgfpathlineto{\pgfqpoint{4.945455in}{2.411813in}}%
\pgfpathlineto{\pgfqpoint{5.017795in}{2.433602in}}%
\pgfpathlineto{\pgfqpoint{5.074159in}{2.447692in}}%
\pgfpathlineto{\pgfqpoint{5.130523in}{2.459385in}}%
\pgfpathlineto{\pgfqpoint{5.186886in}{2.468799in}}%
\pgfpathlineto{\pgfqpoint{5.243250in}{2.476055in}}%
\pgfpathlineto{\pgfqpoint{5.299613in}{2.481272in}}%
\pgfpathlineto{\pgfqpoint{5.355977in}{2.484568in}}%
\pgfpathlineto{\pgfqpoint{5.412341in}{2.486064in}}%
\pgfpathlineto{\pgfqpoint{5.487492in}{2.485465in}}%
\pgfpathlineto{\pgfqpoint{5.562644in}{2.482160in}}%
\pgfpathlineto{\pgfqpoint{5.637795in}{2.476433in}}%
\pgfpathlineto{\pgfqpoint{5.712947in}{2.468566in}}%
\pgfpathlineto{\pgfqpoint{5.790909in}{2.458448in}}%
\pgfpathlineto{\pgfqpoint{5.884848in}{2.444050in}}%
\pgfpathlineto{\pgfqpoint{5.997576in}{2.424329in}}%
\pgfpathlineto{\pgfqpoint{6.147879in}{2.395454in}}%
\pgfpathlineto{\pgfqpoint{6.429697in}{2.340745in}}%
\pgfpathlineto{\pgfqpoint{6.542424in}{2.321347in}}%
\pgfpathlineto{\pgfqpoint{6.636364in}{2.307321in}}%
\pgfpathlineto{\pgfqpoint{6.730303in}{2.295829in}}%
\pgfpathlineto{\pgfqpoint{6.805455in}{2.288832in}}%
\pgfpathlineto{\pgfqpoint{6.880606in}{2.284094in}}%
\pgfpathlineto{\pgfqpoint{6.918182in}{2.282660in}}%
\pgfpathlineto{\pgfqpoint{6.918182in}{2.282660in}}%
\pgfusepath{stroke}%
\end{pgfscope}%
\begin{pgfscope}%
\pgfpathrectangle{\pgfqpoint{1.000000in}{0.330000in}}{\pgfqpoint{6.200000in}{2.310000in}}%
\pgfusepath{clip}%
\pgfsetrectcap%
\pgfsetroundjoin%
\pgfsetlinewidth{1.505625pt}%
\definecolor{currentstroke}{rgb}{0.580392,0.403922,0.741176}%
\pgfsetstrokecolor{currentstroke}%
\pgfsetdash{}{0pt}%
\pgfpathmoveto{\pgfqpoint{1.281818in}{2.282660in}}%
\pgfpathlineto{\pgfqpoint{1.488485in}{2.173844in}}%
\pgfpathlineto{\pgfqpoint{1.620000in}{2.107138in}}%
\pgfpathlineto{\pgfqpoint{1.732727in}{2.052685in}}%
\pgfpathlineto{\pgfqpoint{1.826667in}{2.009810in}}%
\pgfpathlineto{\pgfqpoint{1.920606in}{1.969661in}}%
\pgfpathlineto{\pgfqpoint{1.995758in}{1.939779in}}%
\pgfpathlineto{\pgfqpoint{2.070909in}{1.912105in}}%
\pgfpathlineto{\pgfqpoint{2.146061in}{1.886835in}}%
\pgfpathlineto{\pgfqpoint{2.221212in}{1.864161in}}%
\pgfpathlineto{\pgfqpoint{2.296364in}{1.844265in}}%
\pgfpathlineto{\pgfqpoint{2.371515in}{1.827327in}}%
\pgfpathlineto{\pgfqpoint{2.427879in}{1.817177in}}%
\pgfpathlineto{\pgfqpoint{2.484242in}{1.809837in}}%
\pgfpathlineto{\pgfqpoint{2.540606in}{1.804249in}}%
\pgfpathlineto{\pgfqpoint{2.596970in}{1.800370in}}%
\pgfpathlineto{\pgfqpoint{2.672121in}{1.797777in}}%
\pgfpathlineto{\pgfqpoint{2.747273in}{1.798027in}}%
\pgfpathlineto{\pgfqpoint{2.822424in}{1.801002in}}%
\pgfpathlineto{\pgfqpoint{2.897576in}{1.806576in}}%
\pgfpathlineto{\pgfqpoint{2.972727in}{1.814618in}}%
\pgfpathlineto{\pgfqpoint{3.047879in}{1.824990in}}%
\pgfpathlineto{\pgfqpoint{3.123030in}{1.837551in}}%
\pgfpathlineto{\pgfqpoint{3.198182in}{1.852150in}}%
\pgfpathlineto{\pgfqpoint{3.292121in}{1.873030in}}%
\pgfpathlineto{\pgfqpoint{3.386061in}{1.896536in}}%
\pgfpathlineto{\pgfqpoint{3.480000in}{1.922338in}}%
\pgfpathlineto{\pgfqpoint{3.592727in}{1.955846in}}%
\pgfpathlineto{\pgfqpoint{3.724242in}{1.997654in}}%
\pgfpathlineto{\pgfqpoint{3.893333in}{2.054090in}}%
\pgfpathlineto{\pgfqpoint{4.193939in}{2.154876in}}%
\pgfpathlineto{\pgfqpoint{4.306667in}{2.190433in}}%
\pgfpathlineto{\pgfqpoint{4.400606in}{2.218212in}}%
\pgfpathlineto{\pgfqpoint{4.494545in}{2.243788in}}%
\pgfpathlineto{\pgfqpoint{4.569697in}{2.262326in}}%
\pgfpathlineto{\pgfqpoint{4.626061in}{2.274934in}}%
\pgfpathlineto{\pgfqpoint{4.663636in}{2.282660in}}%
\pgfpathlineto{\pgfqpoint{4.738788in}{2.292189in}}%
\pgfpathlineto{\pgfqpoint{4.832727in}{2.301450in}}%
\pgfpathlineto{\pgfqpoint{4.926667in}{2.308197in}}%
\pgfpathlineto{\pgfqpoint{5.017795in}{2.312692in}}%
\pgfpathlineto{\pgfqpoint{5.111735in}{2.315342in}}%
\pgfpathlineto{\pgfqpoint{5.224462in}{2.316228in}}%
\pgfpathlineto{\pgfqpoint{5.355977in}{2.314726in}}%
\pgfpathlineto{\pgfqpoint{5.506280in}{2.310566in}}%
\pgfpathlineto{\pgfqpoint{5.712947in}{2.302361in}}%
\pgfpathlineto{\pgfqpoint{6.072727in}{2.287809in}}%
\pgfpathlineto{\pgfqpoint{6.260606in}{2.282790in}}%
\pgfpathlineto{\pgfqpoint{6.429697in}{2.280419in}}%
\pgfpathlineto{\pgfqpoint{6.617576in}{2.280085in}}%
\pgfpathlineto{\pgfqpoint{6.880606in}{2.282277in}}%
\pgfpathlineto{\pgfqpoint{6.918182in}{2.282660in}}%
\pgfpathlineto{\pgfqpoint{6.918182in}{2.282660in}}%
\pgfusepath{stroke}%
\end{pgfscope}%
\begin{pgfscope}%
\pgfpathrectangle{\pgfqpoint{1.000000in}{0.330000in}}{\pgfqpoint{6.200000in}{2.310000in}}%
\pgfusepath{clip}%
\pgfsetrectcap%
\pgfsetroundjoin%
\pgfsetlinewidth{1.505625pt}%
\definecolor{currentstroke}{rgb}{0.549020,0.337255,0.294118}%
\pgfsetstrokecolor{currentstroke}%
\pgfsetdash{}{0pt}%
\pgfpathmoveto{\pgfqpoint{1.281818in}{2.282660in}}%
\pgfpathlineto{\pgfqpoint{1.563636in}{2.222365in}}%
\pgfpathlineto{\pgfqpoint{1.732727in}{2.188487in}}%
\pgfpathlineto{\pgfqpoint{1.864242in}{2.164170in}}%
\pgfpathlineto{\pgfqpoint{1.995758in}{2.142144in}}%
\pgfpathlineto{\pgfqpoint{2.108485in}{2.125427in}}%
\pgfpathlineto{\pgfqpoint{2.221212in}{2.110986in}}%
\pgfpathlineto{\pgfqpoint{2.333939in}{2.099077in}}%
\pgfpathlineto{\pgfqpoint{2.427879in}{2.091465in}}%
\pgfpathlineto{\pgfqpoint{2.521818in}{2.086672in}}%
\pgfpathlineto{\pgfqpoint{2.615758in}{2.083822in}}%
\pgfpathlineto{\pgfqpoint{2.728485in}{2.082850in}}%
\pgfpathlineto{\pgfqpoint{2.841212in}{2.084405in}}%
\pgfpathlineto{\pgfqpoint{2.953939in}{2.088321in}}%
\pgfpathlineto{\pgfqpoint{3.066667in}{2.094420in}}%
\pgfpathlineto{\pgfqpoint{3.198182in}{2.104042in}}%
\pgfpathlineto{\pgfqpoint{3.329697in}{2.116056in}}%
\pgfpathlineto{\pgfqpoint{3.480000in}{2.132273in}}%
\pgfpathlineto{\pgfqpoint{3.649091in}{2.153027in}}%
\pgfpathlineto{\pgfqpoint{3.855758in}{2.180814in}}%
\pgfpathlineto{\pgfqpoint{4.344242in}{2.247590in}}%
\pgfpathlineto{\pgfqpoint{4.494545in}{2.265567in}}%
\pgfpathlineto{\pgfqpoint{4.626061in}{2.279205in}}%
\pgfpathlineto{\pgfqpoint{4.701212in}{2.285196in}}%
\pgfpathlineto{\pgfqpoint{4.832727in}{2.292459in}}%
\pgfpathlineto{\pgfqpoint{4.945455in}{2.296890in}}%
\pgfpathlineto{\pgfqpoint{5.092947in}{2.300563in}}%
\pgfpathlineto{\pgfqpoint{5.262038in}{2.302346in}}%
\pgfpathlineto{\pgfqpoint{5.449916in}{2.302007in}}%
\pgfpathlineto{\pgfqpoint{5.694159in}{2.299124in}}%
\pgfpathlineto{\pgfqpoint{6.147879in}{2.290852in}}%
\pgfpathlineto{\pgfqpoint{6.523636in}{2.285075in}}%
\pgfpathlineto{\pgfqpoint{6.805455in}{2.282916in}}%
\pgfpathlineto{\pgfqpoint{6.918182in}{2.282660in}}%
\pgfpathlineto{\pgfqpoint{6.918182in}{2.282660in}}%
\pgfusepath{stroke}%
\end{pgfscope}%
\begin{pgfscope}%
\pgfsetroundcap%
\pgfsetroundjoin%
\pgfsetlinewidth{1.003750pt}%
\definecolor{currentstroke}{rgb}{0.000000,0.000000,0.000000}%
\pgfsetstrokecolor{currentstroke}%
\pgfsetdash{}{0pt}%
\pgfpathmoveto{\pgfqpoint{3.214239in}{0.435000in}}%
\pgfpathquadraticcurveto{\pgfqpoint{2.994650in}{0.435000in}}{\pgfqpoint{2.775062in}{0.435000in}}%
\pgfusepath{stroke}%
\end{pgfscope}%
\begin{pgfscope}%
\pgfsetbuttcap%
\pgfsetmiterjoin%
\definecolor{currentfill}{rgb}{0.800000,0.800000,0.800000}%
\pgfsetfillcolor{currentfill}%
\pgfsetlinewidth{1.003750pt}%
\definecolor{currentstroke}{rgb}{0.000000,0.000000,0.000000}%
\pgfsetstrokecolor{currentstroke}%
\pgfsetdash{}{0pt}%
\pgfpathmoveto{\pgfqpoint{3.271964in}{0.338549in}}%
\pgfpathcurveto{\pgfqpoint{3.306686in}{0.303827in}}{\pgfqpoint{4.136162in}{0.303827in}}{\pgfqpoint{4.170884in}{0.338549in}}%
\pgfpathcurveto{\pgfqpoint{4.205606in}{0.373272in}}{\pgfqpoint{4.205606in}{0.496728in}}{\pgfqpoint{4.170884in}{0.531451in}}%
\pgfpathcurveto{\pgfqpoint{4.136162in}{0.566173in}}{\pgfqpoint{3.306686in}{0.566173in}}{\pgfqpoint{3.271964in}{0.531451in}}%
\pgfpathcurveto{\pgfqpoint{3.237241in}{0.496728in}}{\pgfqpoint{3.237241in}{0.373272in}}{\pgfqpoint{3.271964in}{0.338549in}}%
\pgfpathclose%
\pgfusepath{stroke,fill}%
\end{pgfscope}%
\begin{pgfscope}%
\definecolor{textcolor}{rgb}{0.000000,0.000000,0.000000}%
\pgfsetstrokecolor{textcolor}%
\pgfsetfillcolor{textcolor}%
\pgftext[x=4.136162in,y=0.435000in,right,]{\color{textcolor}\rmfamily\fontsize{10.000000}{12.000000}\selectfont \(\displaystyle \Delta =\) -0.3 inch}%
\end{pgfscope}%
\begin{pgfscope}%
\pgfsetbuttcap%
\pgfsetmiterjoin%
\definecolor{currentfill}{rgb}{0.800000,0.800000,0.800000}%
\pgfsetfillcolor{currentfill}%
\pgfsetlinewidth{1.003750pt}%
\definecolor{currentstroke}{rgb}{0.000000,0.000000,0.000000}%
\pgfsetstrokecolor{currentstroke}%
\pgfsetdash{}{0pt}%
\pgfpathmoveto{\pgfqpoint{0.965278in}{0.358599in}}%
\pgfpathcurveto{\pgfqpoint{1.000000in}{0.323877in}}{\pgfqpoint{2.720682in}{0.323877in}}{\pgfqpoint{2.755404in}{0.358599in}}%
\pgfpathcurveto{\pgfqpoint{2.790127in}{0.393321in}}{\pgfqpoint{2.790127in}{0.668784in}}{\pgfqpoint{2.755404in}{0.703506in}}%
\pgfpathcurveto{\pgfqpoint{2.720682in}{0.738228in}}{\pgfqpoint{1.000000in}{0.738228in}}{\pgfqpoint{0.965278in}{0.703506in}}%
\pgfpathcurveto{\pgfqpoint{0.930556in}{0.668784in}}{\pgfqpoint{0.930556in}{0.393321in}}{\pgfqpoint{0.965278in}{0.358599in}}%
\pgfpathclose%
\pgfusepath{stroke,fill}%
\end{pgfscope}%
\begin{pgfscope}%
\definecolor{textcolor}{rgb}{0.000000,0.000000,0.000000}%
\pgfsetstrokecolor{textcolor}%
\pgfsetfillcolor{textcolor}%
\pgftext[x=1.000000in, y=0.580049in, left, base]{\color{textcolor}\rmfamily\fontsize{10.000000}{12.000000}\selectfont Max combo: 1.0D + 1.0L0}%
\end{pgfscope}%
\begin{pgfscope}%
\definecolor{textcolor}{rgb}{0.000000,0.000000,0.000000}%
\pgfsetstrokecolor{textcolor}%
\pgfsetfillcolor{textcolor}%
\pgftext[x=1.000000in, y=0.428043in, left, base]{\color{textcolor}\rmfamily\fontsize{10.000000}{12.000000}\selectfont ASCE7-16 Sec. 2.4.1 (LC 2)}%
\end{pgfscope}%
\begin{pgfscope}%
\pgfsetroundcap%
\pgfsetroundjoin%
\pgfsetlinewidth{1.003750pt}%
\definecolor{currentstroke}{rgb}{0.000000,0.000000,0.000000}%
\pgfsetstrokecolor{currentstroke}%
\pgfsetdash{}{0pt}%
\pgfpathmoveto{\pgfqpoint{3.233027in}{0.933993in}}%
\pgfpathquadraticcurveto{\pgfqpoint{3.013438in}{0.933993in}}{\pgfqpoint{2.793849in}{0.933993in}}%
\pgfusepath{stroke}%
\end{pgfscope}%
\begin{pgfscope}%
\pgfsetbuttcap%
\pgfsetmiterjoin%
\definecolor{currentfill}{rgb}{0.800000,0.800000,0.800000}%
\pgfsetfillcolor{currentfill}%
\pgfsetlinewidth{1.003750pt}%
\definecolor{currentstroke}{rgb}{0.000000,0.000000,0.000000}%
\pgfsetstrokecolor{currentstroke}%
\pgfsetdash{}{0pt}%
\pgfpathmoveto{\pgfqpoint{3.290751in}{0.837542in}}%
\pgfpathcurveto{\pgfqpoint{3.325474in}{0.802820in}}{\pgfqpoint{4.154949in}{0.802820in}}{\pgfqpoint{4.189672in}{0.837542in}}%
\pgfpathcurveto{\pgfqpoint{4.224394in}{0.872264in}}{\pgfqpoint{4.224394in}{0.995721in}}{\pgfqpoint{4.189672in}{1.030443in}}%
\pgfpathcurveto{\pgfqpoint{4.154949in}{1.065165in}}{\pgfqpoint{3.325474in}{1.065165in}}{\pgfqpoint{3.290751in}{1.030443in}}%
\pgfpathcurveto{\pgfqpoint{3.256029in}{0.995721in}}{\pgfqpoint{3.256029in}{0.872264in}}{\pgfqpoint{3.290751in}{0.837542in}}%
\pgfpathclose%
\pgfusepath{stroke,fill}%
\end{pgfscope}%
\begin{pgfscope}%
\definecolor{textcolor}{rgb}{0.000000,0.000000,0.000000}%
\pgfsetstrokecolor{textcolor}%
\pgfsetfillcolor{textcolor}%
\pgftext[x=4.154949in,y=0.933993in,right,]{\color{textcolor}\rmfamily\fontsize{10.000000}{12.000000}\selectfont \(\displaystyle \Delta =\) -0.2 inch}%
\end{pgfscope}%
\begin{pgfscope}%
\pgfsetbuttcap%
\pgfsetmiterjoin%
\definecolor{currentfill}{rgb}{0.800000,0.800000,0.800000}%
\pgfsetfillcolor{currentfill}%
\pgfsetlinewidth{1.003750pt}%
\definecolor{currentstroke}{rgb}{0.000000,0.000000,0.000000}%
\pgfsetstrokecolor{currentstroke}%
\pgfsetdash{}{0pt}%
\pgfpathmoveto{\pgfqpoint{0.965278in}{0.375574in}}%
\pgfpathcurveto{\pgfqpoint{1.000000in}{0.340852in}}{\pgfqpoint{2.390820in}{0.340852in}}{\pgfqpoint{2.425542in}{0.375574in}}%
\pgfpathcurveto{\pgfqpoint{2.460265in}{0.410297in}}{\pgfqpoint{2.460265in}{0.676500in}}{\pgfqpoint{2.425542in}{0.711222in}}%
\pgfpathcurveto{\pgfqpoint{2.390820in}{0.745944in}}{\pgfqpoint{1.000000in}{0.745944in}}{\pgfqpoint{0.965278in}{0.711222in}}%
\pgfpathcurveto{\pgfqpoint{0.930556in}{0.676500in}}{\pgfqpoint{0.930556in}{0.410297in}}{\pgfqpoint{0.965278in}{0.375574in}}%
\pgfpathclose%
\pgfusepath{stroke,fill}%
\end{pgfscope}%
\begin{pgfscope}%
\definecolor{textcolor}{rgb}{0.000000,0.000000,0.000000}%
\pgfsetstrokecolor{textcolor}%
\pgfsetfillcolor{textcolor}%
\pgftext[x=1.000000in, y=0.580049in, left, base]{\color{textcolor}\rmfamily\fontsize{10.000000}{12.000000}\selectfont Max combo: 1.0L0}%
\end{pgfscope}%
\begin{pgfscope}%
\definecolor{textcolor}{rgb}{0.000000,0.000000,0.000000}%
\pgfsetstrokecolor{textcolor}%
\pgfsetfillcolor{textcolor}%
\pgftext[x=1.000000in, y=0.437303in, left, base]{\color{textcolor}\rmfamily\fontsize{10.000000}{12.000000}\selectfont L only deflection check}%
\end{pgfscope}%
\end{pgfpicture}%
\makeatother%
\endgroup%

\end{center}
\caption{Deflection Envelope}
\end{figure}
Tl Deflection Check: 
$\Delta_{max} = -0.26 {\color{darkBlue}{\mathbf{ \; in}}} = \cfrac{L}{681} < \cfrac{L}{1.0}  \;  \mathbf{(OK)}$\\
\bigbreak
Ll Deflection Check: 
$\Delta_{max} = -0.19 {\color{darkBlue}{\mathbf{ \; in}}} = \cfrac{L}{933} < \cfrac{L}{1.0}  \;  \mathbf{(OK)}$\\
\bigbreak
\vspace{-30pt}
%	---------------------------------- REACTIONS ---------------------------------
\section{Reactions}
The following is a summary of service-level reactions at each support:
\begin{table}[ht]
\caption{Reactions at Supports}
\centering
\begin{tabular}{l l l l l }
\hline
Loc. & Type & D & L0 & L1\\
\hline
0 {\color{darkBlue}{\textbf{ft}}} & Shear & 4.0 {\color{darkBlue}{\textbf{kip}}} & 9.0 {\color{darkBlue}{\textbf{kip}}} & -0.0 {\color{darkBlue}{\textbf{kip}}}\\ 
15 {\color{darkBlue}{\textbf{ft}}} & Shear & 3.4 {\color{darkBlue}{\textbf{kip}}} & 8.4 {\color{darkBlue}{\textbf{kip}}} & 1.0 {\color{darkBlue}{\textbf{kip}}}\\ 
25 {\color{darkBlue}{\textbf{ft}}} & Shear & -0.1 {\color{darkBlue}{\textbf{kip}}} & -2.2 {\color{darkBlue}{\textbf{kip}}} & 1.1 {\color{darkBlue}{\textbf{kip}}}\\ 
25 {\color{darkBlue}{\textbf{ft}}} & Moment & 1.0 {\color{darkBlue}{\textbf{kft}}} & 6.8 {\color{darkBlue}{\textbf{kft}}} & -1.8 {\color{darkBlue}{\textbf{kft}}}\\ 
\hline
\end{tabular}
\end{table}
\end{document}